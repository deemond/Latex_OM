\section{Бортовой журнал технического состояния воздушного судна}

В АО «ЮТэйр» используются бортовые журналы ВС отечественного производства (ОП) и бортовые журналы ВС иностранного производства (ИП) CL 300.

\paragraph{} Бортовой журнал технического состояния ВС отечественного производства

Бортовой журнал самолёта предназначен для контроля за техническим состоянием и оформлением приёма-передачи воздушного судна. При выполнении полета бортовой журнал должен находиться на борту ВС.

В авиакомпании используется типовой бортовой журнал, введенный указанием МГА от 01.11.1975г. №161, включающий в себя XI разделов.

\paragraph{} Бортовой журнал технического состояния CL 300 

Бортовой журнал ВС CL 300 предназначен для контроля за техническим состоянием ВС и полнотой выполнения работ, предусмотренных программой технического обслуживания данного ВС, оформления процедуры приема-передачи имущества и документации ВС между ИТП и ЛС или между экипажами (когда ВС не передается ИАС).

В бортовом журнале (далее – БЖ) вносятся также отказы и неисправности АТ, обнаруженные в полете экипажем (далее - ЛС) и техническим составом (далее – ИТП) во время технического обслуживания ВС (далее - ТО) в (вне) аэропорта с базовой станцией ТО, и информация ИТП о способе их устранения.

Примечание. Порядок использования бортовых журналов ОП и ИП описан в Приложении А 8.11.


\section{Документы находящиеся на ВС во время полёта}

\subsection{Общие положения}

На воздушном судне должны находиться документы, которые члены экипажа воздушного судна предъявляют по требованию уполномоченных должностных лиц. Эти документы условно разделяются на судовые и полётные.
К группе судовых относятся документы:
	подтверждающие соответствие Авиакомпании и ВС требованиям авиационного законодательства;
	необходимые экипажу в нормальных, аварийных и нештатных ситуациях.
Отличительными особенностями судовых документов являются длительный срок действия и постоянное нахождение на борту ВС.
К группе полётных относятся документы:
	подтверждающие квалификацию членов экипажа;
	используемые экипажем для подготовки и выполнения полёта.
Отличительными особенностями полётных документов являются:
	кратковременный (только на предстоящий пролёт, либо серию полётов) срок действия;
	нахождение на ВС только на период выполнения полёта так как на ВС их доставляет экипаж;
	члены экипажа несут ответственность за актуализацию документов. 
Подготовка судовой документации - Оформление (проверка) документа перед размещением на ВС в соответствие с действующими правилами.
Комплектация судовой документации - Доставка документов на борт ВС, размещение их в установленном месте.
Ведение судовой документации – поддержание документации в состоянии пригодном для пользования, замена листов, обложек, пришедших в негодность, а также внесение изменений и дополнений в экземпляры документов.
Полеты ВС без действующих судовых и полетных документов запрещаются.
Перечень судовой и полетной документации определяется документами уполномоченного органа в области ГА и приказами (указаниями) директора авиакомпании. (Приложение А-8.7 к настоящей главе и Часть А-13 «Отчетная документация о полете»). 
Судовые и полётные документы могут находиться на ВС на бумажных или электронных носителях.

\subsection{Судовые документы}

8.6.2.1. Учет и хранение судовой документации, организация ее приема-передачи 
(1) Комплектация судовых документов ВС осуществляется согласно Приложению А-8.7.
(2) Комплектация портфеля судовыми документами осуществляется ведущим инженером по судовой документации Управления поддержания лётной годности ВС ИАС Авиакомпании (далее – УПЛГ ВС ИАС Авиакомпании). 
(3) Сотрудники Авиакомпании, ответственные за первоначальную подготовку (замену) судовых документов (Приложение А-8.7 к настоящей главе) передают их в УПЛГ ВС ИАС Авиакомпании в готовом для размещения на ВС виде и в нужном количестве, при необходимости с приложением инструкции по внесению и с указанием сроков размещения на ВС.
Копии документов заверяются лицом, ответственным за подготовку документа, при этом ответственный за подготовку документа должен быть наделен данным полномочием в соответствии с приказом Авиакомпании. 
(4) Должностные лица УПЛГ ВС ИАС Авиакомпании, маркируют судовые документы государственным и регистрационным знаками экземпляра ВС. 
(5) Ведущий инженер по судовой документации организовывает доставку портфеля с судовой документацией к месту базирования ВС и осуществляет контроль по размещению портфеля на ВС.
(6) Ответственность за сохранность судовой документации и бортового журнала несет работник, принявший бортовой журнал и портфель с судовой документацией:
	бортинженер (бортмеханик), 2й пилот – во время выполнения рейса;
	начальник участка ТО ВС (руководитель ТО ВС) – при выполнении оперативного технического обслуживания;
	должностное лицо УПЛГ ВС ИАС Авиакомпании – при передаче судовой документации и бортового журнала в ИАС, при длительных перерывах в полетах, выполнении периодического технического обслуживания и хранении ВС.
Ответственным за сохранность судовой документации, находящейся на борту ВС, является ИТП или член экипажа, принявший самолет под роспись в бортовом журнале. 
(7) Прием-передача судовой документации от ИТП экипажу, между членами экипажа и ИТП, осуществляется при приеме-передаче ВС и подтверждается подписями в разделе X бортового журнала «Прием-передача ВС».
При приеме специалист проверяет наличие документов, согласно Перечню судовой документации, расписывается за прием, а сдающий расписывается за сдачу. 
(8) Ответственный за сохранность судовой документации обязан:
	при приеме судовой документации проверять ее комплектность по Перечню;
	не оставлять судовую документацию без присмотра.
(9) В случае обнаружения при приеме-передаче ВС некомплектности, порчи судовой документации, должностное лицо, обнаружившее некомплектность, сообщает об этом в УПЛГ ВС ведущему инженеру по судовой документации для организации оперативного доукомплектования судовой документации. 
(10) При нахождении ВС на периодическом ТО, начальник участка ТО ВС (руководитель ТО ВС) передает портфель с судовыми документами в УПЛГ ВС ИАС Авиакомпании для проверки комплектности и актуальности судовой документации ведущим инженером по судовой документации.
(11) В период ремонта ВС судовая документация хранится в УПЛГ ВС ИАС Авиакомпании. При необходимости, до поступления ВС из ремонта, судовая документация данного ВС может быть передана другому ВС и маркирована его государственным и регистрационным знаками.  
 (12)	Экипаж ВС и ИТП обязаны обеспечить надлежащее использование, хранение и прием-передачу судовых документов. 
В промежуточных аэропортах (аэропортах временного базирования) судовая документация предъявляется для проверки только по требованию инспекторского состава уполномоченного органа воздушного транспорта или Авиакомпании в присутствии экипажа. 
Примечание: Допускается хранение судовой документации на воздушном судне, если оно в установленном порядке сдается под охрану САБ авиапредприятия, а на оперативной точке (временном аэродроме) – должностному лицу службы охраны Заказчика.
8.6.2.2. Особенности ведения судовой документации
(1) Документы, которые УПЛГ ВС получает от других подразделений Авиакомпании
Для внесения и обновления судовой документации на самолетах, отдел летных стандартов летной службы Авиакомпании предоставляет в УПЛГ ВС документы и ревизии к следующим документам:
	Руководство по летной эксплуатации ВС;
	Сборник рекомендаций для экипажа по действиям в особых случаях полета;
	Карты контрольных проверок; 
	Листы осмотра ВС экипажем;
	Сертификат эксплуатанта коммерческих воздушных перевозок и эксплуатационные спецификации;
	Сертификат эксплуатанта авиационных работ и эксплуатационная спецификация;
	Свидетельство АОН с приложениями;
	Формы донесений;
	Документы по авиационной безопасности;
Документ, поправка к документу или ревизия передаются в УПЛГ ВС ведущему инженеру по судовой документации в готовом для размещения на ВС виде, необходимом количестве бумажных экземпляров и при необходимости с краткой инструкцией по внесению в бортовой экземпляр документа. Срок размещения на ВС 15 календарных дней с момента поступления документа в УПЛГ ВС.
Руководство по производству полетов. 
РПП разрабатывается летной службой Авиакомпании и является основным нормативным документом авиакомпании, регламентирующим организацию летной работы и правил выполнения полетов.
Бортовой экземпляр РПП хранится в электронной системе бортовой документации Electronic Flight Bag (EFB). Членам экипажа организован доступ к EFB через персональные электронные устройства (IPAD). 
Ответственным за размещение актуальной версии бортовых экземпляров РПП в EFB является начальник отдела аэронавигационного обеспечения лётной службы Авиакомпании.
Порядок внесения поправок в бортовые экземпляры РПП и их плановых поверок изложены в Руководстве по производству полетов Авиакомпании (глава А-0).
Руководство по летной эксплуатации (РЛЭ).
Держателем контрольного экземпляра РЛЭ и ответственным за его ведение является отдел летных стандартов, который вносит изменения (дополнения) в контрольный экземпляр и обеспечивает тиражирование в необходимом количестве и передачу на бумажных носителях изменений (дополнений) в УПЛГ ВС для дальнейшего размещения в рабочих экземплярах на ВС.
Внесение изменений в бортовые экземпляры РЛЭ выполняет ведущий инженер по судовой документации или ответственный сотрудник из числа сотрудников летной службы либо ИАС, в течение 15 календарных дней с момента поступления изменений в УПЛГ ВС. 
Сверка рабочих экземпляров РЛЭ с контрольным экземпляром проводится 2 раза в год при подготовке ВС к ОЗП и ВЛП. 
Сборник рекомендаций по действиям экипажа в аварийных ситуациях. 
Держателем контрольного экземпляра Сборника рекомендаций по действиям экипажа в аварийных ситуациях (далее – Сборник) и ответственным за его ведение в авиакомпании является специалист отдела летных стандартов, который вносит изменения (дополнения) в контрольный экземпляр и обеспечивает тиражирование бумажных экземпляров и передачу в необходимом количестве изменений (дополнений) в УПЛГ ВС ИАС для размещения на ВС.  
Сертификат эксплуатанта коммерческих воздушных перевозок и авиационных работ с эксплуатационными спецификациями, а также свидетельство АОН с приложениями.
Копии данных документов, заверенные печатью Авиакомпании, передают в УПЛ ВС. Ведущий инженер по судовой документации в течение 5 дней обеспечивает их размещение на борту ВС и в течение 15 календарных суток при нахождении ВС на аэродромах оперативного базирования. В любом случае Эксплуатационные спецификации должны быть размещены на борту ВС до начала эксплуатация ВС в новых условиях.
 Страховые полисы.
В состав судовой документации входят страховые полисы обязательного страхования ответственности Авиакомпании перед третьими лицами, пассажирами и груза на каждое ВС, а также страхование членов экипажа от несчастного случая. Заказ и продление страховых полисов осуществляет коммерческий отдел Авиакомпании. Специалисты отдела передают заверенные копии страховых полисов, в необходимом количестве в УПЛГ ВС, не позднее 7 рабочих дней до начала их срока действия, для своевременного размещения на борту самолета. 
Свидетельство о регистрации гражданского ВС.
Выдается на весь календарный срок службы гражданского воздушного судна уполномоченным органом гражданской авиации, после регистрации его в Государственном реестре гражданских ВС.
Замена Свидетельства, выдача дубликата или внесение изменений производится в следующих случаях:
	изменение собственника ВС;
	изменение назначения ВС после его переоборудования;
	порчи или утери Свидетельства.
На ВС находится оригинал Свидетельства. Допускается размещение на ВС копии Свидетельства в течении 30 дней с даты его выдачи.
Информация по инженерно-авиационному обеспечению полетов воздушных судов (выписка из РОТО).
Выписка из РОТО разрабатывается ИАС Авиакомпании и является основным нормативным документом авиакомпании, регламентирующим организацию технического обслуживания и ремонта авиационной техники.     Сверка данного документа производится ведущим инженером по судовой документации УПЛГ ВС на основании информации, предоставленной ОТК ИАС. 
Карта контрольных проверок самолета и листы осмотра ВС экипажем.
Управляются отделом летных стандартов на основании РЛЭ ВС.
Количество экземпляров карт контрольных проверок и листов осмотра ВС экипажем на борту ВС, должно соответствовать количеству членов экипажа типа ВС.
Документы по авиационной безопасности.
Документы по авиационной безопасности, для размещения их на ВС, поступают в УПЛГ ВС ИАС Авиакомпании от специалиста по авиационной безопасности Авиакомпании. Об изменении документов, выходе новой ревизии, специалист по авиационной безопасности оповещает всех заинтересованных лиц посредством электронной почты и предоставляет на бумажном носителе необходимое количество экземпляров для размещения/замены на ВС, с указанием сроков размещения. Документы по авиационной безопасности размещаются на ВС в виде копий.
Санитарный журнал.
Контроль санитарного состояния воздушных судов осуществляется в соответствии с требованиями РПП Авиакомпании, РОТО и документа КД-ДП-В5.015 «Организация санитарно-противоэпидемических мероприятий на воздушных судах Группы «ЮТэйр».
 Проверка санитарного состояния воздушных судов проводится:
	постоянно - перед каждым вылетом;
	периодически с записью в Санитарном журнале воздушного судна – не реже одного раза в 3 месяца.       
Дополнительно периодическая проверка проводится после трудоемких регламентных работ, при поступлении воздушного судна с завода, при подготовке к полетам в осенне-зимний и весеннее-летний периоды.
При выполнении полетов за пределами Российской Федерации:
	ответственным за организацию и контроль проверки санитарного состояния воздушного судна назначается командир авиагруппы;
	лицом, уполномоченным проводить контроль санитарного состояния воздушного судна с записью в санитарном журнале воздушного судна, назначается бортпроводник (бортоператор). 
Организация оперативной замены санитарного журнала ВС в случае окончания, утраты или порчи, обеспечивается инженерно-авиационной службой.
           (2)  Документы, выданные авиационными властями Сертификат летной годности 
                                                                гражданского ВС.
Выдается (продлевается) уполномоченным органом гражданской авиации на период действующих сроков службы (ресурсов) экземпляра ВС его технического состояния, но на срок не более двух лет.
Порядок выдачи, замены и продления срока действия Сертификата летной годности ВС регламентируется Федеральными авиационными правилами. 
 На ВС находится оригинал СЛГ. Допускается размещение на ВС копии СЛГ в течение 30 дней от даты его выдачи.
Удостоверение о годности ГВС по шуму на местности.
Выдается и меняется уполномоченным органом в области гражданской авиации сроком на 2 года.
Продление срока действия Удостоверения в процессе эксплуатации производит межрегиональное территориальное управление Росавиации.
На ВС находится оригинал Удостоверения. Допускается размещение на ВС копии Удостоверения в течение 30 дней от даты его выдачи.
Разрешение на бортовую радиостанцию.
Выдается уполномоченным органом в области гражданской авиации. Срок действия документа – бессрочный. 
На ВС находится оригинал Разрешения. Допускается размещение на ВС Разрешения в течение 30 дней от даты его выдачи.

\subsection{Полётные документы}

(1)	Комплектация пакета полётных документов осуществляется членами экипажа ВС. Командир ВС осуществляет контроль полноты и качества полётных документов. 
(2)	Лица, ответственные за первоначальную подготовку (замену) полётных документов (Таблица А8.6-Т1) передают их соответствующему члену экипажа ВС в готовом для размещения на ВС виде и в нужном количестве. 
(3)	Комплектация полётных документов ВС осуществляется согласно Таблицы А8.6-Т1. 
(4)	Доставку документов на ВС осуществляют лица, указанные в столбце «Ответственный за комплектацию ВС» Таблицы А8.6-Т2.
(5)	Ответственность за сохранность полётных документов несу лица, указанные в столбце «Ответственный за комплектацию ВС» Таблицы А8.6-Т1.
(6)	Ответственный за сохранность полётных документов обязан при получении документов проверить их актуальность.
(7)	В случае обнаружения неактуальности документов член экипажа сообщает об этом командиру ВС, который принимает меры к получению нового документа, или подтверждения его актуальности. 
Примечание: запрещается выполнение полёта с каким-либо неактуальным полётным документом.
(8)	Экипаж ВС обязан обеспечить надлежащее использование, хранение и передачу полётных документов после завершения выполнения полёта. 
                                                                                                                                                 Таблица А8.6-Т1
№
п/п	Наименование документа	Ответственный за первоначальную подготовку (замену)	Ответственный за комплектацию ВС	Места размещения документа
1	Свидетельства членов экипажа ВС	Каждый член экипажа	Каждый член экипажа	Личные вещи члена экипажа
2	Приложения к свидетельству членов экипажа ВС	Каждый член экипажа	Каждый член экипажа	Личные вещи члена экипажа
3	Медицинские заключения, подтверждающее соответствие членов экипажа требованиям к состоянию их здоровья	Каждый член экипажа	Каждый член экипажа	Личные вещи члена экипажа
4	Задание на полет	Командир лётного подразделения	Командир ВС	Кабина экипажа
*5	Генеральная декларация (при международных полетах)	Подразделение государственной пограничной охраны в аэропорту вылета	Командир ВС	Кабина экипажа
*6	Сводная загрузочная ведомость	Провайдер наземного обслуживания ВС в аэропорту вылета	Второй пилот	Кабина экипажа
*7	Пассажирская ведомость (манифест)	Провайдер обслуживания пассажиров в аэропорту вылета	Второй пилот	Кабина экипажа
*8	Грузовая ведомость (манифест)	Провайдер обработки грузов в аэропорту вылета	Второй пилот	Кабина экипажа
*9	Документ, содержащий информацию об опасном грузе, предусмотренный ФАП 141	Провайдер обработки грузов в аэропорту вылета	Второй пилот	Кабина экипажа
10	Метеорологическая информация, необходимая для выполнения полета	Провайдер метеорологического обеспечения в аэропорту вылета	Командир ВС	Кабина экипажа
11	Актуализированная аэронавигационная информация, касающаяся запланированного полета	Начальник отдела аэронавигационного обеспечения	Начальник отдела аэронавигационного обеспечения	EFB (IPAD)
12	Аэронавигационные (полетные) карты	Начальник отдела аэронавигационного обеспечения	Начальник отдела аэронавигационного обеспечения	EFB (IPAD)
13	Рабочий план полета	Начальник отдела аэронавигационного обеспечения	Второй пилот (штурман)	Кабина экипажа
Примечания:
*	документы присутствуют на ВС при необходимости, продиктованной особенностью выполнения рейсов, типом ВС, иными причинами.



\section{Послеполетные разборы}

\subsection{Общие положения}

8.7.1.1. Основной целью послеполетного разбора является анализ и оценка выполненного полета на основе докладов членов экипажа и изучения полетной документации. 
Необходимость проведения послеполетного разбора обусловлено спецификой летной работы. 
Послеполетный разбор в экипаже должен быть местом подробного обсуждения членами экипажа допущенных отклонений, ошибок и нарушений. В процессе разбора должны быть определены причины отклонений и ошибок, выработаны рекомендации по их предупреждению в последующих полетах. 
Послеполетный разбор в экипаже проводит командир ВС (проверяющий) после каждого выполненного полетного задания в кабине экипажа. Послеполетный разбор может быть проведен позднее, но обязательно до начала следующего полета. 
Послеполетный разбор проводится в порядке и объеме, определенным технологией работы членов экипажа, и складывается из: 
а)	анализа каждым членом экипажа своей работы при подготовке к полету и его выполнении;
б)	разбора командиром ВС ошибок и нарушений, допущенных членами экипажа;
в)	оценки командиром ВС работы каждого члена экипажа и выполненного полета в целом;
г)	указаний и рекомендаций командира ВС по предупреждению повторяемости ошибок и нарушений.
Анализ деятельности летного экипажа должен строиться по схеме: отклонение - ошибка – причина.
8.7.2. Процедура послеполетного разбора
8.7.2.1. Командир ВС объявляет о начале разбора и предоставляет членам экипажа слово для доклада в следующей последовательности:
а)	Бортпроводник/бортоператор
б)	Бортмеханик;
в)	Штурман;
г)	Второй пилот.
8.7.2.2. После доклада бортпроводника (бортоператора) о замечаниях и пожеланиях пассажиров, (грузоотправителей), о замечаниях по работе бытового, бортового погрузочного оборудования и оценки работы - дальнейший разбор проводится в составе летного экипажа.
Бортмеханик докладывает КВС о работе материальной части в полёте, о замеченных недостатках по её лётно-технической эксплуатации (какие допущены отклонения или нарушения с анализом их причин), замечания к себе и членам экипажа, об остатке и расходе топлива, записывает в бортжурнал замечания по работе ВС и остаток топлива. Также оформляет справку о работе АТ в полёте, информирует ИТП о состоянии авиационно й техники, осматривает самолёт, докладывает КВС о результатах осмотра, сдаёт ВС в ИАС. Штурман докладывает о налёте за рейс, в том числе ночью, о работе пилотажно-навигационного оборудования и замеченных недостатках по его использованию, об отклонениях от установленных правил полёта и замечаниях по взаимодействию в полёте, осматривает ВС согласно РЛЭ и сдаёт документацию и штурманское снаряжение.
Второй пилот докладывает КВС о выполнении своих обязанностей, замечаниях и недостатках в технике пилотирования, ведения радиосвязи, о взаимодействии в полёте, замечания по лётной эксплуатации самолёта, осматривает самолёт, докладывает о результате осмотра, заполняет задание на полёт и служебную документацию.
Доклад должен содержать оценку работы оборудования и систем ВС, замечания, полученные со стороны служб, оценку собственной работы и техники пилотирования, свои замечания и предложения по улучшению работы, замечания к другим членам летного экипажа.
8.7.2.3. Командир ВС анализирует доклады членов экипажа, указывает на допущенные ошибки, помогает вскрывать причины отклонений и ошибок, оценивает работу членов экипажа, отмечает положительные и отрицательные моменты в работе, ставит задачу на предстоящий период, проверяет правильность оформления полетной документации и производит запись в задании на полет в раздел «Послеполётный разбор» и в задание кабинного экипажа.
