8.8. Летные процедуры
8.8.1. Общие требования
Перед полетом члены экипажа ВС проходят предполетную подготовку в установленном объеме.
Командир ВС, член экипажа имеет право отказаться от выполнения задания на полет, если он не в состоянии его выполнить.
8.8.1.1. Основные принципы управления ресурсами экипажа:
(1) Каждый член экипажа считает: "Безопасность начинается с меня". Поэтому каждый член экипажа без колебаний высказывает свое мнение, когда возникшая ситуация или чьи-то действия представляются ему угрожающими безопасности полетов. Члены экипажа считают категорически неприемлемыми пассивность и равнодушие по отношению к любым нарушениям или недобросовестным действиям. Каждый член экипажа должен иметь мужество сказать - «Нет», если другие члены экипажа толкают его на осознанное нарушение уста-новленных правил и законов летной деятельности.
(2) Каждый член экипажа принимает ответственность за свои решения и действия. Все члены экипажа берут на себя ответственность за преодоление своих недостатков: лени, трусости, пассивности, ложного чувства товарищества. Они также берут на себя ответственность за совершенствование коллективной

деятельности в кабине, что поможет избежать 'многих проблем в процессе летной работы.
(3) Центральной фигурой в обеспечении безопасности полетов является командир воздушного судна. Поэтому именно он осознает всю полноту своей ответственности. Именно он противостоит неблагоприятным обстоятельствам, соблазну нарушать правила полетов. Главная задача командира так организо¬вать взаимодействие в своем экипаже, чтобы максимально мобилизовать его ресурс, чтобы найти то единственно правильное решение, которое практически всегда есть в экипаже.
(4) Командир поощряет высказывание мнений другими членами экипажа, особенно, если возникли сомнения и разногласия, он активно анализирует свои действия и действия подчиненных. Побуждает других членов экипажа к конструктивной дискуссии, в том числе и к анализу собственных действий. Лучший способ реализации власти - мобилизовать наличные ресурсы членов экипажа.
(5) Обязанность всех членов экипажа и, в первую очередь, командира - создать в экипаже обстановку взаимного уважения. Постановка вопросов, обсуждение проблем и поиск коллективного решения не считаются ни чрезмерной осторожностью, ни пустой тратой времени. Если один из членов экипажа, вне зависимости от возраста и опыта, чувствует, что какие-либо действия не безопасны, остальные члены экипажа считают своим долгом разобраться в причинах его опасении и лишь тогда эти действия продолжить. Такой подход создает уверенность в информированности и владении си¬туацией всеми членами экипажа и а готовности каждого к любой экстремальной ситуации.
(6) Каждый член экипажа, и в особенности командир, старается услышать другого. Каждый ищет в заданном ему вопросе не подвох или признаки подры¬ва авторитета, а искреннюю заинтересованность в обеспечении безопасности полета. В полете не должно быть места раздраженному тону, взаимным обвинениям, насмешкам и тем более оскорблениям. Каждый член экипажа должен помнить, что его амбиции - прямая угроза жизни пассажиров. Святая обязан¬ность командира - немедленно погасить любые возникающие в полете трения, переведя их в конструктивную дискуссию.
(7) Каждый член экипажа твердо уверен: никакие возможные экономиче¬ские потери не оправдывают риска неблагополучного завершения полета. По¬этому задержка взлета или уход на запасной аэродром не воспринимается им как досадная неприятность или недостаток.
8.8.2. Распределение обязанностей летного экипажа
8.8.2.1. Распределение обязанностей членов летного экипажа во время взлета, набора высоты, горизонтального полета, снижения, захода на посадку, посадки указаны в РЛЭ (AFM, FCOM) и в Инструкции по взаимодействию и технологии работы экипажа самолета данного типа (SOP).
8.8.2.2. Члены летного экипажа должны осуществлять взаимную перекрестную проверку действий друг друга и подтверждение правильности выполнения важнейших технологических операций, включая:
а)	изменения конфигурации ВС (положения шасси, закрылков, интерцепторов, спойлеров);
б)	установки задатчиков на высотомерах, указателях скорости;
в)	установки значений барометрического давления на высотомерах;
г)	установки задатчиков заданного эшелона полета на высотомерах;
д)	передачу управления воздушным судном от одного пилота другому;
ж) проверку правильности данных, вводимых в AFS/FMS и радиотехнические системы ВС на этапах перед взлетом и перед заходом на посадку;
и) проверку правильности расчетов масс и центровок ВС и данных, вводимых в AFS/ FMS;
к) проверку правильности расчета ЛТХ и правильности ввода их в AFS / FMS.
Процедуры проведения перекрестного контроля (crosscheck) опубликованы в Инструкциях по взаимодействию и технологиях работы членов экипажей (SOP) типов ВС (РПП, Часть В, типа ВС, гл.2).
8.8.2.3. Не пилотирующий пилот ведет инструментальный и визуальный контроль за выполнением полета. В случае потери работоспособности пилотирующего пилота, не пилотирующий пилот должен взять на себя управление ВС. 
8.8.2.4. В процессе руления и в полете, на высотах ниже 3000 м (10000ft), а также в других критических фазах полета, летный экипаж должен выполнять только те обязанности, которые связаны с непосредственным управлением ВС. 
При этом, разговоры в кабине летного экипажа, не связанные с выполнением технологических операций, запрещаются. 
Под критическими фазами полета подразумеваются этапы:
а)	управление ВС на земле, включая руление и буксировку – от начала подготовки кабины к вылету до выключения двигателей после посадки;
б)	взлет и набор высоты;
в)	заход на посадку и посадка;
г)	полет ниже 3000 м (10000 ft), исключая полет на эшелоне;
д)	смена эшелона полета;
ж)	рубежи приема диспетчерского разрешения. 
8.8.2.5. На установленных рубежах (на земле и в воздухе), члены летного экипажа проверяют готовность к выполнению очередного этапа полета по контрольным листам, разработанным для каждого типа ВС, и содержат стандартные и аварийные процедуры. 
Основным средством организации дополнительного контроля за выполнением наиболее ответственных операций, определяющих готовность ВС и экипажа к очередному рубежу или этапу полета и непосредственно влияющих на безопасность полета, является карта контрольной проверки.
Контроль по карте контрольной проверки проводится только после того, как каждый член экипажа доложит о завершении подготовки в соответствии с листом контрольного осмотра.
Контроль по карте контрольной проверки является обязательным комплексом операций, проводимых экипажем под руководством и ответственностью командира ВС на предписанных рубежах при выполнении полетов любого назначения. Процедура контроля по карте контрольной проверки должна обязательно обеспечиваться перекрестным контролем, когда выполняющий очередную операцию член экипажа должен получить подтверждение правильности выполнения от другого члена экипажа.
Чтение вслух соответствующего раздела карты производится членом экипажа в соответствии с Инструкцией по взаимодействию и технологией работы экипажа, который после поступления последнего доклада по последнему пункту зачитываемого раздела карты докладывает командиру ВС о завершении контроля по соответствующему разделу карты.
Пункты контрольной карты зачитываются раздельно, громким голосом, исключая случаи, когда инструкции для данного типа ВС предусматривают выполнение определенной части контрольного перечня молча. 
Последующий пункт не должен зачитываться, пока не будет должным образом проверен предшествующий ему пункт. Подлежит соблюдению точная терминология контрольного перечня, используемого в кабине экипажа.
8.8.2.6. На пассажирских воздушных судах закрытие дверей и грузовых люков после посадки пассажиров осуществляет бортмеханик, а их открытие после заруливания на стоянку осуществляют члены экипажа в соответствии с требованиями РЛЭ ВС (AFM, FCOM) и Инструкции по взаимодействию и технологии работы экипажа.
8.8.2.7. На грузовых воздушных судах закрытие дверей осуществляется бортмехаником, закрытие грузового люка, а также эксплуатацию специального погрузочно-разгрузочного обо¬рудования осуществляет бортоператор, а открытие дверей и установку трапа после заруливания на стоянку осуществляют члены экипажа в соответствии с требованиями РЛЭ ВС и Инструкции по взаимодействию и технологии работы экипажа. Грузовой люк (рампу) открывает бортоператор.
8.8.2.8. Время и очередность приема пищи членами летного экипажа в полете оп¬ределяет командир ВС. Не допускается одновременный прием пищи двумя пилотами.
8.8.2.9. Использование посадочных фар ВС определяется командиром ВС в зависимости от условий полета, орнитологической обстановки, учебно-тренировочного задания на полет и в соответствии с РЛЭ ВС (AFM, FCOM).
8.8.2.10. Распределение обязанностей членов летного экипажа во время взлета, набора высоты, горизонтального полета, снижения, захода на посадку и посадки указаны в РЛЭ ВС (AFM, FCOM) и в Инструкции по взаимодействию и технологии работы экипажа самолета данного типа. 
8.8.2.11. Командир воздушного судна является старшим на борту ВС, осуществляет общее руководство работой экипажа, обеспечивает выполнение полетных процедур в соответствии с Инструкцией по взаимодействию и технологией работы экипажа, РЛЭ ВС (AFM, FCOM) и РПП авиакомпании (доп.см.п.4.3, А-4 РПП). Экипаж должен в течение всего полета иметь достаточно ясное представление об окружающей его воздушной обстановке, имея в виду не только те ВС, которые находятся в пределах видимости, но и те, которые могут находиться в данном районе или месте пересечения воздушных трасс.
8.8.2.12. Основными обязанностями пилотирующего пилота являются пилотирование, контроль за управлением и осуществление навигации. Он должен контролировать полет, работу систем ВС и двигателей и быть в любой момент готовым к переходу на ручной режим управления. 
Непилотирующий пилот ведет инструментальный и визуальный контроль за выполнением полета. В случае потери работоспособности пилотирующего пилота, непилотирующий должен взять управление ВС на себя.
8.8.2.13. В зависимости от конкретных условий полета пилоты должны, как правило, чередовать функции пилотирующего и непилотирующего пилота в равной пропорции.
8.8.2.14. Для регулирования рабочей нагрузки на экипаж ВС при заходе на посадку в приборных метеоусловиях ночью рекомендуется использовать распределение обязанностей между пилотами, при котором: 
а)	максимально используются автоматизированные режимы работы систем ВС;
б)	при невозможности использования в полном объеме автоматизированных режимов работы систем ВС обязанности пилотирующего пилота выполняет командир ВС;
в)	при заходе на посадку в условиях ниже минимума САТ I обязанности пилотирующего пилота выполняет командир ВС, второй пилот выполняет обязанности непилотирующего, если иное не предусмотрено SOP.
Ответственность за равномерное распределение рабочей нагрузки между членами летного экипажа несет командир ВС.
8.8.3. Смена члена летного экипажа в полете
Общие положения
8.8.3.1. В том случае, когда плановая продолжительность полётной смены превышает максимально допустимую, установленную Главой А -7 РПП, для выполнения функций члена экипажа в полёте на время регламентированного технологического перерыва вводятся дополнительные члены экипажа (далее – увеличенный состав экипажа).
8.8.3.2. Продолжительность полетной смены увеличенного состава летного экипажа устанавливается в зависимости от числа дополнительных членов летного экипажа и количества посадок, запланированных заданием на полет согласно Таблицы А7.7 – Т1 Главы А7 РПП.
8.8.3.3. В увеличенном экипаже командиром является пилот, имеющий квалификацию КВС (инструктор), на которого оформлено задание на полет. Он занимает левое пилотское сиденье, несет ответственность за подготовку и выполнение полёта.
8.8.3.4. Замена члена лётного экипажа в полёте производится на высоте полёта не менее 3000м (10000ф), за исключением случая замены члена экипажа, потерявшего работоспособность.
8.8.3.5. Порядок смены членов экипажа и смены экипажей в длительном полёте
В длительном полёте смена членов экипажа, в том числе КВС, производится по указанию КВС только в прямолинейном горизонтальном полёте и при условиях, что:
а)	смена предусмотрена заданием на полёт;
б)	заменяющий член экипажа участвовал в предполётной подготовке;
в)	оставшееся время до пролёта очередного пункта ОД или пересечения трассы – не менее 5 минут;
г)	оставшееся время до начала снижения – не менее 10 минут.
8.8.3.6. Перед заменой в полёте член экипажа знакомится с навигационными особенностями полёта, работой оборудования на рабочем месте по информации заменяемого, и после разрешения командира ВС занимает рабочее место и докладывает: «Рабочее место занял, параметры полёта контролирую».
8.8.3.7. После доклада, ответственность за дальнейшее безопасное выполнение этого полёта ложится на члена экипажа, занявшего рабочее место.
8.8.3.8. В процессе смены остальные члены экипажа должны находиться на своих рабочих местах. Смена членов экипажа в полёте производится поочередно с интервалом не менее 5 мин. с соблюдением указанных условий.
8.8.3.9. В двойном экипаже смена полного состава производится, как указано, в пунктах 8.8.3.6, 8.8.3.7, 8.8.3.8.
           8.8.3.10. Смена командира ВС или второго пилота на крейсерском этапе полета
В состав усиленного экипажа для смены пилотов может быть включён дополнительный пилот, имеющий квалификацию командира ВС, допущенный к полётам с правого пилотского сиденья, или квалификацию «инструктор» на данном типе ВС.
8.8.3.11. Дополнительный пилот, указанный в п. 8.8.3.10, в полёте может занимать левое или правое места пилотов, по указанию КВС. При этом он несет ответственность за выполнение функций соответственно командира ВС или второго пилота, согласно РЛЭ на период подмены основного члена экипажа.
8.8.3.12. Порядок смены пилотов в длительном полёте
В длительном полёте смена командира ВС или второго пилота производится по указанию КВС при условиях и в порядке, предусмотренном подразделом 8.8.3. настоящего раздела РПП. В процессе смены пилота управление ВС должно осуществляться от автопилота.
8.8.3.13. В процессе смены пилота, другой пилот сохраняет полный доступ к органам управления ВС, контролирует режим полёта (находится в активной позе в готовности к управлению в штурвальном режиме).
8.8.4. Распределение обязанностей членов кабинного экипажа
8.8.4.1. Распределение обязанностей членов кабинного экипажа дает необходимое понимание выполнения процедур для исполнения обязанностей в штатных и аварийных ситуациях и устанавливается в зависимости от назначенного по заданию на полет порядкового номера в экипаже.
8.8.4.2. Распределение обязанностей бортпроводников (бортоператоров) в зависимости от номера по полетному заданию по выполнению процедур при подготовке и выполнении полета, а также при действиях в аварийных ситуациях описаны в Технологии работы бортпроводников (бортоператоров). 
8.8.4.3. Распределение обязанностей членов кабинного экипажа при аварийных ситуациях основаны на положениях РЛЭ конкретного типа ВС «Аварийное расписание» и РПП авиакомпании глава В -11по типу ВС.
8.8.4.4. В случае потери работоспособности одного из членов кабинного экипажа КВС производит перераспределение обязанностей по обслуживанию пассажиров и на случай возникновения угрозы безопасности полета в аварийных ситуациях.
Перераспределение обязанностей производится с учетом квалификации конкретных членов экипажа.
8.8.5. Взаимодействие летного и кабинного экипажей
8.8.5.1. Порядок взаимодействия летного и кабинного экипажа в нормальном полете
На период выполнения рейса бортпроводник (бортоператор) подчиняется КВС и несет ответственность за подготовку и работу в рейсе, обеспечение безопасности пассажиров и пассажирской кабины.
Перед вылетом, в период предполетной подготовки командир ВС:
а)	принимает доклад СБ о составе кабинного экипажа и его готовности к рейсу;
б)	представляется и информирует о составе летного экипажа;
в)	сообщает расчетное время полета и коммерческую загрузку;
г)	информирует о наличии опасных грузов на борту ВС и определяет порядок действий кабинного экипажа при возникновении аварийной ситуации;
д)	согласовывает порядок доведения информации пассажирам из кабины экипажа;
е)	определяет условные сигналы связи бортпроводника (бортоператора) с кабиной летного экипажа;
ж)	проводит брифинг с кабинным экипажем, информируя об особенностях предстоящего полета: предполагаемых зонах турбулентности, дополнительных мерах по обеспечению авиационной безопасности, особенностях полета над большими водными пространствами, в полярных широтах, горными массивами и т.д.
Примечание. Брифинг с кабинным экипажем может проводиться до прибытия на ВС или на борту ВС. 
Бортпроводник (далее БП) обязан, по прибытию командира ВС на самолет: 
а)	доложить:
	о ходе подготовки к полету;
	об отклонениях от технологического графика;
	обо всех обнаруженных недостатках в ходе подготовки ВС;
	о наличии и исправности БАСО;
	о готовности кабинного экипажа и пассажирского салона к посадке пассажиров.
б)	согласовать с командиром ВС:
	размещение в салонах ВС больных, инвалидов, несопровождаемых детей, депортированных пассажиров и сотрудников безопасности;
	порядок загрузки багажных помещений и размещение опасных грузов (при необходимости);
	порядок входа в кабину летного экипажа, условные сигналы;
	время и очередность приема пищи членами летного экипажа.
После окончания посадки пассажиров БП докладывает командиру ВС:
	наличие перевозочных документов;
	точное количество пассажиров на борту;
	коммерческую загрузку и ее размещение в багажных отсеках;
	наличие служебной корреспонденции, опасных грузов, оружия и/или боеприпасов;
	наличие депортированных, больных пассажиров, инвалидов и сопровождающих их лиц;
	наличие несопровождаемых детей.
БП должен получить от КВС разрешение на закрытие дверей (при выполнении этих функций).
Перед взлетом, до занятия исполнительного старта и перед посадкой на установленном рубеже, командир ВС должен:
а)	установленным сигналом проинформировать кабинный экипаж о необходимости занять свои места и пристегнуться привязными ремнями;
б)	получить доклад БП о готовности пассажирской кабины к взлету/посадке.
Проводить переговоры между летным и кабинным экипажем на высоте ниже 3000м. (10000ф.) запрещается, кроме случаев возникновении на борту нестандартных ситуаций, влияющих на безопасность полета.
В полете БП немедленно докладывает КВС о возникновении на борту любых нестандартных ситуаций, влияющих на безопасность полета, а также об использовании аварийно-спасательного оборудования в полете.
При возникновении признаков отказа или неисправности электрооборудования пассажирского салона ВС, бортпроводник обязан отключить электрооборудование с немедленным докладом командиру ВС.
При подходе к зоне турбулентности командир ВС оповещает кабинный экипаж включением табло «ЗАСТЕГНУТЬ РЕМНИ» (FASTEN SEAT BELTS), при подходе кзоне умеренной или сильной турбулентности, дополнительно дает информацию по внутрисамолетной связи: «Бортпроводникам занять свои места».
По этому сигналу бортпроводники должны:
а)	прекратить обслуживание пассажиров;
б)	занять свои (или ближайшие свободные) места;
в)	застегнуть привязные ремни.
После заруливания на стоянку и выключения двигателей, командир ВС дает команду на открытие дверей самолета отключением табло «ЗАСТЕГНУТЬ РЕМНИ» («FASTEN SEAT BELTS»).
При любых задержках или отклонениях от плана полета (вынужденная посадка, полет в зоне ожидания, уход на запасной аэродром и т.д.) командир ВС информирует БП и пассажиров о причинах, времени задержки и дальнейших действиях.
8.8.6. Распределение обязанностей в аварийных ситуациях
8.8.6.1. Пилотирующий пилот должен нести ответственность, главным образом, за управление ВС и осуществление контроля за выполнением полета в то время, когда другие члены экипажа выполняют свои обязанности, связанные с локализацией аварийной ситуации. 
Когда второй пилот пилотирует ВС, он выполняет все функции, обозначенные ПАУ (пилот активно управляющий).
Передача управления воздушным судном должна производиться установленным, исключающим ошибки, образом в соответствии со стандартными эксплуатационными процедурами. 
8.8.6.2. Командир ВС, исходя из ситуации, сохраняет решающее право на перераспределение функций в экипаже на любом этапе полета. Такое перераспределение может носить временный характер или на весь период полета.
В зависимости от обстоятельств командир ВС может назначить дополнительные обязанности всем членам летного и кабинного экипажа. 
8.8.6.3. Действия члена экипажа вне зоны его ответственности разрешаются только по команде командира ВС.
8.8.6.4. Основным средством организации контроля за выполнением наиболее ответственных операций, необходимых для выполнения экипажем в нормальных условиях эксплуатации для обеспечения максимальной безопасности полета, готовности ВС и экипажа к очередному рубежу или этапу полета является Карта контрольных проверок.
Пункты Карты контрольных проверок зачитываются раздельно, громким голосом.
Последующий пункт не должен зачитываться, пока не будет должным образом проверен предшествующий ему пункт. 
Процедура контроля по карте контрольной проверки должна обязательно обеспечиваться перекрестным контролем, когда выполняющий очередную операцию член экипажа должен получить подтверждение правильности выполнения от другого члена экипажа.
В случае отсутствия инструкций в контрольных листах и РЛЭ типа ВС (AFM, FCOM), окончательное решение о выполнении процедур по управлению ВС принимает командир ВС.
В любом случае окончательное решение по действиям в аварийной ситуации принимает командир ВС.
8.8.6.5. Экипаж обязан осуществлять перекрестные проверки «Сrosscheck» (при помощи ответов) перед приведением в действие критических средств управления ВС, включая:
а)	снижение уровня тяги у отказавшего двигателя с помощью рычага управления двигателем;
б)	изменения конфигурации;
в)	установка курса, высоты, высотомера и воздушной скорости;
г)	передача управления ВС;
д)	изменения в AFS/FMS и средствах радио навигации во время вылета или фазы захода на посадку;
е)	расчеты рабочих характеристик, включая данные AFS/FMS;
ж)	включение переключателей управляющими перекрывными топливными кранами и переключателями, управляющими Расходом топлива;
з)	включение пожарной системы и огнетушителей;
и)	выключение генераторов.
8.8.6.6. На воздушных судах, оборудованных самолет¬ным переговорным устройством, переговоры в кабине летного экипажа ведутся с его использованием, за исключением, когда ВС оборудовано микрофо¬нами постоянной записи. Устройства записи переговоров в пилотской кабине и системы регистрации полетных данных не должны намеренно выключаться, за исключением случаев, когда это необходимо в целях сохранения данных, связанных с авиационными происшествиями или инцидентами.
8.8.7. Стандартные выражения и терминология
Для исключения неправильного понимания в кабине экипажа или при ведении связи с наземным персоналом, используются следующие правила:
а)	при выполнении стандартных процедур, таких как выполнение карты контрольных проверок, должен использоваться русский или английский язык в соответствии с требованиями РЛЭ (AFM, FCOM) воздушного судна;
б)	должна использоваться стандартная терминология и сигналы;
в)	при возникновении трудностей во взаимодействии между летным и наземным персоналом, используется любой удобный для понимания языков.
Для стандартизации процедур «Сrosscheck» - перекрестных проверок, повышения уровня координации и осведомленности членов экипажа при выполнении операционных процедур в полете в Авиакомпании используются стандартные отклики (ответы) - «Standard callouts», которые употребляются для того, чтобы:
а)	подать команду, поставить задачу;
б)	подтвердить получение команды, задачи;
в)	подтвердить выполнение операций по карте контрольной проверки (Check list);
г)	оповещать об изменениях показаний приборов;
д)	информировать о приближении параметров полета к предельно - допустимым и выходе их за пределы;
е)	идентифицировать конкретные события.
Порядок применения стандартных ответов (откликов) (Standard callouts) при выполнении полетов на конкретном типе ВС описан в Инструкциях и технологиях работы экипажей (SOP) типов ВС (РПП, Часть В, типа ВС, гл.2).
8.8.8. Ведение радиосвязи
8.8.8.1. Радиообмен с диспетчером ОВД ведет командир ВС или, по его поручению, другой член летного экипажа, в соответствии с установленными правилами и фразео¬логией. Радиообмен должен внимательно прослушиваться, как минимум двумя членами экипажа в течение всего полета. По возможности производить записи полученных диспетчерских разрешений.
Отступление от установленных правил допускается при ситуации, угрожаю¬щей безопасности полета. Ведущий радиосвязь должен убедиться в правильности принятой информации, сверив ее восприятие другими членами экипажа. При выявлении разного понимания информации запросить повторения информации.
Перед началом передачи прослушать наличие радиообмена на подлежащей использованию частоте, убедиться в отсутствии сигнала занятости наземного канала связи, чтобы исключить возможность возникновения помех уже ведущейся передаче.
8.8.8.2. Экипаж воздушного судна обязан немедленно сообщить органу ОВД о наблюдаемых опасных метеорологических явлениях, опасных сближениях с воздушными судами и другими материальными объектами и других опасных для полета обстоятельствах.
Летный экипаж докладывает диспетчеру ОВД:
а)	об условиях руления или полета, препятствующих выполнению Правил выполнения полетов и требований руководства по летной эксплуатации ВС;
б)	о выполняемых вертикальных маневрах по рекомендациям БСПС, а также о восстановлении условий, заданных в указании или диспетчерском разрешении ОВД, после разрешения конфликтной ситуации;
в)	о пролете пунктов обязательного донесения доклад может не производиться при получении экипажем информации от органа ОВД о радиолокационном опознавании ВС;
г)	о времени начала снижения с эшелона крейсерского полета;
д)	об условиях полета по запросу диспетчера ОВД;
е)	о входе в аэродромный круг движения при полетах по правилам полетов по приборам, если до этого не было выдано разрешение на посадку;
ж)	о выполнении иных указаний, установленных органом ОВД.
Особое внимание при ведении радиосвязи обращается при получении инструкций на предварительном старте, полетах в горной местности, при изменении курса, изменении радиочастот, изменении эшелона или высоты полета.
Экипаж ВС при получении диспетчерских указаний обязан повторить:
а)	сообщения, отличающиеся от типовых или требующие изменения принятого решения (или задания на полет);
б)	разрешения или запрещения на пересечение ВПП, взлёта, занятие исполнительного старта, захода на посадку, посадки, изменения эшелона высоты) полета и т.д.;
в)	значение принятого и установленного на высотомере давления;
г)	значение контрольной высоты;
д)	значение заданного времени;
е)	заданный эшелон (высоту) полёта;
ж)	заданный курс полета;
з)	значение МПУ ВПП, номера ВПП;
и)	заданную скорость полета или число «М»;
к)	значение заданной частоты (номера) канала связи.
8.8.9. Политика применения автоматических систем управления ВС
8.8.9.1. Автоматические системы управления ВС применяются в целях: 
а) повышения безопасности полета; 
б) уменьшения рабочей нагрузки на экипаж; 
в) увеличения операционных возможностей членов летного экипажа; 
г) увеличения ситуационной осведомленности; 
д) улучшения условий для принятия решений; 
ж) оказания помощи экипажу в действиях по управлению рисками. 
8.8.9.2. Пилотирование ВС и эксплуатация автопилота осуществляется в строгом соответствии с РЛЭ (AFM, FCOM).
Как правило, автоматическое управление полетом следует применять на максимально возможном уровне. 
Пилоты (члены летного экипажа) должны быть подготовлены для использования всех уровней автоматизации, определения условий понижения уровня автоматизации полета и иметь навыки для перехода с одного уровня на другой. 
8.8.9.3. Вход в зону RVSM и полёт в ней выполняются только с включённым автопилотом. При отказе автопилота или появлении неисправности, не позволяющей выдерживать высоту с установленной точностью, экипаж сообщает органу ОВД, который вправе потребовать выхода ВС из зоны RVSM.	
Вне зоны RVSM автопилот может выключаться для балансировки ВС или тренировки пилота в ручном пилотировании ВС, в том числе на этапе захода на посадку.
8.8.9.4. В условиях слабого или умеренного сдвига ветра или турбулентности на всех этапах полёта рекомендуется выполнять полёт в режиме автоматического управления с включённым автопилотом. При наличии информации о возможном сдвиге ветра или турбулентности (болтанке) на этапах взлёта или захода на посадку рекомендуется увеличить скорость полёта в соответствии с требованиями РЛЭ (AFM, FCOM).
8.8.9.5. Как правило, автоматическое управление полетом следует применять на максимально возможном уровне. Выбранный уровень автоматизации для конкретных условий полета должен обеспечивать оптимальное распределение рабочей нагрузки между членами летного экипажа (пилотами), постоянный контроль за профилем полета и положением ВС в пространстве. 
8.8.9.6. В любой нештатной ситуации, требующей отклонения от стандартных эксплуатационных процедур, автопилот рекомендуется включать для снижения операционной нагрузки при выполнении процедур по локализации отказов и неисправностей. 
8.8.9.7. Основные принципы использования автоматических систем управления ВС: 
а)	пилотирующий пилот должен постоянно контролировать соответствие работы автоматической системы с траекторией движения ВС; 
б)	если системы автопилота не работают, как ожидается, изменить уровень автоматизации или отключить эту функцию; 
в)	члены экипажа должны быть информированы о любых изменениях установок в автоматической системе. Если пилотирующий пилот самостоятельно определил изменение режима, он обязан оповестить об этом экипаж; 
г)	оперативные действия по управлению профилем полета и режимами работы автоматической системы осуществляет пилотирующий пилот с немедленным докладом экипажу; 
д)	после включения какого-либо автоматического режима члены экипажа должны убедится, что выбранный режим включился и его индикация соответствует заданной; 
е)	все команды и информация должны быть немедленно подтверждены членом экипажа, к которому они адресованы; 
ж)	любой выбранный уровень автоматизации полета не может исключать ведения визуальной осмотрительности; 
з)	автоматические системы управления ВС рекомендуется применять в районе аэродромов с высокой интенсивностью, предполетный и предпосадочный брифинг должен включать все особенности применения автоматических систем, распределение обязанностей и ответственности членов экипажа ВС; 
и)	пилоты должны контролировать минимальную высоту включения/отключения автоматических режимов управления в соответствии с РЛЭ типа ВС; 
к)	автопилот и директорное управление полетом рекомендуется использовать для захода на посадку при видимости на ВПП менее 1200 м (4000 ft); 
8.8.9.8. При выполнении полёта в автоматическом режиме на ВС ИП члены экипажа осуществляют перекрёстную проверку («Сrosscheck») состояния навигационных систем и изменяемых навигационных элементов полёта. В процедуре перекрестного контроля должны участвовать два члена экипажа. Один вводит данные в навигационную систему, другой контролирует вводимые данные в соответствии с CFP, планом полета, таблицами установочных данных, полетной картой, схемой и т.п. и соответствующий отклик ВС на ввод команд и данных:
а)	при включении автопилота от бортовой навигационной системы (FMS) проверяется соответствие курса следования заданному путевому углу с учётом угла сноса и сохранения заданных высоты и скорости полёта (сохранения поступательной и вертикальной скорости при наборе высоты или снижении);
б)	при пролёте ПОД проверяется точность разворота на расчётный курс следования и соответствие отображаемых на индикаторе данных навигационному расчёту полёта;
в)	на этапах схемы прибытия (STAR) и выхода (SID) оба пилота непрерывно контролируют соответствие навигационных параметров полёта установленным схемам и своевременность занятия высот (эшелонов) к заданным рубежам;
г)	при заходе на посадку один пилот контролирует параметры полёта, находится в готовности отключить автопилот и перейти к ручному режиму пилотирования при появлении значительных (более, чем на оценку удовлетворительно) отклонений от заданного режима полёта, а другой пилот контролирует соответствие навигационных параметров полёта установленной схеме.
Перекрестный контроль заключается в коллективном контроле местоположения самолета с использованием основных и дополнительных средств навигации. При этом каждый член экипажа (пилот) обязан своевременно информировать о замеченных отклонениях от заданных траекторий полета, удалениях до очередных пунктов маршрута, расчетном времени пролета.
