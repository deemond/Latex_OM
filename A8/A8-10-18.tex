8.10.	Послеполетные работы
8.10.1.	Общие положения
8.10.1.1. Послеполётные работы начинаются с момента выключения двигателей после заруливания ВС на стоянку и включают с себя выполнение послеполётных процедур, предписанных РЛЭ ВС и РПП Авиакомпании, оформление отчетной документации о полёте и проведение послеполётного разбора.
Выключение двигателей, оборудования, систем ВС осуществляется в соответствии с РЛЭ ВС (AFM, FCOM).       
8.10.1.2. Во время выхода пассажиров из самолета члены летного экипажа остаются на своих рабочих 
местах.
8.10.1.3. Командир ВС принимает доклад по замечаниям от кабинного экипажа, записывает замечания о работе систем и оборудования в бортовой журнал ВС или контролирует записи, заверяя их своей подписью.
8.10.1.4. Летному и кабинному экипажам запрещается покидать борт ВС до тех пор, пока они не будут убеждены, что все необходимые операции по обеспечению стоянки выполнены, и замечания по работе систем и оборудования воздушного судна переданы диспетчеру по планированию полетов ОКВР (отдел контроля выполнения рейсов):
 раб.тел. +7 391 270-50-83;
 раб. тел. +7 391 275-20-37;
 моб.тел. +7 912 990 40 86 (круглосуточный);
 электронная почта mail.ru: pds-trh@utair.ru, dispаtcher-trh@utair.ru
8.10.1.5. После выполнения экипажем процедур, предписанных РЛЭ (AFM, FCOM) и РПП Авиакомпании, оформления отчетной документации о полёте, командир ВС проводит послеполетный разбор в экипаже.
8.10.1.6. Продолжительность послеполетных работ, после завершающего полёта полетной смены, составляет не более 30 минут. При необходимости выполнения послеполетных работ в большем объеме КВС обязан сделать отметку в разделе «Особые отметки» задания на полет.
8.10.1.7. Прием-передача ВС и находящегося на нем имущества после полета изложены в Главе 10 (Приложение А10.7).
8.11.	Выполнения полетов по ПВП и ППП
Правила и условия, при которых выполняются полеты по ППП и ПВП приведены в главе 12 части А РПП. Переход от полета по ПВП к ППП осуществляется при ухудшении метеоусловий, при несоответствии условиям для полета по ПВП. При этом КВС обязан согласовать свои действия и эшелон полета с органом ОВД, диспетчер которого обязан обеспечить установленный интервал между ВС.
8.12.	Навигационные процедуры
8.12.1. Общие положения
8.12.1.1. Навигационные процедуры устанавливаются в целях: 
а)	обеспечения полета в навигационном отношении;
б)	обеспечения единообразного, эффективного и надежного метода эксплуатации навигационного оборудования;
в)	выполнения настоящих стандартных навигационных процедур всем персоналом летных экипажей авиакомпании;
г)	оптимального распределения нагрузки на членов летного экипажа при подготовке и выполнении полета.
8.12.1.2. Точность и надежность навигации, в том числе выдерживание установленных RNP при выполнении полетов в условиях зональной навигации, достигается за счет:
а)	соответствующей подготовки летного персонала авиакомпании;
б)	поддержания навигационного оборудования ВС в исправном состоянии за счет налаженного технического обслуживания;
в)	преимущественного использования автоматических режимов полета;
г)	установки на ВС оборудования СНС и своевременного обновления базы данных оборудования СНС;
д)	выдерживания установленных требуемых навигационных характеристик при полетах в условиях зональной навигации.
8.12.1.3. Экипажу предоставляется право выбора основных и дополнительных навигационных средств, и методов выдерживания заданных траекторий в зависимости от конкретных условий выполнения полета. В общем случае все навигационное оборудование должно эксплуатироваться согласно РЛЭ ВС (AFM, FCOM), технологии работы экипажа и настоящих процедур. Применение процедуры «перекрестного контроля» во всех случаях, когда это предусмотрено, является обязательным. 
Примечание: Допускается использование переносных спутниковых приемоиндикаторов, допущенных авиакомпанией в качестве информационного средства для обеспечения навигационных процедур. 
8.12.1.4. Для использования оборудования RNAV, независимо от его статуса на борту ВС, экипажи (пилоты и штурманы) должны пройти соответствующую подготовку и иметь допуск к полетам с использованием оборудования ВС, обеспечивающим соблюдение правил и процедур RNAV. 
8.12.1.5. Допускается применение экипажем средств навигации (включая GNSS) в объеме, предусмотренном РЛЭ ВС (AFM, FCOM).
8.12.1.6. В процессе выполнения полета экипаж обязан постоянно контролировать соответствие предварительно выполненного расчета фактическим навигационным параметрам полета.
8.12.1.7.	Перечень навигационного оборудования на борту ВС, включая необходимое для полетов с использованием навигации, основанной на характеристиках
Перед полетом по ППП экипаж удостоверяется в том, что на борту ВС имеются в работоспособном состоянии:
а)	магнитный компас;
б)	хронометр или часы, указывающие время в часах, минутах и секундах;
в)	барометрический высотомер;
г)	система указания приборной воздушной скорости, оборудованная устройством, которое предотвращает выход из строя вследствие конденсации воды или обледенения;
д)	указатель пространственного положения (авиагоризонт);
е)	указатель скольжения;
ж)	указатель поворота, кроме случаев, когда ВС оборудовано тремя авиагоризонтами;
з)	указатель курса ВС (гирокомпас или иная система, выполняющая аналогичные функции);
и)	индикатор неисправности электропитания гироскопических приборов;
к)	указатель температуры наружного воздуха;
л)	указатель вертикальной скорости набора высоты и снижения;
м)	оборудование для обеспечения навигации в зоне подхода и захода на посадку по приборам (например: ILS, MLS, АРК, GNSS, VOR, DME);
н)	радиооборудование для ведения связи в диапазоне 118-137 МГц, в том числе на частоте 121,5 МГц;
о)	при нахождении на ВС с герметизированной кабиной более 2-х человек - оборудование для обнаружения грозы;
п)	на самолетах, ограничение скорости которых выражается числом Маха - средство отображения числа Маха; 
р)	подсветка для всех пилотажных приборов и оборудования, имеющих большое значение для безопасной
эксплуатации воздушного судна и используемых летным экипажем воздушного судна;
с)	при выполнении полета в системе РBN стационарное оборудование СНС, FMS или другое оборудование, допущенное к полетам в требуемых навигационных спецификациях;
т)	автономный переносной фонарь на рабочем месте каждого члена экипажа воздушного судна;
у)	при выполнении полёта в воздушном пространстве RVSM - две независимые системы измерения высоты, один приемоответчик ВОРЛ, передающий данные о высоте полета, система предупреждения об отклонении от заданной высоты полета, автоматическая система выдерживания высоты полета.
Допускается выполнение требований, содержащихся в подпунктах «д», «ж», «и», «к» и «н» настоящего пункта путем использования комбинированных приборов или комплексных командных пилотажных систем, при условии сохранения такой же гарантии от полного отказа, как и предусмотренной для каждого из отдельных приборов.
8.12.2.	Подготовка и выполнение полетов в условиях зональной навигации (RNAV)
8.12.2.1. Положение ВС определяется автоматически бортовой системой RNAV, использующей в качестве навигационных датчиков информацию от наземных (VOR/DME, DME, LOC), спутниковых (GPS, ГЛОНАСС), или автономных бортовых систем (INS, IRS, IRU). Зональная навигация позволяет осуществлять полеты по точкам на трассе, не привязанным к наземным РТС, что значительно повышает эксплуатационную гибкость и эффективность.
С добавлением к RNAV функциональной возможности мониторинга эксплуатационных характеристик (RNP) и выдачи предупреждений экипажу ВС в случае, когда ANP> RNP, стала возможной еще большая оптимизация использования воздушного пространства и возможность выполнения заходов на посадку RNAV GNSS (GPS).
8.12.2.2. Концепция PBN представляет собой переход от навигации, основанной на датчиках, к навигации, основанной на характеристиках. Требования к характеристикам указываются в навигационных спецификациях, в которых также определяется какие навигационные датчики и оборудование можно использовать для соблюдения этих требований к характеристикам
8.12.2.3.	Два вида навигационных спецификаций RNAV, RNP:
а)	Спецификация RNAV – навигационная спецификация, основанная на зональной навигации, которая не включает требования к контролю за выдерживанием и выдачей предупреждений о несоблюдении характеристик, обозначаемая префиксом RNAV, например, RNAV 5, RNAV 1.
б)	Спецификация RNP – навигационная спецификация, основанная на зональной навигации, которая включает требование к контролю за выдерживанием и выдачей предупреждений о несоблюдении характеристик, обозначаемая префиксом RNP, например, RNP 4, RNP АРСН.
Требования к характеристикам бортовой системы RNAV должны определяться в виде точности, целостности, эксплуатационной готовности, непрерывности функциональных возможностей, необходимых для выполнения предполагаемых полетов в контексте концепции конкретного воздушного пространства.
Для полетов в океаническом, удаленном воздушном пространстве, по маршруту и в районе аэродрома спецификация RNP обозначается RNP X, например: RNP 4.
Спецификация RNAV обозначается RNAV X, например: RNAV 1. Если в двух навигационных спецификациях используется одно и тоже значение X, то для их отличия можно использовать префикс, например, Advanced -RNP1 (усовершенствованные) и Basic-RNP 1 (базовые).
Для обозначений RNP и RNAV, выражение -X (где оно приводится) указывает на точность боковой навигации в морских милях, которая должна выдерживаться, по крайней мере, в течение 95% полетного времени всеми ВС, выполняющими полеты в пределах данного воздушного пространства, по маршруту или по схеме полета.
8.12.2.4. Навигационные спецификации захода на посадку охватывают все участки захода на посадку по приборам. Спецификации RNP обозначаются путем использования RNP в качестве префикса и последующего сокращенного текстуального индекса, например, RNP APCH или RNP AR APCH. Спецификаций RNAV для захода на посадку не существует.
Применение навигационной спецификации по этапам полета
                                                                                                                                    Таблица А8.12-Т1

Навигационная     спецификация	Этап полета
	
Маршрутный океанический
/ удаленный	
Маршрутный
континентальный	
Прибытие	Заход на посадку	Вылет
				Начальный	
Промежу- точный	
Конечный	Уход на второй круг1	
RNAV10	10							
RNAV 52		5	5					
RNAV2		2	2					2
RNAV 1		1	1	1	1		1	1
RNP4	4							
RNP 2	2	2						
RNP 1 3			1	1	1		1	1
Усоверш.
RNP (A-RNP) 4	2 5	2 или 1	1	1	1	0,3	1	1
RNP АРСН6				1	1	0, 37	1	
RNP AR АРСН				1-0,1	1-0,1	0,3-0,1	1-0,1	
RNP 0,3 8		0,3	0,3	0,3	0,3		0,3	0,3
Примечания: 
а)	Применяется только после достижения запаса высоты над препятствием в 50м (40м, Кат, Н) после начала набора высоты. 
б)	RNAV 5 является маршрутной навигационной спецификацией, которая может использоваться на начальном участке STAR за пределами 30 м. миль и выше MSA. 
в)	Спецификация RNP1 используется только на маршрутах STAR, SID, начальных и промежуточных участках IАР и при уходе на второй круг после начального этапа набора высоты. За пределами 30 м. миль от КТА значение точности для выдачи предупреждения становится равным 2 м. милям. 
г)	A - RNP также допускают определенный диапазон масштабируемых значений точности боковой навигации RNP. 
д)	Факультативное - требует более высокого значения непрерывности. 
е)	Для спецификации RNP АPCH может реализоваться GNSS с baro-VNAV и SBAS. 
ж)	RNP 0,3 применяется к RNP АРСН. Разные требования к угловым характеристикам применимы только к RNP АРСН. 
з)	Спецификация RNP 0,3 предназначена главным образом для полетов вертолетов. 
и)	Спецификациям RNAV10, RNAV5, RNAV1 соответствуют прежние названия RNP10, BRNAV, PRNAV соответственно.
8.12.2.5. Экипажи, выполняющие полеты в зону действия RNAV 5, должны пройти специальную подготовку по утвержденной программе.
8.12.2.6. В качестве основного средства навигации при выполнении полета в системе RNAV 5 должно использоваться стационарное оборудование СНС, или другое оборудование, допущенное к полетам в системе RNAV. В целях контроля полета использование оборудования КУРС-МП, СД, АРК в качестве дополнительных навигационных средств при выполнении полета в системе RNAV является обязательным. Указанное оборудование в полете должно быть включено и настроено на ближайшие маяки VOR, DME, NDB. 
8.12.2.7. Вылет с неисправным оборудованием зональной навигации в зону действия RNAV 5 допускается после согласования разрешения на полет.
Вылет с неисправным оборудованием, обеспечивающим RNAV 5 из аэропорта за пределами России в или через зону RNAV разрешается только по согласованию с Евроконтролем с соблюдением установленных для таких полетов ограничений.
8.12.2.8. При использовании СНС в зоне RNAV 5 аэронавигационная база данных должна быть действующей. Ответственность за наличие действующей базы данных на борту ВС при полете в зоне действия RNAV 5 несет командир ВС, а обеспечение организует старший штурман летной службы.
8.12.3.	Стандартные навигационные процедуры (СНП)
8.12.3.1.При выполнении каждого полета экипаж ВС обязан соблюдать правила навигации, основные из которых:
а)	комплексное использование бортовых и наземных навигационных средств, обеспечивающих наибольшую точность, надежность и безопасность;
б)	выдерживание маршрута полета с требуемой точностью, которое обеспечивается использованием навигационного оборудования, обладающего необходимой точностью навигации для соответствующих регионов полетов, спецификаций RNAV, RNP, установленных в определенном воздушном пространстве, а также своевременным выявлением отклонений ВС от маршрута и принятием мер по их устранению;
в)	перед каждым заходом на посадку экипаж определяет порядок настройки и использования бортовых и наземных радиотехнических средств навигации для контроля за точностью выдерживания траектории STAR и выхода на конечный этап захода на посадку;
г)	при выполнении маневра захода на посадку экипаж должен постоянно знать местоположение ВС относительно ВПП, использовать для контроля все имеющиеся в распоряжении средства и методы навигации независимо от вида захода на посадку (визуально, радиолокационным наведением (векторением), с использованием точных и неточных посадочных систем);
д)	при обнаружении отклонений от маршрута, превышающих максимально допустимые значения, выход на линию заданного пути осуществляется незамедлительно;
е)	в случае обхода опасных метеоявлений допускается отклонение от заданного маршрута по согласованию с органом ОВД;
ж)	выдерживание безопасных высот на всех этапах полета обеспечивается выполнением требований документов ИКАО и подраздела 8.1.2 настоящего РПП;
з)	выдерживание высот полета, эшелонов и скоростей, заданных органом ОВД, обеспечивается непрерывным перекрестным контролем со стороны экипажа ВС.
8.12.3.2.	Перекрестный контроль
В любой процедуре перекрестного контроля должны участвовать два члена экипажа. При этом один работает с оборудованием, а второй диктует или проверяет вводимые, или контролируемые данные по актуальным источникам (плану полета в соответствии с рассчитанным OFP, полетной карте, схеме и т.д.). 
Во всех случаях, когда полет выполняется по маршруту, вновь составленному или уже имеющемуся в памяти бортового комплекса, он предварительно должен быть проверен. При проверке выполняется процедура перекрестного контроля. 
Один член экипажа должен диктовать данные из справочного материала, а другой - их вводить. После завершения процедуры ввода данных произвести контроль введенной информации.
Один член экипажа должен считывать и произносить вслух введенные данные, другой - подтверждать ее соответствие справочной документации. 
Перед коррекцией системы счисления необходимо убедиться в правильной работе средства, по которому будет производиться коррекция, сравнением его показаний с другими навигационными средствами. 


8.12.3.3.	Контроль плана полета
Во всех случаях, когда полет выполняется по маршруту, вновь составленному или уже имеющемуся в памяти СНС, он предварительно должен быть проверен. При проверке выполняется процедура перекрестного контроля.
8.12.3.4.	Активизация плана полета в СНС
Полет по маршруту производить, как правило, по активному плану полета, имеющему, в общем случае, запас пунктов не менее 4-х. В целях исключения грубых ошибок при редактировании маршрута полета в СНС он должен начинаться и заканчиваться аэропортами (вылета и посадки).
Перед активизацией нового плана в полете в условиях дефицита времени - проверить, как минимум, первых два участка плана с последующей проверкой всего плана в общем порядке.
8.12.3.5.	Выдерживание линии заданного пути 
В общем случае выдерживание лзп производить согласно действующим технологиям работы экипажей и РЛЭ ВС. Штурману (пилоту, не выполняющему активное пилотирование) - при каждой эволюции вс докладывать экипажу причину разворота, сторону разворота и новый ЗПУ (например, «...проходим старицу, разворот влево на ЗПУ 102...»). Пилотам - подтверждать принятую от штурмана (пилота, не выполняющего активное пилотирование) информацию (например, «...понял..102») и контролировать процесс выполнения разворота на новый ЗПУ. 
При необходимости выполнения отворотов от ЛЗП, например, в случае обхода гроз, предупреждать экипаж о развороте (например, «...влево на курс 85, обход грозы»).
Внимание! При полете в зоне RNAV 5 не допускать уклонения от ЛЗП по индикатору CDI более 1 м. мили. В случае непреднамеренного уклонения ВС за указанные пределы - произвести интенсивный выход на ЛЗП с кренами 10-12° и углами выхода 15-20°. 
При получении указания или разрешения ОВД о полете прямо на какой-либо пункт:
а)	установить крен 10 -12° в сторону предполагаемого разворота, если требуется разворот на угол более 30°;
б)	выбрать из базы данных (активного FPL или любой другой пункт) объект спрямления. Если такого пункта в базе данных нет, то ввести его координаты вручную, применяя при этом процедуру ввода координат с перекрестным контролем;
в)	активизировать функцию «DIRECT TO» и следовать в пункт наведения с контролем полета по CDI, при этом, если после разворота есть боковое уклонение, то, независимо от его значения, выйти на линию, соединяющую точку, от которой началось спрямление и пункт спрямления. 
Выполнение схем выхода (SID), подхода (STAR) и элементов схем захода на посадку (Approach) до точки FAF.
Использование СНС как основного средства навигации при выходе, подходе и заходе на посадку разрешается только при выполнении схем SID, STAR, Approach, имеющих пометку «RNAV» или «GPS». Оборудование СНС при выполнении традиционных схем SID, STAR, Approach (без пометки “RNAV”) должно использоваться только как дополнительное средство получения навигационной информации.
Перед вылетом:
а)	подгрузить к активному плану полета предполагаемую схему SID, если они имеются в базе данных и срок действия ее не истек;
б)	убедиться в том, что первый участок полета выбран правильно;
в)	ввести QNH аэродрома вылета 
Контроль работоспособности СНС
Контроль точности работы навигационного оборудования осуществляется экипажем ВС в процессе предполетной проверки, в процессе руления и занятия исполнительного старта путем сравнения текущего курса с курсами РД и ВПП.
Во всех случаях при мигании табло «MSG» тот член экипажа, кто первый заметил появление нового сообщения, должен прочитать его, нажав клавишу «МSG». Если новое сообщение свидетельствует об отсутствии RAIM или каком-либо отказе, необходимо предупредить об этом экипаж и включить секундомер. 
Если в результате произведенных проверок становится очевидным то, что СНС не может быть использовано в качестве средства навигации - выполнять «Процедуры при отказах навигационных средств».
8.13.	Выполнение полетов в воздушном пространстве RVSM
Сокращения:
ВП RVSM (RVSM Airspace) – воздушное пространство, в котором применяется RVSM;
СВЭ RVSM – средства вертикального эшелонирования, обеспечивающие соблюдение функциональных и точностных требований к оборудованию ВС для выполнения полета в ВП RVSM;
Тр. ВП RVSM (RVSM Transition Airspace) – транзитное (переходное) воздушное пространство в пределах границ ВП RVSM, в котором осуществляется изменение системы эшелонирования;
AAD (Assigned Altitude Deviation) – отклонение от заданной высоты (разрешенного эшелона), допускаемое экипажем при управлении воздушным судном;
ASE (Altimetry System Error) – погрешность системы измерения высоты полета;
CFL (Cleared Flight Level) – разрешенный эшелон полета;
CVSM (Conventional Vertical Separation Minima) – традиционные минимумы вертикального эшелонирования в 2000 футов между эшелонами FL290 и FL410 (включительно);
FLAS (Flight Level Allocation Scheme) – схема распределения эшелонов;
RVSM (Reduced Vertical Separation Minima) – сокращенный минимум вертикального эшелонирования в 1000 футов между эшелонами FL290 и FL410 (включительно);
TVE (Total Vertical Error) – суммарная погрешность в выдерживании высоты полета (разрешенного эшелона).
8.13.1.	Подготовка к полетам
8.13.1.1. При проведении подготовки к полету в воздушном пространстве RVSM дополнительно выполняется:
а)	проверка наличия допуска к RVSM ВС, на котором планируется выполнять полет;
б)	проверка прохождения подготовки и допуска к полетам в ВП RVSM планируемых регионов полетов у членов экипажа;
в)	определение точек входа и выхода в/из ВП RVSM, ознакомление с FLAS, транзитными и буферными зонами RVSM, в которых требуется смена эшелонов RVSM;
г)	повторение эксплуатационных процедур при выполнении полета с использованием СВЭ RVSM;
д)	повторение фразеологии и действий экипажа при попадании в особые ситуации, связанные с возможностью выполнять полет в ВП RVSM;
е)	ознакомление с FPL на предстоящий полет в ВП RVSM, а также планирование альтернативного FPL на случай замены ВС на не допущенное к RVSM или отказа СВЭ RVSM.
8.13.1.2. При проведении предполетной подготовки в базовом аэропорту дополнительно выполняется:
а)	уточнение № ВС, выделенного для выполнения полета, проверка наличия ограничений по MEL (Aircraft Status для RVSM);
б)	получение подтверждения полетного диспетчера об обеспечении полета FPL. Если планируемое для выполнения полета ВС не допущено к RVSM, то полетный диспетчер представляет в органы ОВД альтернативный FPL;
в)	уточнение у полетного диспетчера возможных ограничений по применению RVSM, которые могут быть введены полномочным органом ОВД по NOTAM или в связи со сложными метеорологическими условиями в районе предстоящего полета;
г)	осмотр приемников статического давления на предмет отсутствия царапин и вмятин.
Наземный контроль СВЭ RVSM проводится независимо двумя членами экипажа.
СВЭ RVSM данного ВС считаются неисправными, а ВС теряет статус «Допущенного к RVSM» ВС, если:
а)	обнаружена неисправность одного из двух или обоих основных высотомеров;
б)	при нахождении на земле при установленном одинаковом давлении аэродрома (QFE или QNH) показания двух основных высотомеров левого и правого пилотов отличаются на величину 75 фут (23) м и более;
в)	неисправность системы контроля и сигнализации отклонения от заданной высоты;
г)	обнаружена неисправность ответчика ATC (на ПУ горит индикатор отказа ответчика) или двух каналов связи ответчика с основными высотомерами.
Вылет с неисправными СВЭ RVSM в соответствии с FPL, составленным для ВС со статусом «Допущенного к RVSM» из базового аэропорта запрещается. В этом случае экипаж сообщает в ЦУП Авиакомпании о неисправности оборудования СВЭ и потере статуса ВС, допущенного к RVSM. Вылет переносится до устранения неисправности, либо полетный диспетчер отменяет действующий FPL и предоставляет альтернативный FPL, составленный для ВС, не имеющего статуса «Допущенного к RVSM».
8.13.2.	Выполнение полетов в ВП RVSM
Разрешение на вход в воздушное пространство RVSM орган ОВД выдает при наличии достоверной информации о допуске ВС к полетам в условиях RVSM. При отсутствии такой информации диспетчер органа ОВД обязан запросить наличие допуска к полетам в условиях RVSM у экипажа ВС фразой «CONFIRM RVSM APPROVED». Экипаж ВС информирует орган ОВД о наличии допуска к полетам в условиях RVSM фразой «AFFIRM RVSM» или об его отсутствии – «NEGATIVE RVSM».
8.13.3.	Выдерживание эшелонов RVSM
8.13.3.1. Выдерживание эшелона RVSM осуществляется по одному из двух основных высотомеров. Осреднение показаний высотомеров на эшелонах RVSM не допускается.
8.13.3.2. Отключение (подавление) сигнализации отклонения от заданной высоты не допускается. Значение эшелона полета (FL), установленное на задатчике, ни при каких обстоятельствах не изменяется в течение всего полета на данном эшелоне.
8.13.3.3. Включается режим автоматической стабилизации высоты. Отключение этого режима допускается только в целях балансировки самолета, в условиях сильной и умеренной болтанки, а также для устранения отклонения от заданной высоты (AAD) на величину более 65 ft (20м). Если колебания высоты в режиме автоматической стабилизации высоты при выдерживании эшелона полета превышают 65 ft (20м), то этот режим работы для эшелонов RVSM считается неисправным, и ВС теряет статус «Допущенного к RVSM».
8.13.3.4. Контроль выдерживаемого эшелона полета осуществляется по второму основному высотомеру сравнением его показаний с высотомером, используемым для выдерживания эшелона полета.
Если расхождение в показаниях основных высотомеров (левого и правого пилота) не превышает 200 ft (60м), то работа высотомеров считается нормальной. Если расхождение в показаниях основных высотомеров превышает 200 ft (60м), то основные высотомеры для эшелонов RVSM считаются неисправными и экипаж действует в соответствии с положениями «Действия при потере ВС статуса «Допущенного к RVSM».
8.13.3.5. При длительном (более 1 часа) полете на одном эшелоне экипаж через каждые 60 минут полета производит:
а)	контроль установки (индикации) стандартного давления на основных и дополнительных высотомерах;
б)	сравнение показаний основного высотомера второго пилота с показаниями основного высотомера командира ВС;
в)	контроль исправности ответчика ATC (индикатор отказа ответчика не горит).
Если обнаружены неисправности или отклонения, превышающие допустимые значения для ВП RVSM, экипаж действует в соответствии с положениями «Действия при потере ВС статуса «Допущенного к RVSM».
8.13.4.	Изменение эшелонов RVSM
8.13.4.1. Изменение эшелонов RVSM производится после получения соответствующего указания или разрешения органа ОВД, при этом экипажу ВС необходимо:
а)	установить новое значение CFL на задатчиках - сигнализаторах высоты СВЭ;
б)	плавным изменением угла тангажа установить вертикальную скорость набора или снижения. При наличии на индикаторе TCAS отметок от близких ВС в радиусе 5 NM (10 км) и при разнице высот ±2000 ft и менее выдерживается вертикальная скорость не более 5 м/с;
в)	занятие нового эшелона выполнять таким образом, чтобы исключить «проскакивание» CFL на величину более 150 ft (45м) независимо от режима управления ВС и условий полета.
8.13.4.2. ВС считается занявшим указанный в разрешении эшелон полета, если информация о высоте полета ВС, основанная на данных о барометрической высоте, свидетельствует о том, что ВС находится относительно заданного эшелона в пределах ±200 ft (±60м) – в воздушном пространстве RVSM или ±300 ft (±90м) – в воздушном пространстве без применения RVSM в течение 15 секунд.
8.13.4.3. После занятия нового эшелона полета в горизонтальном полете по команде командира ВС сравниваются показания двух основных высотомеров и определяется величина расхождения в их показаниях.
Командир ВС дополнительно убеждается в том, что при занятии нового CFL сигнализация об отказе ответчика, отклонении от заданной высоты и о расхождении в показаниях основных высотомеров отсутствует.
8.13.5.	Действия при потере воздушным судном статуса «Допущенного к RVSM»
8.13.5.1. Воздушное судно теряет статус «Допущенного к RVSM», если при проверке на земле или в полете проявился один или несколько перечисленных ниже отказов или отклонений в работе СВЭ RVSM:
а)	техническая неисправность (отказ, непрохождение теста встроенного контроля) хотя бы одного из двух основных высотомеров;
б)	при нахождении на земле при установленном одинаковом давлении аэродрома (QFE или QNH) показания двух основных высотомеров левого и правого пилотов отличаются на величину 75 фут (23) м и более;
в)	в полете показания двух основных высотомеров левого и правого пилотов отличаются на величину, превышающую 200 ft (60м);
г)	обнаружена неисправность системы контроля и сигнализации отклонения от заданной высоты;
д)	обнаружена неисправность ответчика, либо по информации органа ОВД становится очевидным, что передаваемая ответчиком высота отличается от фактической высоты, выдерживаемой по основному высотомеру, на величину, превышающую 300 ft (90 м), а переход на другой канал высоты не исправил ситуацию;
е)	по любой причине невозможно включение режима автоматической стабилизации высоты, либо при включении этого режима наблюдаются отклонения в стабилизации высоты, превышающие 65 ft (20м);
ж)	от органа ОВД получено сообщение о том, что в соответствии с показаниями системы контроля характеристик выдерживания высоты, полет выполняется с TVE, превышающей 300 ft (90м), и (или) ASE, превышающей 245 ft (75 м).
з)	8.13.5.2. Если ВС теряет статус «Допущенного к RVSM» при нахождении в ВП RVSM, экипаж немедленно сообщает органу ОВД о потере статуса из-за отказа оборудования фразой «UNABLE RVSM DUE EQUIPMENT» и действует в соответствии с диспетчерскими указаниями или разрешениями.
8.13.5.3. Орган ОВД, при получении от экипажа ВС информации о потере статуса RVSM обеспечивает вертикальное эшелонирование 2000 ft (600м) между данным ВС и любыми другими ВС, выполняющими полеты в воздушном пространстве RVSM, а также до следующего пункта передачи управления между смежными органами ОВД принимает меры по сохранению упорядоченного потока воздушного движения и задает данному ВС эшелон полета ниже эшелона полета 290.
8.13.5.4. При принятии решения о продолжении полета до аэродрома назначения экипажу следует учитывать, что наиболее вероятно снижение ниже ВП RVSM (FL 280 и ниже) и что снижение будет выполняться в зависимости от воздушной обстановки.
8.13.5.5. Во всех случаях, когда отказы или отклонения в работе СВЭ RVSM, приведшие к потере статуса «Допущенного к RVSM», устранены или более не проявляются, считается, что статус «Допущенного к RVSM» восстановлен. Если до этого органу ОВД уже было сообщено о потере статуса, то независимо от того, снижено ВС или еще нет, экипаж немедленно сообщает о восстановлении статуса «Допущенного к RVSM» фразой «READY TO RESUME RVSM» и при необходимости, запрашивает требуемый эшелон полета


8.13.6.	Действия при умеренной или сильной турбулентности в ВП RVSM
8.13.6.1. При попадании в зону умеренной или сильной турбулентности, требующей отключение автопилота, или вследствие которой отклонения от заданного эшелона полета (AAD) превышают 65ft (20 м), экипаж немедленно сообщает об этом органу ОВД фразой «UNABLE RVSM DUE TO TURBULENCE» и действует в соответствии с диспетчерскими указаниями или измененным разрешением.
8.13.6.2. В этом случае ВС не теряет статус «Допущенного к RVSM», но в зависимости от ситуации, диспетчер ОВД может либо применить минимум вертикального эшелонирования в 2000 ft (600м), что потребует изменения эшелона полета, либо изменить направление полета для создания бокового эшелонирования с ВС, следующими на смежных эшелонах RVSM.
8.13.6.3. При наличии информации о сильной турбулентности диспетчер органа ОВД может прекратить применение RVSM в полном объеме или для определенного диапазона эшелонов и (или) соответствующего района.
8.13.6.4. При прекращении или ослаблении турбулентности, когда становится возможным включение режима автоматической стабилизации высоты и точное выдерживание эшелона RVSM, и если до этого органам ОВД уже было сообщено о болтанке, экипаж немедленно сообщает о восстановлении выдерживания требований RVSM фразой «READY TO RESUME RVSM» и действует в соответствии с указаниями или разрешениями диспетчера.
8.13.7.	Действия при нарушениях требований RVSM
8.13.7.1. Каждый случай нарушения требований в отношении выдерживания CFL RVSM в воздушном пространстве RVSM расследуется полномочными органами ОВД. К таким случаям относятся:
а)	TVE, равная или превышающая 300 ft (90 м);
б)	ASE, равная или превышающая 245 ft (75м);
в)	AAD, равная или превышающая 300 ft (90 м).
8.13.7.2. Если при сообщении об AAD в результате проверки показаний основных высотомеров стало очевидным, что причиной такой ситуации явилась ложная информация о высоте, передаваемая ответчиком, об этом немедленно сообщить органу ОВД (рекомендуемая фраза- «TRANSPONDER´S ALTITUDE REPORTING ERROR»). При такой ситуации нарушение требований RVSM не фиксируется, и для органов ОВД ВС не теряет статуса «Допущенного к RVSM».
Если переключение на другой канал высоты ответчика не устранило ошибку, то органами ОВД ВС рассматривается как «ВС с отказавшим ответчиком по каналу высоты» с применением соответствующих процедур обслуживания таких ВС.
8.14.	Процедуры при отказах навигационных средств 
8.14.1.	Общие положения
8.14.1.1. Навигационными средствами (в контексте данного пункта РПП) считаются системы или приборы, которые обеспечивают:
а)	программирование и/или выдерживание траектории полета, контроль прохождения рубежей, расчет времени; и/или
б)	предоставление информации органам ОВД о месте ВС, векторе путевой скорости, расчетном времени пролета контрольных рубежей и т.д.; и/или
в)	использование как одного из датчиков навигационных комплексов и систем или как средства их коррекции (автоматической или ручной). 
8.14.1.2. Отказы навигационных средств подразделяются на:
а)	сигнализируемые, при которых экипажу выдается предупреждение об отказе;
б)	несигнализируемые, при которых предупреждение об отказе не выдается, но экипаж определил, что навигационная информация либо ложная, либо неустойчивая, либо в ее достоверности невозможно убедиться.
8.14.1.3. Экипаж должен уметь однозначно интерпретировать сигнализируемые отказы, а также своевременно распознавать и классифицировать несигнализируемые отказы навигационных средств.
Экипаж должен уметь оценивать влияние отказа навигационного средства на навигационные характеристики ВС в целом и, при выполнении полета в условиях установленных требуемых навигационных характеристик (RNP), способность их дальнейшего выдерживания.
ОТКАЗЫ НАВИГАЦИОННЫХ СРЕДСТВ В ПОЛЕТЕ СЧИТАЮТСЯ ОСОБОЙ СИТУАЦИЕЙ И В СЛУЧАЕ, ЕСЛИ НЕТ УГРОЗЫ ИЛИ ФАКТА ПОТЕРИ ОРИЕНТИРОВКИ, НЕ ДОЛЖНЫ ВЫЗЫВАТЬ ЭКСТРЕННЫХ ИЛИ ЧРЕЗВЫЧАЙНЫХ МЕР СО СТОРОНЫ ЭКИПАЖА. 
8.14.1.4. Угроза потери ориентировки, когда ни экипаж, ни орган ОВД не могут контролировать положение ВС в пространстве, это чрезвычайная ситуация, требующая применения соответствующих процедур. 
ПОТЕРЯ ОРИЕНТИРОВКИ, КОГДА НИ ЭКИПАЖ, НИ ОРГАН ОВД НЕ ИМЕЮТ И НЕ МОГУТ ВОССТАНОВИТЬ КОНТРОЛЬ ЗА ПОЛОЖЕНИЕМ ВС В ПРОСТРАНСТВЕ, ЯВЛЯЕТСЯ СИТУАЦИЕЙ БЕДСТВИЯ И ТРЕБУЕТ ПРИМЕНЕНИЯ СООТВЕТСТВУЮЩИХ ПРОЦЕДУР И МЕР.
8.14.1.5. Восстановление контроля положения ВС в пространстве экипажем или органом ОВД отменяет ситуацию бедствия, связанную с потерей ориентировки, но не отменяет чрезвычайную ситуацию и необходимость выполнения соответствующих процедур. Данное требование обусловлено возможным состояние стресса экипажа и вероятностью повторной потери ориентировки.
8.14.2.	Отказ систем счисления координат
8.14.2.1. ПРИ ПОЛНОМ ОТКАЗЕ СИСТЕМ СЧИСЛЕНИЯ КООРДИНАТ ЭКИПАЖУ НЕОБХОДИМО:
а)	перейти на любой приемлемый способ выдерживания ЛЗП (см. Процедура А.4);
б)	оценить способность выдерживания маршрута полета с помощью работоспособных навигационных средств и принять решение о возврате (посадке на промежуточном аэродроме) или следовании на аэродром назначения;
в)	если целесообразно - доложить органу ОВД об отказе навигационного средства и принятом решении, например - фразой «...отказ навигационной системы. Следую по плану»;
г)	фиксировать в штурманском бортжурнале время полета каждого ППМ и путевой угол (курс) следования. В момент пролета ППМ пускать секундомер;
д)	запросить радиолокационную поддержку у органа ОВД;
е)	отклонения от заданного маршрута с целью обхода гроз производить только в сторону, где имеется достаточно наземных РТС, которые могут обеспечить контроль за МС с помощью бортовых средств или наземных РЛС;
ж)	при неустойчивой работе бортовых радиосредств или отсутствии радиолокационной поддержки от органа ОВД - изменить направление полета в сторону наземных РТС или органов ОВД, о чем немедленно доложить органу ОВД;
з)	в целях увеличения дальности действия УКВ-радиосредств (радиолокационного наблюдения) - занимать наиболее высокие эшелоны полета. 
Примечание: При устойчиво работающей СНС, отказ системы счисления не требует применения каких-либо процедур, если экипаж имеет допуск к полетам с использованием СНС. 
8.14.3.	Отказ курсовой системы
При полном отказе курсовой системы - экипажу, независимо от состава и состояния остального оборудования:
а)	оценить возможность продолжения полета на аэродром назначения и принять решение о продолжении полета или производстве вынужденной посадки;
б)	использовать дублирующие индикаторы курса;
в)	доложить органу ОВД об отказе курсовой системы и принятом решении (например, «.....отказ курсовой системы, решение: возвращаюсь в Казань»);
г)	запросить радиолокационную поддержку у органа ОВД;
д)	если целесообразно - увеличить высоту полета в целях увеличения дальности действия УКВ-радиосредств (радиолокационного наблюдения);
е)	проанализировать погоду на аэродроме предполагаемой посадки и наличие радиолокационного обслуживания на нем. Если посадка на запланированном аэродроме по метеоусловиям может вызвать затруднения (хуже минимума для визуального захода) - подобрать наиболее подходящий аэродром и следовать на него;
ж)	отклонения от заданного маршрута с целью обхода гроз производить только в сторону, где имеется достаточно наземных РТС, которые могут обеспечить контроль за МС с помощью бортовых средств или наземных РЛС;
з)	при неустойчивой работе бортовых радиосредств или отсутствии радиолокационной поддержки от органа ОВД - изменить направление полета в сторону наземных РТС или органов ОВД, о чем немедленно доложить органу ОВД;
и)	по возможности избегать длительного полета в облаках и, если ожидается наступление сумерек или темноты, произвести посадку на любом подходящем аэродроме до наступления темноты на нем.
8.14.4.	Отказ бортовых радиотехнических навигационных средств
8.14.4.1. Отказ бортовых радиотехнических навигационных средств не требует применения каких-либо специальных процедур, если отказавшее средство не использовалось или не предполагает быть использовано как основное навигационное средство на каком-либо этапе полета.
8.14.4.2. При отказе радионавигационного средства, имеющего статус «основного», экипажу:
а)	определить средство, которое может заменить отказавшее для целей навигации (например, СНС может заменить любое отказавшее средство, кроме ILS);
б)	определить влияние отказа на навигационные возможности ВС в целом (например, отказ двух комплектов КУРС-МП приводит к невозможности захода на посадку по системе ILS, а отказ только одного комплекта - к заходу на посадку по ILS по минимуму I и II Кат. ICAO);
в)	наметить систему захода на посадку, определить минимум для захода на посадку с отказавшим средством, проанализировать погоду аэродрома назначения и запасного, принять решение о возможности продолжения запланированного полета с отказавшим средством или посадке на любом подходящем аэродроме.
8.14.4.3. Информировать орган ОВД об отказе нужно только в том случае, если отказ существенно влияет на навигационные возможности ВС в целом и приводит к невозможности выполнения полета согласно FPL (RPL) и выполнения захода на посадку по объявленной в ATIS или предлагаемой ОВД схеме захода.
8.14.4.4.	При полетах в зоне действия правил и процедур RNAV 5:
(1) КРАТКОВРЕМЕННАЯ (ДО 5-ТИ МИНУТ) ПОТЕРЯ RAIM СНС.
При кратковременной потере RAIM СНС какие-либо специальные процедуры не выполнять, и, включив секундомер, произвести контроль работы СНС. 
(2) Потеря RAIM СНС на время от 5-ти до 10-ти минут.
При потере RAIM на время более 5-ти минут:
а)	скорректировать текущие координаты на бортовых навигационных вычислителях по любым имеющимся достоверным данным, в том числе по информации на СНС;
б)	производить перекрестный контроль полета, используя все имеющиеся средства навигации;
в)	в случае значительных расхождений в навигационной информации от СНС и от дублирующих (дополнительных) средств - перейти на контроль полета по дублирующим средствам (КУРС-МП, СД, АРК), о чем немедленно доложить органу ОВД фразой «... TUM 1031 NEGATIVE RNAV...»;
г)	выполнять указания ОВД в отношении полета без каких-либо пререканий и предложений;
д)	использовать СНС только как дополнительное средство навигации. 
ПОТЕРЯ RAIM НА ВРЕМЯ БОЛЕЕ 10-ТИ МИНУТ.
Потеря RAIM на время более 10-ти минут приравнивается к потере навигационной функции СНС и его использование в качестве основного или дополнительного средства навигации - запрещается.
При этом:
а)	перейти на контроль полета по дублирующим средствам навигации со следующими приоритетами:
	VOR/DME;
•	NDB;
•	НВУ;
б)	доложить органу ОВД об отказе фразой «... TUM 1031 NEGATIVE RNAV...», повторяя это сообщение при каждой смене частоты связи с органом ОВД; 
в)	выполнять указания ОВД в отношении полета;
г)	предпринять все возможные попытки для восстановления работы СНС, которые предусмотрены в данной ситуации в РЛЭ и инструкции по эксплуатации СНС.
(3) ОТКАЗ СНС.
При появлении признаков отказа или потере функции навигации вследствие перезапуска процессора, постановки помех, программного или аппаратного отказа KLN-90B выполнить следующие действия:
а)	если работа СНС не восстанавливается в течение 60-90 сек, то СНС выключить и через 1 - 2 мин включить повторно. Если при этом работа СНС не восстановилась самолетовождение осуществлять по другим навигационным средствам. 
При этом: 
б)	перейти на контроль полета по дублирующим средствам навигации; 
в)	доложить органу ОВД об отказе фразой «... TUM 1031 NEGATIVE RNAV...», повторяя это сообщение при каждой смене частоты связи с органом ОВД;
г)	выполнять указания ОВД в отношении полета;
Во всех случаях отказов или неуверенности в работе основных бортовых навигационных средств экипажам запрашивать радиолокационную поддержку полета фразой: “...TUM 1031 NEGATIVE RNAV, REQUEST RADAR VECTORING.
Потеря статуса RVSM
(4) Воздушное судно теряет статус допущенного к RVSM, если при проверке на земле или в полете проявился один или несколько перечисленных ниже отказов или отклонений в работе СВЭ RVSM:
а)	неисправность (отказ, непрохождение теста встроенного контроля) одного из двух или обоих основных высотомеров;
б)	при нахождении на земле и при установленном одинаковом давлении аэродрома (QFE или QNH) показания двух основных высотомеров левого и правого пилотов отличаются на величину 75 ft (23 м) и более;
в)	в полете показания двух основных высотомеров левого и правого пилотов отличаются на величину, превышающую 200 ft (60 м);
г)	неисправность ответчиков вторичной радиолокации, либо по информации органа ОВД становится очевидным, что передаваемая ответчиком высота отличается от фактической высоты, выдерживаемой по основному высотомеру, на величину, превышающую 300 ft (90 м), а переход на другой канал высоты ACAS не исправил ситуацию;
д)	по любой причине невозможно включение режима автопилота стабилизации высоты полета, либо при включении этого режима наблюдаются отклонения в стабилизации высоты, превышающей 65 ft (20 м);
е)	от органа ОВД получено сообщение о том, что в соответствии с показаниями системы контроля характеристик выдерживания высоты полет выполняется TVE, превышающий 300 ft (90 м), и /или ASE, превышающей 245 ft (75 м).
8.15.	Процедуры установки давления на высотомерах
8.15.1.	Общие положения
8.15.1.1. При выполнении полетов используются барические уровни начала отсчета высот, соответствующие следующим видам атмосферного давления:
а)	стандартное атмосферное давление (далее – QNE) – 760 мм. рт.ст.;
б)	давление аэродрома (далее - QFE);
в)	давление аэродрома, приведенное к среднему уровню моря по МСА (QNH аэродрома);
г)	минимальное из приведенных к среднему уровню моря по МСА давлений в пределах района ЕС ОрВД (установленного участка района ЕС ОрВД) (далее – QNH района).
Давление аэродрома, передаваемое экипажу ВС, может относиться либо к уровню контрольной точки аэродрома, либо к уровню рабочего порога ВПП.
8.15.1.2. Отсчет барометрической высоты полета воздушного судна производится при полётах:
а)	в зоне взлёта и посадки аэродрома по атмосферному давлению на уровне рабочего порога ВПП (QFE) или давлению, приведённому к уровню моря (QNH): при вылете - от взлета до высоты перехода и на высоте перехода, при прилёте - ниже эшелона перехода и до посадки; или до установленного рубежа зоны взлёта и посадки аэродрома (при вылете) и с установленного рубежа зоны взлёта и посадки аэродрома (при прилете), если планом полёта пересечение высоты перехода/эшелона перехода не предполагается;
б)	выше высоты перехода, на и выше эшелона перехода – по стандартному атмосферному давлению (QNE);
в)	ниже нижнего эшелона при полётах по ПВП – по давлению, приведенному к уровню моря Рприв.мин. по маршруту.
8.15.2.	Установка давления на шкалах барометрических высотомеров 
8.15.2.1. При вылете на высотомерах устанавливается значение высоты «0», если опубликованные схемы/маршруты выхода предусматривают выдерживание высот относительно уровня рабочего порога ВПП (по давлению QFE) или значение высоты равное относительному превышению порога ВПП над уровнем моря, если опубликованные схемы/маршруты выхода предусматривают выдерживание высот по давлению, приведенному к уровню моря (QNH).
8.15.2.2. Установка атмосферного давления на шкалах барометрических высотомеров в последовательности: КВС, второй пилот, штурман (при наличии в экипаже).
Производится:
а)	при вылете: с атмосферного давления на уровне рабочего порога ВПП (QFE) или давления приведённого к уровню моря (QNH) на стандартное атмосферное давление (QNE) – при пересечении высоты перехода или на Р прив. мин. по маршруту при пересечении установленного рубежа зоны взлёта и посадки аэродрома вылета;
б)	при прилете: со стандартного атмосферного давления (QNE) или с Рприв.мин. (QNH района) по маршруту на атмосферное давление на уровне рабочего порога ВПП (QFE) или давление, приведённое к уровню моря (QNH), в соответствии с опубликованными маршрутами/схемами прибытия – при пересечении эшелона перехода или установленного рубежа зоны взлёта и посадки аэродрома прибытия.
8.15.2.3. При выполнении полета на ВС отечественного производства оборудованных барометрическими высотомерами (кроме ВС оборудованных ВБЭ СВС) экипаж учитывает поправки высотомеров в соответствии со следующими нормами:
а)	Показания высотомеров с отклонениями δН1 и δН2 по абсолютной величине до 60 м на эшелонах с интервалами 300 м и до 100 м на эшелонах с интервалами 600 м можно считать нормальными. 
б)	Если отклонения δН1 и δН2 по абсолютной величине превышают соответственно 60 и 100 м, то в полете производится осреднение показаний.
в)	Для ВС, на которых установлено не менее трех высотомеров, вторичная поправка вычисляется по формуле ΔН = - (δН1 +δН2) / 3, а ВС, имеющих два высотомера, по формуле ΔН = δН1/2.
г)	Если после вычисления значение вторичной поправки более 20 м - необходимо определить новое значение заданной высоты (только для основного высотомера) Н2пр=Н1пр + ΔН. Новое значение заданной высоты для основного высотомера должно быть записано в штурманский бортовой журнал Если отклонения δН1 или δН2 на эшелонах с интервалами 300 м по абсолютной величине превышают 100 м, а на эшелонах с интервалами 600 м превышают 200 м, то КВС обязан сообщить диспетчеру ОВД о невозможности точного выдерживания заданной высоты полета. Решение по какому высотомеру производить выдерживание заданного эшелона принимает КВС в зависимости от конкретных условий.
8.15.2.4. На ВС оборудованных системой воздушных сигналов (СВС) или электромеханическими высотомерами, экипаж обязан выдерживать высоты по приборам имеющими выход на самолетный ответчик. 
Процедуры перекрестного контроля при перестановке давления на высотомерах выполняются в соответствии с Инструкцией по взаимодействию и технологией работы экипажа (SOP) типа ВС. (Дополнительно смотри п. 8.3.1.9). Контроль работоспособности высотомерного (истинность показаний, допустимые значения погрешностей и т.д.) оборудования производится в соответствии с РЛЭ и ВС.
8.15.2.5. Применяемые единицы измерения:
а)	атмосферное давление: миллиметры ртутного столба (мм.рт.ст.), милибары (mb), гектопаскали (hрa), инчи (in). Стандартным значением атмосферного давления (QNE) считается: 760 мм.рт.ст.,1013,2 mb, 1013,2 hрa, 29,92 in;
б)	высота полёта может измеряться в метрах или футах. 
При выполнении полётов и подготовке к ним применяются единицы измерения в соответствии с опубликованными в документах аэронавигационной информации сведениям для воздушного пространства, в котором выполняется полёт.
8.15.2.6. Посадка на неконтролируемом аэродроме (площадке), давление на котором органом ОВД экипажу не предоставлено, производится по минимальному атмосферному давлению по маршруту (участку) полета, приведенному к уровню моря. При этом экипаж должен учитывать высоту рельефа местности расположения аэродрома (площадки).
8.15.2.7. Перед вылетом с неконтролируемого аэродрома (площадки), давление на котором органом ОВД экипажу не предоставлено, установка шкалы давления барометрического высотомера производится:
а)	для взлета QNH – или установкой абсолютной высоты аэродрома.
8.15.2.8. При полете по маршруту за пределами района аэродрома перевод шкал давления барометрических высотомеров членов летного экипаж ВС, установленным РЛЭ производится:
а)	при пересечении высоты перехода района в наборе высоты- с давления QNH района на стандартное давление (QNЕ);
б)	при пересечении эшелона перехода района в снижении - со стандартного атмосферного давления (QNЕ) на давление QNH района. 
На горных аэродромах осуществляются по QNH, при этом необходимо:
а)	перед взлетом установить известное значение QNH. Показания высотомера, соответствующие абсолютной высоте аэродрома, в этом случае принимаются за «условный нуль», относительно которого производится набор заданной высоты полета;
б)	перед посадкой на контролируемом аэродроме диспетчер ОВД сообщает экипажу абсолютную высоту аэродрома и значение QNH, которое устанавливается на высотомере. В этом случае высотомер будет показывать абсолютную высоту полета, а в момент приземления – абсолютную высоту аэродрома над уровнем моря.
8.15.2.9. При наборе заданного эшелона полета воздушным судном, выполняющим полет по маршруту ниже нижнего (безопасного) эшелона, перевод шкалы давления барометрического высотомера с минимального давления, приведенного к уровню моря, на стандартное давление производится на высоте перехода района ЕС ОрВД, сообщаемой экипажу органом ОВД, осуществляющим обслуживание полета данного ВС. 
8.15.2.10. Выдерживание высот полёта от соответствующего барического уровня осуществляется по приборам, индицирующим их значение в единицах измерения установленных для воздушного пространства, в котором выполняется полёт. При использовании высотоизмерительного оборудования, имеющего возможность изменения индикации единиц измерения (ft,м), применение единиц измерения не соответствующих установленным для воздушного пространства в котором выполняется полёт, является недопустимым.
8.15.3.	Предоставление информации
8.15.3.1. Соответствующие органы ОВД в любой момент времени имеют в своем распоряжении для передачи на борт находящихся в полете ВС по запросу информацию, необходимую для определения самого нижнего эшелона полета, который обеспечит достаточный запас высоты над местностью на маршрутах или участках маршрутов, в отношении которых эта информация требуется.
Примечание: Если это предписывается на основе региональных аэронавигационных соглашений, такая информация может включать климатологические данные.
8.15.3.2. Центры полетной информации и районные диспетчерские центры имеют в своем распоряжении для передачи на борт находящихся в полете ВС по запросу соответствующее количество донесений о QNH или прогнозов атмосферного давления в отношении районов полетной информации и диспетчерских районов, находящихся в их ведении.
8.15.3.3. Эшелон перехода включается в диспетчерские разрешения на заход на посадку, когда это предписывается соответствующим полномочным органом, или по запросу пилота.
8.15.3.4. В разрешения захода на посадку или разрешения на вход в аэродромный круг, а также в разрешения на руление, выдаваемые вылетающим ВС, включаются данные для установки высотомера по QNH, за исключением тех случаев, когда известно, что ВС уже получило эту информацию.
8.15.3.5. Экипажи ВС обеспечиваются данными для установки высотомера по QFE по запросу или на постоянной основе в соответствии с договоренностью на местах, данные соответствуют QFE на превышении аэродрома.
8.16.	Правила эксплуатации задатчика - сигнализатора заданной высоты
Система сигнализации высоты эшелонирования предназначена для предупреждения экипажа о подходе ВС к заданной высоте и об отклонении от нее. 
Члены экипажей ВС обязаны контролировать выдерживание заданных высот с помощью систем предупреждения о высоте. Порядок эксплуатации систем предупреждения о высоте изложен в руководствах по летной эксплуатации РЛЭ (AFM, FCOM) в части описания систем приборного оборудования.
Назначение системы предупреждения о высоте состоит в том, чтобы предупредить экипаж автоматическим визуальным и/или звуковым сигналом о достижении или освобождении заданной высоты или эшелона полёта. Работа системы должна гарантировать точное выдерживание высоты на всех этапах полёта. Система предупреждения о высоте должна использоваться как регистратор деблокирования высоты, а не как устройство для напоминания доклада высот.
При срабатывании сигнализации об отклонении от заданной высоты (эшелона) полета или при обнаружении отклонения по показаниям высотомеров необходимо сличить показания высотомеров, определить высотомер с наиболее правильными показаниями и, руководствуясь его показаниями, скорректировать высоту полета.
8.17. Использование системы предупреждения опасного сближения с землей 
         СРПБЗ/ EGPWS, ССОС 
8.17.1. Инструкции и требования по обучению методам предотвращения столкновения исправного ВС с землей, принципы использования системы предупреждения о близости земли (СРПБЗ)
Для допуска членов летного экипажа самолета к выполнению своих функций они должны пройти подготовку по программам «Подготовки летных экипажей по предотвращению авиационных происшествий категории CFIT».
8.17.1.1.	Принципы использование системы предупреждения о близости земли (СРПБЗ)
Система раннего предупреждения близости земли (РПБЗ) обеспечивает своевременное оповещение экипажа о возникновении условий полета, развитие которых может привести к непреднамеренному опасному сближению ВС с земной или водной поверхностью, а также о пролете характерных высот на снижении и при заходе на посадку. 
(1) Назначение системы:
Предотвращение столкновения исправных и управляемых воздушных судов с земной поверхностью.
Основные функции:
а)	предупреждение опасного сближения с землей;
б)	отображение характера подстилающей поверхности и искусственных препятствий на дисплее GPWS;
в)	оценка местности в направлении полета;
г)	предупреждение о преждевременном снижении;
д)	сигнализация прохода высоты 150м;
ж)  предоставление информации на дисплее GPWS о препятствиях.
Используемые базы данных:
а)	цифровые модели рельефа (ЦМР), как способа представления электронных данных о местности;
б)	общие характеристики ЦМР;
в)	характеристики ЦМР на различных этапах полета;
г)	аэронавигационные данные, применяемые в GPWS;
д)	данные об искусственных препятствиях, которые могут угрожать безопасности полетов ВС.
Сигнал предупреждения или тревоги системы РПБЗ информирует экипаж ВС о том, что траектория полета воздушного судна по высоте или направлению отличается от запланированной.
Принцип работы системы основан на сравнении текущего превышения траектории движения ВС над земной поверхностью с минимально допустимым превышением относительно цифровой модели рельефа местности, заложенной в базе данных, и последующим формированием предупреждающих сигналов или команд.
Ответная реакция экипажа на все сигналы тревоги или предупреждения, выдаваемые системой, должна быть немедленной и правильной, установление причины активации имеет второстепенное значение.
(2) При срабатывании сигнализации системы GPWS ПКУ дублирует команду или сигнал предупреждения системы GPWS. ПАУ предпринимает превентивные меры до прекращения сигнализации системы GPWS. При непринятии ПАУ своевременных адекватных сложившейся ситуации действий ПКУ берет управление на себя, информирует об этом летный экипаж. В дальнейшем ПКУ предпринимает превентивные меры до прекращения сигнализации системы GPWS. Решение о последующем распределении функций ПАУ и ПКУ между пилотами принимает КВС в тот момент времени, который он сочтет приемлемым для сложившейся ситуации.
(3) При выдаче команды «DON'Т SINK–проверить вертикальную скорость, высоту полета. При необходимости перевести ВС в набор высоты, двигатели на взлетный режим - убедиться в отключении сигнализации.
(4) При появлении сигнала «TOO LOW, TERRAIN» - проконтролировать высоты полета по барометрическому и радио высотомерам, проверить положение шасси, механизации крыла – выпустить их при необходимости или перевести ВС в набор высоты. Убедиться в отключении сигнала.
(5) При появлении сигнала «TOO LOW, GEAR» - проверить положение шасси, проконтролировать высоту полета по радиовысотомеру. Если шасси не выпущено, необходимо перевести ВС в набор высоты, двигатели на взлетный режим. Убедиться в отключении сигнала.
(6) При появлении сигнала «TOO LOW, FLAPS» - проверить положение закрылков, проконтролировать высоту по радиовысотомеру. Если механизация не выпущена, перевести ВС в набор высоты, двигатели на взлетный режим. Убедиться в отключении сигнала.
(7) При появлении сигнала «TERRAIN, TERRAIN» - проверить высоту по радиовысотомеру, при необходимости, перевести ВС в набор высоты, двигатели на взлетный режим. Убедиться в отключении сигнала.
(8) При появлении сигнала «TERRAIN, PULL UP» -проверить высоту полета визуально и по радиовысотомеру.
При необходимости, энергично перевести ВС в набор высоты, не допуская превышения ограничений по перегрузке и углу атаки, двигатели перевести на взлетный режим. Убедиться в отключении сигнала.
(9) При появлении сигнала «SINK RATE» - проверить вертикальную скорость снижения, высоту полета по барометрическому и радио высотомерам. При необходимости, уменьшить вертикальную скорость снижения. Убедиться в отключении сигнала.
(10) При появлении сигнала «SINK RATE, PULL UP» - немедленно, энергично перевести ВС в набор высоты, не допуская превышения ограничений по перегрузке и углу атаки, а двигатели на взлетный режим. Убедиться в отключении сигнала.
(11) При появлении сигнала «GLIDE SLOPE» - проверить наличие сигнализации и индикации об отклонении от глиссады, отключить автопилот, перейти на ручное управление. Продолжить заход на посадку в директорном режиме или уйти на второй круг.
(12) При появлении сигнала «ALTITUDE ALERT» - сравнить высоту на барометрическом высотомере с высотой на РВ, проверить правильность установки на барометрическом высотомере давление и заданной высоты.
(13) При заходе на посадку в аэропортах, не включенных в базу данных EGPWS, отключить режим работы, связанный с анализом в направлении полета.
8.17.1.2. Конкретные процедуры реагирования экипажа на сигналы системы GPWS представлены в FCOM, QRH типа ВС.
8.17.1.3. Если сработало аварийное предупреждение системы EGPWS (GPWS warning), оно должно быть доложено диспетчеру ОВД немедленно и оформлено донесение в Авиакомпанию, используя бланк РПП (Приложение A-11.1. «Форма донесения об опасном сближении»), после завершения рейса.

8.17.2.	Система предупреждения опасного сближения с землей (ССОС)
8.17.2.1. Предназначена для предупреждения экипажа об опасном сближении ВС с землей.
Сигналы предупреждения системы сигнализации опасной скорости сближения с землей (ССОС) могут считаться неадекватными ситуации только в том случае, когда визуальные условия полета дают экипажу возможность немедленно в этом убедиться. Соответственно, когда условия полета не позволяют экипажу немедленно определить адекватность срабатывания сигнализации ССОС (ночью или днем, при отсутствии визуальных условий полета), экипаж должен незамедлительно предпринять исправляющие действия, без потери времени на выяснение причины срабатывания ССОС.
8.17.2.2. Действия экипажа в полете при срабатывании системы ССОС:
а)	на взлете немедленно прекратить снижение и перевести самолет в набор высоты;
б)	на снижении над равнинной местностью немедленно уменьшить вертикальную скорость снижения до безопасного значения;
в)	в горизонтальном полете или на снижении над холмистой или горной местностью энергично перевести самолет в набор высоты, а двигатели на взлетный режим, осуществлять контроль за режимом полета по АУАСП и не допускать выхода за допустимые значения перегрузки и угла атаки;
г)	если экипажу неизвестен рельеф местности, над которой производится полет, необходимо действовать в соответствии с рекомендациями предыдущего пункта;
д)	при выполнении предпосадочного маневра после выпуска шасси и механизации при срабатывании системы ССОС немедленно уменьшить вертикальную скорость снижения и проконтролировать правильность выдерживания заданного профиля снижения, а при полете по глиссаде и срабатывании сигнализации ССОС необходимо немедленно выполнить маневр ухода на второй круг;
е)	при полетах на малых высотах, при наличии болтанки, а также при подходе к аэродрому со сложным рельефом поверхности, на предпосадочной прямой (пролет над препятствием), в том числе, при полете по глиссаде с углом наклона более 3 градусов возможно кратковременное срабатывание (не более 2-3 сек) сигнализации ССОС;
ж)	если при выполнении маневра захода на посадку на высотах ниже 600 метров непрерывно звучит сирена от системы ССОС и горят табло «ОПАСНО ЗЕМЛЯ» и «ВЫПУСТИ ШАССИ» - НЕМЕДЛЕННО ВЫПОЛНИТЬ МАНЕВР УХОДА НА ВТОРОЙ КРУГ!
з)	действия экипажей ВС на всех этапах полета указаны в РЛЭ соответствующего типа.
8.18. Бортовая система предупреждения столкновения с воздушными судами (TCAS)
8.18.1. Принципы, инструкции, правила и требования к обучению методам 
                      предупреждения   столкновений и использований БСПС
Для допуска членов летного экипажа ВС к выполнению своих функций они должны пройти первоначальную подготовку по «Программе подготовки летного состава к летной эксплуатации бортовой системы предупреждения столкновений (БСПС)», утвержденной уполномоченным органом РФ в области ГА.
Для поддержания уровня подготовленности члены летного экипажа должны регулярно с периодичностью один раз в год проходить подготовку в летном подразделении по «Программе теоретической подготовки по выполнению маневров и действий при срабатывании БСПС(TCAS)», включая сдачу экзамена, и тренировку на летном тренажере, включая проверку.
В целях повышения уровня безопасности полетов, в соответствии с требованиями ICAO ВС, эксплуатируемые Авиакомпанией оборудуются бортовой системой предупреждения столкновения в воздухе TCAS.
Система TCAS является дополнительным к визуальному контролю и службе ОВД средством для предупреждения столкновения с ВС, оборудованными ответчиками с режимами «RBS» (A.C.S).
Система предупреждения столкновений в воздухе предназначена:
а)	для обнаружения и определения в ближней зоне полета данного ВС местонахождения других ВС, оборудованных ответчиками УВД, работающими в режимах «RBS», «АС», «SRA»;
б)	выдачи экипажу, в случае необходимости, соответствующих визуальных и речевых команд на выполнение необходимых вертикальных маневров в целях исключения возможности столкновения с конфликтными самолетами;
в)	передачи на наземные пункты ОВД информации о барометрической высоте полета, код рейса (SQUAWK), назначенный органом ОВД и спецсигналов;
г)	система TCAS используется при полетах на ВС, имеющих соответствующее оборудование, на внутренних и на международных линиях.
Эксплуатация системы ТCAS-II, TCAS-2000 выполняется в соответствии с требованиями РЛЭ ВС (AFM, FCOM).
Системой ТCAS-II, ТCAS-2000 управляет командир ВС, а второй пилот по согласованию с КВС наблюдает воздушную обстановку на своем приборе TVI.
Визуальное наблюдение за воздушной обстановкой осуществляют КВС и второй пилот.
8.18.1.1.	Информация о воздушной обстановке «ТА»
Не следует делать попытки маневрирования единственно на основе информации о воздушной обстановке «ТА». Сообщения «ТА» используются, для помощи визуальному опознаванию встречного ВС и предупреждению о возникновения угрозы.

8.18.1.2.	Рекомендации по принятию решения «RA»
Внимание! Указание «RA» является приоритетным перед командой диспетчера!
Все «RA» (корректирующие или превентивные) должны быть выполнены, если это не представляет угрозу для безопасности полета ВС. Если далее «RA» меняются - пилот должен вновь правильно на них отреагировать.
Примечание. Предупреждения о сваливании и сдвиге ветра и предупреждения системы сигнализации о близости земли имеют приоритет по отношению к рекомендациям TCAS.
Не должны использоваться вертикальные скорости, которые превышают, указанные в «RA». Это уменьшит возможность чрезмерного отклонения по высоте, что может привести к столкновению с другими ВС.
Пилоты выполняют «RA» даже в том случае, если она противоречит указанию органа управления воздушным движением относительно выполнения определенного маневра.
Пилот не должен выполнять маневр в направлении, противоположном, указанному в «RA». Это наиболее важно, так как система может согласовывать «RA», выданные этому ВС и другому ВС, оборудованному такой же системой.
Пилоты как можно скорее, с учетом рабочей нагрузки на летный экипаж, уведомляют соответствующий орган ОВД о любой «RA», которая требует отклонения от последнего указания или разрешения диспетчера.
Маневр по возвращению к выданному прежде диспетчерскому разрешению должен быть начат немедленно после сообщения системы: «Clear of conflict».
8.18.1.3.	Фразеология стандартного радиообмена экипажа с ОВД
                                                                                                                                                        Таблица А8.18-Т1
Ситуация	Уведомление диспетчера пилотом	Отчет диспетчера
		
…после начала отклонения летным экипажем от	*r) RA TCAS	*r) TCAS RA
любого разрешения или указания органа ОВД с	*s) ПОНЯЛ	*s) ROGER
целью соблюдения рекомендации по разрешению		
угрозы столкновения (RA) БСПС (радиообмен		
между пилотом и диспетчером)		
… после выполнения маневра в соответствии с	*t) КОНФЛИКТНАЯ СИТУАЦИЯ	*t) CLEAR OF CONFLICT
	РАЗРЕШЕНА, ВОЗВРАЩАЮСЬ НА	RETURNING TO (assigned
RA БСПС (радиообмен между пилотом и	(заданный диспетчером уровень)	clearance)
диспетчером)	*u) ПОНЯЛ (или альтернативные указания)	*u) ROGER (or alternative
		
		instructions)
		
… после выполнения маневра в соответствии с	*v) КОНФЛИКТНАЯ СИТУАЦИЯ	*v) CLEAR OF CONFLICT
RA БСПС и возобновления выполнения	РАЗРЕШЕНА (заданный диспетчером	(assigned clearance)
выданного органом ОВД разрешения или	уровень) ЗАНЯЛ	RESUMED
указания (радиообмен между пилотом и	*w) ПОНЯЛ (или альтернативные	*w) ROGER (or alternative
диспетчером)	указания)	instructions)
		
… после получения разрешения или указания	*x) ВЫПОЛНИТЬ НЕ МОГУ, RA TCAS	*x) UNABLE, TCAS RA
органа ОВД, противоречащего RA БСПС, летный	*y) ПОНЯЛ	*y) ROGER
экипаж выполняет RA и непосредственно		
информирует орган ОВД (радиообмен между		
пилотом и диспетчером)		
* диспетчер может выдать новое разрешение.
** экипаж должен уведомить диспетчера о невозможности следовать его инструкциям
8.18.1.5. Если произошло срабатывание системы TCAS в режиме RA, оно должно быть доложено диспетчеру ОВД и оформлено донесение в Авиакомпанию, используя бланк настоящего РПП, Приложение A-11.1 «Форма донесения об опасном сближении после завершения рейса.
