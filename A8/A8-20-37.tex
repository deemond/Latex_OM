8.20.	Полеты в особых условиях 
К полетам в особых условиях относятся:
а)	полеты при неблагоприятных атмосферных условиях;
б)	полеты в горной местности:
	при безопасной высоте полета 3000м и более, на малых и предельно малых высотах;
	полеты по ПВП над безориентирной местностью, если основным средством навигационной ориентировки является визуальная ориентировка;
	полеты по ПВП в полярных районах, над пустынями и джунглями;
	полеты по ПВП над водным пространством;
	полеты по ПВП в условиях сложной орнитологической обстановки.
8.20.1.	Полеты при неблагоприятных атмосферных условиях
К неблагоприятным атмосферным условиям относятся:
а)	грозовая деятельность;
б)	сильные осадки;
в)	повышенная электрическая активность атмосферы;
г)	обледенение;
д)	турбулентность;
е)	сдвиг ветра;
ж)	облака вулканического пепла;
з)	пыльные и песчаные бури.
8.20.1.1.	Грозовая деятельность. Полеты в зонах грозовой деятельности, сильных ливневых осадков
(1) При принятии решения на вылет с пересечением зоны грозовой деятельности и сильных ливневых осадков КВС обязан учитывать:
а)	характер гроз (внутримассовые, фронтальные);
б)	расположение и перемещение грозовых (ливневых) очагов, возможные маршруты их обхода.
(2) При наличии в районе аэродрома вылета мощно - кучевой и кучево - дождевой облачности экипаж обязан с помощью бортовой РЛС осмотреть зону взлета и выхода из района аэродрома, оценить возможность взлета и определить порядок обхода мощно - кучевой, кучево - дождевой облачности и зон сильных ливневых осадков.
При подходе ВС к зоне грозовой деятельности (сильных ливневых осадков) КВС обязан оценить возможность продолжения полета, принять решение на обход зоны, согласовав свои действия с органом ОВД.
Запрещается преднамеренно входить в кучево - дождевую (грозовую), мощно-кучевую облачность и сильные ливневые осадки.
(3) При визуальном обнаружении в полете мощно - кучевых и кучево - дождевых облаков, разрешается обходить их на безопасном удалении, исключающем попадание воздушного судна в кучево-дождевые (грозовые) и мощнокучевые облака. При невозможности обхода указанных облаков на заданной высоте разрешается визуальный полет под облаками или выше их.
Полеты под кучево – дождевыми (грозовыми) и мощно - кучевыми облаками при крайней необходимости могут выполняться только днем над равнинной местностью по ПВП без входа в зону ливневых осадков. При этом высота полета ВС должна быть не менее безопасной высоты (эшелона) полета, а принижение ВС от нижней границы облаков - не менее 200 м.
Полеты над кучево- дождевыми (грозовыми) и мощно- кучевыми облаками могут выполняться на высоте полета, обеспечивающей пролет воздушного судна над верхней границей облаков с превышением не менее 500м.
(4) При обнаружении в полете мощно – кучевой, кучево - дождевой облачности бортовой РЛС разрешается обходить эти облака на удалении не менее 15км от ближней границы засветки. Пересечение фронтальной облачности с отдельными грозовыми очагами может производиться в том месте, где расстояние между границами засветок на экране бортового радиолокатора не менее 50км.
(5) Для обхода зон опасных метеоявлений отклонения от заданного маршрута разрешаются по согласованию с диспетчером ОВД. При невозможности обхода зон опасных метеоявлений КВС обязан немедленно вывести ВС из опасного для полета района, возвратиться в пункт вылета или произвести посадку на ближайшем аэродроме.
(6) Полеты по ППП в зоне грозовой деятельности без бортовых РТС обнаружения грозовых очагов при отсутствии наземного радиолокационного контроля запрещаются.
(7) Вертикальные вихри (смерчи), связанные с кучево-дождевыми облаками, обнаруженные визуально, экипаж обязан обходить на удалении не менее 30км.
(8) Экипаж обязан прекратить снижение и уйти на второй круг, если наблюдаются опасные метеоявления, представляющие угрозу для выполнения посадки:
а)	в условиях сильных ливневых осадков в виде дождя с интенсивностью, ухудшающей метеорологическую видимость до значения менее 600м без использования бортового радиолокатора и системы заблаговременного предупреждения о сдвиге ветра.
б)	потерян визуальный контакт с огнями приближения (огнями ВПП) или наземными ориентирами при снижении с ВПР (DA(H)) / МВС (MDA(H)) до минимально допустимой, согласно РЛЭ ВС, высоты ухода.
8.20.1.2.	Полеты при повышенной электрической активности атмосферы
(1) Признаками сильной электризации ВС являются:
а)	шумы и треск в наушниках;
б)	беспорядочные колебания стрелок радиокомпасов;
в)	искрение на остеклении кабины экипажа и свечение концов крыльев в темное время суток.
Возникновение электризации наиболее вероятно в слое облаков в интервале температур от +5 до -10 градусов.
(2) КВС при появлении признаков сильной электризации докладывает об этом органу ОВД. При этом необходимо выключить одну УКВ радиостанцию ночью, кроме того, включить освещение кабины экипажа.
Изменение высот полета в зонах повышенной электризации необходимо выполнять с повышенной вертикальной и уменьшенной поступательной скоростью полета в соответствии с требованиями РЛЭ ВС (AFM, FCOM).
После выхода из слоя облаков выполнить горизонтальную площадку продолжительностью 5 - 10 сек. до входа в другой слой.
(3) В случае поражения ВС разрядом атмосферного электричества экипажу необходимо:
а)	проконтролировать параметры работы двигателей;
б)	проверить работу электрооборудования;
в)	осмотреть ВС в целях обнаружения повреждений;
г)	при обнаружении отказов и неисправностей действовать в соответствии с РЛЭ ВС (AFM, FCOM).
8.20.1.3.	Обледенение. Полеты в условиях обледенения
(1) Полеты в условиях обледенения на ВС, не допущенных к эксплуатации в этих условиях, запрещаются.
(2) На всех этапах полета противообледенительная система должна быть включена до входа в зону возможного обледенения.
(3) Если принятые экипажем меры по борьбе с обледенением оказываются неэффективными и безопасное продолжение полета в этих условиях не обеспечивается, КВС обязан, применив сигнал срочности, по согласованию с органом ОВД, (изменить высоту маршрут) полета для выхода в район, где возможно безопасное продолжение полета, или следовать на запасный аэродром.
(4) При полёте самолёта в условиях обледенения с выпущенной механизацией крыла на передних кромках механизации возможно образование льда, который остается даже при включенной противообледенительной системе самолёта.
При посадке на покрытую слякотью ВПП слякоть может попасть на механизацию. В условиях отрицательных температур полная уборка механизации может привести к её повреждению.
Во избежание повреждений механизации не сброшенным льдом и слякотью в процессе её уборки после посадки, при заруливании ВС на стоянку, предкрылки и закрылки рекомендуется оставлять в промежуточном положении. Это даёт возможность техническому или наземному персоналу оперативно проконтролировать состояние поверхностей ВС и, в необходимых случаях, принять меры для устранения обледенения.
8.20.1.4.	Турбулентность. Сильная болтанка. Полеты в условиях сильной турбулентности
(1) При попадании ВС в сильную болтанку КВС обязан доложить органу ОВД и принять меры для выхода из опасной зоны, в том числе изменить высоту, а при невозможности, произвести посадку на запасном аэродроме.
(2) Перед входом в зону возможной болтанки и при внезапном попадании в нее пассажиры должны быть пристегнуты к креслам привязными ремнями.
(3) Вертикальные вихри, не связанные с облаками и обнаруживаемые визуально, экипаж обязан обходить стороной. Вертикальные вихри (смерчи), связанные с кучево-дождевыми облаками, обнаруживаемые визуально, экипаж обязан обходить на удалении не менее 30 км от их видимых боковых границ. 
(4) Полеты в условиях орографической турбулентности
В горах особенно активно протекают процессы образования кучевых, кучево-дождевых облаков, осадков, гроз.
Сложная структура ветра, мощная облачность, ливневые осадки, сильные грозы, обледенение наветренной стороне гор, интенсивная турбулентность на подветренной стороне значительно усложняют условия полетов в горах.
При взлете с высокогорных аэродромов при высоких температурах следует помнить, что длина разбега самолета и взлетная дистанция увеличиваются, при отрыве возможно влияние турбулентности. При полете над горной местностью на высотах, близких к минимальным безопасным в условиях сильного ветра, барометрические высотомеры могут завышать истинные значения высот на величины, указанные в Таблице А8.20-Т1.
                                                                         Таблица А8.20-Т1
Скорость ветра км/ч (узлов)	Превышение местности м (ft)
37 (20)	17 (53)
74 (40)	62 (201)
111 (60)	139 (455)
148 (80)	247 (812)
Выполнить точный расчет при учете данной поправки не представляется возможным.
Командиру ВС предоставляется право оценить, являются ли условия местности, сила и направление ветра таковыми, что необходимо делать поправку на ветер. Поправку на скорость ветра следует применять в дополнение к стандартным поправкам на температуру и информировать об этом орган ОВД.
(5) Полеты в зоне струйного течения
Наиболее характерные черты метеорологических условий полетов в зонах струйных течений – это большие скорости ветра и значительная турбулентность воздуха в области резких перепадов скорости ветра (сдвигов ветра), наблюдающихся обычно на периферии струйного течения. На циклонической стороне струйного течения горизонтальные сдвиги ветра могут в полтора раза превышать сдвиги, наблюдающиеся на его теплой стороне. Зоны вертикальных сдвигов ветра распределены в струйном течении неравномерно, что создает очаговый характер турбулентных зон.
а)	Болтанка в зоне струйного течения большей частью бывает на его циклонической стороне. В зоне струйного течения болтанка иногда бывает при ясном небе. Она наиболее интенсивна в зоне расходимости воздушных течений. Болтанка усиливается в тех районах, где тропопауза имеет большой наклон.
б)	При полете в струйном течении на высотах, близких к потолку возможно попадание в область значительных положительных отклонений температуры от стандартной. В этих случаях возможен выход самолета на закритические углы атаки и срывные режимы, самопроизвольная потеря высоты («проваливание»).
в)	При подготовке к полету необходимо тщательно анализировать расположение струйных течений и учитывать их при выборе маршрута.
г)	При попадании в полете в зону сильной болтанки, связанной со струйным течением, необходимо включить табло «пристегнуть ремни», уменьшить скорость полета до рекомендуемых РЛЭ ВС (AFM, FCOM). В случае необходимости, по согласованию с диспетчером, изменить высоту или маршрут полета.
8.20.1.5.	Сдвиг ветра
(1) Слабый сдвиг ветра – изменение направления и (или) скорости ветра в пространстве, включая восходящие и нисходящие потоки, до – 2 м/с на 30м высоты.
Умеренный сдвиг ветра – изменение направления и (или) скорости ветра в пространстве, включая восходящие и нисходящие потоки, от 2 до 4 м/с на 30м высоты.
Сильный сдвиг ветра - изменение направления и (или) скорости ветра в пространстве, включая восходящие и нисходящие потоки, от 4 до 6 м/с на 30м высоты.
Очень сильный сдвиг ветра - изменение направления и (или) скорости ветра в пространстве, включая восходящие и нисходящие потоки, 6 м/с и более на 30м высоты.
(2) При взлете и заходе на посадку в условиях сдвига ветра необходимо:
а)	увеличить расчетные скорости в соответствии с требованиями РЛЭ ВС (AFM, FCOM);
б)	осуществлять повышенный контроль за изменением поступательной и вертикальной скоростей и немедленно парировать отклонения от расчетных параметров и заданной траектории полета;
в)	при заходе на посадку немедленно уйти на второй круг с использованием взлетного режима и следовать на запасной аэродром, если для выдерживания заданной траектории снижения требуется увеличение режима работы двигателей до номинального и (или) после пролета высоты 300м (1000ф) вертикальная скорость снижения увеличилась на 3 м/с и более от расчетной.
(3) Если произошло попадание ВС в сильный сдвиг ветра, оно должно быть доложено диспетчеру ОВД и оформлено донесение в Авиакомпанию, используя бланк настоящего РПП, Приложение A-11.1 «Формы донесения: об авиационном происшествии, инциденте, опасном сближении» после завершения рейса.
Взлет и заход на посадку в условиях сильного и очень сильного сдвига ветра запрещается.

8.20.1.6.	Облака вулканического пепла
(1) Экипажу ВС необходимо предпринять все возможные меры для того, чтобы избежать попадания ВС в облако вулканического пепла, которое чрезвычайно опасно как для двигателей, так и для фюзеляжа ВС.
Вулканический пепел чрезвычайно абразивен по своей природе, поскольку состоит из твердых и острых частиц скальных пород. При полете в облаке пепла любые поверхности конструкции самолета, находящиеся впереди, подвергаются повреждению.
Вулканический пепел, попавший в реактивный двигатель, может привести к немедленному ухудшению его характеристик и отказу двигателя.
Из всех аэрозольных продуктов, присутствующих в вулканическом облаке, самым опасным веществом для ВС является серная кислота.
Обычно вулканические облака имеют белый цвет и становятся коричневыми лишь в тех случаях, когда они имеют высокое содержание пепла. В сухой атмосфере они принимают слегка голубоватый цвет. Поэтому визуальный контакт с вулканическим облаком можно получить только в дневное время. На экране бортового радиолокатора они не дают отметки.
Информация о вулканической активности, включая информацию об извержении вулканов и облаках вулканического пепла передается экипажам ВС, находящимся в полете диспетчером органа ОВД. Данная информация передается в форме NOTAM и SIGMET сообщений при подготовке к вылету.
(2) Одной из главных рекомендаций экипажам, попавшим в вулканическое облако, является снижение режима работы двигателей до предельно возможного, чтобы сократить до минимума попадание всех материалов вулканического облака в двигатели и выйти из облака как можно быстрее.
При попадании воздушного судна в облако вулканического пепла экипаж может ожидать:
а)	появление в кабине дыма и вулканической пыли;
б)	резкий запах похожий на запах горящей электропроводки;
в)	сильные статические разряды вокруг остекления кабины экипажа;
г)	многочисленные отклонения в работе двигателей, в том числе помпаж, увеличение температуры выходящих газов, срыв пламени;
д)	возможность отказа двигателя;
е)	ненадежные показания воздушной скорости;
ж)	повреждение системы герметизации и электросистемы;
з)	ночью – появление огней «святого Эльма» и других статических разрядов, сопровождаемых ярким свечением оранжевого цвета.
Дополнительно рекомендуются следующие процедуры:
а)	отключить автомат тяги, если он был включен;
б)	включить зажигание на постоянную работу;
в)	включить максимальный отбор воздуха от двигателей, в том числе систему кондиционирования ВС и противообледенительную систему ВС и двигателей для снижения давления в двигателях.
Дополнительно необходимо выполнить соответсвующие аварийные процедуры, изложенные в РЛЭ ВС. При успешном выходе ВС из облака вулканического пепла, необходимо произвести посадку на ближайшем пригодном аэродроме.
В случае наблюдения извержения вулкана или облаков вулканического пепла, либо попадания в облака вулканического пепла, экипаж ВС немедленно сообщает об этом диспетчеру органа ОВД и после посадки заполняет соответствующий бланк в письменной форме.
Первоначальный доклад должен содержать следующую информацию:
а)	позывной воздушного судна;
б)	местоположение воздушного судна;
в)	время;
г)	эшелон полета;
д)	местоположение, пеленг, расстояние до зоны вулканической деятельности;
е)	температура;
ж)	ветер в точке доклада.
Письменный доклад должен быть направлен в ближайший метеооффис при первой возможности.
8.20.1.7.	Пыльные и песчаные бури
(1) При встрече с пыльной (песчаной) бурей на маршруте экипаж обязан обходить ее визуально или проходить над ней. В случае попадания в пыльную (песчаную) бурю перейти на полет по ППП и принять меры по выходу из нее.
(2) Изменение высоты или маршрута полета воздушного судна в целях обхода пыльной (песчаной) бури разрешается только по согласованию с диспетчером органа ОВД, за исключением случаев, когда требуются немедленные действия по отвороту от курса с последующим докладом органу ОВД.
(3) Полеты на малых, предельно малых высотах, заход на посадку и посадка в условиях пыльной (песчаной) бури при сильной болтанке запрещается.
8.20.2.	Полеты в горной местности
8.21.2.1. Полеты при безопасной высоте полета 3000м и более, на малых и предельно малых высотах на самолетах Авиакомпании не выполняются.
8.21.2.2. В случае, если после взлета невозможно выполнить набор безопасной высоты (эшелона) полета до установленного рубежа, набор высоты производиться над аэродромом по установленной схеме.
8.21.2.3. При полете по ППП снижение ВС ниже нижнего (безопасного) эшелона разрешается выполнять только после пролета установленного рубежа начала снижения, при знании экипажем точного местонахождения.
8.21.2.4. При отсутствии непрерывного радиолокационного контроля или неустойчивой работы бортового навигационного оборудования или неустойчивой двусторонней радиосвязи снижение ВС ниже безопасного эшелона с последующим заходом на посадку по установленной схеме разрешается выполнять только после выхода ВС на РТС аэродрома и определения его местонахождения.
8.20.3.	Полеты по ПВП над безориентирной местностью
(если основным средством ориентировки является визуальная ориентировка)
При полетах над безориентирной местностью экипаж ВС должен учитывать особенности физико-географических, навигационных и метеорологических условий.
При подготовке к полету в малоориентирной местности экипаж ВС дополнительно обязан:
а)	отметить на карте характерные ориентиры по маршруту (тропы, русла рек, озера и т.п.), а также удаленные ориентиры, которые могут быть использованы для визуальной ориентировки;
б)	проконсультироваться по вопросам ведения ориентировки с другими экипажами, имеющими опыт полетов по данному маршруту.

8.20.4.	Полеты по ПВП в полярных районах, над пустынями и джунглями
Полеты по ПВП в полярных районах, над пустынями и джунглями на самолетах Авиакомпания не выполняет.
8.20.5.	Полеты по ПВП над водным пространством
Полеты над водным пространством по ПВП на самолетах Авиакомпания не выполняет.
8.20.6.	Полеты по ПВП в условиях сложной орнитологической обстановки
8.20.6.1. В аэропортах, полеты в которые осуществляет Авиакомпания, организовано орнитологическое обеспечение производства полетов.
8.20.6.2. Для предупреждения столкновений ВС с птицами экипаж принимает следующие меры:
а)	Перед принятием решения на вылет КВС учитывает информацию диспетчера ОВД об орнитологической обстановке в районе аэродрома.
б)	На исполнительном старте после получения информации от диспетчера СДП или по каналу АТИС об усложнении орнитологической обстановки КВС оценивает возможность выполнения взлета. Взлет в этих условиях производится с включенными фарами. Экипажу сообщается место обнаружения птиц и направление их полета (по возможности).
в)	В полете, в случае обнаружения скопления птиц, экипаж должен обходить их стороной или над ними. Особенно внимательным экипаж должен быть во время встречи в воздухе с крупными хищными птицами, которые могут проявить агрессивность по отношению к ВС.
г)	При подходе к аэродрому посадки после получения информации от органа ОВД о сложной орнитологической обстановке или при визуальном обнаружении птиц экипажу необходимо:
	повысить осмотрительность;
	включить фары;
	повысить контроль параметров работы двигателей;
	при необходимости уйти на второй круг.
8.20.6.3. В случае, если при снижении на предпосадочной прямой экипаж получил информацию диспетчера ОВД о скоплении птиц, угрожающих безопасности полета, и обнаружении их визуально, он должен принять все меры для предотвращения столкновения с птицами, вплоть до ухода на второй круг.
При невозможности произвести посадку в сложной орнитологической обстановке командиру ВС рекомендуется произвести посадку на запасном аэродроме.
8.20.6.4. Экипажи ВС, заметившие во время полета скопления птиц, представляющих опасность для полетов, немедленно передают информацию о них соответствующему диспетчеру ОВД.
8.20.6.5. Для выявления и устранения последствий экипаж передает специалистам ИАС необходимые сведения о случаях столкновений с птицами.
Специалисты ИАС выявляют и передают в ИБП сведения о повреждениях ВС из-за столкновений с птицами.
8.20.7.	Полеты в зонах значительных температурных инверсий
Радиационные, адвективные и орографические инверсии могут привести к существенному расслоению потоков по вертикали и образованию значительных вертикальных сдвигов ветра в пределах пограничного слоя; при этом максимум скорости ветра обычно приходится на верхнюю границу слоя приземной инверсии и изменения направления ветра с высотой до 60 градусов и более, иногда поворот на 180 градусов. Эти особенности необходимо иметь ввиду при выполнении полетов в слоях инверсии, кроме того пилотирование дополнительно может осложняться низкой облачностью, осадками, туманом.
8.21.	Особые случаи в полете
Экипаж ВС обязан немедленно сообщить органу ОВД о наблюдаемых опасных метеорологических явлениях, опасных сближениях с воздушными судами и другими материальными объектами и других опасных для полета обстоятельствах. По запросу органа ОВД экипаж ВС обязан информировать его об условиях полета.
8.21.1.	Экипаж, как только станет возможным, передает сигналы бедствия «MAYDAY- MAYDAY MAYDAY» в следующих аварийных ситуациях:
а)	пожар на воздушном судне;
б)	отказ двигателя (ей), приводящий к невозможности продолжения полета на Н не ниже безопасной;
в)	захват воздушного судна;
г)	угроза взрыва на борту ВС;
д)	вынужденная посадка вне аэродрома на ВС, не предназначенном для выполнения посадок вне аэродрома;
е)	экстренное снижение;
ж)	нарушение прочности ВС;
з)	полная потеря устойчивости и/или управляемости ВС;
и)	потеря ориентировки;
к)	аварийный остаток топлива (см. 8.19.4.).
8.21.1.1.	Пожар на воздушном судне
При возникновении пожара на ВС экипаж обязан:
а)	приступить к экстренному снижению и одновременно применить все доступные средства для ликвидации пожара;
б)	включить сигнал бедствия.
В зависимости от сложившейся обстановки продолжить полет до ближайшего аэродрома либо произвести посадку вне аэродрома, действуя в соответствии с требованиями РЛЭ.
8.21.1.2.	Отказ двигателя (й), приводящий к невозможности продолжения полета на высоте не ниже безопасной
Если, в случае отказа двигателя, продолжение полета на работающих двигателях не представляется возможным и возникла реальная угроза безопасности полета, КВС обязан:
а)	выполнить действия в соответствии с требованиями РЛЭ;
б)	включить сигнал бедствия;
в)	произвести посадку на любом аэродроме или вне аэродрома, если в сложившейся обстановке такая посадка представляет меньшую угрозу безопасности, чем полет до ближайшего аэродрома.
8.21.1.3.	Захват воздушного судна
При акте незаконного вмешательства на борту экипаж ВС обязан любыми способами попытаться уведомить орган ОВД об этом, а также о любых отклонениях от текущего плана полета, вызванных этими обстоятельствами. Установить на ответчике вторичной радиолокации режим «А» и код 7500.
Дополнительно смотри глава А-10. Приложение А-10.1. «Памятка экипажу воздушного судна по действиям в чрезвычайной обстановке».
8.21.1.4.	 Угроза взрыва на борту воздушного судна
При получении информации о наличии на борту ВС взрывного (зажигательного) устройства командир ВС и экипаж должны действовать в соответствии с инструкциями, изложенными в «Памятке экипажу воздушного судна по действиям в чрезвычайной обстановке» (глава А-10, Приложение А-10.1).
8.21.1.5.	Вынужденная посадка вне аэродрома на ВС, не предназначенном для выполнения таких посадок или на аэродроме, не предназначенном для посадки данного типа ВС
В случае крайней необходимости при невозможности продолжения полета КВС имеет право принять решение о выполнении вынужденной посадке вне аэродрома или на аэродроме, не предназначенном для посадки данного типа ВС. Приняв такое решение, он обязан, по возможности, сообщить органу ОВД о предполагаемом месте и времени посадки.
О предстоящей вынужденной посадке вне аэродрома командир ВС предупреждает всех членов экипажа и информирует пассажиров.
В случае вынужденной посадки ВС командир ВС руководит действиями лиц, находящихся на борту ВС, до передачи своих полномочий представителям службы поиска и спасения.
Дополнительно смотри главу А-11 настоящего Руководства.
8.21.1.6.	Экстренное снижение
Предельной высотой, на которой обеспечивается нормальная жизнедеятельность человека, является высота 4500м, выше этой высоты наступает кислородное голодание. При разгерметизации ВС в полете для предотвращения явлений кислородного голодания, нарушения психической деятельности и функций организма необходимо немедленно предпринять меры по немедленному снижению, либо обеспечению людей на борту ВС дыханием кислородом.
В случае внезапной разгерметизации ВС на крейсерской высоте полета необходимо немедленно перевести самолет в снижение и уменьшить высоту полета до 3000 - 4000м за время не более 3,5 мин.
Помимо разгерметизации кабины, причиной немедленного снижения может служить возникновение пожара на самолете, а также отказ всех генераторов, когда для быстрейшего запуска ВСУ необходимо снизится до высоты, на которой этот запуск возможен.
Процедура выполнения экстренного снижения включает в себя ряд действий, присущих для любого типа ВС, выполняющего крейсерский полет на высотах более 4500м и состоит из следующих основных элементов:
а)	членам летного экипажа одеть кислородные маски (в последовательности, определяемой РЛЭ);
б)	установить режим малого газа двигателям;
в)	перевести самолет на снижение;
г)	для увеличения Vy снижения полностью выпустить интерцепторы, шасси (если предусмотрено РЛЭ);
д)	установить и контролировать max допустимую приборную скорость полета.
При выполнении экстренного снижения для ухода с воздушной трассы производится отворот, как правило, вправо на 30º. Через 1,5 – 2 мин полета устанавливается курс полета параллельно ВТ.
С началом экстренного снижения включить сигнал бедствия, установить код 7700, доложить диспетчеру ОВД.
Детально процедура выполнения экстренного снижения описана в РЛЭ, QRH конкретного типа ВС.


8.21.1.7.	Нарушение прочности ВС. Полная потеря устойчивости и/или управляемости ВС
При потере прочности, полной потере устойчивости, управляемости командир ВС обязан:
а)	включить сигнал бедствия и установить код 7700;
б)	действовать в соответствии с требованиями РЛЭ, FCOM, QRH.

8.21.1.8.	Потеря ориентировки
Ориентировка считается полностью потерянной, если в результате принятых мер местонахождение ВС не определено.
Ориентировка считается временно потерянной, если в результате принятых мер местонахождение ВС определено.
При потере ориентировки командир ВС обязан:
а)	включить сигнал «Бедствие» и код 7700;
б)	передать по радио сигнал «May Day, May Day, May Day,»;
в)	доложить органу ОВД об остатке топлива и условиях полета; по разрешению органа ОВД занять наивыгоднейшую высоту для обнаружения ВС наземными РТС и экономичного расхода топлива;
г)	применить наиболее эффективный в данных условиях (рекомендованный для данного района полетов) способ восстановления ориентировки, согласуя свои действия с органом ОВД;
д)	если восстановить ориентировку не удалось, заблаговременно, не допуская полной выработки топлива и наступления темноты, произвести посадку на любом аэродроме или выбранной с воздуха площадке;
е)	при потере ориентировки снижение ниже безопасной высоты (эшелона) полета запрещается;
ж)	при потере ориентировки вблизи Государственной границы РФ командир ВС должен немедленно взять курс от Государственной границы вглубь территории РФ.
В приграничной полосе выполнять маневры для восстановления ориентировки запрещается.
8.21.1.9.	Отказ радиолокационных средств ОВД и радиотехнического обеспечения полетов на аэродроме посадки
При получении сообщения от органа ОВД (управление полетами) об отказе радиолокационных средств в районе ОВД командир ВС, выполняющий полет в данном районе обязан:
а)	при полете по ППП – продолжить полет, соблюдая заданные высоту (эшелон) и скорость;
б)	при полете по ПВП – усилить осмотрительность;
в)	следить за воздушной обстановкой по радиообмену воздушных судов и органов ОВД.
При отказе средств РТО посадки на аэродроме и невозможности по метеорологическим условиям визуальной посадки КВС обязан уйти на второй круг (выполнить процедуру прерванного захода на посадку) и следовать на запасной аэродром. При невозможности ухода на запасной аэродром из-за недостатка топлива или неисправности авиационной техники командир ВС принимает решение на выполнение посадки.
8.21.2.	Экипаж, как только станет возможным, информирует орган ОВД при необходимости с применением сигнала срочности «PAN-PAN-PAN» о следующих сложных ситуациях:
а)	отказ двигателя(ей), не приводящий к невозможности продолжения полета на Н не ниже безопасной;
б)	потеря радиосвязи при полете в контролируемом воздушном пространстве;
в)	попадание ВС в зону опасных для полета метеорологических явлений;
г)	выполнение посадки ВС в условиях ниже минимума для посадки в случаях, не позволяющих продолжать полет до другого аэродрома;
д)	ухудшение устойчивости и/или управляемости ВС;
е)	внезапное ухудшение состояния здоровья лица на борту ВС, требующее медицинской помощи, которая не может быть оказана на борту ВС;
ж)	отказы систем ВС, приводящие к невозможности выполнения полета до аэродрома назначения;
з)	минимальный остаток запаса топлива (см. 8.19.3.4.).
8.21.2.1.	Отказ двигателя (й), не приводящий к невозможности продолжения полета на Н не ниже безопасной
Экипаж обязан действовать в соответствии с требованиями РЛЭ, FCOM, AFM, QRH.
8.21.2.2.	Потеря радиосвязи при полете в контролируемом воздушном пространстве
Смотри главу А-12 настоящего Руководства.
8.21.2.3.	Попадание ВС в зону опасных для полета метеорологических явлений
К неблагоприятным атмосферным условиям для полетов относятся:
а)	грозовая деятельность;
б)	сильные осадки;
в)	повышенная электрическая активность атмосферы;
г)	обледенение;
д)	турбулентность;
ж)  сдвиг ветра;
и)  облака вулканического пепла;
к)  пыльные и песчаные бури.
К ОПАСНЫМ для полета метеорологическим явлениям и условиям относятся указанные в РЛЭ метеорологические явления и условия, полеты в которых запрещаются.
При встрече с опасными метеоявлениями по маршруту полета КВС обязан без промедления доложить органу ОВД об опасных условиях полета и принять меры для их обхода. При невозможности обхода изменением маршрута или высоты полета экипаж обязан возвратиться на аэродром вылета или произвести посадку на ближайшем запасном аэродроме (при наличии осадков в виде дождя с интенсивностью, ухудшающей метеорологическую видимость до величины менее 600м на ВС с возможностью использования бортового радиолокатора и системы заблаговременного предупреждения о сдвиге ветра разрешается выполнение взлета и посадки).
8.21.2.4.	Выполнение посадки ВС в условиях ниже минимума для посадки
(В случаях, не позволяющих продолжать полет до другого аэродрома)
В случае, когда к моменту прибытия ВС погода в районе аэродрома оказалась ниже установленного минимума для выполнения посадки и нет возможности по запасу топлива и состоянию авиационной техники произвести посадку на запасном аэродроме, диспетчер органа ОВД данного аэродрома обязан принять все возможные меры для обеспечения посадки ВС. В этом случае КВС имеет право принять решение на выполнение посадки.
8.21.2.5.	Ухудшение устойчивости и/или управляемости ВС
При ухудшении устойчивости и/или управляемости ВС командир должен доложить органу ОВД и действовать согласно требованиям РЛЭ, AFM, FCOM, QRH.
8.21.2.6.	Основанием для вынужденной посадки при внезапном ухудшении состояния здоровья лица на борту ВС, когда медицинская помощь не может быть оказана в полете, могут быть следующие симптомы:
а)	осложненные роды (кровотечение с большой кровопотерей);
б)	внезапная и длительная потеря сознания, сопровождающаяся нарушением дыхания, снижением или повышением артериального давления, которые не поддаются лечению имеющимися на борту ВС медикаментозными средствами;
в)	не снижающиеся сильные и длительные боли в грудной клетке, за грудиной, в области сердца, не снимаемые нитроглицерином и валидолом;
г)	сильная головная боль с головокружением, тошнота, рвота при высоком артериальном давлении, не снижающиеся имеющимися на борту ВС медикаментозными средствами;
д)	острые боли в животе, напряженность и болезненность при нажатии на брюшную стенку;
е)	травматические повреждения, открытая рана грудной клетки, артериальное кровотечение, переломы костей таза;
ж)	острое психическое заболевание, сопровождающееся неадекватным поведением (буйством);
з)	острое инфекционное заболевание с угрожающими жизни больного симптомами (прогрессирующее обезвоживание организма), проявляющееся непрерывным поносом и рвотой, высокой (39,5 - 40º) температурой тела, потерей сознания, воспалением легких с нарушением функции дыхания, появлением крови в мокроте или легочным кровотечением).
(1) Действия экипажа ВС при предъявлении пассажиром жалоб на состояние здоровья в полете изложены в РПП, глава А -11, Раздел 11.15.
(2) В случае принятия решения на выполнение вынужденной посадки с целью сохранения жизни больного пассажира, КВС обязан:
а)	информировать диспетчера органа ОВД, под управлением которого он находится, о принятом решении;
б)	связаться с диспетчерским пунктом аэродрома вынужденной посадки и передать информацию о требуемой медицинской помощи, сообщив (по возможности) симптомы заболевания.
(3) Решение о возможности продолжения рейса с больным пассажиром после оказания ему медицинской помощи (не связанной с необходимостью госпитализации) принимает медицинский работник аэропорта вынужденной посадки.
8.21.2.7.	Отказы систем ВС, приводящие к невозможности выполнения полета до аэродрома назначения
При отказе систем (агрегатов) ВС, вызывающих необходимость изменения плана полета, в том числе к вынужденной посадке, командир ВС обязан:
а)	при полете по ППП по возможности перейти на полет по ПВП, а когда нет уверенности в безопасности такого перехода, но имеется техническая возможность продолжения полета по ППП, продолжить полет по ППП или следовать указаниям органа ОВД.
8.22. Спутный след 
Во время полета за самолетом создается аэродинамический след или спутный поток, мощность которого зависит от размеров и масса воздушного судна, скорости и высоты полета, угла атаки (перегрузки). Спутный след обычно устойчив. 
В случае попадания самолета в спутный след впереди летящего самолета, воздействие следа может оказаться настолько сильным, что органами управления самолета будет невозможно преодолеть это воздействие. 
Необходимо учитывать, что при полёте ниже тропопаузы влияние спутного следа на ВС при попадании в него возрастает. 
Спутный след образуется: 
а)	реактивной струей двигателя; 
б)	пограничным слоем, сбегающим с поверхности самолета; 
в)	концевыми вихрями крыла, связанными с образованием подъемной силы; 
г)	нисходящего потока от несущего винта. 
Для предотвращения попадания в спутный след впереди летящего ВС, а также в целях парирования последствий попадания в турбулентность в следе от следующего впереди ВС может применяться процедура полета с оперативным боковым смещением. 
В полёте, по возможности, путём визуального наблюдения за попутным воздушным судном необходимо определить пути обхода спутного следа. 
Если необходимо избежать турбулентность в следе, используется один из трех имеющихся вариантов: 
а)	полет по осевой линии; 
б)	смещение на 2 км (1 м. миля); 
в)	смещение на 4 км (2 м. мили) вправо. 
Во избежание попадания в спутный след от встречного (попутного) ВС отклонение от линии пути необходимо производить в сторону откуда дует ветер, согласовывая свои действия с диспетчером ОВД. 
Для согласования смещений экипажи воздушных судов могут выходить на связь с другими воздушными судами на частоте, предназначенной для связи «воздух – воздух» между экипажами воздушных судов. 
Для исключения случаев попадания воздушных судов в спутный след в Авиакомпании установлены минимальные временные интервалы между взлетом и посадкой: 
а)	при полетах с одной ВПП или параллельных ВПП, расстояние между осями, которых менее 1000м - не менее 45 секунд. 
Минимальные временные интервалы при взлете с одной ВПП, или параллельных ВПП, расстояние между осями, которых менее 1000м, устанавливаются: 
а)	для ВС с максимальной взлетной массой менее 7000 кг, следующих за ВС с максимальной взлетной массой более 7000 кг, – не менее 3 мин.; 
б)	для ВС с максимальной взлетной массой более 7000 кг, следующих за ВС с максимальной взлетной массой 136 000 кг и более, - не менее 2-х минут; 
в)	во всех остальных случаях – не менее 1 минуты. 
При взлете ВС с максимальной взлетной массой менее 136000кг со средней части ВПП или параллельных ВПП, расстояние между осями, которых менее 1000 м, за ВС с максимальной взлетной массой 136 000кг и более, взлетающим от ее начала, минимальный временной интервал устанавливается не менее 3 минут. 
Минимальные временные интервалы при посадке на одну ВПП или параллельные ВПП, расстояние между осями, которых менее 1000м, устанавливаются: 
а)	для ВС с максимальной взлетной массой менее 7000 кг, следующих за ВС с максимальной взлетной массой более 7000кг – не менее 3 минут; 
б)	для ВС с максимальной взлетной массой более 7000кг, следующих за ВС с максимальной взлетной массой 136000 кг и более – не менее 2 минут; 
в)	во всех остальных случаях – не менее 1 минуты.
Если произошло попадание ВС в спутный след, оно должно быть доложено диспетчеру ОВД и оформлено донесение в Авиакомпанию, используя бланк настоящего РПП, Приложение А-11.1 «Формы донесения: об авиационном происшествии, инциденте, опасном сближении» после завершения рейса. 
8.23. Космическая и солнечная радиация 
Воздушные суда Авиакомпании не выполняют полеты на высотах более 15000 м (49000 ft).
8.24. Расположение членов экипажа в полете
8.24.1. Летный экипаж
8.24.1.1. Не допускается нахождение в кабине летного экипажа лиц, не связанных с выполнением задания на полет.
8.24.1.2. Левое пилотское сидение занимает командир ВС. Исключения составляют случаи ввода в строй пилотов командирами ВС, а также случаи проверок командиров ВС на правом пилотском сидении.
8.24.1.3. Во время руления и в полете, на высотах ниже 3000м (10000 ft), а также в других критических фазах полета, каждый член летного экипажа должен находиться на штатном рабочем месте и выполнять свои обязанности в кабине пилотов.
8.24.1.4. Во время остальных этапов полета каждый член летного экипажа, от которого требуют служебные обязанности, должен находиться на рабочем месте, если его покидание не стало необходимым по служебным или физиологическим нуждам, при условии, что, как минимум, один пилот всегда находится на рабочем месте, владеет ситуацией, управляет воздушным судном, его кресло находится в рабочем положении, обеспечивающим непосредственный доступ к органам управления ВС.
8.24.1.5. Командир ВС на протяжении всего полета обязан находиться на своем рабочем месте.  Кратковременно оставлять рабочее место ему разрешается при благоприятных условиях полета. В этом случае воздушным судном управляет второй пилот, а остальные члены экипажа должны находиться на своих рабочих местах. На период замены пилотов автопилот должен быть включен.
8.24.1.6. Выходить из кабины пилотов разрешается кратковременно, но не более, чем одному члену экипажа, а в особой ситуации - по решению командира ВС.
8.24.1.7. При двухчленном экипаже - на время покидания кабины одним из пилотов, с целью предотвращения доступа в кабину посторонних лиц один специалист кабинного экипажа должен находиться в пилотской кабине, не занимая рабочего места пилота.
Членам экипажа оставлять свои рабочие места без разрешения командира воздушного судна запрещается.
8.24.2. Кабинный экипаж
8.24.2.1. Члены кабинного экипажа во время взлета и посадки, а также в случае необходимости находятся на служебных креслах.
8.24.2.2. В особых ситуациях полета при аварийной эвакуации пассажиров, члены экипажа занимают места в салоне ВС, предусмотренные аварийным расписанием в соответствии с РПП (РЛЭ, AFM, FCOM). Сигналом к выполнению аварийного расписания является остановка ВС.
8.24.3. Потеря работоспособности членами экипажа в полете
8.24.3.1. Основаниями предполагать о возможной потере работоспособности членами экипажа являются:
а)	проблемы с сердцем, жалобы на боли в груди, слабость, учащенное сердцебиение, тошнота, бледность, повышенное потоотделение, постоянное зевание, поверхностное учащенное дыхание;
б)	ситуация, когда член экипажа вразумительно не реагирует на любые голосовые вызовы, связанные с существенным отклонением от стандартных процедур или от стандартного профиля полета;
в)	ситуация, когда один из членов экипажа не отвечает на запросы другого, особенно после повторного запроса при чтении карты контрольных проверок (Check list), отвечает неверно или не отвечает совсем.
Потеря работоспособности, ухудшение здоровья, ранение, внезапная смерть члена экипажа или пассажира в полете является особым случаем. О таком случае и принятом решении экипаж обязан известить орган ОВД.
8.24.3.2. При утрате работоспособности командиром ВС предпринять следующие действия:
а)	другой пилот обязан взять на себя управление и обеспечить сохранение безопасного профиля полета. Если случай произошёл на конечном этапе захода на посадку, выполнить уход на второй круг и, при необходимости, выполнить полёт в зону ожидания;
б)	включить автопилот;
в)	запросить, при необходимости, помощь бортпроводников для эвакуации утратившего работоспособность командира ВС. Если это возможно, удалить его из кресла. Если такой возможности нет - зафиксировать его в кресле, передвинув сидение на максимально возможное расстояние от органов управления;
г)	при наличии дополнительного пилота, порядок замены определяет пилот, принявший на себя обязанности командира ВС. В горизонтальном полёте произвести замену в экипаже.
8.24.3.3. Потеря работоспособности командиром воздушного судна, если экипаж состоит из одного пилота, приведет к катастрофическим последствиям.
При появлении симптомов возможной потери работоспособности:
а)	включить сигнал бедствия;
б)	информировать орган ОВД;
в)	произвести немедленную посадку.
8.24.3.4. При потере работоспособности командиром ВС в экипаже, состоящем из двух пилотов, обязанности КВС должен выполнять второй пилот и, в зависимости от метеоусловий, принять решение о следовании до первого пункта посадки или посадке на ближайшем запасном аэродроме.
8.24.3.5. Если экипаж состоит из трёх и более пилотов (увеличенный экипаж или наличие проверяющего), то при потере работоспособности командиром ВС его обязанности выполняет пилот более высокой среди оставшихся пилотов квалификации, который занимает место КВС и выполняет его обязанности.
8.24.3.6. При потере работоспособности пилота, занимающего правое пилотское сидение - заменить его другим пилотом. Если нет возможности удалить его с рабочего места - зафиксировать в кресле, передвинув сидение на максимально возможное расстояние от органов управления.
8.24.3.7. При потере работоспособности командиром ВС(пилотом) в двухчленном экипаже старший бортпроводник или назначенный им другой подготовленный проводник должен постоянно находиться в кабине экипажа и выполнять указания КВС (пилота, принявшего на себя обязанности КВС). По его команде зачитывать Check list.
8.24.3.8. При потере работоспособности старшим бортпроводником его обязанности выполняет бортпроводник № 2.
8.24.3.9. В случае потери работоспособности члена экипажа в полете, командир ВС (член экипажа, принявший на себя обязанности командира ВС) обязан, по возможности, организовать оказание ему медицинской помощи, принять решение о продолжении или прекращении полета.
8.24.3.10. Заход на посадку планировать с минимальной рабочей нагрузкой, избегать предельных режимов при заходе на посадку и ситуаций, когда может потребоваться уход на второй круг.
Подготовить экипаж и пассажиров к возможной аварийной посадке. Выполнять обычный заход на посадку и посадку. После выполнения безопасной посадки остановить ВС на ВПП или в согласованной безопасной зоне. Выполнить установленные процедуры и выключить двигатели. Завершить установленные процедуры. Принять участие совместно со спасателями в эвакуации пострадавшего и пассажиров.
8.25. Использование ремней безопасности экипажем и пассажирами
Все воздушные суда Авиакомпании оборудованы местами и поясными привязными ремнями для каждого пассажира и привязной системой для каждого члена экипажа.
8.25.1. Члены экипажа
8.25.1.1. Во время взлета посадки, при попадании ВС в зону турбулентности, а также, если КВС сочтет это необходимым, в интересах безопасности, каждый член экипажа (пилоты обязательно) должен быть пристегнут всеми имеющимися на рабочем кресле ремнями безопасности (плечевыми и поясными). Членам летного экипажа (не пилотам) в случаях, когда ремни создают помехи для исполнения ими своих служебных обязанностей, допускается отстегнуть плечевые ремни (поясные ремни должны оставаться пристёгнутыми).
Во время других фаз полета, каждый член летного экипажа при нахождении на рабочем месте должен быть пристегнут поясным ремнем безопасности, а кресло, как минимум, одного из пилотов находится в положении, обеспечивающем непосредственный доступ к органам управления ВС.
8.25.1.2. Члены кабинного экипажа должны быть пристегнуты привязными ремнями при выполнении взлета и посадки, а также во всех случаях по требованию командира ВС.
8.25.1.3. Во время отдыха на специально отведенных для этого местах члены экипажа также должны быть пристегнуты поясными ремнями.
8.25.1.4. Информация для бортпроводников о начале обслуживания пассажиров передается командиром ВС.
8.25.1.5. При подходе к зоне слабой турбулентности, после того как летный экипаж оповестит кабинный экипаж включением табло «ЗАСТЕГНИТЕ РЕМНИ», бортпроводник-бригадир уточнения у КВС по внутрисамолетной связи информацию о предполагаемой длительности и интенсивности турбулентности.
При прогнозировании умеренной или сильной турбулентности КВС оповещает членов кабинного экипажа включением табло «ЗАСТЕГНИТЕ РЕМНИ», информирует пассажиров и бортпроводников по системе Р/А.
При внезапном попадании в зону умеренной или сильной турбулентности КВС оповещает членов кабинного экипажа включением табло «ЗАСТЕГНИТЕ РЕМНИ», дополнительно дает информацию по системе Р/А «БОРТПРОВОДНИКАМ ЗАНЯТЬ СВОИ МЕСТА»
По этому сигналу бортпроводники должны:
а)	немедленно прекратить обслуживание пассажиров;
б)	закрепить съемное бытовое и буфетно-кухонное оборудование;
в)	занять свои (или ближайшие свободные) места;
г)	застегнуть ремни безопасности;	
д)	бортпроводник-бригадир занимает свою или ближайшую станцию (ближайшее свободное место), застегивает ремни безопасности, информирует пассажиров о прохождении зоны турбулентности и необходимости застегнуть ремни.
8.25.1.6. После заруливания на стоянку и выключения двигателей, командир ВС дает команду на открытие дверей самолета отключением табло «ЗАСТЕГНУТЬ РЕМНИ» («FASTEN SEAT BELTS»).
8.25.2. Пассажиры
8.25.2.1. Пассажиры должны быть пристегнуты привязными ремнями от начала выруливания (буксировки) до набора (пересечения) эшелона FL 100(3000м) и от эшелона FL 100(3000м) и ниже при снижении с эшелона до заруливания на стоянку, а также всегда по требованию КВС.
8.25.2.2. Перед взлетом и посадкой, во время руления, а также в случае необходимости, КВС должен убедиться (по докладу бортпроводника), что все пассажиры занимают свои места и пристегнуты ремнями безопасности.
8.26. Доступ в кабину летного экипажа
Работа экипажа должна проводиться в условиях минимальной возможности доступа в кабину экипажа.
При перевозке пассажиров дверь пилотской кабины должна быть закрыта на запорное устройство с момента окончания посадки пассажиров и закрытия входных дверей перед вылетом и до открытия входных дверей ВС для высадки пассажиров после полета. Порядок доступа в кабину летного экипажа определяется КВС по установленным сигналам с соблюдением мер предосторожности.	
Дополнительное запорное устройство применяется по решению КВС в чрезвычайных ситуациях. (см. А-10).
8.27. Использование свободных мест экипажа
Использование свободных мест экипажа
Помимо выполняющего свои рабочие функции летного экипажа и лиц, включаемых в задание на полет с правом нахождения в кабине летного экипажа в кабину экипажа могут допускаться по разрешению КВС бортпроводники при выполнении взлета и посадки, например в случае, когда все сиденья в салоне заняты и при условии, что минимальное количество бортпроводников, согласно требованиям РЛЭ ВС (AFM, FCOM), находятся в салоне.
Для ознакомления бортпроводника с практической работой лётного экипажа во время полёта по разрешению КВС он может находиться в кабине пилотов. При этом КВС обязан провести дополнительный брифинг в объеме достаточном для обеспечения безопасности на всех этапах полета.
В интересах безопасности КВС обладает постоянно действующими полномочиями запретить нахождение какого бы то ни было лица в кабине экипажа.
8.28. Подготовка пассажирской кабины к полету определяет:
а)	контроль наличия, комплектности и исправности аварийно-спасательного оборудования;
б)	осмотр багажных и пассажирских помещений в целях определения их санитарного состояния и обнаружения посторонних предметов и забытых вещей;
в)	проверка наличия и состояния пассажирского и бытового оборудования: салонов, штор, занавесок, обивки кресел, ковров, состояние буфетов-кухонь и туалетных комнат;
г)	проверка исправности привязных ремней (выборочно), работы регуляторов;
д)	откидывание спинок кресел, индивидуальных столиков, вентиляции, СГУ и СПУ, сигнализации вызова бортпроводника и чистоты пепельниц;
е)	проверка исправности всех видов освещения (только при подключенном наземном питании и с разрешения бортинженера, бортмеханика);
ж)	проверка наличия воды и хим. жидкости в туалетах, наличие и исправность кислородных приборов и пожарного оборудования;
з)	проверка наличия на борту ВС аварийно-спасательных средств, переносных огнетушителей, кислородных приборов и масок;
и)	приемка бытового съемный инвентарь и средства обслуживания пассажиров в полете, бытового мягкого имущества;
к)	приемка бортпитания, бортпосуды, контроль его размещения в буфете-кухне;
л)	экипировка пассажирских салонов и туалетных комнат;
м)	проверку наличия документов установленного образца у лиц, осуществляющих подготовку ВС к вылету.
Перед началом посадки пассажиров.
Необходимо проверить правильность установки трапа; при этом ограничительные ремни разрешается снимать с проемов дверей только на период посадки пассажиров.
Процедуры посадки и высадки всех категорий пассажиров осуществляется согласно РЛЭ, РЦЗ 
При встрече и размещении пассажиров в салонах ВС:
а)	встретить пассажиров у трапа или у входной двери;
б)	оказать помощь пассажирам при переходе с трапа в самолет;
в)	оказать помощь в размещении верхней одежды и ручной клади;
г)	в рядах рядом с аварийными выходами или на креслах с ограничениями не размещаются следующие категории пассажиров:
	несовершенно летние подростки без сопровождения;
	дети и младенцы;
	беременные женщины;
	инвалиды;
	слепые или глухие;
	престарелые или слабые пассажиры;
	заключенные, депортированные;
	тучные пассажиры.
Размещение инвалидов
Инвалиды должны быть размещены как можно ближе к аварийным выходам, но без нарушения вышеприведенных инструкций.
Запрещается размещать инвалидов там, где они могут препятствовать выполнению своих обязанностей бортпроводникам.
Костыли, трости и т.д. должны быть размещены так, чтобы не препятствовать передвижению по салону, и, в то же время, быть "под рукой" инвалидов во время полета. Во время аварийной ситуации инвалиды должны оставаться на своих местах на борту. Помощь им будет оказана назначенными пассажирами или членами экипажа.
Слепые и глухие пассажиры должны быть проинструктированы перед полетом отдельно.
Необходимо убедиться, что они поняли, как действовать в аварийной ситуации.
VIР пассажиры, семейные группы.
Дополнительные услуги повышенной комфортности не предоставляются.
По возможности, семьи и группы должны размещаться вместе, а матери и дети в одном ряду.
Проверка салона перед взлетом и посадкой.
Пассажиры должны находиться на своих местах и быть пристегнутыми ремнями безопасности.
Индивидуальные столики должны быть убраны.
Спинки кресел должны находиться в вертикальном положении.
Аварийные выходы должны быть не заблокированы багажом.
Туалеты - свободны от пассажиров и багажа
Ручная крупногабаритная кладь размещена на закрытых полках для ручной клади или под креслом.
Никто не должен курить.
Стеллажи для ручной клади и верхней одежды закрыты.
Кухни должны быть убраны, и находиться в безопасном состоянии.
Доложить о выполненной проверке бригадиру бортпроводников.
Размещение ручной клади в салоне.
В качестве ручной клади принимаются вещи, вес и габариты которых позволяют безопасно разместить их в салоне ВС. В качестве ручной клади может перевозиться: для пассажиров эконом – класса – не более одного места багажа, массой до 5 кг, размерами не более 55х40х20 см.Там, где возможно, ручная кладь должна размещаться на багажных полках, снабженных дверками. В случае ограничений по весу – это указывается. (тяжеловесный багаж – вес одного места выше 32 кг) Ручная кладь большого размера, которая не может быть уложена на багажных полках, должна размещаться в багажном отделении. Негабаритный багаж – 1 место превышает 203 см. в сумме трёх измерений. Ручная кладь, содержащая бутылки с жидкостью не может размещаться на багажных полках, т.к. жидкость может протечь.
Ручная кладь может быть размещена под сиденьями пассажирских кресел. Это возможно, если ручная кладь не закрывает проход между рядами кресел.
Ручная кладь не может быть размещена у перегородок салона, а также между креслами рядов, расположенных у аварийных выходов.
Запрещается размещение ручной клади в туалетных комнатах;
Ручная кладь не должна преграждать доступ к аварийно - спасательному оборудованию и аварийным люкам. Бортпроводники отвечают за то, чтобы аварийно-спасательное оборудование, расположенное на багажных полках, не было перекрыто ручной кладью пассажиров или членов экипажа, а также кухонным оборудованием.
Проверка салонов самолета в течение рейса
Требования по безопасности по время полета:
а)	постоянный контроль в салоне ВС в целях обеспечения мер авиационной безопасности в полете, в соответствии с «Памяткой экипажу по действиям в чрезвычайной обстановке» № 66/И ДСП;
б)	контроль за соблюдение правил поведения пассажиров на борту ВС, предупреждение курения в полете;
в)	процедуры обеспечения эвакуации пассажиров в аварийной обстановке:
	информирование пассажиров о предстоящей аварийной посадке;
	информирование пассажиров о наличии на борту ВС аварийно-спасательного оборудования и правилах его применения;
	подбор и подготовка пассажиров для оказания помощи экипажу во время эвакуации пассажиров;
	принятие мер к недопущению паники среди пассажиров;
	проверка правильности принятия рекомендуемых поз и застегивания ремней безопасности;
	организация эвакуации пассажиров с приведением в действие АСО самолета согласно РЛЭ самолета.
Проверка туалетов:
а)	производится через каждые 30 минут;
б)	запрещается курить;
в)	проверка мусорных корзин для безопасности и на предмет тлеющих сигарет;
г)	проверка чистоты и экипировки, при необходимости документировать.
Проверка салонов:
а)	проверка салона при ночном полете производится каждые 15 минут;
б)	при проверке используется персональный фонарик, если включено только дежурное освещение;
в)	контролируется выполнение запрета на курение на борту;
г)	своевременное оказание требуемых услуг пассажирам.
При внезапном попадании в сильный турбулентный поток или при предупреждении из кабины пилота о скором попадании в турбулентный поток (это может быть сделано по бортовой системе связи или включением табло «ЗАСТЕГНУТЬ РЕМНИ») необходимо:
а)	занять ближайшие места и пристегнуть ремни безопасности;
б)	проинструктировать пассажиров и пристегнуть их ремни;
в)	оставаться на своем месте до получения инструкций от командира или до выключения табло «ЗАСТЕГНУТЬ РЕМНИ»; «НЕ КУРИТЬ».
Подготовка и проверка на безопасность салонов в транзитном аэропорту.
Проверка карманов кресел.
Проверка полок для ручной клади.
Проверка наличия подголовников на креслах.
Проверка чистоты туалетов.
Проверка чистоты салонов, наличия привязных ремней на креслах пассажиров.
При обнаружении на борту забытого пассажирами багажа, сведения передаются бортпроводнику, который в свою очередь информирует об этом наземный персонал аэропорта.
Освещение салона в темное время суток.
Во время взлета – посадки включается дежурное освещение, для того чтобы в случае аварийной посадки пассажирам легче было адаптироваться в темноте. Согласно новому ВК – освещение 10%.
Требования безопасности по окончании полета.
Бортпроводники должны выходить из самолета на трап только после его окончательной установки по команде лиц, ответственных за подгон и отгон трапа.

После посадки самолета бортпроводники обязаны:
а)	занять свои места у дверей, проконтролировать установку трапа;
б)	оказать помощь пассажирам при сходе на трап;
в)	после выхода последнего пассажира установить на двери ограничительные ремни, сдать буфетно-кухонное оборудование и неиспользованные продукты питания;
г)	не допускается спускаться по трапу прыгая через несколько ступенек; во время спуска по трапу следует держаться за поручень;
д)	от самолета по перрону следовать безопасным путем с учетом рулящих ВС, работающих двигателей, вращающихся лопастей винтов, движущегося спецавтотранспорта, перронной механизации.
Обслуживание пассажиров после посадки
После посадки самолета и информации, бортпроводники помогают пассажирам одеться, оказывают помощь в сборах, прощаются и благодарят их за полет.
После выхода пассажиров, бортпроводники осматривают салоны на предмет забытых вещей.
Найденные вещи сдаются представителю авиакомпании в аэропорту.
Процедуры при заправке топлива при нахождении пассажиров на борту ВС.
Определяются приказом МГА от 21.06.85г. № 133 и включают в себя:
а)	Нахождение на месте дозаправки аэродромных средств пожаротушения и пожарно - спасательного расчета.
б)	Наличие у основных входов трапов (не мене 2х) двери открыты и к ним должен обеспечиваться свободный доступ.
в)	Включение табло «не курить».
г)	Информирование пассажиров о недопустимости создания источников воспламенения (курение, пользование зажигалкой, включение – выключение освещения и иных способов).
д)	Нахождение у каждой двери члена экипажа.
8.29. Требования по безопасности в кабинах ВС
8.29.1. Размещение кабинного оборудования и ручной клади
8.29.1.1. На всех этапах полета кабинное оборудование, ручная кладь размещаются в соответствии с установленными правилами безопасности таким образом, чтобы не представлять собой угрозу безопасности полета или здоровью любого лица, находящегося на борту ВС.
Ручную кладь следует размещать только под креслом, находящимся перед пассажиром, или на багажных полках так, чтобы полки можно было закрыть и исключить непроизвольное перемещение в условиях полета; все предметы, непомещающиеся под креслом или на багажной полке, необходимо сдать в багажный отсек.
Груз, почта, багаж размещаются в багажных отделениях ВС и надежно закрепляются швартовочными сетками. В аэропортах, где отсутствуют договорные обязательства по установлению ответственности представителя за прием/сдачу коммерческой загрузки, бортпроводник принимает коммерческую загрузку (груз, почту, багаж).
Мягкий инвентарь и средства обслуживания должны быть размещены в специально предусмотренных местах и надёжно закреплены перед взлётом и посадкой.
При проведении контроля салона на безопасность перед вылетом бортпроводники должны убедиться в том, что кухонное оборудование закреплено, полки и гардеробы закрыты, ручная кладь размещена в соответствии с требованиями обеспечения безопасности полета.
8.29.1.2. При длительных задержках вылета, если пассажиры находятся в салоне ВС, при низких температурах наружного воздуха разрешается предоставление пледов пассажирам бизнес-класса. Во время взлёта пледы должны быть убраны в безопасное место.
Предоставление сервисных услуг пассажирам, в том числе и выдачу пледов, разрешается осуществлять после или через 10 минут после взлёта.
Сбор предоставленных пассажирам пледов перед посадкой производится после включения светового табло «Застегнуть ремни».
8.29.1.3. В кабине летного экипажа не допускается нахождение предметов, ограничивающих управление ВС, нормальную эксплуатацию систем и оборудования ВС и деятельность членов экипажа.
Личные вещи экипажа, размещаемые в кабине экипажа, должны быть закреплены штатными устройствами.
При отсутствии возможности закрепления, личные вещи располагаются в пассажирском салоне в соответствии с правилами размещения ручной клади пассажиров.
Служебная документация, необходимая экипажу в полёте, размещается в кабине экипажа в местах, где возможна её фиксация от перемещения на взлёте и посадке и где она была бы доступна к использованию с рабочего места члена экипажа.
Командир ВС перед вылетом должен лично проверить и убедиться, что находящиеся в кабине экипажа предметы правильно размещены и надёжно закреплены.
8.29.2.	Инструктаж пассажиров
На всех этапах полета Авиакомпания (член кабинного экипажа, при его отсутствии в экипаже – КВС):
а)	в устной форме информирует пассажиров в отношении правил безопасности;
б)	обеспечивает каждого пассажира картами по безопасности, которые в иллюстрированной форме содержат инструкции по применению аварийно-спасательного оборудования и аварийного покидания ВС.
Информационная работа командира ВС должна быть согласована со старшим бортпроводником в целях исключения дублирования информации.
8.29.2.1. Перед взлетом
Паcсажиры информируются:
а)	о расстоянии, времени в пути, скорости и высоте полета;
б)	при выполнении совместных рейсов по «код – шерингу» дается дополнительная информация.
Пассажиры предупреждаются:
а)	о запрете курения на протяжении всего полета;
б)	приведении кресла в вертикальное положение;
в)	правилах размещении ручной клади и запрете размещения жидкостей на верхних полках;
г)	ограничениях по использованию персонального радиоэлектронного оборудования.
Старший бортпроводник (КВС) знакомит пассажиров:
а)	с порядком использования ремней безопасности и, если есть, плечевых ремней, включая то, как правильно их застегнуть и расстегнуть;
б)	расположением и правилами применения кислородного оборудования;
в)	расположением аварийных выходов;
г)	при необходимости - расположением и порядком использования спасательных жилетов;
д)	расположением и содержанием инструкций по безопасности.
8.29.2.2. После взлета
Пассажиры информируются о:
а)	высоте, скорости, температуре за бортом и в салонах самолета.
Пассажирам напоминается по мере необходимости:
а)	правила в отношении курения;
б)	правила использования плечевых и/или привязных ремней безопасности;
в)	правила заполнения миграционных карт прибывающим иностранным гражданами.
8.29.2.3. Перед посадкой
Пассажирам напоминается по мере необходимости:
а)	правила в отношении курения;
б)	правила использование привязных и плечевых ремней безопасности;
в)	о приведении спинок кресел в вертикальное положение и о необходимости убрать откидные столики и зафиксировать их на спинке впереди стоящего кресла;
г)	расположение и использование средств аварийного покидания ВС (канаты, надувные трапы и т.д.);
д)	расположение и закрепление ручной клади (переразмещения);
е)	ограничения по использованию персонального радиоэлектронного оборудования.
8.29.2.4. После посадки
Пассажирам напоминается:
а)	правила в отношении курения;
б)	использование привязных/плечевых ремней безопасности;
в)	порядок выхода из самолета.
8.29.3.	Аварийный брифинг
Надлежащим образом проведенный аварийный инструктаж пассажиров позволяет избежать шока, паники и сохранить жизни людей. При возникновении в полете аварийной ситуации командир ВС лично информирует пассажиров о создавшейся обстановке. Он должен сделать это спокойно и профессионально, тем самым, убеждая пассажиров в способности экипажа справиться с аварийной ситуацией. Только в том случае, если создавшаяся обстановка не позволяет командиру ВС лично провести брифинг с пассажирами, он может поручить эту работу другому члену экипажа.
Содержание последующего подробного инструктажа пассажиров, осуществляемого кабинным экипажем, определяется создавшимися обстоятельствами и проводится в соответствии со специальной инструкцией.
Старший бортпроводник несет ответственность за его проведение (дополнительно смотри главу А -11).
8.29.4.	Курение на борту
Для членов экипажа ВС и пассажиров запрещено курение и потребление табака, а также использование электронных сигарет в любых помещениях на борту воздушного судна Авиакомпании во время нахождения его на земле или в полете.
Пассажиры должны быть проинформированы о запрете курения на борту воздушных судов Авиакомпании.
Табло «НЕ КУРИТЬ» (“NO SMOKING”) должно быть включено с момента посадки пассажиров до покидания ВС последним пассажиром после завершения рейса.



8.29.5.	Готовность салона к взлету и посадке 
До взлета и посадки старший бортпроводник обязан убедиться в готовности пассажирского салона к взлету (посадке) и доложить об этом командиру ВС.
Готовность пассажирского салона к взлету (посадке) означает:
а)	все пассажиры находятся на своих местах с пристегнутыми ремнями безопасности (дети на руках у родителей);
б)	в салоне и на кухне все предметы закреплены;
в)	пути эвакуации свободны;
г)	шторки на иллюминаторах открыты;
д)	освещение салона притушено.
8.29.6.	Меры, принимаемые к недисциплинированным пассажирам в полете
8.29.6.1. Командир ВС должен действовать в соответствии с полномочиями, предоставленными ему Токийской конвенцией 1963 г. и Воздушным Кодексом РФ.
Авиакомпания обеспечит командиру ВС полную поддержку во всех случаях использования этих полномочий.
8.29.6.2. Член экипажа, обнаруживший нарушение правил поведения, делает нарушителю устное замечание.
В случае подчинения нарушителя требованиям члена экипажа дальнейших действий не требуется.
Если пассажир продолжает нарушать правила поведения, бортпроводник информирует КВС об инциденте.
8.29.6.3. Командир обязан потребовать от нарушителя исполнения установленных правил поведения на борту ВС.
Командир ВС имеет право применять все необходимые меры, в том числе меры принуждения, в отношении лиц, которые своими действиями создают непосредственную угрозу безопасности полета ВС и отказываются подчиняться распоряжениям командира ВС.
При продолжении инцидента командир ВС по действующему каналу радиосвязи должен передать информацию представителю Авиакомпании в пункте посадки.
8.29.6.4. Представитель Авиакомпании в соответствии с действующими в пункте посадки правилами информирует соответствующие службы для проведения мероприятий по привлечению нарушителя к ответственности и лично участвует в их проведении.
Представитель Авиакомпании сообщает об инциденте и его последствиях в службу авиационной безопасности аэропорта посадки.
8.29.6.5. При принятии решения о выполнении вынужденной посадки, связанной с инцидентом на борту, в аэропорту, не имеющем представителя Авиакомпании, КВС связывается с аэропортом посадки, передает информацию об инциденте и действует в соответствии с правилами, установленными в данном аэропорту.
8.29.6.6. Предусмотрено документальное протоколирование инцидента, происшедшего на борту ВС, для принятия последующих мер административного воздействия к нарушителю:
а)	в случае отказа пассажиру в перевозке соответствующие работники Авиакомпании и правоохранительных органов заполняют «Рапорт об отказе в перевозке по причине нарушения «Правил поведения» и предоставляют данную информацию в САБ для ведения учета данного рода нарушений;
б)	в случае, если после получения от членов экипажа устного замечания с разъяснением о возможности последующих санкций пассажир продолжает вести себя на борту ВС деструктивно, членами экипажа заполняется «Рапорт о нарушении «Правил поведения…» на борту ВС», для передачи нарушителя «Правил поведения» в правоохранительные органы. Необходимые для этого документы оформляются сотрудниками Авиакомпании по установленной законодательством соответствующего государства форме, при этом максимально подробно описывается содержание факта нарушения, по возможности, фиксируются показания свидетелей, а также собираются вещественные доказательства;
в)	учет и анализ данных по нарушению пассажирами «Правил поведения» ведется руководителем подразделения по АБ в целях соответствующего обучения персонала Авиакомпании и придания данных фактов максимальной огласке в средствах массовой информации.
8.29.6.7. Разрешение на применение мер принуждения к выполнению требований КВС предусмотрено ВК РФ:
а)	в соответствии с пунктом 1.2. статьи 58 Воздушного кодекса РФ, КВС имеет право применять все необходимые меры, в том числе меры принуждения, в отношении лиц, которые своими действиями создают непосредственную угрозу безопасности полета ВС и отказываются подчиняться распоряжениям КВС. Применение мер принуждения к выполнению распоряжения командира ВС допускается только после предъявления нарушающему «Правила поведения…» пассажиру письменного распоряжения о необходимости выполнения требований командира ВС (См. главу А-10);
б)	если после вручения ему данного распоряжения пассажир не прекращает своего деструктивного поведения, КВС дает указание членам экипажа на применение мер принуждения (применением средств сдерживания) к выполнению распоряжений командира ВС. При этом применение каких-либо специальных средств не допускается.
8.29.7.	Расторжение договора перевозки
При совершении ближайшей запланированной или вынужденной посадки: в случае нарушения пассажиром правил поведения на борту ВС, создающего угрозу безопасности полета, либо угрозу жизни или здоровью других лиц, а также в случае невыполнения пассажиром распоряжений командира ВС, предъявленных в соответствии со статьей 58 ВК РФ, Авиакомпания в лице своего уполномоченного представителя имеет право в одностороннем порядке расторгнуть заключенный с данным пассажиром договор воздушной перевозки без какой-либо компенсации его стоимости путем отказа в дальнейшем пользования услугами компании (согласно статье 107 Воздушного кодекса РФ) и передать такого нарушителя в правоохранительные органы для применения к нему соответствующих санкций.
8.29.8.	Инструктаж лиц, сопровождающих грузы 
В случае необходимости нахождения на борту грузового воздушного судна сопровождающих грузы лиц, они должны пройти инструктаж перед выполнением рейса и ознакомлены с размещением и использованием аварийного оборудования, в том числе:
а)	ремней безопасности;
б)	аварийных выходов;
в)	спасательных жилетов (индивидуальных плавательных средств), если предусмотрено;
г)	кислородных масок;
д)	спасательного оборудования коллективного пользования.
В кабине летного экипажа не допускается нахождение предметов, ограничивающих управление воздушным судном, нормальную эксплуатацию систем и оборудования воздушного судна и деятельность членов экипажа воздушного судна. 
8.30.	Перевозка багажа и груза членов экипажа ВС
Осуществляется в соответствии с приказом директора АО «ЮТэйр»» «О провозке ручной клади, багажа и груза членами экипажей на воздушных судах аАО «ЮТэйр»” при выполнении полетных заданий». Норма бесплатного провоза ручной клади для каждого члена летного (кабинного) экипажа и лиц, вписанных в задание на полет – не более 20 кг.
8.31.	Всепогодные полеты
Термин Всепогодные полеты означает выполнение руления, взлета, подхода и посадки в условиях, когда контакт с визуальными ориентирами затруднен в связи с метеорологическими условиями.
Для того, чтобы выполнять полеты в условиях ограниченной видимости, эксплуатант должен следовать введенным на аэродроме процедурам в условиях ограниченной видимости (LVР).
Взлет в условиях ограниченной видимости (LVTO) – взлет при значениях RVR менее 400 м. (См. раздел «Взлет»).
Схемы заходов на посадку по приборам классифицируются следующим образом:
а)	NPA – неточные заходы на посадку по приборам для выполнения двухмерных заходов (2D) по типу А;
б)	APV – заходы с вертикальным наведением для выполнения трехмерных заходов (3D) по типу А;
в)	РА – точные заходы на посадку для выполнения трехмерных заходов (3D) по типу А или В.
8.32. Правила выполнения полетов увеличенной дальности воздушными судами с двумя газотурбинными двигателями (ETOPS, NON ETOPS)
Авиакомпания не сертифицирована и не выполняет полёты по правилам ETOPS.
8.33. Эксплуатация воздушных судов с не устраненными отказами и неисправностями или с отклонениями от стандартной конфигурации ВС
8.33.1. Общие положения
Экипаж ВС несет ответственность за своевременное и правильное занесение в бортовой журнал обнаруженных отказов и неисправностей.
К полету допускается только исправное ВС, имеющее достаточный для выполнения полета остаток ресурса, отвечающее техническим условиям, прошедшее установленную эксплуатационными документами проверку и подготовку.
Решение о выполнении полета с отказавшим оборудованием или при отклонениях от стандартной конфигурации принимается, если это не повлияет на безопасность полёта, а устранение неисправности в данный момент времени невозможно технически или экономически нецелесообразно.
При вылете с допустимыми неисправностями, согласно перечню, экипаж обязан ознакомиться с записями в бортжурнале, изучить особенности выполнения полета при существующих отказах, неисправностях и убедиться, что все необходимые работы выполнены и оформлены документально. 
В зависимости от типа ВС, в эксплуатации применяются перечни допустимых отказов и неисправностей, перечни минимального оборудования или Minimum Equipment List (ПДО/ПМО/MEL). 
Перечень отклонений от стандартной конфигурации (Configuration Deviation List – CDL) хранятся на борту ВС в составе РЛЭ ВС (AFM, FCOM) в Части В РПП или в виде отдельных сборников.
Ответственность за ведение контрольных экземпляров перечисленных документов возлагается на Летную службу авиакомпании. 



8.33.2. Перечень минимально исправного оборудования «МINIMUM EQUIPMENT LIST»
8.33.2.1. Перечень минимально исправного оборудования (MEL) применяется на ВС иностранного производства. На воздушных судах, разработанных или эксплуатировавшихся в СССР (Ан-74, Ан-24/26, Ан-2), применяется перечень допустимых отказов и неисправностей, который является составной частью РЛЭ, имеет ограниченную сферу и дает возможность его использования при вылете самолета до аэропорта базирования либо аэропорта, где имеется сертифицированный персонал для выполнения ремонта. 
Перечень минимально исправного оборудования (MEL) Авиакомпании составляется c учетом установленного на конкретном ВС оборудования и возможно более жесткими требованиями, чем основной перечень минимального оборудования (MMEL), разработанный изготовителем воздушного судна.
При этом принимаются во внимание, как требования федеральных властей, так и конкретные особенности эксплуатации ВС в Авиакомпании.
Вылет из базового аэропорта с не устраненными отказами и неисправностями на воздушном судне запрещается.
Запрещается применение перечня допустимых отказов и неисправностей для ВС, выпускаемых в полет после Ф-Б или периодического ТО.
Экипаж воздушного судна несет ответственность за своевременное и правильное занесение в бортовой журнал обнаруженных отказов и неисправностей. 
При подготовке самолета к вылету с отложенной неисправностью инженерно-технический персонал обязан:
а)	определить характер и причину выявленной и записанной в бортжурнале неисправности или отказа;
б)	убедиться в том, что данная неисправность или отказ не окажут влияния на работу других систем, агрегатов или оборудования;
в)	отключить (надежно изолировать) отказавший блок, агрегат, узел;
г)	результаты проделанной работы оформить в бортовом журнале и карте-наряде.
Окончательное решение о продолжении рейса принимает КВС с учетом конкретных условий предстоящего полета.
В случае принятия решения на вылет с использованием перечня допустимых отказов делается запись в бортжурнале (разделы VI и III) и карте – наряде за подписями КВС и уполномоченного представителя ИАС.
8.33.2.2.	Перечень минимального исправного оборудования для ВС иностранного производства
Перечень минимально исправного оборудования «МINIMUM EQUIPMENT LIST» (MEL) для ВС иностранного производства можно применять при вылете из транзитного, конечного аэропортов и из базового аэропорта.
На ВС иностранного производства MEL издается отдельным сборником.
(1) Структура MEL
В разделе 1 «Введение» описаны действия экипажа при выполнении полета с применением MEL.
Перечень составлен по функциональным системам самолета. В перечне указываются:
а)	системы и оборудование, агрегаты, их коды;
б)	допустимые периоды эксплуатации самолета с не работоспособным оборудованием;
в)	перечень необходимых дополнительных процедур (О), (М).
Для ВС иностранного производства установлен календарный срок (REPAIR INTERVAL), в течение которого неисправность должна быть устранена, в часах и днях: 
а)	Категория A – Определенный период времени для устранения неисправности не установлен, однако неисправности данной категории должны быть устранены в течение определенного периода времени, указанного в колонке «Примечание» (минимальные требования и необходимые процедуры).
б)	Категория В (72 часа – три дня).
в)	Категория С (240 часов – десять дней).
г)	Категория D (120 дней). 
Отсчет времени ведется, начиная с 00.00 календарных суток, следующих за сутками, когда была обнаружена неисправность:
а)	количество агрегатов данного типа и одинакового назначения, установленных на самолете (NUMBER INSTALLED);
б)	минимальное количество исправных агрегатов данного типа и одинакового назначения, с которыми разрешается вылет самолета (NUMBER REQUIRED FOR DISPATCH);
в)	условия и ограничения, которые должны быть соблюдены при вылете самолета с данным видом неисправного оборудования (REMARKS OR EXEPTIONS);
г)	ответственность за обеспечение условий и ограничений возлагаемая на экипаж самолета - (О);
д)	ответственность за обеспечение условий и ограничений возлагаемая на наземный инженерно-технический персонал - (М);
е)	ответственность наземного инженерно-технического персонала и экипажа самолета – (МО).
А также даются и другие ограничения, замечания, исключения и рекомендации, касающиеся условий эксплуатации систем ВС.
(2) При принятии решения на вылет с допустимыми неисправностями оборудования ВС КВС по бортовому журналу знакомится с записью неисправности, пунктом МЕL, открытым ИАС и дополнительными процедурами (О) или (М). На воздушном судне по Hold Item List (HIL) убедиться, что количество исправного оборудования (агрегатов) соответствует MEL, в кабине экипажа установлена (при необходимости) табличка с указанием отказавшей системы и агрегата, а также срока устранения неисправности. В бортовом журнале и, где это предусмотрено, карте-наряде делается запись с указанием предпринятых в соответствии с MEL мер и действий (О), (М).
            8.33.2.3.	Перечень допустимых отказов и неисправностей для ВС отечественного производства
Настоящий перечень определяет допустимые отказы и неисправности самолета, с которыми разрешается продолжать полет до ближайшего аэродрома где присутствует сертифицированный инженерно-технический персонал или следовать до аэродрома базирования. 
Для воздушных судов, разработанных или ранее эксплуатировавшихся в СССР, во исполнение положений п. 5.71.1. Приказа МТ РФ от 31.07.2009 №128 «Подготовка и выполнение полетов в гражданской авиации Российской Федерации» основанием для разработки ПДО/ПМО являются требования РЛЭ. 
8.33.2.4.	Применение MEL
Применение ПМО (MEL)
Решение о применении ПМО (MEL) принимает КВС в случае выявления неисправности оборудования:
а)	экипажем ВС при выполнении предыдущего полета;
б)	экипажем ВС при выполнении предполетного контроля систем и оборудования;
в)	наземным инженерно-техническим персоналом при проведении технического обслуживания перед полетом или после полета.
При принятии решения на вылет и в полете с допустимыми неисправностями оборудования воздушного судна КВС должен руководствоваться рекомендациями ПМО (MEL). Перед полетом убедиться, что количество исправного оборудования (агрегатов) соответствует ПМО (MEL), в кабине экипажа установлена (при необходимости) табличка с указанием отказавшей системы и агрегата, а также срока устранения неисправности. В бортовом журнале и, где это предусмотрено, карте-наряде делается запись с указанием предпринятых в соответствии с ПМО (MEL) мер и действий, с указанием аэропорта, в котором неисправность должна быть устранена. 
Записи из бортового журнала переносятся в Hold Item List (HIL). 
В случае отсутствия в аэропорту сертифицированного инженерно-технического персонала (ИТП), оформление бортового журнала и HIL возлагается на командира ВС. Экипаж может обратиться за консультацией к сертифицированному ИТП базового аэропорта по любым доступным каналам связи. Для решения оперативных вопросов экипаж связывается с диспетчером ОКВР.
В случае отсутствия возможности выполнения предусмотренных ПМО (MEL) процедур экипажем, для выполнения соответствующих процедур в транзитный аэропорт доставляется сертифицированный ИТП.
По прилету в аэропорт, где есть сертифицированный ИТП, повторно выполняются все рекомендации, предусмотренные ПМО/MEL. 
Порядок подготовки ВС к вылету (необходимый объем информации, представляемой экипажу, внесение соответствующих записей в бортжурнал, карту-наряд и установление предупреждающих табличек в кабине) такой же, как и в случае подготовки к вылету с применением ПМО/MEL.
Чтобы принять решение, командиру следует учитывать:
а)	тип полета или полетное время;
б)	нагрузку на экипаж;
в)	ограничения;
г)	аэропорты вылета и назначения;
д)	максимальную относительную высоту; 
е)	метеорологические условия;
ж)	реакцию членов экипажа.
Во всех случаях окончательное решение на вылет ВС, на котором имеются не устраненные неисправности, принимает КВС с учетом условий предстоящего полета, оборудования аэродромов взлета и посадки, степени готовности (квалификации) экипажа.
Командиру ВС предоставляется право отказаться от полета, если он считает, что неисправности ВС окажут влияние на безопасность полета в конкретных условияx.
При неясности или двояком понимании отдельных пунктов, любые сомнения относительно интерпретации MEL должны быть решены КВС в сторону усиления (большего ограничения) требований MEL.
Командиру ВС запрещается принимать решение на вылет, а ИТП передавать экипажу ВС с отложенными неисправностями (отказами), не оформленными в установленном порядке согласно MEL.
Рубежом применения MEL при подготовке ВС к вылету является момент начала руления.
Внимание!
Если отказ произошел в процессе руления, командир ВС может принять решение на продолжение рейса, учитывая характер отказа и возможности безопасного выполнения полета. В этом случае, следует воспользоваться рекомендациями MEL и QRH. Такое решение на продолжение рейса командир ВС может принять до взлета ВС, если в примечаниях к пункту MEL отсутствует процедура (М), есть возможность выполнения процедуры (О), проконсультировавшись, при необходимости, с сертифицированным ИТП.
В случае отказа в полете, положения MEL не применяются. 
(2) Схема применимости MEL по этапам полета.
 
8.33.3. Перечень отклонений от стандартной конфигурации CONFIGURATION DEVIATION 
                                                                        LIST (CDL)
Перечень отклонений от стандартной конфигурации самолета определяет панели, лючки, створки, обтекатели и т.п., которые могут отсутствовать в конструкции ВС без ущерба для летной годности, но которые влияют на летно-технические и эксплуатационные характеристики ВС (например, ухудшение комфорта из-за дополнительного шума).
8.33.4. Применение MEL/CDL в аэропортах, где отсутствует квалифицированный 
                                                      технический персонал
Указанное применение возможно только в случае отсутствия в условиях применения пункта MEL процедуры (М) или при наличии процедуры (М), не предусматривающей никаких действий по ТО, кроме отключения АЗС или установки выключателей/переключателей в заданное положение. В противном случае в аэропорт доставляется сертифицированный ИТП для выполнения соответствующих процедур или устранения неисправности.
В случае если отказ или неисправность обнаружены на аэродроме, где нет сертифицированной организации по ТО, КВС, любым доступным способом, сообщает об отказе или неисправности диспетчеру ОКВР, специалистам отдела ПЛГ иностранных ВС УПЛГ ВС ИАС и, при необходимости, в Инспекцию по безопасности полетов Общества.
Отдел ПЛГ иностранных ВС УПЛГ ВС ИАС сообщает о выявленном дефекте в контрактную организацию по ТО ВС для оценки неисправности и определения необходимых действий по ТО. 
В любом случае, КВС делает запись о неисправности в бортжурнал (TLB) ВС, докладывает диспетчеру   ОКВР с предоставлением, по возможности, копии страницы бортжурнала с описанием неисправности любым доступным способом.
Диспетчер ОКВР информирует об этом уполномоченного представителя ИАС, где выполняются следующие работы:
а)	сертифицированный специалист ИТП по ТО знакомится с записью в бортжурнале (TLB), проводит консультации с экипажем, анализирует неисправности на основе РЛЭ (AFM, FCOM) и РТЭ (АММ) и др.;
б)	в зависимости от характера неисправности, влияния её на безопасность полета и трудоёмкости устранения принимается решение о возможности продолжения эксплуатации ВС с отложенной неисправностью по MEL/CDL или необходимости устранения неисправности;
в)	при невозможности применения соответствующего пункта МЕL/CDL или устранения неисправности на месте, согласно РОТО, делается доклад руководителю организации по ТО для принятия решения о возможности дальнейшей эксплуатации ВС и информация передается диспетчеру ОКВР;
г)	Диспетчер ОКВР информирует командира ВС о допуске ВС к эксплуатации с неисправностью по МЕL/CDL или о решении руководителя организации по ТО о дальнейшей эксплуатации ВС при невозможности применения соответствующего пункта МЕL/CDL, или устранения неисправности на месте;
д)	руководитель организации по ТО передает диспетчеру ОКВР и командиру ВС, с использованием возможных каналов связи, радиограмму о возможности эксплуатации ВС в соответствии с пунктом MEL/CDL.
Командир ВС вкладывает в бортжурнал (TLB), полученную РД и, по рекомендациям организации по ТО, выполняет операции в соответствии с требованиями пункта MEL.
В зависимости от конкретных условий полёта и ограничений, возникающих в связи с открытием пункта MEL/CDL, при проведении предполётной подготовки КВС принимает окончательное решение на выполнение полёта с открытым пунктом МЕL/CDL. При возникновении условий или каких-либо причин, угрожающих безопасности полёта с данной неисправностью, КВС имеет право отложить вылет до изменения условий или устранения неисправности.
В листе TLB, где внесена запись о замечании на ВС, в разделе «ACTION \ WORK PERFORMED» КВС записывает решение на вылет: «Вылет рейса ЮТ ХХХ в соответствии с РД №ХХХХХХХ, указывает имя и фамилию сертифицирующего персонала, его личный номер, номер одобренной организации по ТО ВС» и вкладывает РД в бортовой журнал (TLB).
 (Пример записи: «Schedule flight UT ХХХ acceptable iaw telex #.XXXXXXXX. Andrey Vasilev, Stamp number UTG 090, BDA/AMO/510).
ПРИМЕЧАНИЕ. Для решения оперативных вопросов в любое время в т.ч., нерабочее, праздничные и выходные дни круглосуточно на связи находится диспетчер по планированию полетов ОКВР:
раб.тел. +7 391 270-50-83
раб. тел. +7 391 275-20-37
моб.тел. +7 912 990 40 86 (круглосуточный)
почта: pds-trh@utair.ru, dispetcher-trh@utair.ru
Номера телефонов для связи с отделом ПЛГ иностранных ВС УПЛГ ВС ИАС: 
+7 (3452) 28-58-13
+79199218295
+79097866777. 
8.33.4.1.	Полеты (перегон) ВС с неисправностью (отказом), не указанной (м) в перечне минимально исправного оборудования MEL/CDL
Такие полеты выполняются без пассажиров в случаях, когда отсутствует возможность выполнения необходимого ТО на данном аэродроме, по специальному разрешению (телеграмме, факсу, E-mail) технического директора или лица его замещающего в конкретном случае и с соблюдением следующих требований:
а)	составлен технический акт, в котором четко определены характер, причины и последствия неисправности, имеется обоснованное заключение о том, что неисправность не влияет на безопасность полета и с указанием конкретного срока устранения неисправности, необходимыми мероприятиями по контролю за неисправностями;
б)	КВС перед каждым полетом имеет полную информацию о неисправности и ее влиянии на полет;
в)	в бортовой журнал (TLB) и «карту – наряд» для ВС, внесены записи о выполненных на ВС работах по обеспечению безопасного выполнения полета с не устраненной неисправностью, а также название и номер документа, разрешающего полет.
Для выполнения перегона ответственные лица летного подразделения, ИАС определяют состав экипажа для выполнения перегона, соответствующий необходимым требованиям (КВС налет более 1000 ч на типе, ВП налет более 500 ч), организуют предварительную подготовку, на которой изучаются статус ВС, дополнительные ограничения и условия, изменения эксплуатационных характеристик ВС, предоставляют соответствующие документы и запрашивают разрешение на выполнение перегона:
а)	у предприятия - изготовителя ВС;
б)	у авиационных властей государства регистрации;
в)	государств, в воздушном пространстве которых будет выполняться перегон (при необходимости);
г)	в инспекции Авиакомпании и в Инспекции ФАВТ.
В случаях, подпадающих под понятие Специальный полет, разрешение выдается уполномоченным органом в области ГА.
8.34. Некоммерческие полеты
8.34.1. Полеты по перегонке воздушных судов
Полеты по перегонке ВС в ремонт, из ремонта и в целях перебазирования производятся днем и ночью в соответствии с требованиями нормативных документов ГА.
Оформление перегонки ВС, состояние которых не отвечает установленным техническим требованиям, (за исключением специальных полетов воздушного судна) производится в порядке, определенном Приказом. ФАВТ от 18.08.2008 г. № 244 «О порядке оформления и выдачи разрешений на выполнение разовых полетов ВС, обусловленных особыми условиями эксплуатации».
8.34.1.1.	Специальный полет воздушного судна (приказ ФСВТ от 17.05.2000 г. №126) 
Это полет (полеты) экземпляра гражданского воздушного судна, которое в настоящее время может не отвечать применяемым требованиям летной годности, но способно совершить безопасный полет (полеты) с целью перелета на базу для ремонта или переоборудования, или технического обслуживания, или на место хранения без права перевозки грузов и пассажиров и выполнения каких-либо работ (РПИП ГА-91, п. 1.3, примечание, Авиационные правила, часть 21, п. 21.197, п.п. (а)).
Специальный полет воздушного судна, организуется и выполняется в случаях:
а)	конструкция и (или) характеристики ВС в настоящее время частично не соответствуют типовой конструкции;
б)	имеющего просроченный срок службы и (или) ресурс (срок действия сертификата (удостоверения) летной годности истек);
в)	в конструкции ВС, системах, агрегатах или силовых установках имеются неисправности и (или) отказы, выходящие за пределы соответствующих разделов РЛЭ ВС, которые невозможно устранить на месте, где в настоящее время находится ВС.
Специальные полеты по Инструкции соответствуют испытательному полету и выполняются по индивидуально разработанной Программе летно-испытательными экипажами ЛИО ГосНИИ ГА или иными летчиками-испытателями, имеющими соответствующие допуски и сертификат (свидетельство летчиков испытателей), прошедшими одобрение на летно-методическом совете ГосНИИ ГА или экипажами, в составе которых включен летчик-испытатель (РПИП-ГА-91 п. 1.3, примечание, Авиационные правила, часть 21, п.11.2, п.п. 11.2.2).
8.34.2.	Испытательные, исследовательские и контрольные полеты (облеты), 
                                                            контрольное руление
Порядок проведения контрольных полетов, облетов и КР приведен в таблице в Приложении А 8.3.
Контрольные полеты (облеты) выполняются в целях проверки исправности и определения годности ВС к эксплуатации, после ремонта, замены двигателя (двигателей) и оборудования, а также для проверки работы радиосветотехнических средств, схем снижения и захода на посадку на аэродроме, которая не может быть выполнена на земле. Воздушные суда, прошедшие ремонт и лётные испытания на заводах, при приёмке их представителями авиакомпании опробуются в полёте экипажами авиакомпании, если облёты предусмотрены воздушным законодательством.
К испытательным, исследовательским полетам, а также к полетам, связанным с испытанием ВС после капитального ремонта на заводах, допускается летный состав в соответствии с требованиями уполномоченного органа в области ГА.
8.34.2.1. Контрольно-испытательные полеты (облеты) после выполнения работ, предусмотренных регламентом технического обслуживания.
Контрольно-испытательный полет выполняются в соответствии с требованиями действующих методических указаний типовых программ и утвержденных уполномоченным органом в области ГА типовых программ.
При отсутствии типовой программы контрольного полета авиакомпанией разрабатывается и утверждается индивидуальная программа его проведения, в которой указывается цель полета, его условия и режимы, параметры, подлежащие проверке, а также состав экипажа и других участников контрольного полета.
В зависимости от программы контрольного полёта в задание на полет могут быть включены работники научно – исследовательских организаций, лица инженерно - технического состава и другие специалисты. Решение о включении их в задание на полет принимается эксплуатантом. 
Совмещать контрольно-испытательный полет с выполнением производственных заданий запрещается, кроме случаев, разрешенных уполномоченным органом в области ГА.
Контрольные полеты (облеты) производятся: для облета ВС днем при видимости не менее 2000 м и высоте нижней границы облаков не менее 200 м, но не ниже минимума, установленного инструкцией по производству полетов на данном аэродроме, 
Полеты выполняются при хорошей видимости горизонта и с превышением над облачностью не менее 1000 м.
Подготовку ВС к контрольному полету осуществляют в соответствии с требованиями ЭД и производственным заданием. В карте-наряде на оперативное ТО ВС и в бортжурнале записывается «Самолет подготовлен к контрольному полету. Вылет разрешаю». Карта-наряд должна иметь отметку «Перед контрольным полетом».
Контрольные полеты (облеты) выполняется экипажем эксплуатационного подразделения по письменной заявке ИАС после проведения, соответствующего ТО и готовности ВС к полетам. Экипажу выдаются бланки Карты контрольного полета, программы контрольно - испытательного полета, а по п.1 программы, бланки протоколов.
Полеты, отмеченные звездочкой (*) в программе КИП, кроме полетов после замены руля высоты или стабилизатора, выполняют экипажи, в состав которого входят в качестве КВС пилот командно-летного или инструкторского состава эксплуатационного подразделения по принадлежности ВС.
Полеты после замены стабилизатора или руля высоты выполняют экипажи, в состав которого входят в качестве КВС пилот командно-летного или инструкторского состава эксплуатационного подразделения по принадлежности ВС, прошедшего тренировку на этих режимах с летчиком - испытателем Гос. НИИ ГА.
Приказом директора авиакомпании на основании записи в летной книжке, сделанной летчиком испытателем Гос. НИИ ГА, лица командно-летного или инструкторского состава допускаются к выполнению этих полетов в качестве командира ВС.
По каждому пункту Программы выполняется один полет, продолжительность которого указана для каждого типа ВС.
Экипаж обязан проверить и оценить работу вновь установленных двигателей (двигателя) или агрегатов в соответствии с методикой, изложенной в программе, с требованиями РЛЭ, Инструкцией по эксплуатации двигателя и т.д.
В полетах, связанных с заменой ОЧК, киля, стабилизатора, закрылков, рулей высоты и направления, элеронов или рулевых машин, экипаж обязан дать качественную оценку устойчивости и управляемости.
Внимание! В случае появления каких-либо отклонений в поведении самолета (тряска, падение усилий на штурвале или педалях, колебания по курсу и крену и т. д.) КВС должен изменить режим полета, прекратить выполнение задания и произвести посадку.
Обо всех отклонениях в поведении самолета и неисправностях, обнаруженных в полете, экипаж обязан сделать запись в бортжурнале ВС, а по п.1 Программы оформить протоколы.
Если после устранения дефектов, выявленных в контрольно - испытательном полете, вновь требуется проверка в полете, то в повторном контрольно-испытательном полете проверяется работа тех агрегатов и систем, по которым были замечания экипажа.
Содержание полета по программе может быть дополнено начальником ИАС совместно с начальником летной службы с соответствующей записью в задании на облет.
По окончании контрольно-испытательного полета произвести техническое обслуживание в соответствии с регламентом и с обязательным осмотром тех агрегатов и систем, из-за которых проводился полет. Карта-наряд на выполнение указанных работ и приложения к ней должны иметь отметку «После выполнения контрольного полета».
Замечания экипажа о выявленных неисправностях, записанных в бортжурнале, должны быть перенесены в карту-наряд и устранены под контролем инженера ОТК. Запись в бортжурнале об устранении неисправностей производит инженер смены. Заключение об исправности, ТО и устранения неисправностей дают непосредственный руководитель работ и специалист, ответственный за их контроль и качество.
На выполненные после контрольного полета работы ТО оформляются в установленном порядке карта-наряд, бортовой журнал ВС и производятся соответствующие записи в карте контрольного полета. После контрольно-испытательного полета производят расшифровку средств записи полетной информации.
Бортовой журнал и карты-наряды на ТО до и после контрольно-испытательного полета направляются для проверки в ОТК, а затем в ИАС. На основании карты контрольного полета, подписанной командиром ВС, оформленных вышеуказанных карт-нарядов, производится запись в формуляр воздушного судна.
Контрольный полет (указывается цель или причина полета) выполнен в течение........минут. Техническое обслуживание выполнено по карте-наряду N...........самолет пригоден к эксплуатации.
8.34.2.2.	Контрольное руление
Контрольное руление ВС производят для проверки работы систем и изделий, которая не может быть выполнена на стоянке.
Перечень обязательных случаев, требующих контрольного руления ВС, устанавливает ЭД.
Авиакомпания вправе принимать решения о выполнении контрольного руления и в других случаях, не входящих в состав обязательных, с учетом устанавливаемых ЭД требований и ограничений.
Контрольное руление выполняют в соответствии с требованиями РПП, РЛЭ типа ВС и утвержденной программой его проведения.
При отсутствии типовой программы контрольного руления авиакомпанией разрабатывается и утверждается индивидуальная программа его проведения, в которой указывается цель контрольного руления, его условия и режимы, параметры, подлежащие проверке, а также состав экипажа и других участников контрольного руления.
Программа контрольного руления ВС составляется специалистом, ответственным за проведение ТО и работ по устранению неисправностей, обуславливающей необходимость контрольного руления и утверждается начальником ИАС.
Утвержденная программа контрольного руления передается экипажу ВС вместе с заданием на контрольное руление.
 Подготовку ВС к контрольному рулению осуществляют в соответствии с требованиями ЭД, программой контрольного руления и производственным заданием. В карте-наряде на оперативное обслуживание ВС и бортовом журнале (VI раздел) записывают: "Самолет подготовлен к контрольному рулению. Руление разрешаю".
В карте-наряде должна быть отметка "Перед контрольным рулением".
О выполнении программы контрольного руления и его результатах записывают в бортовом журнале и программе контрольного руления.
Объем технического обслуживания ВС после контрольного руления, порядок производства работ и оформлении технической документации определяется требованиями ЭД и производственным заданием. Карта-наряд на выполнение указанных работ должна иметь отметку "После выполнения контрольного руления".
Программа контрольного руления с заключением экипажа за подписью КВС прилагается к карте-наряду.
Заключение об исправности ВС после выполнения работ ТО и устранения неисправностей дают непосредственный руководитель работ и специалист ответственный за общий контроль их качества.
8.34.3.	Учебные и тренировочные полеты
8.34.3.1.	Тренировочные полеты
Запрещается выполнять тренировочные полеты на ВС, не оборудованном системой двойного управления.
При выполнении тренировочных полетов, если одно из пилотских кресел занято тренируемым (обучаемым), то второе кресло должно быть занято лицом, имеющим достаточную квалификацию и обладающим свидетельством пилота с квалификационными отметками «инструктор» типа, класса, соответствующими воздушному судну, на котором выполняется полет.
Запрещается имитировать аварийную обстановку или нештатные ситуации при осуществлении перевозки пассажиров или груза.
При тренировочных полетах на борту ВС может находиться не более 2 тренируемых экипажей или 4 пилотов.
При выполнении полетов с выключением двигателя (двигателей) или на предельных режимах на борту ВС может находиться только 1 тренируемый экипаж.
Состав экипажа при тренировочных полетах определяется, исходя из цели задания на полет. Тренирующий (тренируемый) выполняет обязанности того из пилотов, рабочее место которого он занимает.
Выполнение взлетов и посадок в аэродромных полетах разрешается при фактических метеоусловиях не ниже соответствующих тренировочных минимумов аэродрома и инструктора, а также наличии запасного аэродрома. При метеоусловиях ниже тренировочного минимума разрешается выполнение заходов на посадку с уходом на второй круг, с высоты не меньшей ВПР(DA(H)) / МВС(MDA(H)), установленной для тренировочных полетов.
8.34.4.	Демонстрационные полеты
Демонстрационные полеты выполняются в соответствии с требованиями, установленными в главе XXVIII «ФАП полетов в воздушном пространстве РФ». К участию в таких полетах допускаются экипажи, прошедшие специальную подготовку.
8.34.5.	Поисковые полеты
8.34.5.1. Организация и проведение спасательных полетов осуществляется в соответствии с Руководством по организации и проведению аварийно-спасательных работ в районе ответственности авиакомпании.
8.34.5.2. Поисковые полеты выполняются на воздушных судах, обеспеченных поисковым и аварийно- спасательным оборудованием. При вылете на поиск на борту ВС должна находиться парашютно-десантная группа (группа спасателей), в том числе медицинский работник.
8.34.5.3. Для проведения спасательных полетов организуется дежурство подготовленных к поиску экипажей, поисковых воздушных судов. Для проведения спасательных полетов, кроме специально выделенных, могут использоваться не оборудованные самолеты.
8.34.5.4. Экипажи воздушных судов авиакомпании, привлекаемые для выполнения поиска и спасания, обеспечения ликвидации чрезвычайных ситуаций, должны пройти соответствующую подготовку. 

8.35. Требования по обеспечению и использованию кислорода в процессе выполнения полета
8.35.1.	Требования по обеспечению и использованию кислорода
8.35.1.1. Кислородная система самолета предназначена для питания кислородом пилотов, управляющих воздушным судном в течение всего высотного полета, аварийного питания членов летного экипажа, кабинного экипажа, а также пассажиров в случаях разгерметизации ВС, а также в терапевтических целях.
8.35.1.2. Командир ВС принимает меры к тому, чтобы обеспечить членов экипажа и пассажиров достаточным количеством кислорода для дыхания при выполнении полета на таких абсолютных высотах, где недостаток кислорода может привести к ухудшению работоспособности членов экипажа или оказать неблагоприятное воздействие на пассажиров.
При наличии автоматической стационарной кислородной системы в салоне не допускать размещения на каждом блоке кресел более одного ребенка, который перевозится без предоставления пассажирского места.
8.35.1.3. Перед полетом члены летного экипажа должны подготовить индивидуальное кислородное оборудование для использования в экстренном случае. В случае покидания рабочего места одним из пилотов ВС пилот, осуществляющий пилотирование на высотах свыше 7600м, обязан надеть кислородную маску и пользоваться ею до возвращения другого пилота на рабочее место, если иное не предусмотрено РЛЭ (AFM, FCOM, QRH).
8.35.1.4. Переносное кислородное оборудование для членов кабинного экипажа и пассажиров должно находиться в максимально заправленном состоянии и постоянно готовым к использованию в течение полета.
8.35.1.5. Воздушные суда Авиакомпании, имеющие герметизируемые кабины для размещения экипажа и пассажиров, и выполняющие полеты на высотах от 10000ft до 13000ft в течение более 30 мин. и на высотах более 4000 м в течение всего времени, оборудованы достаточным количеством кислородных масок и переносных кислородных приборов, расположенных таким образом, чтобы в случае разгерметизации обеспечить немедленный доступ всех членов летного и кабинного экипажа к кислороду.
8.35.1.6. При выполнении полета, когда условная высота в салоне превышает 10000 ft (атмосферное давление в салоне меньше 700 ГПа) в течение более чем 30 минут, а также в течение любого периода, когда условная высота в салоне достигает 13000ft. (атмосферное давление в салоне менее 620ГПа), члены летного экипажа обязаны пользоваться для дыхания индивидуальными кислородными масками.
8.35.1.7. Воздушные суда Авиакомпании, имеющие герметизируемые кабины, оборудованы визуальной и звуковой сигнализацией для пилотов об опасном падении давления (разгерметизации) внутри воздушного судна
8.35.2.	Запас кислорода 
8.35.2.1. Экипаж воздушного судна обязан контролировать наличие запаса кислорода перед полетом и его использование при дыхании членами экипажа и пассажирами:
а)	при полетах на высотах, где барометрическая высота в кабине составляет от 3000 до 4000 м более 30 минут – всех членов экипажа и, по крайней мере,10% пассажиров, в течение времени, превышающего 30 минут;
б)	при полете на высотах, где барометрическая высота в кабине превышает 4000м (13 000ф) - в течение всего времени для всех членов экипажа и пассажиров;
в)	для ВС с герметизированными кабинами при полетах выше барометрической высоты 7600м (25 000ф) или при полетах ниже 7600м (25 000фут), если ВC не может безопасно снизиться в течение 4 минут до высоты, где барометрическая высота в кабине составит 4000м (13 000ф), должно обеспечиваться не менее 10 минутного запаса кислорода для всех членов экипажа и пассажиров для использования его в случае экстренного снижения (до 75 м/сек – 15 000ф/мин) при падении давления в кабине.
Время снижения с максимального эшелона полета пассажирских ВС до высоты 4000м (13 000ф) занимает не более 90 сек, что является допустимым временем пребывания пассажиров в условиях разгерметизации.
8.35.3.	Оборудование, защищающее органы дыхания экипажа
8.35.3.1.	Летный экипаж
Рабочие места всех членов летного экипажа оснащаются оборудованием (дымозащитными масками), защищающим глаза, нос и рот и обеспечивающим подачу кислорода не менее чем на 15 минут, сохраняющим возможность членов летного экипажа осуществлять внутреннюю и внешнюю радиосвязь.
Дополнительно в кабине экипажа ВС находится легкодоступный любому члену летного экипажа комплект переносного оборудования, предназначенный для защиты глаз, носа и рта (баллон с дымозащитной маской) и обеспечивающий подачу кислорода для дыхания не менее чем на 15 минут.
8.35.3.2.	Кабинный экипаж
Воздушное судно оснащается следующим оборудованием:
а)	переносное оборудование, расположенное на каждой станции бортпроводника, защищающее глаза, нос и рот и обеспечивающее подачу кислорода не менее, чем на 15 мин. для каждого основного номера кабинного экипажа;
б)	один дополнительный комплект защитного оборудования, расположенный рядом с огнетушителем.
8.35.3.3. Процедура использования кислородных масок лётным экипажем в аварийной ситуации, условия, вызывающие необходимость применения кислородных масок и процедура использования изложены в FCOM, QRH конкретного типа ВС.
8.36.	Наблюдения и донесения с борта воздушного судна
Экипаж воздушного судна, находящийся в полете по запросу органа ОВД сообщает сведения о фактической погоде, а также, при возможности, о наблюдении им опасных метеорологических явлений. Сведения о погоде, полученные от экипажей воздушных судов, используются метеорологическими подразделениями при подготовке оперативных прогнозов погоды.
Наблюдения с бортов воздушных судов подразделяются на следующие виды:
а)	наблюдения на этапе набора высоты, снижения;
б)	наблюдения при полете по воздушной трассе или району выполнения авиационных работ;
в)	специальные и другие нерегулярные наблюдения с борта;
г)	наблюдения по форме AIREP при выполнении международных полетов.
8.36.1.	Наблюдения на этапе набора высоты, снижения
8.36.1.1. Экипажи ВС во время набора высоты (снижения) сообщают данные о высоте нижней и верхней границы облаков, наличии обледенения, турбулентности, сдвига ветра, а также ветре на 100 м и высоте круга. Указанные сведения передаются диспетчеру при наборе высоты после достижения безопасной высоты или во время снижения.
8.36.1.2. В тех случаях, когда аэродромным метеорологическим органом выпущено предупреждение о сдвиге ветра в зонах набора высоты или захода на посадку, который фактически не наблюдается, экипаж воздушного судна сообщает об этом диспетчеру УВД по возможности в кратчайший срок.
8.36.2.	Наблюдения при полете по воздушной трассе или району выполнения авиационных работ
8.36.2.1. Указанные наблюдения проводятся во всех случаях, когда имеют место сильная турбулентность, обледенение или другие условия (явления), которые, по мнению командира воздушного судна могут влиять на безопасность полета других воздушных судов. Информация о наличии указанных условий и явлений передается сразу после их обнаружения.
8.36.3.	Специальные и другие нерегулярные наблюдения
Наблюдения поводятся в случаях, когда аэродромный метеорологический орган, обеспечивающий полеты в районе УВД, через который следует воздушное судно, запрашивает определенные данные.  
 
8.36.4.	Наблюдения с борта воздушного судна по форме AIREP
Порядок проведении наблюдений при выполнении международных полетов по форме AIREP с борта воздушного судна, регистрации данных наблюдений, их передачи с борта воздушного судна и дальнейшего распространения определен специальной инструкцией.
8.36.5.	Регистрация данных бортовых наблюдений
8.36.5.1. Данные наблюдения с борта воздушного судна, кроме тех, которые указаны в п3, регистрируются на специальном бланке «бортовая погода», если экипаж ВС был обеспечен таким бланком перед вылетом.
8.36.5.2. Сообщения с борта воздушных судов, получаемые аэродромным метеорологическим органом через орган ОВД, регистрируются в специальном журнале. 
8.37.	Перечень услуг, предоставляемых пассажирам на борту воздушного судна
8.37.1.	Информационно – справочное обслуживание
Информационно – справочное обслуживание – своевременное и достоверное информирование пассажиров о правилах поведения на борту воздушного судна, правилах пользования аварийно-спасательным оборудованием, предоставляемых услугах, условиях полёта. Информирование пассажиров проводится посредством СГУ, трафаретов, световых табло, по информационным листкам и индивидуально. Данный вид обслуживания проводится на всех этапах обслуживания пассажиров. Информация должна быть краткой и по существу, в соответствии с утверждёнными текстами информаций.
8.37.2.	Индивидуальное обслуживание
Индивидуальное обслуживание проводится на всех этапах полета и включает в себя:
а)	оказание услуг каждому пассажиру в течении всего полёта:
б)	особое внимание к больным, инвалидам, пожилым людям и пассажирам с детьми;
в)	включение индивидуального освещения и вентиляции;
г)	ответы на вопросы пассажиров, беседы с пассажирами;
д)	установка и оснащение детских люлек;
е)	обслуживание не сопровождаемых детей.
В течении всего полёта бортпроводник должен уделять внимание пассажирам, не оставлять салоны без внимания более чем на 10 минут, предупреждать возможные неприятности, устранять имеющиеся неудобства.
8.37.3.	Медицинское обслуживание
Медицинское обслуживание – оказание первой доврачебной помощи пассажирам на борту воздушного судна с применением средств и препаратов, содержащихся в бортовой медицинской аптечке. Все бортпроводники должны иметь элементарные навыки оказания первой медицинской помощи. В течении всего полёта бортпроводник должен наблюдать за пассажирами, обращая внимание на специфические внешние признаки, характеризующие состояние здоровья человека. Особое внимание должно уделяться больным, пассажирам преклонного возраста и детям.
                                  
                                                             --------конец главы-----
