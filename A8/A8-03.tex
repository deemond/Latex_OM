\section{Методика расчета минимальных высот полета}

В целях предотвращения столкновений ВС с наземными препятствиями авиакомпания устанавливает правила расчета и использования минимальных безопасных высот полета и эшелонов полета.

Минимальные высоты полета контролируются по барометрическим высотомерам (основным и контрольным), установленным по давлению QNH, QFE или QNE в соответствии с методикой, приведенной в данном разделе. 
 
Предварительный расчет минимальных высот полета производится специалистом отдела аэронавигационного обеспечения (АНО) авиакомпании для всех постоянных маршрутов полета, эксплуатируемых основных и запасных аэродромов для условий МСА. Рассчитанные значения минимальных высот полета могут корректироваться в случаях, когда изменились данные АНИ, на которых основаны их расчеты, либо при обнаружении любых неточностей в расчетах.

Экипаж в процессе предварительной или предполетной подготовки должен самостоятельно определить значения минимальных безопасных высот в соответствии с "Методикой определения минимальных высот полета".

Независимо от имеющихся в распоряжении экипажа сведений о заранее рассчитанных минимальных высотах полета, на борту ВС при выполнении полета должны быть такие документы АНИ, которые позволят экипажу самостоятельно определить минимальную высоту полета при любом возможном отклонении от запланированного маршрута полета.

\subsection{Минимальные высоты при полетах по РФ}

\paragraph{} За исключением случаев, в которых это необходимо при осуществлении взлета, посадки или указанных в пункте 8.3.1.2 запрещено выполнять полет воздушного судна: 
а)	над территориями населенных пунктов и над местами скопления людей при проведении массовых мероприятий - ниже высоты, допускающей в случае отказа двигателя аварийную посадку без создания чрезмерной опасности для людей и имущества на земле, и ниже высоты 300 м над самым высоким препятствием в пределах горизонтального радиуса в 500 м вокруг данного воздушного судна;
б)	в местах, не указанных в пункте 8.3.1.1(а) на расстоянии менее 150 м от людей, транспортных средств или строений. 

\paragraph{} Полеты с отклонением от требований пункта 8.3.1.1. (б) разрешены в случаях, когда это не создает опасности для людей или имущества на земле при выполнении авиационных работ или летном обучении под наблюдением пилота–инструктора. 

В зависимости от правил полетов, рельефа местности, скорости полета устанавливается минимальный запас высоты полета (МЗВ) воздушного судна над наивысшими препятствиями в соответствии с Таблицей А8.3-Т1.

В связи с тем, что измерение высот полета производится барометрическими высотомерами, которые при изменении температуры относительно МСА имеют методическую погрешность и при температурах наружного воздуха менее стандартной +15ºС завышают значение измеренной высоты, необходимо, при выполнении захода на посадку при отрицательных температурах наружного воздуха, высоты ниже высоты круга (контрольных точек FAP, FAF) выдерживать с учетом температурной поправки (ΔНt). 
Температурная поправка определяется с помощью Таблицы А8.3-Т2 или по формуле в Приложении А8.1.

На аэродромах Российской Федерации, при расчете схем захода на посадку, для публикации в документах АНИ значения высот круга и выше (Н IAF, Н БВП, Н MSA, Н круга, Н FAF в случаях, когда она равна Н круга), рассчитываются с учетом минимальных температур на аэродроме, отмеченных за период многолетних наблюдений и не требуют корректировки по фактическим условиям полета. 

Безопасная высота полета (БВП) 

Безопасная высота полета (БВП) (прежнее обозначение МБВ) - высота полета, исключающая столкновение ВС с земной (водной) поверхностью или препятствиями на ней. Обеспечивает запас высоты над поверхностью или препятствиями не менее 300м. В публикуемых в документах АНИ значениях БВП Российских аэродромов учтена температурная поправка по результатам многолетних наблюдений за температурой воздуха на конкретном аэродроме.

Значение БВП, указанное в скобках – по QFE в метрах; без скобок – по QNH в футах. 

Опубликованное в документах АНИ значение БВП действительно в радиусе 50 км от КТА или радионавигационного средства (указано в публикации). 

Зона действия БВП может быть разбита на секторы.

Инструментальные погрешности барометрических высотомеров зависят от многих факторов, основными из которых являются: точность изготовления, упругое последствие анероидной коробки, гистерезис – разница прямого и обратного хода, дисбаланс передаточно-множительного механизма, люфты и др.
Электромеханический барометрический высотомер обеспечивает более точное измерение относительной высоты полета за счет разгрузки чувствительного элемента с помощью электрической следящей системы. 

Инструментальные поправки сведены в Таблицы поправок для учета их при выполнении полета.

Современные высотоизмерительные системы обладают незначительными инструментальными погрешностями, что позволяет не учитывать их при выдерживании заданных высот полета (РЛЭ, AFM, FCOM типа ВС).

\begin{center}
    === To be continued ===
\end{center}
