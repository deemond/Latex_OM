
\section{Подготовка к полетам}

Каждому полету должна предшествовать подготовка экипажей. 

Лицо летного состава может назначаться в состав экипажа ВС для выполнения полетов после ознакомления с законами, правилами и процедурами, касающимися его обязанностей в соответствии с планируемыми районами полетов, используемыми аэродромами и соответствующими аэронавигационными средствами.

\subsection{Предварительная подготовка}


Целью предварительной подготовки является подготовка пилотов ВС к предстоящему полету (полетам) по маршрутам и аэродромам. 

Планирование, организация и контроль проведения предварительной подготовки осуществляются командиром летного подразделения.

Все лица, входящие в состав экипажа, независимо от занимаемой должности и опыта работы, обязаны пройти подготовку и проверку готовности к полету в соответствии с требованиями данного РПП, стандартов и рекомендаций ИКАО.
Предварительная подготовка является одной из основных форм повышения уровня профессиональной подготовки летного состава, в процессе которой проводится изучение вопросов, касающихся предстоящего полетного задания, подготовка документов и материалов, необходимых для выполнения полета, отрабатываются действия в ожидаемых ситуациях, а также проводится контроль готовности.

Сроки и методы проведения предварительной подготовки:
\begin{enumerate}[label=\alph*), ref=\alph*]
\item Предварительная подготовка в зависимости от цели предстоящего полетного задания проводится в полном составе экипажа или индивидуально с каждым членом экипажа лицами инструкторского, командно-летного состава с участием необходимых специалистов не позднее дня накануне вылета в следующих случаях:
    \begin{itemize}    
        \item перед первым самостоятельным полетом пилота на данном типе воздушного судна в качестве командира ВС;
        \item перед первым полетом пилота в качестве командира ВС по данной трассе, маршруту, району полетов; 
        \item перед полетом по специальному заданию (литерные рейсы);
        \item перед выполнением нового вида авиационных работ;
        \item для КВС при полетах по маршрутам и на аэродромы, которые требуют особых навыков или знаний - не реже одного раза в течение последовательных 12 месяцев; 
        \item перед облетом ВС; 
        \item при сезонной подготовке; 
        \item по истечению у ЧЭ 12 месяцев после выполнения предыдущего полета в качестве члена летного экипажа ВС, инспектирующего пилота или наблюдателем в кабине летного экипажа:
        \begin{itemize} 
            \item в установленном районе;
            \item по маршруту или на аэродромы, которые требуют особых навыков или знаний. 
Примечание: В Авиакомпании к маршрутам и аэродромам, которые требуют особых навыков или знаний относятся:
            \item маршруты категории «В»; 
            \item аэродромы категории «В» и «С».
        \end{itemize}
        \item при систематических полетах по данным трассам или при выполнении данного вида работ – один раз в 6 месяцев (как правило, совмещается с ВЛП и ОЗП).
    \end{itemize}
\item Предварительная подготовка экипажа вне мест базирования в случае изменения задания на полет (полеты по новым трассам, маршрутам, районам) проводится командиром ВС. В этом случае отметка о проведении предварительной подготовки вносится в задание на полет. При этом ответственность за полноту и качество подготовки несет командир ВС.
\end{enumerate}

Предварительная подготовка проводится в полном составе назначенного на полет летного экипажа:
\begin{enumerate}[label=\alph*), ref=\alph*]
    \item перед первым самостоятельным полетом пилота на данном типе воздушного судна в качестве командира ВС; 
    \item перед первым полетом пилота в качестве командира ВС по данной трассе, маршруту, району полетов; 
    \item перед полетом в целях проверки воздушного судна после выполнения технического обслуживания (облетом ВС).
\end{enumerate}

В остальных случаях предварительная подготовка может проводиться раздельно (в разные дни) для КВС и других членов экипажа под руководством инструктора (по специальности).

Допускается проведение предварительной подготовки непосредственно перед вылетом, с учетом требуемого объема подготовки и установленного режима труда и отдыха.

Лицо, проводившее предварительную подготовку, несет персональную ответственность за ее полноту, качество, оформление документации и доведения информации до руководства летного подразделения о внесение итогов подготовки в информационную систему.

При выполнении полетов в закрепленном составе экипажа предварительные подготовки проводятся в полном составе назначенного на полет экипажа.

Ответственность за качество организации и проведения предварительной подготовки, учет при планировании летной работы, исключающий возможность вылета неподготовленного экипажа к выполнению задания, возлагается на командира летного подразделения (лицо его замещающее).

В завершение предварительной подготовки проводится контроль готовности к выполнению полета с розыгрышем полета, контролем знаний порядка использования MEL/CDL в части отложенных неисправностей, анализа сочетаний отложенных неисправностей по MEL/CDL, расчета взлетно-посадочных характеристик и принятия решения на полет. 

В перечень контрольных вопросов для предварительной подготовки входят вопросы, отражающие порядок совместных действий членов экипажа в различных условиях полета или при отказе авиационной техники, с учетом эксплуатируемых воздушных трасс, маршрутов, особенностей полетов. В зависимости от цели предварительной подготовки должны быть разработаны варианты конкретных вопросов, обязательных для изучения экипажем.

Заключение о готовности экипажа (ЧЭ) к полетам на установленные аэродромы, маршруты и районы, дает лицо, проводившее предварительную подготовку, на основании проведенного им контроля готовности по процедурам и особенностям выполнения предстоящего полета (полетов) и уверенности в том, что каждый ЧЭ в достаточной мере знает намеченный маршрут и аэродромы, которые могут быть использованы для взлета и посадки.

Лица командно-летного и инструкторского состава, принимавшие участие в организации и проведении предварительной подготовки, считаются прошедшими подготовку вместе с экипажем.

Контроль качества проведения предварительных подготовок обеспечивается в процессе их проведения, при розыгрыше полетов.

\subsection{Обеспечение предварительной подготовки}


Обеспечение предварительной подготовки осуществляется командиром летного подразделения, лицом его замещающим, с привлечением лиц КЛС.
Допускается проведение предварительной подготовки и контроля готовности по средствам видеоконференцсвязи для ЧЭ, находящихся вне мест базирования (нахождение в служебной поездке, в командировке и т. д.), в соответствии с суточным планом полетов.

Во всех случаях, ответственность за своевременность и качество личной подготовки возлагается персонально на каждого члена экипажа ВС

\subsection{Содержание предварительной подготовки}


Предварительная подготовка проводится в полном объеме и предусматривает:
\begin{itemize}
    \item уяснение задачи предстоящего полета (полетов);
    \item подбор и подготовку документации, необходимой для выполнения полета (полетов);
    \item изучение маршрута полета, его географических и климатических особенностей;
    \item изучение аэродрома(ов) назначения, запасных по документам АНИ (горных аэродромов и аэродромов, расположенных в приграничной полосе, по инструкциям по производству полетов в районе аэродрома);
    \item изучение расположения навигационных средств по маршруту полета, порядка и особенностей их
использования;
    \item изучение рубежей приема-передачи управления между пунктами ОрВД по маршруту полета и порядка ведения радиосвязи;
    \item изучение запретов, ограничений использования воздушного пространства и приграничной полосы (при полетах в приграничных районах) с выделением ее на полетной карте;
    \item изучение особенностей эксплуатации бортовых систем применительно к конкретным условиям предстоящего полета (полетов);
    \item изучение порядка взаимодействия членов экипажа в особых случаях полета на всех этапах его выполнения применительно к конкретным условиям предстоящего полета (полетов);
    \item изучение инструкции по производству полетов по воздушной трассе, маршрутно-справочных данных, норм заправки топлива на полет и планируемой предельной коммерческой загрузки;
    \item изучение инструкции по предотвращению захвата, угона и причинения ущерба самолету, акта незаконного вмешательства (АНВ);
    \item изучение правил использования самолетных аварийно-спасательных средств;
    \item при систематических полетах по данной трассе или виду авиационных работ предварительная подготовка может проводиться по решению командира летного подразделения не в полном объеме – учитываются только особенности полетов в предстоящий период, а также изменения в документах АНИ;
    \item при подготовке к полету на горный аэродром производится тренировка на тренажере по схеме 
данного аэродрома, если такая тренировка не проводилась при очередной ежеквартальной тренировке экипажа на тренажере;
    \item дополнительно тренаж в кабине ВС по взаимодействию и технологии работы членов экипажа при выполнении специальных видов авиационных работ (пожаротушение, десантирование, перевозка грузов, посадка на площадки, подобранные с воздуха);
    \item проводится розыгрыш полета.
\end{itemize}

Ответственность за качество организации и проведения предварительной подготовки, учет при планировании летной работы, исключающий возможность вылета неподготовленного экипажа, возлагается на командира летного подразделения (лицо, его замещающее).

 


\subsection{Контроль готовности экипажа к полету}


Предварительная подготовка завершается контролем готовности экипажа к полету, который проводится лицами инструкторского состава, с привлечением лиц КЛС (старшим группы при базировании в отрыве от базы) или лицом его замещающим.

Перечень контрольных вопросов разрабатывается специалистами летного подразделения для каждого региона полетов, вида полетов, вида авиационных работ и включает в себя особенности выполнения полетов по типовым маршрутам (аэродромам). В перечень контрольных вопросов включаются вопросы, отражающие особенности взаимодействия членов экипажа в полете и в особых случаях, применительно к конкретным условиям предстоящего полета (полетов). 

Ответственность за объективность оценки готовности экипажа (членов экипажа) несет лицо инструкторского состава (КЛС) осуществляющее контроль готовности.

\subsection{Учет предварительной подготовки}


Оформление и учет предварительных подготовок осуществляется на бланках установленной формы, которые хранятся в летном подразделении. При проведении предварительной подготовки вне мест базирования – оформляется в задании на полет в разделе «Особые отметки» за подписью КВС. Форма записи: «Дата», «Предварительная подготовка проведена по маршруту…..», экипаж к полету готов, подпись КВС».

Информация о сроках проведения подготовки вносится в информационную систему.

\subsection{Предварительная подготовка кабинного экипажа}


Предварительная подготовка проводится:
\begin{itemize}
    \item перед первым полетом в качестве стажера;
    \item перед первым самостоятельным полетом;
    \item перед полетом по специальному заданию;
    \item перед полетом в другой климатической (географической) зоне;
    \item перед направлением на работу с местом базирования на другом аэродроме;
    \item перед допуском к полетам в качестве инструктора;
    \item при сезонной подготовке. 
\end{itemize}

Предварительная подготовка кабинного экипажа проводится в полном составе. Допускается проведение подготовки несколькими экипажами одновременно, за исключением - перед первым самостоятельным полетом в качестве инструктора.

Предварительная подготовка проводится в те же сроки, что и подготовка летных экипажей.

К проведению предварительной подготовки привлекаются специалисты служб Авиакомпании.

Предварительная подготовка к полету кабинного экипажа предусматривает:
\begin{itemize}
    \item уяснение задачи предстоящего полета;
    \item подбор и подготовку необходимой документации и справочного материала;
    \item оказание первой помощи, организацию эвакуации и действия по обеспечению выживания в районах с различными климатическими условиями при посадке на сушу и воду. Взаимодействие с летным экипажем в особых случаях:
    \item изучение особенностей обслуживания пассажиров в предстоящий период работы.
\end{itemize}

Предварительную подготовку организует и проводит ведущий специалист кабинного состава лётной службы или должностное лицо, его замещающее. Ответственность за полноту и качество подготовки несут лица, проводившие подготовку.

Факт проведения и учет предварительных подготовок осуществляется на бланках установленной формы, которые хранятся в летном подразделении.

\subsection{Контроль организации и качества проведения предварительной подготовки}


Контроль организации и качества проведения предварительной подготовки в летных подразделениях осуществляется вышестоящим командно-летным и инспекторским составом.

Эффективность и качество проведения предварительной подготовки зависит от умения командно-летного состава подразделения правильно ее организовать и проводить. Поэтому вышестоящий командно-летный и инспекторский состав обязан осуществлять систематический контроль за организацией и качеством проведения предварительной подготовки в летных подразделениях.

Контроль включает:
\begin{itemize}
    \item сроки проведения и ход выполнения плана предварительных подготовок экипажей (членов экипажей) подразделения;
    \item качество проведения и обеспечения подготовок, перечни контрольных вопросов;
    \item ведение учетной документации.
\end{itemize}

При обнаружении недостатков в проведении предварительной подготовки должностное лицо, осуществляющее контроль, обязано внести соответствующие коррективы, а при необходимости - лично подготовить и провести предварительную подготовку с экипажем (членом экипажа) как показательную, с привлечением командно-летного и инструкторского состава.

О результатах проведения контроля или показательной предварительной подготовки должностное лицо, проводившее ее, докладывает своему непосредственному начальнику.