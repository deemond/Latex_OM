8.9. Выполнение полета
8.9.1. Буксировка воздушного судна 
8.9.1.1. Буксировка на место запуска производится: 
а)	по разрешению органа ОВД и руководителя буксировкой; 
б)	в соответствии с установленной на данном аэродроме схемой наземного движения; 
в)	при наличии непрерывной двухсторонней связи между руководящим буксировкой лицом и экипажем воздушного судна по переговорному устройству, по радио или визуально с помощью установленных сигналов воздушного судна и с органом ОВД. 
8.9.1.2. Пересечение и занятие ВПП или РД при буксировке ВС производится по разрешению органа ОВД. 
8.9.1.3. При пересечении, занятии ВПП или РД летный экипаж и/или лица, осуществляющие буксировку: 
а)	соблюдают визуальную и радиоосмотрительность; 
б)	докладывают органу ОВД об освобождении ВПП или РД. 
Безопасность буксировки обеспечивается лицом, руководящим буксировкой. 
8.9.1.4. Буксировка воздушного судна выполняется с включенными аэронавигационными огнями и проблесковыми маяками (если предусмотрено РЛЭ (AFM, FCOM) типа ВС). 
Примечание: Положение ног пилотов на органах управления в процессе руления должно исключить возможность непреднамеренного торможения ВС.
8.9.2. Запуск двигателя (двигателей) воздушных судов
8.9.2.1. Запуск двигателей, прогрев на повышенных оборотах, опробование двигателей ВС выполняется по разрешению органа ОВД с докладом получения последней информации ATIS и производится под контролем технического персонала: 
а)	на стоянке (в том числе у аэровокзала); 
б)	на участках РД, отведенных для этой цели; 
в)	на специально оборудованной площадке, определенной в установленном порядке для данного аэродрома; 
г)	в процессе буксировки, если это не противоречит требованиям инструкции по эксплуатации ВС (РЛЭ) и аэронавигационного паспорта или инструкции по буксировке воздушных судов аэродрома. 
8.9.2.2. Перед началом запуска двигателя (двигателей) на воздушном судне: 
а)	удостовериться в безопасности людей и отсутствии посторонних предметов, которые могут быть повреждены или представлять опасность при запуске. При невозможности лично убедиться в безопасности запуска, получить необходимую информацию от лица, руководящего с земли запуском двигателей по переговорному устройству, по радио или визуально с помощью установленных сигналов;
б)	технологические процедуры согласно РЛЭ должны быть выполнены с обязательной проверкой по карте контрольных проверок; 
в)	включить проблесковые маяки (если предусмотрено РЛЭ типа ВС).
8.9.2.3. При запуске двигателей воздушного судна летным экипажем ВС поддерживается двухсторонняя связь (проводная, радио, визуальные сигналы) с лицом наземного персонала, обеспечивающим выпуск воздушного судна.
8.9.2.4. Запрос члена летного экипажа на запуск двигателя на контролируемом аэродроме или запуск двигателя на неконтролируемом аэродроме с целью производства полета свидетельствует о принятии решения КВС о начале полета.
8.9.3. Руление
8.9.3.1. Перед началом руления экипаж должен ознакомиться со схемой руления на аэродроме, маркировкой маршрута руления, должны быть выполнены все технологические процедуры согласно РЛЭ ВС (AFM, FCOM) с проверкой по карте контрольных проверок. 
8.9.3.2. На контролируемом аэродроме руление выполняется пилотом после получения от органа ОВД соответствующего разрешения на руление и информации о схеме руления по аэродрому. Орган ОВД, управляющий движением воздушного судна по аэродрому, информирует экипажи воздушных судов о взаимном расположении воздушных судов, в том числе и следующих по одному маршруту при рулении в условиях видимости менее 400 м. Пилоту органом ОВД может передаваться другая информация, необходимая для обеспечения безопасности руления или буксировки.
8.9.3.3. Осуществляя маневрирование на земле перед взлетом ВС и после посадки во избежание столкновения с другими ВС или препятствиями на земле, КВС и члены экипажа должны использовать все имеющиеся ресурсы для понимания в любой момент времени места нахождения ВС на рабочей площади аэродрома. 
Такими ресурсами могут быть: 
а)	различные аэродромные знаки, курсовые указатели, указатели РД и ВПП; 
б)	маркировки поверхности аэродрома; 
в)	светосигнальное оборудование маршрутов руления; 
г)	освещение отдельных зон территории аэродрома. 
При подготовке к полету экипаж должен в достаточной степени изучить схемы руления на аэродромах вылета, назначения, запасных, предварительно наметить возможные варианты маршрута руления.
При получении разрешения (указания) по порядку руления, если маршрут руления достаточно сложен, произвести запись основных пунктов информации во избежание ошибок при рулении. 
В процессе руления отслеживать прохождение маршрута руления, контролировать его соответствие полученному указанию от диспетчера руления. При возникновении каких-либо затруднений при рулении запрашивать уточняющую информацию у диспетчера. 
Руление ВС осуществлять с включенными проблесковыми маяками, рулежным светом фар, ночью с включенными аэронавигационными огнями (АНО). 
Для обеспечения руления в условиях пониженной видимости, особенно ночью, ориентироваться по огням светооборудования РД, ВПП, осевым огням РД, ВПП. При затруднениях в правильности выполнения маршрута руления запросить машину сопровождения. 
Пересечения ВПП, занятие ВПП для взлета производить только по разрешению диспетчера. При занятии ВПП убедиться визуально, прослушиванием радиообмена, путем анализа информации БСПС (TCAS) в отсутствии какого-либо ВС на предпосадочной прямой. 
Взлет производить с включенными на максимальный режим взлетно-посадочными фарами, проблесковыми маяками, ночью – с включенными АНО. 
После посадки и освобождения ВПП запросить указания по рулению. 
На неконтролируемых аэродромах и площадках перед началом руления воздушного судна КВС осуществляет осмотр летного поля и выбирает маршрут буксировки, руления. 
8.9.3.4. Выруливание с места запуска двигателя (двигателей) выполняется с разрешения: 
а)	диспетчера ОВД; 
б)	ответственного лица, обеспечивающего выпуск воздушного судна, а при его отсутствии по решению командира ВС.
В начале руления экипаж воздушного судна проверяет работоспособность тормозной системы. 
8.9.3.5. Экипажу воздушного судна запрещается начинать и продолжать руление, если: 
а)	давление в тормозных системах не соответствует эксплуатационным ограничениям или имеются другие признаки неисправности тормозов; 
б)	на контролируемом аэродроме не получено разрешение органа ОВД или органа управления движением на перроне; 
в)	безопасность руления не обеспечивается из-за наличия препятствий, неудовлетворительного состояния места стоянки или рулежных дорожек. 
г)	отсутствует автомашина сопровождения, если сопровождение при рулении обязательно или выполняется по запросу экипажа. 
8.9.3.6. Руление производится на минимально возможной тяге двигателей для уменьшения воздействия шума и реактивных струй двигателей. 
Руление выполняется по маршруту, указанному диспетчером ОВД. В процессе руления наличие непрерывной двухсторонней связи с органом ОВД обязательно. О невозможности выполнить заданный маневр командир ВС должен доложить органу ОВД. 
Пересечение критических зон посадочных маяков либо ВПП производится по отдельному разрешению органа ОВД. После пересечения (освобождения) ВПП по маршруту руления командир ВС обязан сообщить об этом органу ОВД.
Независимо от полученного указания органа ОВД, перед пересечением, занятием ВПП или рулежной дорожки летный экипаж воздушного судна и (или) лица, осуществляющие буксировку воздушного судна, обязаны убедиться в безопасности маневра.
Если на аэродроме предусмотрена система управления рулением ВС с применением огней на РД, экипаж обязан строго выдерживать заданный огнями маршрут и быть готовым прекратить руление при загорании сигналов остановки. 
8.9.3.7. Скорость руления
Выбирается командиром ВС в зависимости от: 
а)	ограничений РЛЭ воздушного судна; 
б)	состояния перрона, РД и ВПП, по которым проходит маршрут руления; 
в)	наличия препятствий по маршруту руления; 
г)	видимости и степени освещенности на маршруте руления; 
д)	других условий, по усмотрению КВС, определяющих безопасность руления. 
Не рекомендуется руление на скорости более 40 км/ч (20 КТ) на прямых участках и 10 км/час (5 КТ) на разворотах, на скользких поверхностях – более 5-10 км/час (3-6 КТ). 
При наличии в аэропорту системы позиционирования на места стоянок (SAFEDOCK) скорость заруливания на стоянку ограничивается согласно документам АНИ. 
В нормальных условиях не рекомендуется применять раздельное торможение и несимметричную тягу двигателей на разворотах. 
Скорость руления в любом случае должна обеспечивать безопасную остановку ВС. 
Командир ВС несет ответственность за обоснованность выбора скорости руления. 
При рулении необходимо следить, чтобы воздушные суда и обслуживающий персонал на перроне не подвергались опасности воздействия реактивной струи или воздушного потока от воздушного винта. 

8.9.3.8.	Предотвращение столкновений
Пилоты при рулении должны вести постоянную визуальную и радио осмотрительность. Остальные члены летного экипажа при рулении также должны вести осмотрительность, если это не затрудняет выполнение их основных обязанностей, и предупреждать КВС о препятствиях по маршруту руления.
При рулении воздушных судов навстречу друг другу их КВС обязаны уменьшить скорость руления до безопасной и, держась правой стороны, разойтись левыми бортами. 
При сближении воздушного судна на пересекающихся направлениях КВС обязан пропустить воздушное судно, двигающееся справа. 
Запрещено обгонять рулящее воздушное судно. 
При рулении под контролем органа ОВД порядок взаимного расхождения воздушных судов на пересекающихся маршрутах определяет орган ОВД. 
Командир ВС несет ответственность за соблюдение правил руления. 
При обнаружении препятствия на маршруте руления командир ВС: 
а)	принимает меры по предотвращению столкновения, вплоть до полной остановки; 
б)	докладывает органу ОВД о наличии препятствия; 
в)	после устранения препятствия продолжает руление с разрешения органа ОВД. 
8.9.3.9.	Применение внешнего светового оборудования ВС 
Руление ночью, а также днем при видимости менее 2000 м осуществляется с включенными аэронавигационными огнями и фарами.
Проблесковые маяки днем и ночью должны быть включены с момента запуска двигателей до момента их останова. 
8.9.3.10.	Сопровождение ВС на рулении
Воздушные суда при рулении в обязательном порядке сопровождаются радиофицированной автомашиной сопровождения: 
а)	днем и ночью – при метеорологической видимости менее 400 м; 
б)	ночью – воздушные суда 1 и 2 класса независимо от метеоусловий; 
в)	если маркировочная разметка рулежных дорожек по маршруту руления или места стоянки хотя бы частично не просматривается из-за наличия снега, льда или по другим причинам; 
г)	по запросу экипажа независимо от времени суток, метеоусловий и класса воздушного судна; 
д)	в зарубежных аэропортах - согласно инструкции данного аэродрома. 
При лидировании воздушных судов безопасную дистанцию между автомашиной сопровождения и лидируемым воздушным судном поддерживает командир ВС. 
Ответственность за предотвращение столкновений ВС с препятствиями при лидировании ВС машиной сопровождения возлагается на командира ВС. 
Заруливание на место стоянки выполняется по:
а)	указаниям диспетчера ОВД;
б)	маркировочной разметке места стоянки;
в)	сигналам и указаниям дежурного по сопровождению;
г)	сигналам встречающего лица ИАС, при его отсутствии по решению командира ВС;
д)	сигналам системы позиционирования и остановки ВС.
При установке воздушного судна не по маркировочной разметке места сто¬янки экипаж воздушного судна:
а)	информирует об этом диспетчера ОВД;
б)	по согласованию с диспетчером ОВД повторяет маневр заруливания на сто¬янку;
в)	вызывает буксир для постановки воздушного судна на место стоянки.
8.9.3.11.	Перегрев тормозов
Во избежание перегрева тормозов после интенсивного использования при посадке, экипаж ВС может запросить диспетчера органа ОВД о необходимости сокращения маршрута руления, если есть такая возможность. Экипаж ВС учитывает возможный перегрев тормозных устройств, колес шасси в процессе руления после посадки, при необходимости прекращает руление, запрашивает облив тормозов водой согласно РЛЭ ВС. 
На стоянке, после установки упорных колодок, по согласованию с наземным персоналом, экипаж ВС выключает стояночный тормоз для ускоренного охлаждения тормозов.
8.9.3.12.	Выключение двигателей
Производится после заруливания на стоянку, с соблюдением технологических процедур согласно РЛЭ воздушного судна. 
Перед выключением двигателей устанавливается стояночный тормоз и проверяются параметры электросети от наземного источника или ВСУ. Момент выключения двигателей определяет командир ВС. 
В случае если от наземного персонала поступила информация о нештатной ситуации, командир ВС принимает меры к экстренному выключению двигателей. 

8.9.3.13.	Использование стояночного тормоза 
Командир ВС должен использовать во всех случаях, указанных в РЛЭ воздушного судна. Установка (снятие) стояночного тормоза согласовывается с наземным персоналом. 
Особое внимание уделяется установке стояночного тормоза перед запуском и выключением двигателей для предотвращения неконтролируемых перемещений ВС. Порядок использования стояночного тормоза при запуске двигателя в процессе буксировки координируется с наземным персоналом. 
Снятие со стояночного тормоза производится только после выключения двигателей и получения информации от наземного персонала об установке упорных колодок. 
При сильном ветре или неудовлетворительном состоянии поверхности стоянки (снег, лед, значительный уклон и т.п.) стояночный тормоз оставляется включенным, а при необходимости командир ВС требует установку дополнительных упорных колодок под все колеса. 
 8.9.3.16.	Световые сигналы, подаваемые с автомашины сопровождения
Рулите за мной – зеленый свет.
Прекратить руление – красный свет.
8.9.4.	Взлет
8.9.4.1. На контролируемых аэродромах взлет осуществляются по разрешению ор¬гана ОВД. Экипаж должен подтвердить разрешение на взлёт. В случае ожидания взлёта на ВПП, но неполучения разрешения в установленный период времени экипаж должен повторно запросить разрешение на взлёт.
8.9.4.2. На аэродроме и посадочных площадках, где нет органов ОВД, взлет выпол¬няется по решению командира ВС. Место начала взлета и его направление опреде¬ляет командир ВС. Информацию о времени, месте 
и направлении взлета командир ВС пере¬дает на частоте органа ОВД, в районе ответственности которого находится ВС. В случаях необходимости длительного занятия ВПП (более 1 мин.), экипаж ВС до ее занятия необходимом времени для подготовки к взлету. Если после выдачи разрешения на взлет прошло более 1 минуты, то экипаж воздушного судна обязан запросить повторное разрешение на взлет.
8.9.4.3. Перед взлетом: 
а)	летный экипаж воздушного судна проверяет установку высотомеров; 
б)	КВС убеждается в готовности воздушного судна и членов экипажа воздушного судна к взлету; 
в)	КВС убеждается в отсутствии наблюдаемых препятствий впереди на ВПП и по траектории взлета; 
г)	КВС убеждается в соответствии фактической погоды минимуму для взлета и фактической погоды, состояния ВПП ограничениям летно-технических характеристик ВС с учетом фактической погоды; 
д)	КВС убеждается в отсутствии по траектории полета зон опасных метеорологических явлений; 
е)	на контролируемом аэродроме КВС получает разрешение на взлет от органов ОВД. 
Запрещается выполнение взлета при наличии информации о сильном дожде, с интенсивностью, ухудшающей метеорологическую видимость менее 600 м., без использования бортового радиолокатора и системы заблаговременного предупреждения о сдвиге ветра. 
При наличии информации о видимости в трех частях ВПП видимость на ВПП (далее - RVR) оценивается командиром ВС в начале разбега, в средней точке и в конце ВПП - по сообщенной органом ОВД или АТИС). 
8.9.4.4. При выполнении взлета располагаемая длина разбега и взлетная дистанция от места начала разбега должны соответствовать потребной дистанции продолженного взлета и длине разбега для фактической взлетной массы воздушного судна и условий взлета. 
Взлет воздушных судов производится, как правило, от начала ВПП. 
Разрешается выполнять взлет не от начала ВПП при условии, если:
а)	это предусмотрено инструкцией по производству полетов на данном аэро¬дроме (аэронавигационным паспортом аэродрома);
б)	располагаемые характеристики летной полосы от места начала разбега соот¬ветствуют потребным для фактической взлетной массы воздушного судна и условиям взлета. 
На неконтролируемых аэродромах место начала взлета и его направление определяет КВС. На неконтролируемых аэродромах перед взлетом КВС обязан передать на частоте органа ОВД, в районе ответственности которого он находится, место и магнитный курс взлета. 
Взлет с кратковременной остановкой на ВПП рекомендуется выполнять на мокрых, обледенелых, заснеженных и покрытых слякотью ВПП, а также в условиях интенсивности воздушного движения на аэродроме по согласованию с органом ОВД. 
Взлет без остановки на ВПП (немедленный взлет) выполняется при дальности видимости на ВПП (RVR) не менее 400м. При этом значения РВД, РДПВ и РДР для расчета взлетных характеристик должны быть уменьшены на 150 м. 
Если к моменту достижения контрольной скорости двигатели не вышли на режим взлетной тяги (не поступил доклад «Режим взлетный»), взлет должен быть прекращен в соответствии с РЛЭ ВС (AFM, FCOM). 
Если фактическая взлетная масса самолета ограничена по условиям располагаемой длины ВПП, выполняется нормальный взлет с выводом двигателей на взлетную тягу до страгивания ВС на исполнительном старте. В этом случае применение взлетного режима обязательно.
Первый разворот после взлета (отворот с курса взлета) выполняется на удалении и высоте, установленной схемой выхода из района аэродрома (SID).
Экипажу воздушного судна с момента начала разбега воздушного судна и до набора высоты 200 метров запрещено вести радиосвязь, а органу ОВД вызывать экипаж воздушного судна, за исключением случаев, когда это необходимо для обеспечения безопасности. 
8.9.4.5.  Экипажу взлетать запрещается, если:
а)	экипаж получил информацию, что взлет создаст помеху воздушному судну, которое выполняет прерванный заход на посадку (уход на второй круг); 
б)	впереди на ВПП (летной полосе) имеются препятствия; 
в)	по курсу взлета имеются опасные метеоявления, скопления птиц, угрожающие безопасности взлета; 
г)	взлетная масса ВС превышает максимальное значение, предусмотренное РЛЭ или эквивалентным ему документом (рассчитанное в соответствии с РЛЭ) для фактических располагаемых дистанций, метеорологических условий и требуемого градиента набора на начальном участке схемы вылета; 
д)	при полетах ночью отсутствует действующее светосигнальное оборудование; 
ж)  фактическая погода ниже установленного для взлета эксплуатационного минимума; 
и)  имеется информация о сильном дожде с интенсивностью, ухудшающей метеорологическую видимость до значения менее 600 м., без использования, при этом, бортового радиолокатора и системы заблаговременного предупреждения о сдвиге ветра; 
к)  скорость ветра у земли с учетом его направления и порывов, а также состояние поверхности ВПП и значение коэффициента сцепления не соответствует установленным РЛЭ данного типа ВС ограничениям; 
л) поверхность ВС покрыта льдом, инеем, мокрым снегом или истек срок действия противообледенительной обработки. 
8.9.4.6. Взлет выполняется КВС или, по указанию КВС, вторым пилотом.
Выполнение взлета вторым пилотом в целях тренировки разрешается:
а)	под контролем пилота – инструктора при соблюдении следующих условий:
	боковая составляющая ветра не более 80% от предельно-допустимой для фактического состояния поверхности ВПП;
	при видимости (видимости на ВПП) не менее 400 м;
б)	под контролем командира ВС:
	при боковой составляющей ветра не более 50% от предельно-допустимой для фактического состояния поверхности ВПП;
	при видимости (видимости на ВПП) не менее 400 м.
8.9.4.7. Взлет воздушного судна производится с включенными фарами, в зависимости от внешних условий и в соответствии с РЛЭ ВС по решению КВС, и до высоты не менее 50 м. 
8.19.4.8. Разрешается взлет при попутном ветре, если это предусмотрено инструк¬цией по производству полетов на данном аэродроме (аэронавигационным паспортом аэродрома) при величине попутной составляющей ветра, не превышающей требова¬ния руководства по летной эксплуатации воздушного судна данного типа.
8.9.4.9.	Взлет в условиях ограниченной видимости (LVTO)
Взлет в условиях ограниченной видимости (LVTO) – взлет при значениях RVR менее 400 м. (на аэродромах Российской Федерации при значениях RVR менее 550 м).
Взлет выполняется командиром ВС при соблюдении следующих условий:
а)	минимум для взлета 200м. (днем и ночью) применяется, если ВПП имеет огни осевой линии;
б)	без огней осевой линии с маркировкой осевой линии (RCLM), включены и работают огни высокой или средней интенсивности (HIRL, MIRL) –минимум для взлета составляет: днем – 300м., ночью – 400м. 
в)	при отсутствии огней ВПП (HIRL, MIRL) или их отказе в работе величина видимости на ВПП применяется не менее:
	для категории ВС А и В – 300 м днем и ночью;
	для категории ВС С – 500 м днем и 700 ночью. 
В этих условиях взлет должен обеспечиваться другой маркировкой или огнями ВПП, достаточными для визуальной ориентировки, беспрерывного наблюдения поверхности ВПП и выдерживания направления в процессе всего разбега ВС при взлете.
г)	Если процедуры при низкой видимости (LVР) не действуют, минимум для взлета всех категорий ВС устанавливается 400 м днем – при наличии маркировки осевой линии ВПП, ночью – при наличии маркировки осевой линии ВПП и работе огней ВПП. Если нет маркировки центральной линии ВПП и не работают огни ВПП, минимум устанавливается не менее 400 м для категории ВС А и В, 700 м для категорий С и только днем.
д)	Минимум для взлета 200 м применяется при коэффициенте сцепления на ВПП не менее 0,5 и боковой составляющей скорости ветра не более половины предельного допустимого значения для взлета данного типа.
е)	Решение на вылет по минимумам, указанным в графах 2,3 и 4, таблицы СП1-Т1 Приложения СП1 принимается по наименьшему значению видимости на ВПП, измеренной вдоль ВПП дистанционными измерителями (регистраторами) дальности видимости. При неработающих дистанционных измерителях или их отсутствии на аэродроме следует руководствоваться значениями видимости, указанными в графах 5 и 6 Таблицы СП1-Т1.
ж)	В тех случаях, когда днем не обеспечено обозначение осевой линии маркировочными знаками по всей длине ВПП, руководствоваться значениями, указанными в графах 4 и 6 Таблицы СП1-Т1 Приложения СП1.
з)	Минимумы для взлета применяются при наличии запасного аэродрома для взлета. Запасным аэродромом для взлета может быть выбран аэродром, на котором фактическая или прогнозируемая погода соответствует пункту 8.4.2.3, а время полета от аэродрома вылета до запасного аэродрома определяется в соответствии с рекомендациями РЛЭ, но во всех случаях не превышает:
	одного часа полета с одним отказавшим (критическим) двигателем для двухдвигательных самолетов;
Решение на вылет без запасного аэродрома для взлета может быть принято при метеоусловиях на аэродроме вылета выше минимума для посадки на нем, при условии, что нет других причин, препятствующих возврату на аэродром вылета.
и)	Заход на посадку и посадка по приборам по категории II и III не разрешается, если не предоставляется информация о RVR. Контрольная RVR определяется по сообщенным значениям RVR в одной или нескольких точках наблюдения за RVR (точка приземления, средняя точка и дальний конец ВПП), используемые в целях определения соблюдения установленных эксплуатационных минимумов. В случае, если используется информация о RVR в разных точках, контрольная RVR представляет собой RVR в точке приземления, при этом RVR в средней точке и в дальнем конце ВПП не менее RVR установленного минимума для взлета.
к)	Значение RVR указывается только тогда, когда это значение не совпадает со значением VIS. В этом случае перед цифровым значением минимума указывается аббревиатура RVR.
8.9.4.10.	Действия экипажа при прерванном и продолженном взлете 
Решение о прекращении или продолжении взлета принимает командир ВС. 
Скорость принятия решения V1 – наибольшая скорость разбега, при которой в случае отказа критического двигателя возможно, как безопасное прекращение, так и безопасное продолжение взлета. 
Расчет скорости принятия решения V1, скорости подъема передней ноги VR и безопасной скорости на взлете V2 производится (контролируется расчет) экипажем ВС перед каждым взлетом согласно РЛЭ воздушного судна (AFM, FCOM) с учетом конкретных условий предстоящего взлета и состояния ВПП.
(1) Прерванный взлет
Решение на прекращение взлета и выполнение всех операций по его прекращению (подача соответствующих команд на использование необходимых для торможения систем ВС) является прерогативой командира ВС в любом случае, независимо от того, кто пилотирует воздушное судно. 
Командир ВС (или по его команде бортмеханик) в процессе разбега держит руку на рычагах управления двигателями до скорости V1, а при принятии решения на прекращение взлета дает команду «РУД О» и предпринимает все необходимые действия по остановке воздушного судна в соответствии с РЛЭ. 
Решение на прекращение взлета может быть принято на усмотрение командира ВС при отказе двигателя или при появлении других неисправностей, угрожающих безопасности полета, если не достигнута скорость принятия решения на продолжение взлёта или если ВС отклонилось от заданного направления настолько, что продолжение разбега не обеспечивает безопасности. Запрещается отрыв воздушного судна от земли на скорости, менее установленной РЛЭ.
В случае прекращения взлёта по причине отказа или неисправности воздушного судна повторный взлет запрещается до выяснения и устранения причин, вызвавших прекращение взлета. 
Если прекращение взлёта не связано с отказом или неисправностью воздушного судна, решение о выполнении повторного взлёта может быть принято КВС, после проведения работ, если они предусмотрены в эксплуатационной документации воздушного судна.
(2) Продолженный взлет
При любых отказах на взлете на скорости выше скорости принятия решения V1 взлет должен быть продолжен. 
Никакие действия не предпринимаются до надежной стабилизации полета, кроме отключения звуковой сигнализации, если она мешает нормальному взаимодействию, до тех пор, пока: 
а)	не закончены нормальные процедуры;
б)	не закончены действия, предусмотренные РЛЭ типа ВС (AFM, FCOM); 
в)	не достигнута высота 400 ft (120 м) AGL в случаях отказа в процессе взлета;
г)	не достигнута высота 1200 ft (400 м) AGL в случае отказа двигателя в процессе взлета на Ан-74;
Ниже высоты 400 ft (120 м) AGL допускается только: 
д)	увеличение тяги двигателей; 
е)	уборка/выпуск шасси, если это не приведет к опасному нарушению балансировки воздушного судна.
Примечание. Для Ан-74ниже высоты 700 ft (200 м) (при отказе двигателя ниже 1200 ft (400 м)) AGL.
8.9.4.11.	Процедуры маневрирования при отказе двигателя на взлете на скорости более V1
При отказе двигателя на взлете после скорости V1 существует несколько видов процедур (ENGINE FAILURE PROCEDURE (EFP)), опубликованных в программе расчета ВПХ, для обеспечения безопасного и эффективного выполнения полета, которые должны быть выполнены экипажем:
а)	ENGINE OUT SID (SPECIAL ENGINE FAILURE PROCEDURE); 
б)	STANDARТ ENGINE OUT PROCEDURE; 
в)	STANDART INSTRUMENT DEPARTURE; 
г)	IMMEDIATE VISUAL RETURN 
а) ENGINE OUT SID (SPECIAL ENGINE FAILURE PROCEDURE) выполняется, если опубликована, при отказе двигателя до пролета точки отворота на начальный пункт схемы ENGINE OUT SID. 
Полет выполняется в соответствии с опубликованной процедурой ENGINE OUT SID. На engine out acceleration altitude (height) переведите ВС в горизонтальный полет для полной уборки механизации крыла в соответствии с FCOM. После уборки механизации крыла продолжите набор требуемой высоты или MSA. Выполните процедуры согласно QRH и информируйте ОВД о принятом решении. 
Примечание. При отказе двигателя после пролета точки отворота на ЕО SID выполняюm предписанный SID. 
б) STANDARТ ENGINE OUT PROCEDURE (выполняется при отсутствии препятствий по курсу взлета по решению КВС, при отсутствии ENGINE OUT SID). 
На курсе взлета набирайте 800 ft над уровнем аэродрома, информируйте ОВД о принятом решении. Переведите ВС в горизонтальный полет для полной уборки механизации крыла в соответствии с FCOM во время разворота на выбранную точку. После уборки механизации крыла продолжите набор требуемой высоты или MSA. Выполните процедуры согласно QRH. При необходимости запросите «векторение» для захода на посадку или полета в зону ожидания. 
в) STANDART INSTRUMENT DEPARTURE (выполняется по решению КВС при отсутствии значительных разворотов после взлета). 
Выполняйте полет в соответствии с опубликованным маршрутом выхода (SID). Переведите ВС в горизонтальный полет на engine out acceleration altitude (height) для полной уборки механизации крыла в соответствии с FCOM. После уборки механизации продолжайте набор требуемой высоты или MSA. Выполните процедуры согласно QRH и информируйте ОВД. 
г) IMMEDIATE VISUAL RETURN (в визуальных условиях при наличии негасимого пожара или бомбы на ВС). 
Если требуется выполнить немедленный возврат на аэродром вылета при полете в визуальных условиях, выполнить процедуры ENGINE FAILURE AFTER V1» до достижения acceleration altitude (height). Перевести ВС в ГП и сохранять механизацию крыла в положении FLAPS 1 до выполнения третьего разворота (начала доворота на посадочный курс).
8.9.4.12.	Использование минимальных безопасных высот в аварийных случаях после взлета
При аварийной обстановке после взлета при маневрировании ВС в пределах зоны ВЗП могут использоваться высоты Нмс (МВС) для визуального захода на посадку соответствующей категории ВС. При этом запас высоты над препятствиями составит:
а)	для ВС категории A и B - 90 м;
б)	для ВС категории С и D-120 м;
При маневрировании ВС за пределами зоны ВЗП, но в пределах 50 км от аэродрома минимальной используемой высотой является высота БВП для горных аэродромов и БВП, уменьшенная на 100 м, для остальных аэродромов. 
Примечание. Под термином «маневрирование» понимается выполнение полета в зоне ВЗП с учетом ограничений данного аэродрома, либо выполнение полета по установленной схеме выхода или входа при принятии решения о возвращения на аэродром вылета.
8.9.4.13.	Маневрирование после взлета в случаях негасимого пожара или бомбы на борту ВС (IMMEDIATE VISUAL RETURN)
При выполнении аварийной посадки после взлета, при наличии негасимого пожара или бомбы на борту ВС, экипажу ВС предоставляется право, исходя из конкретных условий полета, использовать, с учетом положений п. 8.9.4.12 следующие варианты схем:




Вариант 1.

                                                 R взп
	
                                                                                   
                                                                  ВПП          S1
                                       

                                                                                              S2
                                                                  



Рис.1


При принятии решения о выполнении экстренной посадки после взлета и до достижения удаления S1 выполняется разворот на 80º, а затем разворот на 260º в противоположном направлении с выходом на посадочную прямую с курсом, обратным посадочному, на удалении S2 (Таблица А-8.9-Т1).








Вариант 2.                     
                                                    R взп
                                                                                  S1
                                                               

                                                                ВПП
                                                      
                                                                                
                                                                                       2R
                                                                                     
                                                                                               S 2
                                                                                      


Рис.2
При выполнении аварийной посадки на участке от удаления 2R до S2 выполняется разворот на 260º, а затем на 80º в противоположном направлении с выходом на посадочную прямую с курсом обратным посадочному на удалении S1 (Таблица А-8.9-Т1).
Вариант 3 




                      


                                                               ВПП

                                                                                                             
                                                                                                                                                                     
                             R взп
                                                                                  S 2
                                                                                      
                                            
Рис.3
При принятии решения о выполнении аварийной посадки на ВПП с рабочим курсом до удаления S2 выполняется разворот на 180º с дальнейшим выполнением визуального захода на посадку (Табл. А8.9-Т1).
Примечания:
а)	При выполнении аварийной посадки выполнение первого разворота на удалениях более S1 для варианта №1 и S2 для варианта №2 приведет к выходу за пределы зоны ВЗП соответствующей категории ВС. В этом случае необходимо использовать НМС ВЗП для более высокой категории ВС.
б)	Использование варианта №2 на удалении менее 2R не позволит выйти на посадочную прямую до порога ВПП.                                                                                                                                     Таблица А-8.9 -Т1
Катего
рия
ВС	Скорость
выполнения
маневра
(км / час / узлы)	R
При крене
15º/30º
(км)	S1
(км)	S2
(км)	2R
(км)	R
ВЗП
(км)	Запас высоты
над
препятствиями
(км)	НМС ВЗП
(минимальная)
(м)		
										
										
										
										
		1,3	1.0	3,6	2,6					
В	210/114	0,6	3.1	4.3	1.2	4,90	90	150		
		1,6	0,1	3,3	3.2					
	230/124	0,7	2.8	4,2	1.4					
		1, 8	2,4	6,0	3,6					
С	250/135	0,8	5,4	7,0	1,6	7,85	120	180		
		2,2	1,2	5,6	4,4					
	270/146	1,0	4,8	6,8	2.0					
		2,5	2,3	7,3	5,0					
	290/157	1,2	6,2	8,6	2,4					
		2,8	1,1	6,7	5,6					
D	310/167	1,3	1,9	8,5	2,6	9,79	120	210		
		3,2	0,2	6,6	6,4					
	330/178	1,5	5,3	8,3	3,0					
		3,6*	-1,0	6,2	7,2					
	350/189	1,7	4,7	8,1	3,4					
		4,0*	-2,2	5,8	8,0					
	370/200	1,9	4,1	7,9	3,8					
Примечание: При выполнении маневра при данных скоростях и кренах необходимо использовать данные по выполнению захода с круга для ВС категории Е, при этом:
Rвзп = 12,82 км, запас над препятствиями = 150 м, Нмс (миним) = 240 м.
8.9.5.  Набор высоты
8.9.5.1. Набор высоты после взлета производится с курсом взлета до высоты над аэродромом не менее: 
а)	50 м на воздушном судне при выполнении авиационных работ, если работы выполняются на высоте 50 м и менее; 
б)	установленной схемой выхода из района аэродрома, но не менее высоты установленной схемой вылета или РЛЭ типа ВС.
Выход воздушного судна из района контролируемого аэродрома осуществляется по установленной схеме или по указаниям органа ОВД. 
При наличии нескольких опубликованных схем выхода орган ОВД заблаговременно информирует экипаж воздушного судна о схеме выхода и первоначально заданной высоте, если она не установлена в аэронавигационной информации.
Дальнейший набор согласованной с органом ОВД высоты производится при полете по установленной схеме выхода из района аэродрома.
8.9.5.2. С момента начала разбега воздушного судна и до набора высоты 200 м экипажу и диспетчеру пункта ОВД не допускается вступать в радиосвязь, за исключением случаев, когда возникает угроза безопасности полета.
8.9.5.3. Разрешение экипажу ВС на выполнение взлета является одновременно разрешением для перехода на связь с диспетчером круга на высоте 200 метров (безопасной или заданной). До набора этой высоты экипаж ВС обязан прослушивать частоту диспетчера старта. Если после выдачи разрешения на взлет прошло более одной минуты, то экипаж ВС обязан запросить повторное разрешение на взлет. 
С момента начала разбега воздушного судна и до набора высоты 200 м экипажу и диспетчеру пункта ОВД не допускается вступать в радиосвязь, за исключением случаев, когда возникает угроза безопасности полета.
8.9.5.4. При взлете на контролируемом аэродроме и получении разрешения бессту¬пенчатого набора заданного эшелона полета доклад диспетчеру ОВД о взлете может не производиться. Экипаж ВС обязан прослушивать частоту ДПК до пересечения заданного эшелона (рубежа передачи ОВД диспетчеру ДПП).
После взлета на контролируемом аэродроме и при невозможности занятия заданного эшелона (высоты) полета к установленному или заданному органом ОВД рубежу, командир ВС информирует об этом соответствующий орган ОВД.
8.9.5.5. В процессе выполнения набора высоты члены летного экипажа должны выполнять все процедуры, предусмотренные РЛЭ ВС (AFM, FCOM), технологией работы и настоящим РПП. 
8.9.5.6. Выполнение установленных стандартных процедур выхода (SID) обязательно. Указания диспетчера органа ОВД выполняются экипажем ВС в том случае, если они не противоречат ограничениям РЛЭ воздушного судна (AFM, FCOM). 
В наборе высоты все члены летного экипажа должны соблюдать визуальную и радиоосмотрительность. 
8.9.5.7. При пересечении высоты перехода в наборе экипажем ВС по команде командира производится установка стандартного давления на высотомерах в соответствии со стандартными операционными процедурами. 
8.9.5.8. При пересечении эшелона FL100 (3000м) экипажем ВС производится контроль работы высотной системы воздушного судна в соответствии со стандартными операционными процедурами. 
8.9.5.9. В наборе высоты, во избежание срабатывания бортовой системы предупреждения столкновений (далее - БСПС (TCAS)), за 1000 ft (300м) до заданного эшелона (высоты) полета экипажем ВС устанавливается вертикальная скорость набора не более 1379 ft в минуту (7м/сек). 
Командир ВС должен контролировать и корректировать вертикальную скорость для поддержания заданного градиента набора высоты.
8.9.5.10. Если воздушное судно не может занять заданный органом ОВД эшелон (высоту) к установленному или заданному органом ОВД месту, экипаж воздушного судна обязан своевременно проинформировать об этом орган ОВД. 
8.9.5.11. В процессе набора высоты при получении команды диспетчера ОВД на занятие (сохранение) высоты (эшелона) полета экипаж воздушного судна подтверждает получение команды на занятие (сохранение) заданной высоты и контролирует правильность установки задатчика высоты.
8.9.5.12. По окончании набора заданного эшелона летный экипаж воздушного судна должен сличить показания барометрических высотомеров.
8.9.5.13. Для выполнения полетов в воздушном пространстве, в котором установлен порядок измерения высот полета в футах, ВС должно быть оборудовано соответствующими высотомерами (футомерами).
8.9.6.	Полеты по воздушным трассам и маршрутам
8.9.6.1. При выполнении полета по маршруту в контролируемом воздушном пространстве выдерживаются требования по точности аэронавигации, установленные для данного района полета и заданные органом ОВД высоты (эшелоны) полета.
Контроль курса, других навигационных элементов полета и ветра выполняется с периодичностью, позволяющей исключить отклонение ВС от заданной траектории на величину, превышающую допустимое значение для данного района полетов. 
В случае непреднамеренных отклонений от текущего плана полета, экипажем воздушного судна предпринимаются следующие действия: 
а)	если воздушное судно отклонилось от линии пути, экипаж воздушного судна корректирует курс воздушного судна в целях быстрейшего возвращения на линию заданного пути; 
б)	если среднее значение истинной воздушной скорости на крейсерском эшелоне между двумя контрольными пунктами не является неизменным или ожидается, что оно изменится на плюс-минус 5% от истинной воздушной скорости, указанной в плане полета, экипаж информирует об этом орган ОВД; 
в)	если обнаружится, что уточненный расчет времени пролета очередного запланированного контрольного пункта отличается более чем на 2 минуты от времени, о котором была уведомлен орган ОВД, экипаж воздушного судна информирует орган ОВД об уточненном времени. 
8.9.6.2. Изменение в полете плана полета в целях изменения маршрута или следования на другой аэродром производится при условии, что, начиная с места, где было произведено изменение маршрута полета, соблюдаются требования по запасу топлива и масла для обеспечения полета (по прибытии на другой аэродром) в течение не менее 30 минут со скоростью полета в зоне ожидания на высоте 450 метров при стандартных температурных условиях. 
8.9.6.3. Полеты по воздушным трассам, МВЛ и маршрутам полета в зависимости от метеорологических условий, типов ВС и их оборудования выполняются по ППП или ПВП на заданных высотах (эшелонах) полета в пределах установленной ширины трассы (МВЛ, маршрута полета).
8.9.6.4. В случае угрозы безопасности полета допускается изменение заданной высоты (эшелона) полета и уклонение от линии заданного пути, при этом экипаж ВС немедленно информирует об этом орган ОВД.
8.9.6.5. Экипаж ВС не позднее, чем за 5 минут до входа в воздушную трассу (МВЛ, маршрут полета) запрашивает разрешение и уточняет условия входа у органа ОВД, осуществляющего обслуживание воздушного движения на воздушной трассе (МВЛ, маршруте полета).
8.9.6.6. Разрешение и условия на вход в воздушную трассу (МВЛ, маршрут полета), при взлете с близко расположенного аэродрома, экипаж ВС запрашивает сразу после взлета.
8.9.6.7. Экипаж ВС не позднее, чем за 5 минут до выхода из воздушной трассы (МВЛ, маршрута полета) получает разрешение и условия выхода от органа ОВД, который будет осуществлять обслуживание воздушного движения после выхода ВС из воздушной трассы (МВЛ, маршрута полета).
8.9.6.8. Вход в воздушную трассу (МВЛ, маршрут полета) и выход из нее (его) производятся в режиме горизонтального полета на предварительно согласованных с органами ОВД эшелонах (высотах).
8.9.6.9. Занятие заданной высоты (эшелона) входа в воздушную трассу (МВЛ, маршрут полета) производится не менее чем за 10 км до ее границы. Изменение высоты (эшелона) полета после выхода из воздушной трассы (МВЛ, маршрута полета) производится на удалении не менее 10 км от границы воздушной трассы или по указанию органа ОВД.
8.9.6.10. Экипаж ВС непрерывно прослушивает канал (частоту) радиосвязи диспетчера того диспетчерского пункта (ДП), на ОВД которого он находится. Переход на радиосвязь с диспетчером другого ДП осуществляется только после получения разрешения на это от диспетчера ДП, в зоне (районе) которого ВС находилось. 
8.9.6.11. При наличии информации о воздушной обстановке на пунктах управления воздушным движением от автоматизированных систем управления воздушным движением или вторичного радиолокатора по указанию органа ОВД, экипаж ВС может быть освобожден от докладов о пролете пунктов обязательного донесения и эшелоне полета. 
8.9.6.12. Запрещено выполнять полет ВС над территориями населенных пунктов и над местами скоплений людей при проведении массовых мероприятий ниже высоты, допускающей, в случае отказа двигателя, аварийную посадку без создания чрезмерной опасности для людей и имущества на земле. 
В случае, когда метеоусловия не позволяют выдерживать установленную высоту полета, обход населенного пункта производится, как правило, с правой стороны, если не установлен иной порядок обхода.
8.9.6.13. Орган ОВД, по запросу экипажа или в случае отклонения ВС от линии заданного пути, при наличии радиолокационного контроля предоставляет экипажу имеющуюся информацию о его местоположении.
8.9.6.14. При возникновении в полёте признаков приближения к зоне опасных метеорологических явлений или получении соответствующей информации КВС обязан принять меры для обхода опасной зоны, если полёт в ожидаемых условиях не разрешён РЛЭ. При невозможности продолжать полёт до пункта назначения из-за опасных метеорологических явлений КВС может произвести посадку на запасном аэродроме или вернуться в пункт вылета. О принятом решении и своих действиях КВС должен, при наличии связи, сообщить органу ОВД, который обязан принять необходимые меры по обеспечению безопасности дальнейшего полёта. 
8.9.6.15. В полете экипаж должен постоянно анализировать аэронавигационную и метеорологическую обстановку по маршруту полета (в районе выполнения авиационных работ), на запасных аэродромах, на аэродроме назначения и запасных аэродромах пункта назначения и вести контроль расхода топлива.
При получении информации об ухудшении метеоусловий ниже установленного минимума или технической неготовности на аэродроме назначения (запасном аэродроме), делающих невозможным совершение безопасной посадки, орган ОВД немедленно сообщает об этом экипажу ВС.
Полет на запасной аэродром может выполняться с оптимальным профилем полета, по кратчайшему расстоянию вне воздушных трасс (по согласованию с органом ОВД). 
На основании анализа аэронавигационной и метеорологической обстановки КВС может выбрать запасной аэродром в полете. 
Полет по ППП продолжается в направлении аэродрома намеченной посадки только в том случае, если самая последняя имеющаяся информация указывает на то, что к расчетному времени прилета посадка на указанном аэродроме или на одном запасном аэродроме пункта назначения может быть выполнена с соблюдением эксплуатационных минимумов для посадки.
При входе в район ОВД, где находится рубеж ухода на запасной аэродром, экипаж ВС обязан информировать орган ОВД о расчетном времени пролета рубежа ухода и выбранном запасном аэродроме. При получении указанной информации, в случае если воздушное судно находится вне зоны вещания автоматизированной системы ВОЛМЕТ, орган ОВД немедленно запрашивает данные о фактической и прогнозируемой погоде, а также подтверждение технической готовности запасного аэродрома и аэродрома назначения к приёму воздушного судна и передаёт эти сведения экипажу воздушного судна;
Решение на продолжение полета до аэродрома назначения с рубежа ухода может быть принято КВС, если последняя информация указывает на то, что: 
а)	прогнозом погоды на аэродроме назначения ко времени прилета предусматриваются метеоусловия, соответствующие требованиям для запасного аэродрома 8.4.2.14; 
б)	есть информация о технической готовности аэродрома назначения к приему ВС.
8.9.6.16. В контролируемом воздушном пространстве при входе в район ОВД на установленном рубеже передачи связи командир ВС сообщает органу ОВД свое местонахождение, высоту полета и получает от него дальнейшие условия для полета.
8.9.6.17. За 30-40 мин. до расчетного времени прибытия экипаж должен установить связь по специальному каналу связи с соответствующей службой аэропорта посадки и (или) авиакомпании и передать информацию об неисправностях или отказах в работе авиатехники выявленных в полете. Передается также информация коммерческого содержания для оптимизации процедур подготовки ВС к дальнейшему полету.
8.9.7.	Снижение, заход на посадку и посадка
8.9.7.1.	Предпосадочная подготовка
Перед снижением для захода на посадку под руководством КВС проводится предпосадочная подго¬товка экипажа и воздушного судна. При продолжительности полета менее 1 чaca часть предпосадочной подготовки по решению командира ВС может быть проведена перед вылетом.
(1) В процессе предпосадочной подготовки экипаж должен:
а)	проверить правильность работы бортовых навигационных средств самолетовождения, включая:
	проверку RNP/ANP;
	проверку правильности работы СНС, НК, FMS и т.п.;
	проверку точности самолетовождения помощью радионавигационных средств (VOR, DME, АРК, РЛС и т.п.);
б)	настроить радионавигационные средства самолетовождения (ILS, VOR, DME, АРК и т.п.);
в)	проверить соответствие фактического местоположения ВС определенному навигационной системой.
(2) Рубеж начала снижения (TOD) рассчитывается с учетом маршрута снижения, фактической высоты полета, ограничений по высотам и скоростям, необходимости применения противообледенительной системы, направления и скорости ветра по высотам, массы воздушного судна. Снижение планировать заблаговременно для исключения чрезмерно крутой траектории. 
(3) Принятая информация АТIS записывается в палетку "Взлет-Посадка" установленного в авиакомпании образца. 
Давление аэродрома по QNH, при необходимости, пересчитывается в давление QFE, как минимум, с обязательным сравнением результатов. На основе полученных данных о фактических метеоусловиях и уточненной массе ВС рассчитываются посадочные характеристики.
В зависимости от переданной информации АТИС экипаж при проведении предпосадочной подготовки по таблице в главе В.0 производит пересчет для установки давления на основных барометрических высотомерах значения давления из миллибар в миллиметры. Производит перерасчет контрольных высот захода на посадку (высота 4го разворота, точка входа в глиссаду, высота пролета дальнего/ближнего привода, высота принятия решения) из значений в футах в значения в метрах. 
На футомере КВС на эшелоне перехода устанавливается давление аэродрома (QNH или QFE), на высотомерах остальных членов экипажей устанавливается пересчитанное давление аэродрома (QNH или QFE). Заход на посадку выполняется в соответствии со схемой захода на посадку, опубликованной в сборнике Jeppesen.
При перерасчете значений давлений из миллибар в миллиметры, контрольных высот из футов в метры, перерасчет производится вторым пилотом и штурманом с последующим контролем КВС. 
После установления полученных данных до выполнения карты контрольных проверок на эшелоне перехода производится перекрестный контроль рассчитанных данных.
(4) Схема захода на посадку располагается в кабине экипажа таким образом, чтобы каждый пилот мог видеть всю необходимую информацию. Маневр захода на посадку выполняется в соответствии с опубликованной схемой и указаниями диспетчера органа ОВД.
(5) При смене ВПП (курса посадки) или возникновении условий, требующих изменения ранее принятых решений, экипажем ВС должна быть проведена дополнительная подготовка и повторная проверка выполненных операций по карте контрольной проверок. 
(6) При выполнении неточного захода на посадку, за исключением заходов с применением метода CDFA, экипаж в процессе предпосадочной подготовки рассчитывает удаление VDP* от торца ВПП и способы определения:
а)	временем полета от FAF;
б)	временем полета от ДПРМ (locator outer with radio marker(LOM));
в)	удалением по DME;
г)	с использованием FMS.
*VDP (точка начала визуального снижения) – точка, расположенная на посадочной прямой, в которой ВС находится на стандартной траектории снижения(УНГ=3°) и высоте, равной минимальной высоте снижения (МВС).
VDP следует рассматривать как последнюю точку на траектории захода на посадку по приборам, с которой может быть выполнено стабилизированное визуальное снижение и посадка на данную ВПП.
В зависимости от МВС при УНГ=3° VDP определяется по Таблицам А 8.9-Т2 и А - 8.9-Т3.
                                                                                                                                               Таблица А 8.9-Т2
МВС	80	90		100		110	120	130		140		150	160	170	180	190	200	210	220	230	240	250
(м)																						
																											
S до																											
ВПП	1200	1400		1600		1800	2000	2200		2400		2600	2800	3000	3100	3300	3500	3700	3900	4100	4300	4500
(м)																											
                                                                                                                     
                                                                                                                                                 Таблица А 8.9-Т3
МВС
(фут)	260	300	330		360		390		430		460		490	520	560	590		620	660	690		720	750	790	820
																									
																											
S до
ВПП
(n. m)																											
	0.7	0.8	0.9		1.0		1.1		1.2		1.3		1.4	1.5	1.6	1.7		1.8	1.9	2.0		2.1	2.2	2.3	2.4
																											
(7) Для реализации процедуры непрерывного снижения при неточном заходе на посадку (CDFA) с неавтоматизированным расчетом траектории снижения (без применения консультативного наведения VNAV) экипаж должен рассчитать вертикальную скорость снижения на конечном участке захода на посадку.
8.9.7.2.	Предпосадочный брифинг
Предпосадочный брифинг командир ВС проводит, как правило, до начала снижения, после завершения всех операций предпосадочной подготовки и получения докладов от членов летного экипажа о готовности к снижению.
При предпосадочном брифинге обсуждаются (но не ограничиваются) следующие элементы:
а)	предполагаемая для посадки ВПП, ее состояние, коэффициент сцепления; тип и состав светотехнического оборудования, огни «визуальной» глиссады, маркировка ВПП, ее длина и ширина, наличие смещенных порогов, расположение РД и карманов для разворота; наличие параллельных (близких по направлению) ВПП, РД, ГВПП, автодороги, мосты и прочие объекты, которые могут быть ошибочно восприняты за назначенную для посадки ВПП;
б)	метеорологические фактические и прогнозируемые условия на маршруте снижения и при заходе на посадку, информация ATIS, наличие опасных метеоявлений, порядок использования ПОС, локатора и других систем ВС;
в)	планируемое положение закрылков в зависимости от внешних условий полета, состояния ВПП, технического состояния ВС
г)	порядок использования систем торможения, в том числе особенности применения автоматического и неавтоматического режима торможения, особенности применения реверса тяги с учетом обстоятельств, ухудшающих условия торможения к моменту расчетного времени посадки;
д)	маневр входа в зону аэродрома (STAR), включая маршрут и схему захода на посадку, ограничения по высотам и скоростям, противошумовые процедуры, безопасные высоты в районе аэродрома и возможные маршруты векторения;
е)	порядок использования высотомеров (QNH / QFE), эшелон перехода;
ж)	система захода на посадку – основная, резервная, эксплуатационный минимум захода на посадку по основной и резервной системам;
з)	действия при отказе основной системы захода на посадку, принятие решения о применении резервной системы; 
и)	настройка радиосредств (частоты, курсы, режимы, положение переключателей и т.д.);
к)	маневр ухода на второй круг, включая маршрут, ограничения по высотам и скоростям, порядок взаимодействия;
л)	запасной аэродром, остаток топлива, максимальное время ожидания, маршрут ухода на запасной аэродром и порядок взаимодействия;
м)	порядок использования автоматизированной системы управления ВС, особенности эксплуатации ВС при заходе на посадку;
н)	предполагаемый маршрут руления после освобождения ВПП;
о)	техническое состояние ВС и его систем, влияние имеющихся неисправностей на предстоящий этап захода на посадку и производство посадки, 
п)	особенности захода на посадку и производство посадки при наличии отложенных неисправностях по ПМО/MEL/CDL;
р)	при наличии в кабине экипажа специалиста (обзёрвер, стажер, проверяющий, контролирующий) дополнительно проводится его инструктаж. 
Уделяется внимание порядку использования откидного сидения, использования ремней безопасности, пользования кислородом, покидания кабины в штатной и аварийной ситуациях.
В процессе проведения командиром ВС предпосадочного брифинга для захода на посадку в условиях минимума САТ II и III дополнительно обсуждаются: 
а)	наличие допуска членов летного экипажа к выполнению захода на посадку САТ II или III; 
б)	состояние аэродрома посадки и его оборудования на соответствие требованиям, установленным для данной категории захода на посадку; 
в)	состояние систем ВС и его оборудования на соответствие требованиям, установленным для данной категории захода на посадку; 
г)	порядок взаимодействия членов летного экипажа при заходе на посадку по указанному минимуму; 
д)	порядок выполнения процедуры ухода на второй круг; 
е)	действия экипажа (процедуры) при отказах систем (оборудования) ВС или оборудования аэродрома посадки на различных этапах захода, определяемых РЛЭ ВС (AFM, FCOM); 
ж)	дополнительные процедуры, стандартная фразеология и отклики, характерные для захода по САТ II или III; 
з)	использование светотехнического оборудования ВС; 
и)	регулировка положения пилотских кресел по высоте и удалению от органов управления. 
При изменении условий захода на посадку (ВПП, STAR, метеорологическая обстановка, система захода и т.п.) командир ВС должен провести дополнительный предпосадочный брифинг
8.9.7.3.	Снижение ВС с эшелона полета
При входе в район ОВД, в котором расположен аэродром посадки, командир ВС (член экипажа по указанию КВС) информирует орган ОВД о выбранном запасном аэродроме. 
Снижение воздушного судна с заданного эшелона (высоты) полета производится по разрешению диспетчера органа ОВД с докладом экипажа о начале снижения. Если указания диспетчера, по мнению командира ВС, не обеспечивают безопасности, следует немедленно запросить изменение полученного указания. 
Вход воздушного судна в район контролируемого аэродрома производится по схеме опубликованной аэронавигационной информации или по указаниям органа ОВД. При наличии нескольких опубликованных схем захода орган ОВД заблаговременно информирует экипаж воздушного судна о схеме захода, по которой следует выполнять полет. 
В процессе снижения экипажи воздушных судов во избежание срабатывания БСПС (TCAS)
выдерживают рекомендованные ограничения по вертикальной скорости не более 7 м/с (1379 ft в минуту) за 300 м (1000 ft) до заданного эшелона (высоты).
(5) В зонах с интенсивным воздушным движением в процессе снижения с эшелона 3000 м (10000 ft) для захода на посадку устанавливается приборная скорость не более 450 км/ч. (250 kt).
При снижении ниже высоты 3000 м (10000 ft) вертикальная скорость снижения не должна превышать значения 15 м/с (3000 ft/м), ниже высоты Н эш.перех. (Н IAF) - не должна превышать 10 м/с (2000 ft/ м), ниже Н твг (Н FAF) – не должна превышать 5 м/с (1000 ft/ м). При этом, если расчетная вертикальная скорость на участке конечного этапа захода на посадку на конкретном аэродроме более 5 м/с, это должно оговариваться на предпосадочном брифинге.
Информация органа ОВД «Снижение без ограничений» является основанием для выдерживания скоростей на усмотрение командира ВС.
Сведения о введении ограничений публикуются в документах аэронавигационной информации. 
Наиболее предпочтительным распределением обязанностей между пилотами при заходе на посадку в сложных метеорологических условиях и невозможности использования автоматизированной системы захода на посадку является вариант активного пилотирования вторым пилотом до ВПР(DA/H). 
На контролируемом аэродроме при невозможности выдерживания парамет¬ров полета заданных диспетчером органа ОВД, невозможности занятия заданного эшелона (высоты) к установленному или заданному рубежу командир ВС своевременно информирует об этом орган ОВД.
При входе в район контролируемого аэродрома командир ВС, выполняющий полет по ПВП, сообщает органу ОВД свое местонахождение, высоту полета и получает от него условия для захода на посадку.
В целях регулирования интервалов между воздушными судами, оказания содействия экипажам по обходу районов с неблагоприятными метеорологическими условиями, орган ОВД может производить векторение, а также задавать режимы поступательных и вертикальных скоростей в допустимых для данного ВС пределах, при этом:
а)	при осуществлении векторения точность выдерживания параметров, задаваемых органом ОВД, обеспечивает летный экипаж воздушного судна с учетом летно-технических характеристик ВС;
б)	векторение обеспечивается посредством указания пилоту конкретных курсов, позволяющих экипажам воздушным судов выдерживать необходимую линию пути; 
в)	в случае, если воздушное судно начинает наводиться с отклонением от ранее заданного маршрута, пилоту сообщается органом ОВД о целях такого наведения и не даются указания на снижение ниже высоты, обеспечивающей предписанный запас высоты над препятствием, в том числе с учетом влияния низких температур;
г)	векторение ВС прекращается органом ОВД после возобновления пилотом самостоятельного самолетовождения на основании выданного диспетчером ОВД указания, содержащего информацию о местоположении воздушного судна, точке выхода на заданный маршрут, магнитном путевом угле и расстоянии до неё; 
д)	момент доворота воздушного судна для выхода на траекторию конечного этапа захода на посадку является окончанием векторения. Разрешение на заход выдается органом ОВД одновременно с последним заданным курсом, сообщения о местонахождении ВС; 
е)	при заходе на посадку по приборам начатое векторение продолжается до выхода самолета на конечный этап захода на посадку по приборам или до начала визуального захода на посадку по разрешению органа ОВД; 
ж)	после получения разрешения на заход лётный экипаж ВС выдерживает последний заданный курс до входа в зону действия средств наведения на конечном этапе захода на посадку, затем без дополнительных указаний органа ОВД выполняет доворот и стабилизацию воздушного судна на линии, заданной средством наведения на продолженном конечном этапе захода на посадку.
Снижение ВС для посадки на горном аэродроме производится:
а)	при полетах по ППП - при наличии радиолокационного контроля или при применении: угломерно-дальномерных систем, или стационарного спутникового приемоиндикатора (при наличии схемы захода на посадку по СНС), при устойчивой работе бортового навигационного обо¬рудования и знания летным экипажем местоположения ВС после пролета маркированного рубежа с соблюдением схемы захода на посадку;
б)	при полетах по ПВП - в соответствии с данными правилами с обязательным применением радиотехнических средств захода на посадку.
Внимание! На горных контролируемых аэродромах полеты по траекториям, задавае¬мым органом ОВД, не допускаются.
8.9.7.4.	Полет в зоне ожидания
(1) Вход в зону ожидания осуществляется по установленному маршруту, а при его отсутствии – по стандартным правилам: параллельный, смещенный или прямой вход.
(2) Правила входа в зону ожидания(ЗО).     
 
1. «parallel entry» - выход на fix ЗО, полет с обратным курсом на предписанной скорости (табл.А8.9-Т4) в течении времени, указанного на схеме или 1мин. на FL до 140 включительно, или 1.5мин.на FL более 140, разворот на fix, при пролете fix вписывание в схему ЗО;
2. «offset entry» - выход на fix ЗО, полет с курсом, отличающимся от обратного на 30º, на предписанной скорости (табл.А8.9-Т4) в течении времени, указанного на схеме или 1мин. на FL до 140 включительно, или 1.5мин.на FL более 140, разворот на fix, при пролете fix вписывание в схему ЗО;
3. «direct entry» - выход на fix ЗО, при пролете fix вписывание в схему ЗО.

       (3) Полет в зоне ожидания производится по установленной схеме или в соот¬ветствии с указанием органа ОВД.

 
В контролируемом воздушном пространстве изменение высоты (эшелона) полета в зоне ожидания производится с разрешения органа ОВД, под управлением которого находится воздушное судно. 
Полет в зоне ожидания выполняется на скорости не более опубликованной для зоны ожидания, а если она не опубликована, то на скорости, не превы¬шающей значений, указанных в таблице А 8.9-Т4
Вход в зону ожидания и выполнение процедуры ожидания выполняется с креном 25 градусов или с угловой скоростью разворота не менее 3 град/с, если на схеме не опубликованы другие данные.
Скорость полета в зоне ожидания выдерживается с точностью не ниже 5 км/ч. (см. табл.А 8.9-Т4).
Скорости полета по прибору в зоне ожидания
                                                                                                                                          Таблица А 8.9-Т4
ВЫСОТА/ЭШЕЛОН	ОБЫЧНЫЕ УСЛОВИЯ	УСЛОВИЯ ТУРБУЛЕНТНОСТИ
ДО FL 140 (4250М) 
ВКЛЮЧИТЕЛЬНО	170 KT (315КМ/Ч) ВС КАТ. А, В
230 KT (425КМ/Ч) ВС КАТ. С, D	170 KT (315 КМ/Ч) ВС КАТ. А И В
280 KT (520*КМ/Ч)
ВЫШЕ FL 140 ДО FL 200 (6100М) ВКЛЮЧИТЕЛЬНО	240 KT (445**КМ/Ч)	
280 KT (520КМ/Ч) ИЛИ 0.8 М
В ЗАВИСИМОСТИ ОТ ТОГО, ЧТО МЕНЬШЕ*
ВЫШЕ FL 200 ДО FL 340 (10350М) ВКЛЮЧИТЕЛЬНО	265 KT (490**КМ/Ч)	
ВЫШЕ FL 340	0.83 М
 Для схем ожидания, связанных со структурой маршрутов, используется скорость полета 450 км/ч.
Скорость 520 км/ч или 0.8 М, рассчитанная из условий турбулентности, используется для полета в зоне ожидания только после предварительного разрешения органа ОВД.
Для схем ожидания, связанных со структурой маршрутов, используется скорость полета 520 км/ч.
(4) Разворот на линию пути удаления начинается в момент выхода ВС в кон¬трольную точку ожидания. 
Разворот на линию пути приближения начинается без упреждения в слу¬чаях, когда начало разворота задается моментом достижения заданного значения на¬вигационного параметра (дальности или пеленга от наземного средства навигации) или моментом пролета над навигационным средством. 
В случаях, когда момент начала разворота на линию пути приближения не указан, то экипаж руководствуется временем полета по линии пути удаления от тра¬верза контрольной точки ожидания, равным для штилевых условий:
1 мин, если эшелон (высота) ожидания не превышает FL 140 (4250 м);
1,5 мин, если эшелон (высота) ожидания превышает FL 140 (4250 м).
В случае невозможности выполнения требований процедуры ожидания, ко¬мандир ВС информирует об этом орган ОВД, под управлением, которого находится воздушное судно.
8.9.7.5.	Заход на посадку
Необходимые условия для начала или продолжения захода на посадку по приборам:
(1) Перед заходом на посадку экипаж ВС обязан проверить правильность установки давления на шкалах давлений барометрических высотомеров и сравнить показания всех высотомеров.
(2) Перед заходом на посадку на контролируемом аэродроме командир ВС должен сообщить органу ОВД: 
а)	выбранную систему захода на посадку и получить разрешение на заход на посадку по выбранной системе, если не предполагается наведение ВС по инициативе органа ОВД;
б)	при выполнении захода на посадку с применением визуального маневрирования (circle-to-land) - об установлении визуального контакта с ВПП и (или) ее ориентирами;
в)	при выполнении визуального захода - об установлении визуального контакта с аэродромом.
(3) Запрещается заход на посадку по ППП ниже установленной в документах аэронавигационной информации высоты начала конечного этапа захода на посадку если:
а)	состояние ВПП не соответствует установленным требованиям;
б)	скорость ветра у земли с учетом его направления и порывов, а также значение коэффициента сцепления превышают установленные ограничения.
(4) Если значение сообщённой метеорологической видимости или контрольной RVR ниже эксплуатационного минимума для посадки, заход на посадку по ППП не продолжается ниже установленной в документах аэронавигационной информации высоты начала конечного этапа захода на посадку. 
(5) Если после пролёта этой высоты получено значение метеорологической видимости или RVR ниже эксплуатационного минимума для посадки, заход на посадку может продолжаться до ВПР (DA/H) или МВС (MDА/H). В этом случае, при условии, что до достижения ВПР (DA/H) или МВС (MDА/H), командиром ВС установлен необходимый визуальный контакт с наземными ориентирами (п. 8.9.7.20), КВС имеет право произвести снижение ниже ВПР (DA/H) или МВС (MDА/H) и выполнить посадку. 
Не допускается устанавливать эксплуатационные минимумы аэродрома для посадки при видимости менее 800м, если не предоставляется информация о RVR.
Заход на посадку и посадка по приборам по минимуму ниже САТ I не разрешается, если не предоставляется информация о RVR. При отсутствии информации о RVR заход на посадку и посадка по САТ I выполняется при метеорологической видимости не менее 800 м.
(6) Контрольная RVR определяется по сообщенным значениям RVR в одной или нескольких точках наблюдения за RVR (точка приземления, средняя точка и конец ВПП), используемых в целях определения соблюдения установленных эксплуатационных минимумов. В случае, когда используется информация о RVR в разных точках, контрольная RVR представляет собой RVR в точке приземления, при этом RVR в средней точке и в дальнем конце ВПП должна быть не менее RVR установленного минимума для взлета. 
Примечание. При отсутствии информации о минимумах для взлета с данным посадочным курсом (взлет с данным курсом запрещен) применять значения минимума для взлета, приведенные в таблице «Условные минимумы для взлета» сборников эксплуатационных минимумов аэродромов, рассчитанные согласно Методике определения минимумов Авиакомпании.
Внимание!
«Условные минимумы для взлёта» используются только для определения RVR в средней точке и дальнем конце ВПП при заходе на посадку и не являются основанием для выполнения процедуры взлёта с ВПП, взлёт с которой запрещён.
Запрещается выполнение посадки при наличии информации о сильном дожде и метеорологической видимости менее 600 м без использования бортового радиолокатора и системы заблаговременного предупреждения о сдвиге ветра. 
(7) В любом случае, КВС прекращает заход на посадку на любом аэродроме, если, по его мнению, не обеспечивается безопасность посадки. 
При отсутствии визуального наблюдения пилотом менее одного наземного ориентира в течение времени, достаточного для оценки пилотом местоположения воздушного судна и тенденции его изменения по отношению к заданной траектории полёта, продолжение захода на посадку ниже DA/H или MDA/H является нарушением минимума для посадки.
Указанными ориентирами являются:
а)	при заходе на посадку с применением визуального маневрирования (маневр «circle-to-land») – любые ориентиры, относительно которых представляется возможным определять положение воздушного судна относительно ВПП. Снижение ниже высоты MDA/H, установленной для визуального маневрирования (маневр «circle-to-land»), допускается только при наличии визуального контакта с порогом ВПП или светосигнальными средствами захода на посадку, связанными с ВПП;
б)	при заходе на посадку в условиях не ниже категории I – система огней приближения или её часть, порог ВПП и его маркировка, входные огни ВПП, огни обозначения порога ВПП, система визуальной индикации глиссады, зона приземления, её маркировка, огни зоны приземления, посадочные огни ВПП.
(8) На контролируемом аэродроме разрешение захода на посадку по опубликованной схеме от органа ОВД может быть выдано в любой момент времени, но не позднее выхода ВС на конечный участок захода на посадку (до входа в глиссаду). Заход на посадку по РМС начинается в точке конечного участка захода на посадку, являющейся точкой входа в глиссаду (FAF, FAP).
(9) При полетах на неконтролируемый аэродром или на контролируемый аэродром, на котором временно не производится обслуживание аэродромного (воздушного и (или) наземного) движения, перед заходом на посадку КВС обязан: 
а)	получить информацию о состоянии ВПП и её пригодности для выполнения посадки от органа ОВД аэродрома или выполнить осмотр ВПП с воздуха;
б)	передать сведения о месте и магнитном курсе посадки на частоте связи органа ОВД, в районе ответственности которого он находится. 
После приземления, при наличии связи с органом ОВД, сообщить ему о посадке. 
(10) Командир ВС может выполнять повторный заход на посадку при:
а)	запасе топлива, достаточном для ухода на запасной аэродром с ВПР(DA(H)) или от точки ухода на второй круг после повторного захода (с сохранением Final Reserve Fuel);
б)	фактических условиях захода на посадку и посадки, соответствующих требованиям ФАП.
(11) При отсутствии на аэродроме посадки радиолокационного контроля или невозможности использования угломерно-дальномерной системы, заход на посадку в условиях ППП по установленной схеме, выполняется после выхода на радионавигационное средство, на котором основана схема, на безопасном эшелоне (высоте).
(12) При одновременном визуальном заходе на посадку двух воздушных судов преимущество совершить посадку первым имеет воздушное судно, летящее впереди, слева или ниже.
8.9.7.6.	Указания по выполнению точных и неточных заходов на посадку по приборам
Сокращения
ILS       система посадки по приборам
MLS     микроволновая система посадки (сантиметрового диапазона)
GLS     система посадки с использованием GBAS
GBAS  наземная система функционального дополнения
PAR     посадочный радиолокатор
SBAS   спутниковая система функционального дополнения
APV     схема захода на посадку с вертикальным наведением
VNAV вертикальная навигация
LNAV боковая навигация
LOC   курсовой радиомаяк
DME   дальномерное оборудование
SRA   заход на посадку с помощью обзорного радиолокатора
NPA   неточный заход на посадку
РА      точный заход на посадку
VOR   всенаправленный ОВЧ-радиомаяк
NDB   ненаправленный радиомаяк
RNAV зональная навигация
PBN   навигация, основанная на характеристиках
RNP   требуемые навигационные характеристики
CDFA заход на посадку с непрерывным снижением на конечном участке
FMS    система управления полетом
Заходы на посадку на конечном этапе схемы захода на посадку разделяются на следующие виды:
а)	точный заход на посадку с использованием ILS, MLS, GLS, GBAS (п.8.9.7.8), PAR, SBAS;
б)	заход на посадку с вертикальным наведением (APV) (с использованием баро - VNAV, APV I и APV II);
в)	неточный заход на посадку (с использованием NPA, LOC c DME, VOR, VOR c DME, 2 NDB, NDB, NDB c DME, VDF, RNAV(LNAV), SPA, который в свою очередь подразделяется на «при наличии FAF» и «при отсутствии FAF».
В «Сборниках эксплуатационных минимумов аэродромов для взлета и посадки самолетов» приведены структурные схемы заходов на посадку для различных:
а)	видов захода на посадку;
б)	методов захода на посадку;
в)	схем захода на посадку по ППП;
г)	классификаций минимальных значений эксплуатационных минимумов;
д)	видов конечных этапов захода на посадку;
е)	способов управления траекторией в вертикальной плоскости при заходах NPA;
ж)	заходов на посадку с применением оборудования RNAV.
8.9.7.7.	Классификация заходов на посадку по приборам
Схемы заходов на посадку по приборам классифицируются следующим образом:
а)	NPA- неточные заходы на посадку по приборам для выполнения двухмерных заходов (2D) по типу А;
б)	APV- заходы с вертикальным наведением для выполнения трехмерных заходов (3D) по типу А;
в)	РА - точные заходы на посадку для выполнения трехмерных заходов (3D) по типу А или В.
Классификация заходов на посадку по приборам исходя из расчетных наиболее низких эксплуатационных минимумов, ниже которых заход на посадку продолжается только при необходимом визуальном контакте с ориентирами, следующим образом:
а)	тип A: минимальная относительная высота снижения или минимальная относительная высота принятия решения составляет 75 м (250 фут) или более;
б)	тип B: относительная высота принятия решения составляет менее 75 м (250 фут). Заходы на посадку по приборам типа B подразделяются на следующие категории:
(1) Категория I (САТ I): относительная высота принятия решения не менее 60м (200 фут) и/либо при видимости не менее 800 м, либо при дальности видимости на ВПП не менее 550м (любой точный заход на посадку при DA/H в 60м (200футов) или выше и минимальной видимости RVR в 550м или более определяется как стандартный заход на посадку по САТ I);
(2) Категория II (САТ II): относительная высота принятия решения менее 60м (200фут), но не менее 30м (100 фут) и/или дальность видимости на ВПП не менее 300 м;
(3) Категория IIIA (САТ IIIA): относительная высота принятия решения менее 30м (100фут) или без ограничений по относительной высоте принятия решения и/или дальность видимости на ВПП не менее 175 м;
(4) Категория IIIB (САТ IIIB): относительная высота принятия решения менее 15м (50фут) или без ограничений по относительной высоте принятия решения и/или дальность видимости на ВПП менее 175 м, но не менее 50м;
(5) Категория IIIC (САТ IIIC): без ограничений по относительной высоте принятия решения и дальности видимости на ВПП.
Внимание!
а)	Перед каждым полетом КВС должен проверить состояние ВС по открытым пунктам MEL в листе отложенных неисправностей (HIL).
б)	Если состояние радио или электротехнического оборудования ВПП не отвечает установленным требованиям или отсутствует информация о видимости на ВПП, для принятия решения на производство посадки необходимо руководствоваться таблицами Приложения СП2 РПП части С «Сборник установленных минимумов для захода на посадку и взлета на аэродромах, разрешенных для самолетов категорий В и C». 
Процедуры выполнения точных заходов на посадку описаны в РПП, Часть В типа ВС, глава 2 «Инструкция по взаимодействию и технология работы членов экипажа» (Standard Оperation Procedure).
8.9.7.8.	Точный заход на посадку с использованием Global Navigation Satellite System 
(GNSS) – вариант GBAS (Ground-Based Augmentation System)
(1) Выполнение захода на посадку.
Точный заход на посадку по GBAS выполняется методом, в значительной степени аналогичным точному заходу на посадку по ILS, с использованием бокового наведения на промежуточном участке до входа в глиссаду, после чего для посадки наряду с боковым наведением начинает и продолжает обеспечиваться вертикальное наведение.
(2) Критерии отображения информации при заходе на посадку по GBAS.
GBAS обеспечивает точный заход на посадку, аналогично заходу на посадку по категории I ILS.
Минимальные функциональные возможности отображения аналогичны ILS и предусматривают индикацию отклонения по курсу, индикацию отклонения в вертикальной плоскости, информацию о расстоянии до порога ВПП и флажки сигнализации отказов. При отсутствии на борту навигационного оборудования пилот не обеспечивается информацией о местоположении и навигационной информацией. Предоставляется лишь информация, обеспечивающая наведение по курсу и глиссаде на конечном этапе захода на посадку.
8.9.7.9.	Заход на посадку с вертикальным наведением (APV) 
Заход на посадку по приборам основанный на характеристиках (PBN), предназначенный для выполнения трехмерных (3D) заходов на посадку по приборам типа A.
Заход на посадку с вертикальным наведением с использованием оборудования баро - VNAV
Барометрическая вертикальная навигация (баро-VNAV) представляет собой навигационную систему, которая выдает пилоту вычисленное вертикальное наведение относительно угла траектории в вертикальной плоскости (VРА), номинальное значение которого составляет 3°.
Обеспечиваемое ЭВМ вертикальное наведение основывается на барометрической абсолютной высоте и определяется в виде VPA, начинающегося на относительной высоте опорной точки (RDH).
Схемы APV/баро-VNAV предназначаются для использования ВС, оборудованными FMS или иными системами RNAV, способными вычислять траектории барометрической VNAV и выдавать отклонения от них на индикатор на приборной доске.
Особенности использования баро-VNAV при низких температурах наружного воздуха аэродрома
а)	схемы бapo-VNAV не разрешается использовать в том случае, когда температура на аэродроме ниже опубликованной минимальной температуры для конкретной схемы, если система управления (FMS) не имеет для конечного этапа захода на посадку утвержденной компенсации низких температур. При наличии такой функции минимальную температуру можно не учитывать при условии, что она находится в пределах минимальной сертифицированной температуры для оборудования. Ниже этой температуры и применительно ВС, не имеющим FMS с утвержденной для конечного этапа захода на посадку компенсацией низких температур, может по-прежнему использоваться схема LNAV при условии, что:
б)	для захода на посадку опубликованы обычная неточная схема с применением RNAV и ОСА/Н APV/LNAV;
в)	пилотом применяется соответствующая поправка к высотомеру на низкую температуру ко всем опубликованным минимальным абсолютным/относительным высотам.
Оборудование баро – VNAV
Оборудование баро-VNAV может применяться при выполнении заходов на посадку и посадок двух различных классов:
а)	Заходы на посадку и посадки с вертикальным наведением
При использовании баро-VNAV наведение в боковой плоскости основывается на навигационных спецификациях RNP APCH и RNP AR APCH. Такие схемы публикуются с указанием абсолютной/относительной высоты принятия решения (DA/H). Их не следует путать с классическими схемами неточного захода на посадку (NPA), в которых устанавливается минимальная абсолютная/относительная высота снижения (MDA/H).
Заходы на посадку и посадки с вертикальным наведением обладают значительным преимуществом по сравнению с операциями, используемыми совместно со схемой неточного захода на посадку, т.к. они основываются на специальных критериях построения схем, не требуя перекрестной проверки ограничений в схеме неточного захода на посадку (например, пролет контрольных точек ступенчатого снижения).
Примечание. На основе этих критериев также решаются проблемы:
 потери высоты после начала ухода на второй круг, когда разрешается использовать DA вместо MDA и этим стандартизируется техника пилотирования при выполнении заходов на посадку с вертикальным наведением и точных заходов на посадку;
 пролета препятствий на этапе захода на посадку и посадки, с учетом температурных ограничений до DA, в результате чего обеспечивается более надежная защита от препятствий по сравнению со схемой неточного захода на посадку.
б)	Неточные заходы на посадку и посадки
В этом случае использовать систему баро-VNAV не требуется, но она может применяться в качестве средства вспомогательного содействия методу CDFA (8.9.7.11(а)). Это означает, что консультативное наведение VNAV используется совместно со схемой неточного захода на посадку.
8.9.7.10.	Неточный заход на посадку (NPA) с использованием 2 NDB (ОСП), NDB c DME, VOR, 
                                        LOC, LOC c DME, VOR c DME, SRA, RNAV (LNAV)
(1) Предназначен для выполнения двухмерных (2D) заходов на посадку по приборам типа A
Примечание. При неточном заходе на посадку по системе ОСП, ОПРС (NDB, VOR) при отсутствии дальномерной системы маяки системы ОСП (NDB) располагаются, как правило, в створе ВПП (но не более ± 3º от МПУ залегания ВПП посадки) по курсу захода перед ВПП. Отдельно стоящий маяк ОПРС (VOR) может использоваться при расположении его как в створе ВПП, так и не в створе ВПП (но не более ± 10º от МПУ залегания ВПП посадки). 
Отдельно стоящий маяк (ОПРС, VOR) может быть использован для захода на посадку при его расположении от 10 км. перед ВПП до 1,5 км. за ВПП. 
В каждом из перечисленных вариантов минимум для захода на посадку по приборам устанавливается в зависимости от расположения маяков (ОСП, ОПРС, NDB, VOR) относительно ВПП и рассчитывается в соответствие с Методикой определения эксплуатационных минимумов аэродромов для взлета и посадки ВС, утвержденной для авиакомпании.
(2) Полеты по схемам неточного захода на посадку могут выполняться с использованием метода захода на посадку с непрерывным снижением на конечном участке (CDFA). Операции по методу CDFA с консультативным наведением при VNAV и вычислением параметров бортовым оборудованием считаются трехмерными (3D) заходами на посадку по приборам с использованием высоты DA (H).
Операции по методу CDFA с вычислением параметров на основе неавтоматизированного расчета требуемой вертикальной скорости снижения считаются двухмерными (2D) заходами на посадку по приборам с использованием высоты МDA (H).
(3) Выход на траекторию захода на посадку и её выдерживание в боковой плоскости осуществляется по информации системы посадки, выдаваемой на пилотажные приборы пилотов, или по командам диспетчера зоны посадки при заходе по локатору.
(4) Контроль пути по дальности производится с использованием радионавигационных средств (DME, GPS), счисление пути - по времени пролёта FAF и по информации органа ОВД.

8.9.7.11.	Способы управления траекторией в вертикальной плоскости при неточных заходах:
а)	Заход на посадку с непрерывным снижением на конечном участке при неточном заходе на посадку Continuous Descent final approach (CDFA)
Заход на посадку с непрерывным снижением на конечном участке при неточном заходе на посадку (CDFA) применяется, если имеется опубликованная на схеме точка FAF или имеется возможность определить точку начала снижения на конечном этапе захода на посадку (ТВГ) или расстояния до торца ВПП с помощью FMS, RNAV, DME, SRA.
Данный способ предусматривает непрерывное стабилизированное снижение в полете на конечном этапе захода на посадку, выполняемое с наведением VNAV и вычислением параметров бортовым оборудованием (для 3D заходов) или на основе неавтоматизированного расчета требуемой вертикальной скорости снижения (для 2D заходов), без промежуточных горизонтальных участков. Вертикальная скорость снижения выбирается, корректируется для обеспечения непрерывного снижения до точки, расположенной на высоте примерно 15м (50фут) над посадочным порогом ВПП, или до точки, где для данного типа ВС должен начинаться маневр выравнивания перед посадкой. Снижение рассчитывается и осуществляется таким образом, чтобы обеспечить пролет на минимальной абсолютной высоте или выше её любых контрольных точек ступенчатого снижения.
УХОД НА ВТОРОЙ КРУГ должен начинаться на высоте, предотвращающей снижение ниже МDA (H), но не менее чем за 15м (50Ф) до достижения МDA(H). При раннем уходе на второй круг летный экипаж не должен набирать высоту больше, чем высота контрольной точки конечного этапа захода на посадку до достижения точки MAPt, чтобы не создавать помеха для остальных участников воздушного движения.
Ни в какой момент времени при полете на конечном этапе захода на посадку с использованием метода CDFA ВС не должно выполнять горизонтального полета.
Любые развороты при уходе на второй круг не начинаются до тех пор, пока ВС не достигнет МАРt. Если ВС достигает МАРt раньше, чем DA(H) (МDA(H)) в МАРt должен быть начат уход на второй круг.
                FAF

       	                                                                                                            MAP

                                                  VDA 3,0º


              Высота начала ухода на второй круг             при 2D ЗАХОДАХ
                         
            
 
              MDA (H) или DA (H) (см.п.п. 8.9.7.10.(3))

                                                                                                                         15м (50 фут).
                         Рис.1
б)	Снижение с постоянным углом
Второй способ рассчитан обеспечить постоянный угол снижения от конечной контрольной точки захода на посадку (FAF) или оптимальной точки на схемах без FAF до опорной точки над порогом ВПП, расположенной на высоте 15м (50 фут). При подходе ВС к MDA(H) принимается решение о продолжении снижения с постоянным углом, либо выполнять выравнивание в горизонтальный полет на или выше MDA(H), в зависимости от визуальных условий.
Если визуальные условия являются адекватными, ВС продолжает снижение до ВПП без какого-либо промежуточного выравнивания в полете.
Если визуальные условия являются неадекватными для продолжения снижения, ВС выполняет выравнивание в полете на или выше MDA(H) и продолжает полет по линии пути приближения до тех пор, пока не окажется в визуальных условиях, достаточных для снижения ниже MDA (H) до ВПП, или пока не выполнит уход на второй круг по достижении опубликованной точки ухода на второй круг.
   FAF                                                                VDP	                             MAP

	


                                            VDA 3,00




MDA (H)            
                
                Высота начала выода в ГП


                                                                                                    15м (50 фут)
                                           Рис.2
в)	Ступенчатое снижение
Предусматривает снижение с повышенной вертикальной скоростью до минимальных высот контрольных точек ступенчатого снижения. Как правило, указывается только одна контрольная точка ступенчатого снижения.
В	схеме для VOR/DME может быть установлено несколько контрольных точек DME, каждая с соответствующей минимальной абсолютной высотой пролета.
Пилот начинает снижение после стабилизации на линии пути, выдерживая самолет на опубликованных соотношениях "расстояние по DME/высота" или выше.
Данный способ является приемлемым до тех пор, пока получаемый градиент снижения остается менее 15% и уход на второй круг начинается по достижении МАРt или до МАРt. Этот способ требует уделять особое внимание контролю абсолютной высоты вследствие высоких вертикальных скоростей снижения на участке до достижения MDA(H), а также на последующем участке вследствие повышенного времени полета в зоне препятствий на минимальной абсолютной высоте снижения.

            FAF


высота пролета контрольной точки                                                           MAP

                                                             VDA 3,0º



MDA (H)

                                   высота начала вывода в ГП

	15м (50 фут)
  
Рис.3
В установленной (расчетной) точке входа в глиссаду самолёт переводится на снижение с расчётной вертикальной скоростью VУ.
Eсли к моменту достижения заданной высоты контрольная точка не пройдена, то необходимо установить режим работы двигателей, который был установлен перед началом снижения, и следовать в горизонтальном полёте до пролета контрольной точки.
Вывод ВС в горизонтальный полет начинается за 20 - 30м до достижения высоты пролета контрольной точки.
Если в контрольной точке высота больше расчётной, то производится увеличение VУ снижения на 0,5 ÷1 м/с.
8.9.7.12.	При достижении МDA(H) при неточном заходе на посадку:
если установлен необходимый визуальный контакт с ВПП и положение ВС в пространстве обеспечивает безопасность посадки, снижение для посадки может быть продолжено, при этом не допускается снижения к ВПП с вертикальной скоростью более 5 м/с;
а)	если визуальный контакт с наземными ориентирами не установлен или визуальный контакт недостаточен для принятия решения на посадку, ВС переводится в горизонтальный полет и следует на высоте не ниже МDA(H) в направлении на точку ухода на второй круг(MAPt) до установления необходимого контакта с ВПП;
б)	если до пролета точки VDP необходимый контакт с ВПП не установлен (см. п. 8.9.7.20) или выполнение посадки небезопасно, командир ВС обязан выполнить уход на второй круг с набором высоты в направлении точки MAPt. Боковое маневрирование по схеме ухода разрешается только после пролета MAPt.
На аэродромах, оснащенных системой визуальной индикации глиссады для выдерживания установленной глиссады снижения к ВПП используются показания огней этой системы.
При снижении ПКУ сравнивает установленные (расчетные) и фактические высоты пролета контрольных точек (ОПРС, ДПРМ, БПРМ, маркеров, удалений до ВПП), определяет уклонения ВС от посадочного курса и сообщает о них экипажу.
При пролете контрольных точек схемы захода экипаж должен производить сверку высоты полета по барометрическим высотомерам, с показаниями радиовысотомера (учитывая превышения рельефа местности).
8.9.7.13.	Методика выполнения неточного захода на посадку с использованием VOR/DME
Выполнение захода на посадку по VOR/DME имеет ряд особенностей и отличий, как в порядке использования радионавигационного оборудования, так и построения самой схемы захода.
При выполнении заходов по маякам (VOR/DME) необходимо руководствоваться следующим:
а)	Требования ICAO к размещению маяка VOR и установление линии пути конечного этапа захода на посадку:
	при отсутствии фиксированной точки конечного этапа захода на посадку (ТВГ) VOR должен быть расположен не далее 1,9 км от ближайшей точки ВПП;
	при наличии фиксированной точки конечного этапа захода на посадку (ТВГ) VOR может располагаться не далее 37 км от аэродрома;
	при расположении VOR в стороне от осевой линии ВПП линия пути конечного этапа захода на посадку устанавливается под углом к продолжению осевой линии ВПП, которая может устанавливаться с пересечением и без пересечения осевой линии ВПП перед порогом. Но при этом должны быть соблюдены условия, указанные на рисунках 1 и 2.





 
Рис. 1. Для схем с пересечением осевой линии ВПП
 
Рис. 2. Для схем без пересечения осевой линии ВПП перед порогом.
б)	Особенности захода на посадку при использовании наведения по дуге DME (DME Arc).
Наведение по дуге DME осуществляется на начальном участке захода на посадку. Параметры схем маневрирования указываются на схемах STAR или APPROACH CHART. Маневрирование по дугам DME используется для захода на посадку как по системе VOR/DME, так и по другим системам (ILS и т.д.).
Процедура захода по дуге DME включает в себя следующие этапы:
	подход по заданному значению радиала/пеленга до фиксированного значения дальности по DME;
	разворот на 90 градусов для выхода на дугу заданного радиала;
	следование по дуге до указанного радиала для последующего разворота на посадочную прямую.
Таблица значений кренов / ∆ R в зависимости от W и радиуса дуги DME Arc
                                                                                                                                                 Таблица А8.9-Т5
А RC (NM)	W КМ/Ч (KN)
А RC (КМ)	222(120)	259(140)	296(160)	333(180)	370(200)	407(220)	444(240)
7	13	2/ 4º	2º / 5º	3º / 6º	4º / 8º	5º / 10º	6º / 12º	7º / 14º
10	19	1º / 3º	2º / 3º	2º / 4º	3º / 6º	3º / 7º	4º / 8º	5º / 10º
13	24	1º / 2º	1º / 3º	2º / 4º	2º / 4º	3º / 6º	3º / 7º	4º / 8º
16	30	- / 2º	1º / 2º	1º / 3º	2º / 4ºº	2º / 4º	2º / 5º	3º / 6º
19	35	- / 1º	1º / 2º	1º / 2º	1º / 3º	2º / 4º	2º / 5º	3º 5º
22	41	- / 1º	1º / 2º	1º / 2º	1º / 3º	2º / 3º	2º / 4º	2º / 5º

  
Рис. 3.	При полете по дуге заданного радиуса ВС следует с креном, который вычисляется на НЛ-10м или по табл.№ 1 в зависимости от заданного радиуса и текущего значения путевой скорости.
Наглядным источником информации для выдерживания постоянного радиуса дуги яв- ляются показания ИКУ. При выполнении «правильной» дуги DME Arc в каждый текущий момент времени значение магнитного курса должно быть больше значения радиала на 90 градусов для правой орбиты и на 90 градусов меньше для левой. 
При использовании навигационных приборов с индикацией КУР при полете по дуге с правым креном значение КУР по маяку VOR (NDB) должно быть 90°, при полете с левым креном - 270°.
Совместное использование показаний DME и значений радиалов (КУР) позволяет траекторию ВС при наличии ветра приблизить к дуге постоянного радиуса.
Величину AR (Рис.3) можно взять из Таблицы А8.9-Т5, исходя из конкретных условий полета 
На ВС, оборудованных FMS, обеспечивающими ввод координат пунктов маршрута, и оснащенными приемниками спутниковой навигации, имеются возможности контроля выполнения маневра по дуге DME Arc. Тем не менее основным источником информации являются значения дальности DME и радиала VOR.
На ВС, оборудованных FMS, целесообразно при подготовке к полету разбить дугу DME на несколько промежуточных участков, в зависимости от величины радиуса дуги и величины воздушной скорости, чтобы обеспечить участки полета по 30-40 сек и произвести программирование этих точек. Полет выполнять по запрограммированным участкам.
При использовании приемников спутниковой навигации при подготовке к полету аналогичным образом рассчитываются и вводятся в маршрут полета промежуточные точки дуги DME. При выполнении захода по дуге DME осуществляется последовательный выход во все запрограммированные точки с использованием параметровнаведения пеленг (bеагing) фактический путевой угол (track) с обязательным контролем выполнения маневра по текущим значениям дальности DME, радиала VOR и магнитного курса.
На картах захода на посадку начало разворота для выхода на посадочную прямую, как правило, указывается фиксированным значением радиала, который в отдельных случаях может указываться по категориям ВС.
в)	Общие правила порядка использования бортового навигационного оборудования при заходе на посадку по системам VOR, VOR/DME, NDB.
Настройка аппаратуры должна быть продумана так, чтобы перед четвертым разворотом на первом комплекте оборудования наведения был настроен ILS (VOR, NDB), т.е. то средство, по которому будет осуществляться заход на конечном участке захода. Затем настройка «разворачивается» в обратную сторону по STAR.
Порядок настройки, последовательность переключений навигационных средств и приборов должны быть продуманы на целесообразность, надежность и наглядность отображения траектории ВС на схеме STAR(SID).
8.9.7.14.	Выполнение визуального захода на посадку
В случае отсутствия на аэродроме радиотехнических средств захода на посадку, планирование полета с учетом визуального захода на посадку на данном аэродроме выполняется при фактических или прогнозируемых метеоусловиях: Н нго не ниже MSA, видимость не менее 5000 м.
Визуальный заход на посадку может начинаться в любой точке маршрута прибытия, любой точке схемы захода на посадку по ППП при установлении необходимого визуального контакта с наземными ориентирами.
Визуальный заход на посадку не выполняется, если экипаж не знает рельефа местности и характерных наземных ориентиров, на которых основана процедура захода на посадку.
Визуальный заход на посадку разрешается выполнять при условии, что командир ВС, имеет действующий допуск к выполнению визуальных заходов на посадку.
На контролируемом аэродроме выполнение визуального захода на посадку требует разрешения органа ОВД.
Органом ОВД выдается разрешение на выполнение визуального захода на посадку воздушному судну, выполняющему полет по ППП, при условии:
а)	экипаж имеет возможность поддерживать визуальный контакт с наземными ориентирами;
б)	сообщаемая нижняя граница облаков соответствует или превышает высоту, на которой начинается начальный участок захода на посадку ВС, получившего такое разрешение, или
в)	экипаж сообщает, что метеорологические условия позволяют выполнять визуальный заход на посадку.
При выполнении визуального захода на посадку экипаж должен выдерживать высоты, опубликованные на схемах этого вида захода на посадку на данном аэродроме. При отсутствии указанной информации экипажу разрешается производить снижение по ППП в зону радиусом 50км (25nm) от КТА до высоты, равной БВП (MSA) (для соответствующего сектора захода) для горного аэродрома или до высоты БВП (MSA), уменьшенной на 100м – для остальных аэродромов. Дальнейшее снижение с целью выполнения визуального захода на посадку возможно при условии установления визуального контакта с ВПП или известными ему характерными наземными ориентирами.
Об установлении визуального контакта с ВПП или известными характерными наземными ориентирами КВС сообщает органу ОВД.
При потере визуального контакта с наземными ориентирами или ВПП на любом этапе визуального захода на посадку снижение прекратить и выполнить полет в сторону ВПП с набором высоты и входом в схему прерванного захода на посадку (ухода на второй круг) по приборам с последующим докладом органу ОВД.


8.9.7.15.	Заход на посадку «с круга» (сircling approach) 
(1) При отсутствии метеоусловий, необходимых для выполнения визуального захода на посадку, а также, если экипаж не знаком в достаточной степени с рельефом местности, характерными наземными ориентирами для выполнения визуального захода на посадку, может быть выполнен заход на посадку с применением процедуры визуального маневрирования (сircling approach). При этом полет по ППП по схеме прибытия производится до входа в зону визуального маневрирования (ЗВМ) на высоте не ниже высоты входа в ЗВМ по ППП с последующим выполнением указанной процедуры.
На контролируемом аэродроме процедура захода на посадку сircling approach выполняется после установления визуального контакта с ВПП и получения разрешения органа ОВД. Доклад КВС о наличии такого контакта является запросом на выполнение процедуры визуального маневрирования для захода на посадку.
При заходе на посадку с применением визуального маневрирования (сircling approach) экипаж обеспечивает нахождение ВС в пределах установленной зоны визуального маневрирования.
Заход на посадку с применением визуального маневрирования применяется на аэродромах, для которых предусмотрена данная процедура захода на посадку.
Заход на посадку с применением визуального маневрирования выполняется при непрерывном визуальном контакте со средой ВПП (порогом ВПП или светотехническими средствами захода на посадку).
Точка начала захода на посадку (Тнвзп) определяется командиром ВС и находится внутри зоны визуального маневрирования. Выпуск шасси и механизации (в промежуточное положение) осуществляется до Тнвзп, а довыпуск механизации крыла в посадочное положение – при довороте на посадочный курс.
На аэродромах с установленной схемой захода на посадку с применением визуального маневрирования, снижение по приборам производится до высоты, установленной в точке начала захода на посадку по предписанной схеме или указанной органом ОВД, а после начала процедуры – до минимальной высоты снижения МВС(MDA(H)).
Снижение ВС ниже минимальной высоты снижения с целью посадки производится после начала разворота на предпосадочную прямую при наличии визуального контакта с ВПП.
При потере визуального контакта с ВПП в любой точке зоны визуального маневрирован ия снижение прекращается и выполняется полет в сторону ВПП с набором высоты и входом в схему прерванного захода на посадку (ухода на второй круг) по приборам с последующим докладом органу ОВД.
(2) При полетах на посадочные площадки по ПВП, где отсутствуют органы ОВД, перед заходом на посадку командир ВС:
а)	выполняет контрольные заходы в целях осмотра площадки и определения ее пригодности к посадке.
б)	При выполнении контролируемого полета:
	сообщает органу ОВД, в районе ответственности которого находится ВС, предполагаемое время, место и магнитный курс посадки;
	информирует орган ОВД, при наличии радиосвязи, о выполнении посадки. 
Командир ВС должен доложить о готовности к посадке при полете на конечном участке захода на посадку на удалении, как правило, не менее 4 км до торца ВПП, но не ниже ВПР(DA(H))/МВС(MDA(H)).
(3) Посадка на контролируемом аэродроме производится с разрешения органов ОВД, которое должно быть получено экипажем:
а)	при точном заходе на посадку, заходе на посадку с использованием бокового и вертикального наведения до достижения высоты 60 м. над аэродромом, но не ниже ВПР (DA/H) или (MDA/H);
б)	при неточном заходе на посадку - до достижения МВС(MDA/H));
в)	при визуальном заходе на посадку, заходе на посадку с круга, заходе на по¬садку по ПВП - после вывода ВС на конечный участок захода на посадку, на удале¬нии от рабочего порога ВПП не менее одного километра.
(4) Минимальные безопасные высоты пролета препятствий, МВС, скорости и радиусы траекторий полета при заходе на посадку «с круга» (сircling approach)
Полет до зоны визуального маневрирования в зоне круга осуществляется по правилам ППП.
Вход в зону аэродрома осуществляется по установленным схемам (STAR) или по траекториям, задаваемым службой ОВД в соответствии с существующими правилами.
При маневрировании ВС в пределах зоны ВЗП могут использоваться высоты Нмс (МВС) для визуального захода на посадку соответствующей категории ВС. При этом запас высоты над препятствиями составит:
а)	для ВС категории A и B - 90 м;
б)	для ВС категории С и D - 120 м;
в)	для ВС категории Е - 150 м;
При маневрировании ВС за пределами зоны ВЗП, но в пределах 30 км от КТА, минимальной используемой высотой является высота входа в зону ВЗП, указанная на схемах визуального захода на посадку.
При этом запас высоты над препятствиями составит не менее 200 м.
При маневрировании ВС в пределах от 30 до 50 км от КТА минимальной используемой высотой является высота БВП для горных аэродромов и БВП, уменьшенная на 100 м, для остальных аэродромов.
Минимальная высота снижения МВС (MDA(H)) в зоне визуального маневрирования публикуется в сборниках АНИ в колонке «Circle-to-land», но при этом MDH устанавливается не менее:
   150м для ВС категории В;	 210м для ВС категории D;
   180м для ВС категории С;	 240м для ВС категории Е.

	                                                                                                                             Таблица А-8.9-Т6
Категория	Vмах при полете	Радиус для построения	Минимальное	Запас высоты над
ВС	в зоне ВЗП (км/ч)	зоны ВЗП (км)		значение МВС(м)	препятствиями в зоне ВЗП (м)
B	250	4.9		150	90
C	335	7.85		180	120
D	380	9.79		210	120
Е	445	12.82		240	150
(5) Основные положения по распределению обязанностей в экипаже при визуальном заходе на посадку, заходе на посадку с применением визуального маневрирования:
Командир ВС осуществляет активное пилотирование:
а)	пилотирует ВС;
б)	выполняет операции согласно технологии работы экипажа;
в)	выполняет посадку или уход на второй круг;
г)	при потере визуального контакта с ВПП или её ориентирами прекращает заход с круга и выполняет маневр по уходу на второй круг.
При потере визуального контакта с наземными ориентирами и/или ВПП на любом этапе визуального захода на посадку снижение прекратить и выполнить полет в сторону ВПП с набором высоты и входом в схему прерванного захода на посадку (ухода на второй круг) по приборам с последующим докладом органу ОВД.
Второй пилот осуществляет контролирующее пилотирование:
а)	контролирует полет по приборам в процессе визуального маневрирования, обращая особое внимание на выдерживание установленной высоты визуального маневрирования, скорости и углов крена;
б)	помогает, при необходимости, активно пилотирующему пилоту, информируя его о расположении ВПП;
в)	выполняет операции согласно технологии работы экипажа.
В любом случае решение о производстве посадки или о прекращении захода на посадку, если не обеспечивается безопасность посадки, и выполнении процедуры ухода на второй круг принимает командир ВС.
Для допуска к выполнению визуальных заходов на посадку КВС должен пройти подготовку по утвержденной программе.
8.9.7.16.	Внеочередной заход на посадку
Воздушному судну, которому требуется немедленная посадка, обеспечивается внеочередной заход на посадку. 
Экипаж воздушного судна, сообщивший органу ОВД о недостаточном остатке топлива для ожидания посадки в порядке общей очереди, имеет преимущественное право в выполнении маневра на снижение и заход на посадку перед другими ВС, кроме воздушных судов, которым требуется немедленная посадка. 
При одновременном визуальном заходе на посадку двух воздушных судов преимущество совершить посадку первым имеет воздушное судно, летящее впереди, слева или ниже.
8.9.7.17.	Снижение на конечном участке захода на посадку
(1) При полете в процессе захода на посадку на высотах ниже точки входа в глиссаду (FAF, FAP), ниже высоты круга (при полетах по России) при отрицательных температурах наружного воздуха экипаж ВС обязан выдерживать высоты с учетом температурной поправки.
(2) Особенности полета на конечном этапе захода на посадку в зависимости от вида и способа управления в вертикальной плоскости описаны в п. 8.9.7.6.
(3) Снижение на конечном участке захода на посадку до высоты принятия решения ВПР (DA(H) или минимальной высоты снижения МВС (MDA(H) осуществляется с расчетной вертикальной скоростью, определяемой гради¬ентом снижения (углом наклона траектории) с учетом ветра или соблюдая процедуры «Уточненной методики захода на посадку» на тех аэродромах, где предусмотрено.
(4) При снижении ниже Н твг (Н FAF, FAP) вертикальная скорость снижения не должна превышать 5 м/с (1000 ft/м), при этом, если расчетная вертикальная скорость на конечном участке захода на посадку, предусмотренная схемой конкретного аэродрома более 5м/с(1000ft/м), это должно оговариваться на предпосадочном брифинге.
(5) На контролируемом аэродроме при радиолокационном наведении разрешение на заход выдается органом ОВД одновременно с курсом выхода на конечный участок захода на посадку.
При осуществлении радиолокационного наведения ответственность за безопасный пролет наземных препятствий возлагается на орган ОВД.
(6) При выполнении захода на посадку по посадочному локатору пилот выпол¬няет команды органа ОВД.
При отклонениях от заданной траектории по направлению пилот не предприни¬мает корректирующих действий без специального указания органа ОВД.
При отклонениях от заданной траектории по высоте (отклонения по глиссаде) пилот предпринимает корректирующие действия на основе предоставляемой орга¬ном ОВД информации даже в том случае, когда конкретных указаний об этом не по¬ступает.
(7) Если радиолокационное наведение начато в отношении прилетающего воздушного судна, оно продолжается довыхода воздушного судна к конечному участку захода на посадку при заходе на посадку по приборам;
а)	получения разрешения на визу¬альный заход при выполнении визуального захода на посадку.
В случае потери наведения по курсу и/или глиссаде при наличии визуаль¬ного контакта с ВПП осуществляется переход на правила визуального захода на посадку.
(8) Осуществление радиолокационного наведения, выполняемого после выдачи разрешения захода на посадку, отменяет ранее выданное разрешение.
(9) КВС прекращает заход на посадку на любом аэродроме в месте, в котором будут нарушены ограничения эксплуатационных минимумов данного аэродрома, или, по мнению КВС, не обеспечивается безопасность посадки.
8.9.7.18.	Процедура стабилизированного захода на посадку
Самолет при осуществлении коммерческой воздушной перевозки должен быть стабилизирован на заданной траектории захода на посадку до высоты 300м (1000ft) относительно порога ВПП в приборных (IMC) метеоусловиях полета и не ниже 150м (500ft) в визуальных метеоусловиях (VMC).
Самолет считается стабилизированным для продолжения захода - находится на расчетной глиссаде и посадочном курсе (на расчетной траектории при визуальном заходе на посадку (заход с круга)) и при этом:
а)	для выдерживания траектории снижения требуется лишь небольшие эволюции по курсу / тангажу;
б)	приборная скорость не превышает расчетного значения плюс 20 км/ч (10 узлов) и не менее расчетной скорости захода на посадку;
в)	создана необходимая посадочная конфигурация согласно РЛЭ ВС (AFM, FCOM), выполнены в полном объеме действия по карте контрольных проверок;
г)	вертикальная скорость снижения не превышает 5 м/сек (1000ft в минуту). Если конечный этап захода на посадку требует выдерживать вертикальную скорость снижения более 5м/сек, это должно быть оговорено на предпосадочной подготовке;
д)	режим работы двигателей соответствует посадочной конфигурации самолета, скорости захода и не должен превышать номинального режима или быть ниже режима, установленного для данных условий;
е)	при выполнении захода на посадку с применением кругового маневрирования (circle-to-land), крен должен быть убран на конечном участке захода на посадку до достижения самолетом высоты 300ft относительно порога ВПП.
Если при заходе на посадку параметры полета соответствуют указанным выше, контролирующий пилот при достижении высоты 1000ft в приборных метеорологических условиях или 500ft в визуальных метеорологических условиях должен объявить: «тысяча футов или пятьсот футов – полет стабилизирован».
Если ВС не стабилизировано при приборных МУ захода на посадку до высоты 1000ft или до высоты 500ft в визуальных МУ полета, контролирующий пилот должен объявить: «полет не стабилизирован» – выполнить маневр ухода на второй круг.
Во всех случаях контролирующий пилот обязан своевременно информировать экипаж, если параметры полета выходят за допустимые пределы безопасности в соответствии со стандартными эксплуатационными процедурами и при достижении при этом минимальной высоты стабилизации выполнить уход на второй круг.
На конечном участке неточного захода на посадку стабилизированная траектория полета должна выдерживаться как можно ближе к требуемому профилю, при этом экипаж должен сохранять и корректировать расчетное значение вертикальной скорости снижения в соответствии со стандартными эксплуатационными процедурами.
8.9.7.19.	Предельно допустимые отклонения при заходе на посадку
  
8.9.7.20.	Надежный визуальный контакт на ВПР (МВС) - выполнение совокупности условий:
а)	четкое, непрерывное наблюдение огней, предметов, форм на земной поверхности;
б)	однозначное понимание, что эти огни, предметы и т.п. имеют отношение к ВПП–являются ее ориентирами;
в)	достаточная дальность, на которую просматриваются эти ориентиры.
г)	в зависимости от типа захода на посадку и установленного для этого захода минимума по видимости, как минимум, один из перечисленных ориентиров отчетливо виден и опознан:
	при заходе на посадку с применением визуального маневрирования («circle-to-land») - любые ориентиры, относительно которых возможно определять положение ВС относительно ВПП.
Снижение ниже MDA (H), установленной для визуального маневрирования, допускается только при наличии визуального контакта с порогом ВПП или светосигнальными средствами захода на посадку, связанными с ВПП;
	при заходе на посадку в условиях не ниже САТ I – система огней приближения или её часть, порог ВПП и его маркировка, входные огни ВПП, огни порога ВПП, система визуальной индикации глиссады, зона приземления, её маркировка, огни зоны приземления, посадочные огни ВПП;
	при заходе на посадку по САТ II или САТ III а* - участок системы огней приближения, состоящий, по крайней мере, из трех последовательных осевых огней системы огней приближения, огни зоны приземления и осевые огни ВПП, посадочные огни ВПП.
Исполнение вышеуказанных условий должно обеспечить командиру ВС возможность оценить положение ВС относительно заданной траектории полета (тангаж, крен, боковое уклонение, путевой угол, наклон траектории), а также возможность завершения захода на посадку без помощи пилотажных приборов.
При отсутствии визуального наблюдения пилотом как минимум одного наземного ориентира в течение времени, достаточного для оценки пилотом местоположения ВС и тенденции его изменения по отношению к заданной траектории полета, продолжение захода на посадку ниже DA/Н или MDA/Н является нарушением минимума для посадки.
(1) Обеспечение информацией об абсолютной высоте и сообщение об абсолютной высоте автоматическими средствами или членами летного экипажа ВС
При полете на конечном участке захода на посадку пилот, осуществляющий активное управление ВС (ПАУ, PF), и экипаж обеспечиваются информацией о текущем изменении абсолютной (относительной) высоты полета. Указанная информация предоставляется автоматически (речевыми информаторами) на самолетах, имеющих такое оборудование, или членом экипажа на самолетах, не имеющих систем речевой информации. Процедура реализации данного условия указана в Инструкции по взаимодействию и технологии работы членов экипажа (SOP) конкретного типа ВС (Часть В РПП).
(2) Принятие решения на продолжение полета ниже (DA(H)/ (MDA(H)) с целью производства посадки
Заход на посадку может быть продолжен ниже высоты принятия решения ВПР (DA/H) при точном заходе на посадку (заходе с вертикальным наведением) или ниже минимальной высоты снижения МВС (MDA/H) при неточном заходе на посадку или при построении визуального маневра захода на посадку, если:
а)	положение ВС на глиссаде (траектории визуального маневрирования при ВЗП) обеспечивает посадку в зоне приземления приемлемым маневрированием по устранению боковых уклонений в пределах разрешенных, с вертикальной скоростью, обеспечивающей перегрузку на посадке в допустимых пределах;
б)	полет самолета стабилизирован при достижении высоты 300 м (1000 ft) над уровнем аэродрома при полете в приборных метеорологических условиях или при достижении высоты 150м (500ft) над уровнем аэродрома при полете в визуальных метеорологических условиях;
в)	поддерживается надежный визуальный контакт не менее, чем с одним из ориентиров, имеющих отношение к ВПП (п. 8.9.7.20).
8.9.7.21.	Условия, при которых командир ВС должен прекратить заход на посадку и уйти 
                                                                    на второй круг:
(1) Командир ВС должен прекратить заход на посадку и выполнить уход на второй круг если:
а)	впереди по траектории полета наблюдаются опасные метеорологические явления;
б)	наблюдаются скопления птиц, представляющие угрозу безопасности посадки;
в)	наблюдаются сильные осадки в виде дождя с интенсивностью, ухудшающей метеорологическую видимость до величины менее 600 м без использования бортового радиолокатора и системы заблаговременного предупреждения о сдвиге ветра;
г)	для выдерживания градиента снижения на глиссаде снижения требуется увеличение режима работы двигателей более номинального, если иное не предусмотрено РЛЭ;
д)	значение RVR на втором и / или третьем участке ВПП ниже эксплуатационного минимума для взлета;
е)	до установления надежного визуального контакта с огнями приближения или другими наземными ориентирами сработала сигнализация ВПР и (или) опасного сближения с землей;
ж)	заход на посадку самолета при осуществлении коммерческой воздушной перевозки не стабилизирован (определение п. 8.9.7.18) при достижении высоты 1000 ft над уровнем аэродрома при полете в приборных метеорологических условиях или при достижении высоты 500ft над уровнем аэродрома при полете в визуальных метеорологических условиях;
з)	до достижения DA/H при точном заходе на посадку, при заходе на посадку с вертикальным наведением или неточном заходе на посадку CDFA не установлен надежный визуальный контакт с огнями подхода (огнями ВПП) или наземными ориентирами;
и)	при неточном заходе на посадку в приборных метеорологических условиях до достижения контрольной точки начала визуального снижения VDP (п. п. (6) пункта 8.9.7.1) не установлен надежный визуальный контакт с огнями подхода (огнями ВПП) или наземными ориентирами (п. 8.9.7.20);
к)	положение ВС в пространстве или параметры его движения относительно ВПП не обеспечивают безопасность посадки;
л)	потерян надежный визуальный контакт с огнями подхода (огнями ВПП) или наземными ориентирами при снижении ниже DA/H или MDA/H;
м)	в воздушном пространстве или на летной полосе появились препятствия, угрожающие безопасности полета;
н)	расчет на посадку не обеспечивает безопасность ее выполнения;
о)	экипаж получил сообщение о фактических условиях посадки, при которых состояние поверхности ВПП и значение коэффициента сцепления с учетом скорости и порывов ветра у земли и его направления не соответствуют ограничениям летно-технических характеристик ВС;
п)	к моменту достижения DA(H) / MDA(H) отклонения от заданной или расчетной траектории, расчетных значений поступательной, вертикальной скоростей снижения превышают установленные ограничения;
р)	до пролета ВПР (DA(H)) или до достижения минимальной высоты снижения (МВС(MDA(H)) не получено разрешение на посадку;
с)	командир ВС не уверен в благополучном исходе посадки;
т)	получена информация о высоте нижней границы облаков менее установленного минимального значения (НГО) на аэродромах, где данный параметр опубликован в таблице минимумов с пометкой «CEILING REQUIRED», согласно требованиям государства, в котором данный аэродром расположен.
Примечание. При отсутствии разрешения на посадку на контролируемый аэродром при достижении высоты 200 ft (60м) над аэродромом, но не ниже DA/H или MDA/H выполняется прерванный заход (уход на второй круг).
Предполагается, что полет будет выполняться по опубликованной схеме ухода на второй круг.
В процессе выполнения неточного захода на посадку после достижения MDA/H если не установлен надёжный визуальный контакт с огнями подхода (огнями ВПП) или наземными ориентирами, КВС имеет право продолжить без снижения полёт до точки VDP, по достижении которой принять решение о посадке или уходе на второй круг.
(3) При выполнении неточных заходов на посадку на аэродромах, где отсутствует информация о расположении точки MAPt, экипаж выполняет процедуру ухода на второй круг по достижению MDA(H) аналогично действиям на DA(H).
Примечание. Допускается полет над точкой начала ухода на второй круг (MAPt) на большей высоте, чем предусмотрено схемой, но не более высоты пролета контрольной точки конечного этапа захода на посадку (FAF).
(4) Пилотирующий пилот обязан немедленно начать маневр ухода на второй круг по информации любого члена экипажа о выходе параметров полета за пределы допустимого, даже если, по его мнению, продолжение захода на посадку и посадка могут быть выполнены безопасно. ВС переводится в набор высоты с немедленным докладом органу ОВД о выполнении маневра, за исключением случаев, когда на схеме захода на посадку имеется информация о возможном срабатывании такой сигнализации.
(5) При снижении на предпосадочной прямой, при отказе оборудования ВС или наземных РТС системы захода на посадку, выбранной на предпосадочной подготовке в качестве основной, командир ВС обязан:
а)	при наличии визуального контакта с ВПП перейти на визуальное пилотирование и продолжить заход, используя для контроля траектории полета ВС все оставшиеся в распоряжении экипажа средства;
б)	при отсутствии визуального контакта с ВПП и наличии у экипажа данных о фактических метеоусловиях не ниже минимума, установленного для захода на посадку по системе, выбранной в качестве резервной:
	если фактическая высота полета в момент отказа более чем на 300 ft выше ВПР(DA(H))/МВС (MDA(H)) резервной системы:
	сохранять траекторию снижения по курсу и глиссаде;
	перейти на резервную систему захода на посадку, перенастроив при необходимости бортовые РТС;
	если при переходе на пилотирование по резервной системе, требуется значительное изменение траектории захода или возникают технологические трудности - выполнить уход на второй круг, провести дополнительную предпосадочную подготовку и повторить заход на посадку по резервной системе;
	если фактическая высота полета в момент отказа превышает DA(H)/MDA(H) схемы захода на посадку по резервной системе на 100м (300 ft) или менее:
	выполнить уход на второй круг;
	провести предпосадочную подготовку для захода по резервной системе и повторить заход;
в)	при отсутствии визуального контакта с ВПП и получении экипажем сведений о фактической погоде ниже минимума, установленного для захода по резервной системе выполнить уход на второй круг и принять решение о возможности выполнения посадки на данном аэродроме.
8.9.8.	Выполнение схем вылета и схем захода на посадку с использованием систем зональной навигации (RNAV) на основе приемников базовой GNSS
8.9.8.1. Схемы вылета и неточного захода на посадку с использованием GNSS основаны на применении систем RNAV, которые могут обеспечиваться различным по составу бортовым оборудованием, начиная от автономных приемников базовой GNSS и заканчивая мультисенсорной системой RNAV, использующей информацию, предоставляемую датчиком базовой GNSS. Летные экипажи должны знать специальные функции оборудования.
8.9.8.2. Следует уделить особое внимание эксплуатационным требованиям, которым должна отвечать GNSS:
а)	навигационная база данных содержит текущую информацию по заходу на посадку по неточным посадочным системам (обновляется в сроки AIRAC - Aeronautical Information Regulation And Control). Регламентирование и контроль аэронавигационной информации. Система заблаговременного уведомления об изменениях аэронавигационных данных по единой таблице дат вступления их в силу. Представляет собой установленный график обновления всех аэронавигационных данных в т.ч. бортовых навигационных баз данных для FMC. В соответствии с этим графиком, аэронавигационная информация обновляется каждые 28 дней – таким образом, в каждом году 13 циклов (Cycles));
б)	заход на посадку, который должно выполнить данное ВС, извлекается из базы данных, и по нему определяется местоположение всех навигационных средств и всех пунктов на маршруте, необходимых для данного захода на посадку;
в)	хранящаяся в базе данных информация предоставляется экипажу в том порядке, в котором она опубликована на схеме данного захода на посадку;
г)	пункты на маршруте в навигационной базе данных не могут быть изменены летным экипажем;
д)	соответствующее бортовое оборудование, необходимое для полета по данному маршруту от аэродрома назначения и до любого запасного аэродрома, и для полета по маршруту захода на посадку в этом аэропорту, должно быть установлено на данном ВС и быть в рабочем состоянии. В рабочем состоянии должны также быть и соответствующие наземные навигационные средства;
е)	данный заход на посадку (схема выхода) выбирается из навигационной базы данных. Кодирование базы данных нуждается в официальной публикации процедур захода;
ж)	должен осуществляться контроль целостности (Receiver autonomous integrity monitoring (RAIM) или аналогичным устройством);
з)	на этапе предполетного планирования полета по ППП:
и)	при полете на запасной аэродром на нем должна также существовать процедура захода на посадку без использования GPS;
к)	на аэродроме назначения в расчетное время прибытия используется прогнозирующее устройство RAIM или аналогичное ему, а также возможности контроля (осуществляемые RAIM или аналогичным устройством);
л)	при использовании запасного аэродрома при взлете или запасного аэродрома на маршруте на них должна иметься, по меньшей мере, одна процедура захода на посадку без использования GPS.
8.9.8.3.	Режимы работы и пределы срабатывания сигнализации
(1) Приемник базовой GNSS работает в трех режимах: режимы полета по маршруту, в зоне аэродрома и захода на посадку, исходя из того, что ВС пилотируется в ручном режиме. Пределы срабатывания сигнализации RAIM автоматически связаны с режимами работы приемника и устанавливаются соответственно на ±3.7, ±1.9 и ±0.6 км (±2.0, ±1.0 и ±0.3 м. мили).
(2) Для каждого направления ВПП аэродрома, предназначенного для выполнения неточного захода на посадку методом зональной навигации по СНС, разрабатываются и публикуются в документах аэронавигационной информации специальные стандартные маршруты прибытия (далее – STAR) и карты захода на посадку.
(3) Траектория STAR и захода на посадку по СНС задается последовательностью контрольных точек пути (оптимально 6, но не более 9 точек). Обязательными точками схемы захода на посадку по СНС являются:
IAWP - контрольная точка начального этапа захода на посадку;
IWP    - контрольная точка промежуточного этапа захода на посадку;
FAWP - контрольная точка конечного этапа захода на посадку;
MAWP- точка ухода на второй круг.
(4) Точкам пути схемы подхода и захода на посадку по СНС присваивается код в соответствии с требованиями международного стандарта Авиационной радионавигационной корпорации ARINC-424 (далее – ARINC) и Европейской конвенции по наименованию точек пути:
а)	точки пути обозначаются буквенным индексом (последние две латинские буквы четырехбуквенного кода ICAO аэродрома) и трехзначным порядковым номером. Последняя цифра номера не должна быть 0 или 5 (например: WW157);
б)	точки пути ухода на второй круг обозначаются как “RW (номер порога ВПП)” (например: RW05).
8.9.8.4.	Предполетные процедуры
(1) Пилот/эксплуатант должен соблюдать специальные процедуры приведения в действие, инициализации и самопроверки оборудования, изложенные в руководстве по эксплуатации воздушного судна.
(2) Пилот должен выбрать соответствующий(ие) аэродром(ы), ВПП/схему захода на посадку и начальную контрольную точку захода на посадку на пульте бортового приемника GNSS для определения наличия RAIM для данного захода на посадку. 
(3) В РЛЭ воздушного судна должен быть указан необходимый порядок выполнения соответствующих операций при заходе на посадку и альтернативных действий применительно к оборудованию FMS.
(4) Первая обязанность пилота - иметь печатную схему данного захода, вылета или подхода и сверять по ней истинность информации. В обязанности пилота также входит проверять дату окончания срока гарантии истинности информации в карточке данных Jeppesen, показываемую при запуске прибора.
8.9.8.5.	Вылет
(1) Настройка оборудования.
Приемник базовой GNSS необходимо установить в режим, соответствующий его использованию при вылете, как это указано на схеме вылета (например, приведенная на карте схема может содержать указание о приемлемости использования режима полета в районе аэродрома, если режим вылета не обеспечивается, выбрав чувствительность CDI ±1,9 км (±1,0 м. мили). Для выполнения полета по опубликованному SID маршруты вылета должны быть загружены в действующий план полета из текущей информации базы навигационных данных. На некоторых участках SID может потребоваться определенное вмешательство пилота, особенно в тех случаях, когда осуществляется радиолокационное наведение на линию пути или требуется выйти на конкретную линию пути до точки пути.
(2) Вылеты по прямой 
В тех случаях, когда ориентация начальной линии пути вылета (α≤15°) определяется местоположением первой точки пути, расположенной после DER, отсутствуют какие-либо особые требования, связанные с приемником базовой GNSS.
(3) Вылеты с разворотом 
Развороты устанавливаются как “разворот в точке пути “флай-бай”, “разворот в точке пути “флай-овер” или “на абсолютной/относительной высоте”. При использовании некоторых систем развороты на абсолютной/относительной высоте не могут кодироваться в базе данных, и в таком случае эти развороты должны выполняться вручную.
В том случае, когда датчик GNSS для FMS выходит из строя и результирующая конфигурация оборудования оказывается недостаточной для выполнения или продолжения полета по схемам, пилот должен немедленно уведомить об этом орган УВД и запросить имеющуюся альтернативную схему, отвечающую возможностям системы RNAV. Следует отметить, что в зависимости от типа сертифицированной используемой FMS, руководства по летной эксплуатации воздушных судов и данные, представленные изготовителями, могут предусматривать возможность продолжения полета.
8.9.8.6.	Схемы захода на посадку на основе GNSS
(1) Как правило, выполнение полета по схеме неточного захода на посадку по приборам с использованием базовой GNSS очень похоже на традиционный заход на посадку. Различия заключаются в навигационной информации, отображаемой блоком управления и индикации оборудования GNSS, и терминологии, используемой для описания некоторых элементов. Выполнение захода на посадку с использованием базовой GNSS обычно представляет собой полет с наведением по соответствующим точкам и не зависит от каких-либо наземных навигационных средств, т.е., иными словами, полет с обеспечением зональной навигации.
(2) Заход на посадку не может выполняться, если параметры захода на посадку по приборам не извлекаются из базы данных бортового оборудования, которая:
а)	содержит все точки пути, указанные на схеме предстоящего захода на посадку;
б)	представляет их в той же последовательности, в которой они указаны на опубликованной карте схемы;
в)	содержит обновленную информацию для текущего цикла AIRAC.
(3) В целях обеспечения правильности отображения базы данных GNSS пилоты должны выполнить проверку корректности индицируемых данных для захода на посадку на основе GNSS после загрузки схемы в действующий план полета по данной схеме. 
(4) Приступая к заходу на посадку по базовой GNSS, необходимо сначала выбрать соответствующий аэропорт, ВПП/схему захода на посадку и начальную контрольную точку захода на посадку (IAF). Пилоты должны всегда знать обстановку для определения пеленга и расстояния до IAF в схеме GNSS до начала выполнения этой схемы. 
(5) Пилотам необходимо полностью выполнять схему захода на посадку от IAF, за исключением случаев, когда им выдано другое конкретное разрешение. 
(6) Командир ВС несет ответственность за:
а)	выдерживание схемы GNSS;
б)	выдерживание установленных безопасных высот пролета препятствий на различных участках схемы, в том числе минимальной высоты снижения;
в)	принятие решения о прекращении неточного захода на посадку по GNSS при неуверенности в достоверности информации аппаратуры СНС и перехода на продолжение захода по резервной системе или об уходе на второй круг.
Примечание. В последнем случае КВС сообщает диспетчеру УВД: «Не могу использовать СНС для захода на посадку, перехожу на заход по (резервная система захода)».
(7) После загрузки схемы захода на посадку в бортовую базу навигационных данных предусматривается выполнение перечисленных ниже действий. В зависимости от типа оборудования GNSS некоторые из этих действий (или все) могут осуществляться автоматически:
а)	По достижении расстояния 56 км (30 м. миль) до контрольной точки аэродрома приемники базовой GNSS будут подавать либо сигнал «активизация», либо, когда система автоматически задействует режим работы, индикацию о том, что воздушное судно находится в районе аэродрома.
б)	При этом сигнале пилот должен включить режим захода на посадку. Некоторые, но не все, типы бортового оборудования GNSS будут включать режим захода на посадку автоматически.
в)	Если пилот рано включает режим захода на посадку (например, когда IAF находится на расстоянии более 56 км (30 м. миль) до контрольной точки аэродрома), чувствительность CDI (Course Deviation Indicator) не меняется, пока не достигается расстояние 56 км (30 м. миль). Это не относится к системам, которые автоматически задействуют режим работы.
г)	Когда одновременно включен режим захода на посадку и воздушное судно находится в пределах 56 км (30 м. миль) от контрольной точки аэродрома, на расстоянии 56 км (30 м. миль) приемник базовой GNSS переключается на чувствительность, соответствующую режиму полета в районе аэродрома и связанную с этим настройку RAIM. Если пилот не обеспечивает включение режима захода на посадку на расстоянии или до достижения расстояния 56 км (30 м. миль) от контрольной точки аэродрома, приемник не переключается на режим полета в районе аэродрома. Критерии пролета препятствий предполагают, что приемник работает в режиме полета в районе аэродрома и соответствующие зоны основаны на этом допущении.
д)	По достижении расстояния 3,7 км (2,0 м. миль) до IAF и при том условии, что режим захода на посадку включен (что должно быть сделано, см. п. c) выше), чувствительность CDI и настройка RAIM плавно меняются до достижения в FAF значений 0,6 км (0,3 м. миль), соответствующих заходу на посадку. Кроме того, появится уведомление “включен заход на посадку”.
е)	Пилот должен проверить наличие сигнала “включен заход на посадку” по достижении или до пролета FAF и выполнить уход на второй круг, если он отсутствует или если он отменен в результате отмены автоматически выбранной чувствительности.
ж)	Если CDI не выставлен по центру в тот момент, когда меняется чувствительность CDI, любое отклонение будет увеличиваться и создавать неверное впечатление о том, что воздушное судно отклоняется еще более, хотя оно может следовать удовлетворительно выдерживаемым курсом. Для избежание такой ситуации пилоты должны обеспечить достаточно установившееся выдерживание по правильной линии пути, по крайней мере, за 3,7 км (2,0 м. миль) до FAF.
(8) Пилот должен знать угол крена/скорость разворота, которые конкретный тип бортового оборудования GNSS использует для расчета упреждения разворота, а также учтены ли в производимых расчетах ветер и воздушная скорость. Эта информация должна содержаться в руководстве по эксплуатации бортового оборудования. При завышенном или заниженном угле крена разворот на курс конечного участка захода на посадку может значительно задержать ориентацию по курсу и обусловить высокие скорости снижения для достижения абсолютной высоты следующего участка.
(9) Пилоты должны уделять особое внимание точной работе бортового оборудования базовой GNSS при выполнении схем полетов в зоне ожидания и в случае дополнительных схем захода на посадку и операций, таких, как стандартные развороты и развороты на 180º. Такие схемы могут потребовать вмешательства пилота для прекращения выставления точек пути приемником или возобновления автоматического выставления навигационного оборудования GNSS после выполнения маневра. Одна и та же точка пути может появляться на маршруте полета последовательно несколько раз (IAF, FAF, MAHF при стандартном развороте/развороте на 180º). Необходимо убедиться в том, что приемник выставлен на соответствующую точку пути участка выполняемой схемы, особенно в том случае, если одна или несколько точек “флай-овер” пропущены (FAF, а не IAF, если не выполняется стандартный разворот). Пилоту может потребоваться обойти одну или несколько точек “флай-овер” одной точки пути с тем, чтобы начать выставление оборудования GNSS в надлежащем месте последовательности точек пути.
(10) Некоторые приемники базовой GNSS могут выдавать информацию об абсолютной высоте. Однако пилот должен выдерживать минимальные абсолютные высоты с использованием барометрического высотомера. Оборудование будет автоматически представлять точки пути от точки IAF до контрольной точки ожидания при уходе на второй круг (MAHF), если только пилот вручную уже не предпринял соответствующие действия. В точке MAPt оборудование может автоматически не выставляться на следующую требуемую точку пути. В этом случае может потребоваться вручную выставить оборудование GNSS на следующую точку пути. При радиолокационном наведении может потребоваться вручную выбрать следующую точку пути с тем, чтобы GNSS правильно использовала соответствующие точки в базе данных и связанные с ними траектории полета.
8.9.8.7.	Конечный участок захода на посадку
(1) Конечный участок захода на посадку с использованием GNSS будет обозначаться точкой пути, которая обычно располагается на расстоянии 9,3 км (5,0 м. миль) от порога ВПП.
(2) Чувствительность индикации курса. 
Чувствительность CDI, связанного с оборудованием GNSS, меняется в зависимости от режима работы. На этапе полета по маршруту до выполнения захода на посадку по приборам отклонение на полную шкалу чувствительности индикатора составляет 9,3 км (5,0 м. миль) по обе стороны от осевой линии.
(3) При включении режима захода на посадку чувствительность индикатора меняется от отклонения 9,3 км (5,0 м. миль) до 1,9 км (1,0 м. миля) на полную шкалу по обе стороны от осевой линии.
(4) На расстоянии 3,7 км (2,0 м. мили) по линии пути приближения к точке FAF чувствительность индикатора начинает меняться до уровня, при котором отклонение на полную шкалу составляет 0,6 км (0,3 м. мили) по обе стороны от осевой линии. Некоторые типы бортового оборудования GNSS могут обеспечивать между точками FAF и MAPt индикацию угловых данных, которая примерно соответствует чувствительности индикации курса при использовании курсового маяка ILS.
(5) Контрольные точки ступенчатого снижения. 
Пролет контрольной точки ступенчатого снижения осуществляется так же, как и в случае захода на посадку на основе наземных средств. Любые необходимые контрольные точки ступенчатого снижения до точки пути ухода на второй круг будут задаваться с помощью расстояний вдоль линии пути. 
В том случае, когда FMS предусматривает возможность вертикальной навигации, содержащаяся в базе навигационных данных схема может включать траекторию полета с постоянным снижением, которая проходит выше профиля схемы ступенчатого снижения в вертикальной плоскости. Использование возможности вертикальной навигации с помощью FMS будет зависеть от подготовленности, обучения и эксплуатационного утверждения летного экипажа.
(6) Градиент/угол снижения. 
Оптимальный градиент/угол снижения составляет 5,2%/3°, однако, если требуется более высокий градиент/угол, его максимально допустимое значение составляет 6,5%/4°30'. 
В тех случаях, когда FMS обеспечивает возможность определения траектории полета в вертикальной плоскости, эта траектория будет задаваться соответствующим углом. Типичное значение угла составляет 3º. В тех случаях, когда на карту наносится профиль постоянного снижения, он изображается с указанием соответствующего угла.
(7) В отношении системы FMS применяются критерии, изложенные в п.п. 8.9.8.6(1), 8.9.8.6(2). Соответствующая разрешающая способность по курсу может обеспечиваться путем выбора летным экипажем соответствующего масштаба электронной карты. В тех случаях, когда выбранные масштабы карты являются непригодными (т. е., слишком большими или разрешающая способность является недостаточной), использование командного пилотажного прибора или FMS/автопилота может позволить улучшить ситуацию.
(8) При отказе аппаратуры СНС на предпосадочной прямой КВС обязан сообщить об этом диспетчеру УВД и принять решение на продолжение захода на посадку по резервной системе или об уходе на второй круг для повторного захода на посадку по резервной системе установленной схемы.
8.9.8.8.	Участок ухода на второй круг
(1) Уход на второй круг с использованием GNSS требует от пилота принятия действий по выставлению приемника базовой GNSS после прохождения MAPt на участок схемы, связанный с уходом на второй круг. Пилот должен досконально знать процедуру задействования конкретного типа оборудования базовой GNSS, установленного на борту воздушного судна, и приступить к принятию соответствующих действий после MAPt. Задействование режима ухода на второй круг до MAPt приведет к немедленному изменению чувствительности CDI на чувствительность в районе аэродрома (±1,0 м. миля) и навигационное наведение будет продолжаться до MAPt. Наведение не будет обеспечиваться после прохождения MAPt или инициировать разворот при уходе на второй круг без вмешательства пилота. Если режим ухода на второй круг не задействован, бортовое оборудование базовой GNSS будет отображать продолжение конечного участка приближения и расстояние вдоль линии пути будет увеличиваться от MAPt, пока он не будет выставлен вручную после прохождения MAPt.
(2) При использовании приемника базовой GNSS система маршрутов ухода на второй круг, в которой первая линия пути проходит по установленному курсу, а не прямо до следующей точки пути, требует от пилота принятия дополнительных действий по установлению курса. На этом этапе полета особенно важно знать все необходимые входные данные.
При использовании FMS выбранная из базы навигационных данных схема обычно включает маршруты ухода на второй круг и поэтому от пилота не требуется принимать какие-либо действия по определению траектории полета при уходе на второй круг.
8.9.8.9.	Уход на запасной аэродром и взаимодействие с органами ОВД 
а)	уход на запасной аэродром выполняется в соответствии со схемами выхода из района аэродрома; 
б)	при принятии решения об уходе в процессе полета на конечном этапе захода на посадку – выполняется процедура ухода на второй круг в соответствии с РЛЭ (AFM, FCOM) типа ВС. 
Процедура ухода на второй круг на контролируемом аэродроме выполняется по установленной схеме или по указанию органа ОВД и предусматривает вывод воздушного судна в точку, с которой возможно выполнить повторный заход на посадку или полет в зону ожидания, или полет на запасный аэродром. 
в)	О принятом решении командир ВС обязан доложить диспетчеру ОВД и дальнейший полет согласовывать с диспетчером ОВД.
8.9.9.	Посадка ВС
8.9.9.1. Экипаж ВС должен получить разрешение на посадку от диспетчера ОВД и подтвердить разрешение:
а)	при точном заходе на посадку, заходе на посадку с использованием бокового и вертикального наведения - до достижения ВПР(DA(H));
б)	при неточном заходе на посадку - до достижения МВС(MDA(H));
в)	при визуальном заходе на посадку, заходе на посадку с применением визуального маневрирования, заходе на посадку по ПВП - после вывода ВС (в процессе доворота) на конечный участок захода на посадку, на удалении от порога ВПП не менее 1000 м.
8.9.9.2. В зависимости от сложившейся ситуации диспетчер ОВД может проинформировать экипаж ВС: «посадка дополнительно». В этом случае разрешение на посадку должно быть дано до пролёта ВПР (МВС), но в любом случае не позднее пролёта рубежа 1000 м от порога ВПП.
8.9.9.3. Порядок перехода от автоматического режима управления полетом к штурвальному и действия экипажа при частичных отказах автоматических систем управления полетов определяется РЛЭ ВС (AFM, FCOM).
8.9.9.4. Ограничения на посадке по компоненту ветра, при снижении Ксц, видимости, определяются РЛЭ ВС (AFM, FCOM) исходя из ограничений использования автоматических систем захода на посадку.
8.9.9.5. Посадка воздушных судов ночью выполняется, как правило, с включенными посадочными фарами. При посадке в тумане и других метеоявлениях, создающих «световой экран», высота включения фар и порядок их использования определяются командиром ВС.
8.9.9.6. При выполнении посадки экипаж ВС выполняет требования РЛЭ ВС (AFM, FCOM) по предупреждению выкатывания за пределы ВПП; возникновению гидроглиссирования на ВПП, покрытой слоем осадков; боковых заносов; юза колес; определению скорости начала торможения.
8.9.9.7. Посадка воздушного судна производится в границах зоны приземления ВПП (площадки) или в пределах установленных нормативов, обеспечивающих оста¬новку воздушного судна после пробега на рабочей площади аэродрома (площадки). 
8.9.9.8. Пролет порога ВПП производится на высоте, рекомендованной РЛЭ ВС, но не ниже 10 метров.
8.9.9.9. В случае отсутствия уверенности командира ВС в выполнении посадки в пределах установленной зоны приземления, должен быть выполнен уход на второй круг.
8.9.9.10. После посадки экипаж ВС должен установить связь с органом ОВД для получения инструкции по освобождению ВПП и маршруту руления.
8.9.9.11. В целях тренировки второму пилоту разрешается выполнять посадку.
а)	Под контролем инструктора при условиях:
	боковая составляющая ветра не более 80% от предельно-допустимой для фактического состояния поверхности ВПП;
	располагаемая посадочная дистанция превышает потребную посадочную дистанцию не менее чем на15%;
	высота нижней границы облаков (вертикальная видимость) на 10 м (30 ft) и более превышает высоту принятия решения (минимальную высоту снижения) применяемого минимума захода на посадку;
	видимость (видимость на ВПП), превышает на 200 м и более минимум захода на посадку (но не менее 550 м).
б) Под контролем командира ВС при условиях:
	боковая составляющая ветра не более 50% от предельно-допустимой для фактического состояния поверхности ВПП;
	располагаемая посадочная дистанция превышает потребную посадочную дистанцию не менее чем на 15%;
	высота нижней границы облаков (вертикальная видимость) на 30 м (100 ft) и более превышает высоту принятия решения (минимальную высоту снижения) применяемого минимума захода на посадку;
	видимость (видимость на ВПП) превышает на 300 м и более минимум захода на посадку (но не менее 800 м).
8.9.9.12.	Зона приземления
Ширина зоны приземления
¼ ширины полосы в обе стороны от осевой линии																					
																	
																							
																							
			20																				
																							
																							
																							
																			
																			
		150м					800 м													
8.9.9.13. Определение минимальных безопасных высот пролета препятствий при выполнении ВЗП.
Вход в зону аэродрома осуществляется по установленным схемам (STAR) или по траекториям, задаваемым службой ОВД в соответствии с существующими правилами.
Полет до зоны визуального маневрирования в зоне круга осуществляется по правилам ПП.
                                                                                                                                          Таблица А8.9-Т7
Категория ВС	V мах при полете в зоне ВЗП (км/ч)	Радиус для построения зоны ВЗП (км)	Минимальное значение Нм.с     (м)	Запас высоты над 
препятствиями в зоне ВЗП (м)
А	185	3,12	120	90
B	250	4.9	150	90
C	335	7.85	180	120
8.9.10.	Ограничения по минимальным скоростям полета
8.9.10.1. На всех этапах полета не допускать уменьшения скорости ниже рекомендованной для соответствующей конфигурации самолета. В случае непреднамеренного снижения скорости до минимальной сработает предупредительная сигнализация (АУАСП). В случаях срабатывания сигнализации немедленно принять меры по увеличению скорости.
8.9.10.2. В полете как с убранными, так и с выпущенными закрылками на разрешенных малых скоростях может быть явление тряски.
8.9.10.3. Выполнение маневра на скоростях, близких к минимальным, требует от пилота повышенного внимания. Маневр (разворот, переход из режима снижения в горизонтальный полет и др.) выполнять с минимальными перегрузками (не допуская срабатывания предупредительной сигнализации (АУАСП) , плавным движением рулей и с креном, не превышающим рекомендаций РЛЭ (AFM, FCOM).
8.9.11.	Ограничения по вертикальным скоростям снижения при заходе на посадку
8.9.11.1. Пилотам следует уделять внимание тому, чтобы при любых полученных диспетчерских разрешениях или начатом быстром снижении не допускать такой потери высоты, при которой достигается уровень ниже безопасной высоты для того района, где выполняется полет.
8.9.11.2. В процессе снижения экипажи ВС во избежание срабатывания БСПС выдерживают рекомендованные ограничения по вертикальной скорости 7 м/с за 300 м до заданного эшелона (высоты).
8.9.11.3. При полёте вблизи поверхности экипаж должен избегать предельных вертикальных скоростей снижения, а на высотах ниже 3000 м выдерживает вертикальную скорость снижения VУ не превышая максимальных значений, указанных в Таблице А8.9-Т8. 

                                                                                                                                                    Таблица А8.9-Т8
                           Высота полёта	Максимальное значение вертикальной
скорости снижения
  Выше 3000м (10000ф)	в соответствии с РЛЭ
  От 3000м (10000ф) до Н эш. перехода (Н IAF)	15 М/С (3000 Ф/МИН)
  От Н эш. перех. (Н IAF) до Н круга	10 М/С (2000 Ф/МИН)
  От Н круга до Н твг (H FAF)	7 М/С (1379 Ф/МИН)
  Ниже Н твг (H FAF)	5 м/с (1000 ф/мин), если иное не   предусмотрено схемой
При этом, если расчетная вертикальная скорость на конечном участке захода на посадку на конкретном аэродроме более 5 м/с, это должно оговариваться на предпосадочном брифинге.
8.9.11.4. Во всех случаях, вертикальные скорости снижения при заходе на посадку не должны превышатьограничений приведенных в РЛЭ ВС (AFM, FCOM).
8.9.11.5. При превышении расчетной вертикальной скорости снижения по глиссаде после ДПРМ на 3 м/с (600 Ф/МИН) и более - немедленно уйти на второй круг с использованием взлетного режима.
8.9.12.	Ограничение высоких скоростей снижения вблизи поверхности земли
Пилотам всегда следует уделять пристальное внимание тому, чтобы при любых полученных диспетчерских разрешениях или начатом быстром снижении не допускать такой потери высоты, при которой достигается уровень ниже безопасной высоты для того района, где выполняется полет.
8.9.13.	Системы спутникового слежения
8.9.13.1. На воздушных судах АО «ЮТэйр» применяются системы спутникового слежения - Indigo Xplorer TS и Ske Trac ISAT-200A, которые обеспечивают эффективное и экономически выгодное отслеживание воздушных судов, а также предоставляют инструменты для надежной связи и сбора данных о выполненных полетов, мониторинга места нахождения ВС в полете в реальном времени наземным диспетчером.
8.9.13.2. Спутниковая система слежения Indigo Xplorer TS применяется на самолетах Ан-2 и Ан-74-200, представляет собой переносной приемопередатчик GPS со сменным аккумулятором, который обеспечивает передачу на сайт компании Indigo данных о местонахождении ВС с отображением на карте, а также текстовые сообщения, сигнал «Тревога».
(1) Внутренний аккумулятор блока Indigo Xplorer TS обеспечивает обмен данных со спутником даже после отключения питания самолетной сети. При необходимости интервал отправки сообщений устанавливается пользователем, но не более 15 минут (по умолчанию – интервал 5 минут).
Программное обеспечение (ПО) по сопровождению полетов отражает текущие и архивные данные
положения, включая широту, долготу, время GPS, относительное положение на известную точку маршрута в табличном формате и на карте.
Отчеты о прохождении полетов сохраняются на сервере производителя ПО и отображаются для текущего
просмотра наземным диспетчером АО «ЮТэйр» в заданном периоде времени (полетное время в день/день/неделю/месяц).
(2) Зарядка аккумулятора производится от сети 220V через разъем - USB Interface, с контролем уровня
зарядки. При вылете ВС зарядка аккумулятора блока должна быть максимальной или достаточной для обеспечения передачи сообщений на всем протяжении полета.
(3) Зарядка аккумулятора, размещение прибора (GPS) в кабине пилотов ВС и эксплуатация его в полете
осуществляется вторым пилотом.
(4) На борту ВС прибор (GPS) должен находиться в специально отведенном для этого месте, в положении, обеспечивающим максимальную видимость неба.
8.9.13.3. На самолетах Ан-74ТК-100 установлен один комплект системы спутникового слежения SkyTrac ISAT- 200A, в состав которого входят: приемопередатчик ISAT-200A, комбинированная антенна, диспетчерский голосовой интерфейс DVI-300, панель дисплея в кабине экипажа CDP-300, съемная внутренняя литиево-ионная батарея.
(1) Диапазон рабочих температур оборудования составляет от - 20 до +55 С. Допускается краткосрочное
понижение и повышение температур, но не более чем на 30 минут, до – 40 и +70 С.
(2) Аккумулятор и система зарядки обеспечивает обмен данных со спутником сети Iridium даже после
отключения бортовой сети.
(3) Программное обеспечение по сопровождению полетов отражает текущие и архивные данные о
положении ВС по маршруту, включая широту, долготу, время GPS, время полета, ожидаемое время прибытия.
(ETA), расстояние относительное положение ВС на известной точке по маршруту, скорость движения, высоту, курс в табличном формате и на карте. 
 
Отчеты о прохождении полетов сохраняются на сервере сети Iridium и отображаются для текущего
просмотра наземным диспетчером АО «ЮТэйр» в заданном периоде времени (полетное время в день/день/
неделю/месяц).
(4) Данные передаются с ВС в любую точку мира через Интернет с помощью спутников сети Iridium и
отображаются на любом компьютере с интернет браузером и доступом в Интернет.
(5) Интервалы отправки сообщений о местонахождении ВС определяются пользователем, но не более 15
минут (по умолчанию – интервал 1 минута).
