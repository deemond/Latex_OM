\section{Принятие решения на вылет и выбор запасных аэродромов}


\subsection{Общие правила}

\paragraph{} КВС принимает решение на вылет на основании:
\begin{itemize}
    \item готовности экипажа к выполнению полета;
    \item технической готовности ВС;
    \item анализа метеообстановки;
    \item анализа воздушной обстановки и обеспечения полета;
    \item анализа адекватности аэродромов вылета, назначения и запасных, включая анализ:
    \begin{itemize}
        \item состояния летного поля, ВПП, рулежных дорожек, перронов;
        \item наличия и работоспособности навигационных средств, состояния светотехнического оборудования;
        \item обеспечения требуемого эксплуатационного минимума для посадки;
        \item наличия ограничений, запретов на выполнение полетов, необходимости предварительного запроса;
        \item соответствия категории аэродрома по уровню требуемой пожарной защиты типу ВС.
    \end{itemize}
\end{itemize}

\paragraph{} Командиру ВС разрешается выбирать для взлета и посадки на самолете площадку, о которой отсутствует аэронавигационная информация, если она осмотрена с земли или подобрана с воздуха и признана командиром ВС удовлетворяющей требованиям РЛЭ ВС. Для допуска к посадкам на площадку, подобранную с воздуха, пилот самолета должен пройти соответствующую подготовку под руководством инструктора.
Коммерческие перевозки пассажиров на самолетах с подбором площадок с воздуха запрещены.

Если при подготовке к полету оказалось, что взлетная масса воздушного судна превышает допустимую для фактических условий на старте, командир ВС имеет право принять решение о переносе вылета или снятии части груза.

Если метеоусловия на аэродромах вылета, назначения и (или) запасных, а также по маршруту в период между первоначальным получением метеорологической информации для принятия решения на вылет и вылетом ВС по информации диспетчера ОВД ухудшились и не соответствуют правилам принятия решения на вылет, командир ВС в этом случае обязан пересмотреть решение на вылет.

При задержке более чем на 20 минут от времени вылета, предусмотренного планом полета, командир воздушного судна обязан получить повторное разрешение на вылет. Метеоинформацию и повторное разрешение на вылет разрешается получать по радио (другим средствам связи).

В случае задержки на 30 мин. и более после расчетного времени начала выполнения контролируемого полета или на 1 час и более начала неконтролируемого полета план полета должен быть изменен или представлен новый план полета. Командир ВС обязан повторно принять решение на вылет.

\paragraph{} Принятие решения на вылет, взлет и посадку воздушного судна производится по наивысшему из установленных минимумов:  
\begin{itemize}
    \item командира воздушного судна (минимума, указанного в задании на полет; при принятии решения по минимуму КВС ниже CAT I, необходимо наличие допуска к таким заходам на посадку у остальных членов экипажа);
    \item аэродрома (эксплуатационного);
    \item воздушного судна.
\end{itemize}

При принятии решения на выполнение взлета (посадки) соответствие фактического ветра установленным ограничениям определяется с учетом его порывов.

Командиру ВС, выполняющему международные рейсы, разрешается принимать самостоятельное решение на продолжение рейса в случаях задержек в иностранных аэропортах с учетом того, чтобы общее рабочее время не превышало установленных норм рабочего времени и, находясь в контакте с представителем Авиакомпании, диспетчером ОКВР, принимать меры по сокращению возможных задержек.

\subsection{Выбор запасных аэродромов при вылете}


\paragraph{} \label{par:altrule2} Для самолетов запасной аэродром при взлете должен выбиратЬся и указыватЬся в рабочем плане полета в тех случаях, когда метеорологические условия на аэродроме вылета равны установленному эксплуатационному минимуму аэродрома для посадки, который может быть применен, или ниже его, или не представляется возможным вернуться на аэродром вылета по другим причинам. 

Для самолетов запасной аэродром при взлете выбирается в пределах следующего расстояния от аэродрома вылета при расчете в стандартных атмосферных условиях, в штиль, с использованием фактической взлетной массы:
\begin{itemize}
    \item для самолетов с двумя силовыми установками - не дальше расстояния, эквивалентного одному часу времени полета на крейсерской скорости с одним неработающим двигателем, определенном в соответствии с РЛЭ ВС (AFM, FCOM) или установленному эксплуатантом времени, но не более 2-х часов полета, если эксплуатант имеет разрешение на полеты по правилам EDTO не менее 120 минут;
    \item для самолетов с тремя или более силовыми установками - не дальше расстояния, эквивалентного 2-м часам времени полета на крейсерской скорости с одним неработающим двигателем.
\end{itemize} 

Запасной аэродром для взлета выбирается при соответствии фактической погоды или прогноза погоды эксплуатационному минимуму аэродрома для посадки, который может быть применен в течение периода времени, начинающегося за 1 час до и заканчивающегося через 1 час после расчетного времени прибытия с учетом ограничений в случае отказа одного двигателя. 

\paragraph{} \label{par:altrule} Кроме случаев, указанных в «Руководстве по выполнению полетов по правилам EDTO», выбирается пригодный для посадки запасной аэродром по маршруту следования таким образом, чтобы с любой точки маршрута до выбранного запасного аэродрома на маршруте время полета с одним отказавшим двигателем в стандартной атмосфере в штиль не превышало 60 минут для ВС с двумя газотурбинными двигателями или 180 минут с тремя и более двигателями.

Запасные аэродромы на маршруте указываются в рабочем плане полета.

\paragraph{} В качестве пригодного для посадки может использоваться аэродром, на котором;
\begin{itemize}
    \item посадочные характеристики ВС позволяют выполнить безопасную посадку;
    \item имеются светотехническое оборудование и средства связи;
    \item имеются метеорологическое и аварийно-спасательное обеспечение, навигационные средства;
    \item имеется хотя бы одна схема захода на посадку по приборам;
    \item имеются необходимые виды и средства обслуживания, соответствующие техническим характеристикам воздушного судна и который находится в рабочем состоянии в ожидаемое время использования.
\end{itemize}

Запасной аэродром должен находиться в рабочем состоянии в ожидаемое время использования.

Запасными аэродромами на маршруте могут быть аэродромы вылета и пункта назначения.

Запасной аэродром по маршруту (3\% ERA) - подходящий аэродром по маршруту, который может потребоваться на этапе планирования, выбранный в целях снижения до 3\% Route Reserve на случай непредвиденных обстоятельств.

Особенности расположения «3\%-го» запасного аэродрома по маршруту (3\% ERA) для уменьшения Route Reserve до 3\% описаны в пункте 8.4.9.7 настоящей главы.


\begin{table}[H]
    \begin{center}
    \caption{Зона оперирования ВС при выборе запасных аэродромов для взлета и по маршруту}
    \begin{tabular}{|p{0.20\textwidth}|p{0.32\textwidth}|p{0.32\textwidth}|}
    \hline
    &\multicolumn{2}{c|}{Максимальное время полета до запасного аэродрома}\\
    \cline{2-3}
    Тип ВС&\multicolumn{2}{c|}{60 минут}\\
    \cline{2-3}
    &Для взлета&Маршрут\\
    \hline
    Ан-24/26	&330км	&330км\\
    \hline
    Ан-74	    &400км	&400км\\
    \hline\hline


    \end{tabular}
    \end{center}
\end{table}

\textbf{Примечание:} \textit{Указанные в таблице границы зоны оперирования и скорости ВС используются только на этапе планирования полета для построения маршрута и не являются эксплуатационным ограничением при выполнении ухода на пригодный для посадки аэродром. В зависимости от крейсерского эшелона полета и ряда других факторов фактическое время полета при уходе на запасной аэродром может превышать установленное максимальное время полета до запасного аэродрома.}

\subsection{Выбор запасного аэродрома пункта назначения}\label{sect:alt}

\paragraph{} При полете по ППП выбирается и указывается в планах полета, по крайней мере, один запасной аэродром пункта назначения, уход на который возможен с ВПР аэродрома назначения или с заранее запланированной точки на маршруте (рубежа ухода), за исключением случаев, когда продолжительность полета не превышает 6 часов, аэродром назначения имеет две независимые ВПП, пригодные для посадки ВС, и получена информация о фактической, прогнозируемой погоде, дающая уверенность в том, что в течение периода времени, начинающегося за 1 час до и заканчивающегося через 1 час после расчетного времени прибытия, видимость будет не менее 5000м, а нижняя граница облаков (вертикальная видимость) будет не ниже 600м и превышать MDH для захода на посадку с применением визуального маневрирования («circle-to-land») не менее чем на 150 м, а в случае, если такая высота не опубликована, то не ниже безопасной высоты в районе аэродрома (секторе захода на посадку).

В качестве запасного аэродрома пункта назначения может использоваться аэродром пункта назначения при наличии двух независимых ВПП.
Примечание. Независимыми ВПП являются две или более ВПП на том же самом аэродроме, расположенные таким образом, что если одна ВПП закрыта, то производство полетов можно обеспечивать с помощью другой.

При расчетной продолжительности полета с рубежа ухода до аэродрома назначения более 2 часов информация о фактической и прогнозируемой погоде на аэродроме назначения должна указывать на то, что в течение периода времени, начинающегося за 2 часа до и заканчивающегося через 2 часа после расчетного времени прибытия, нижняя граница облаков (вертикальная видимость) и видимость будут соответствовать требованиям настоящего раздела, но не ниже 200 м и не менее 2500 м соответственно.

\paragraph{} \label{par:alt1}За исключением случаев, указанных в \hyperref[par:alt2]{пункте \ref*{par:alt2}}  настоящего раздела, полет по ППП не начинается до тех пор, пока КВС не будет получена информация, указывающая на то, что:
\begin{itemize}
    \item условия на аэродроме намеченной посадки к расчетному времени прилета будут соответствовать эксплуатационному минимуму аэродрома или превышать их;
    \item условия на запасном аэродроме пункта назначения, если таковой требуется, к расчетному времени прилета будут соответствовать при планируемом заходе на посадку:
    \begin{itemize}
        \item по САТ II и/или III* - нижней границе облаков (вертикальной видимости) не ниже 60 м, видимости (RVR) - не менее эксплуатационного минимума аэродрома для посадки при САТ I;
        \item по радиомаячным системам инструментального захода на посадку (кроме САТ II и/или III*) – нижней границе облаков (вертикальной видимости) не ниже MDH для захода по схеме неточного захода на посадку, видимости(RVR)-не менее эксплуатационного минимума захода по схеме неточного захода на посадку;
        \item при заходе по схеме неточного захода на посадку-нижней границе облаков (вертикальной видимости) превышающей MDH для захода по схеме неточного захода на посадку не менее чем на 50м, видимости (видимости на ВПП) – превышающей эксплуатационный минимум для посадки при выполнении захода по схеме неточного захода на посадку не менее чем на 500 м;
        \item с применением визуального маневрирования («circle-to-land») - нижней границе облаков (вертикальной видимости) превышающей MDH для захода на посадку с применением визуального маневрирования не менее чем на 100 м, видимости, превышающей эксплуатационный минимум для захода на посадку с применением визуального маневрирования не менее чем на 1000 м.
    \end{itemize}
\end{itemize}

\paragraph{} \label{par:alt2}При отсутствии информации о метеорологических условиях (МУ) аэродрома назначения или при наличии информации, свидетельствующей о погоде ниже минимума для посадки к расчетному времени прибытия, выбираются два запасных аэродрома пункта назначения с МУ, соответствующими требованиям \hyperref[par:alt1]{пункта \ref*{par:alt1}}  настоящего раздела РПП, или один запасной аэродром, на котором видимость будет не менее 5000 м, а нижняя граница облаков (вертикальная видимость) будет не ниже 450 м. и превышать MDH для захода на посадку с применением визуального маневрирования не менее чем на 150м, а если такая высота не опубликована, то не ниже БВП в районе аэродрома (в секторе захода на посадку).

В качестве указанной информации используются сведения, полученные от полномочного метеорологического органа, которые Авиакомпания признает достоверными.

При выборе запасных аэродромов используются эксплуатационные минимумы аэродрома для посадки, применимые для конкретной ВПП с учетом направления и скорости ветра.

Аэродром вылета может быть запасным аэродромом для взлета, на маршруте или для аэродрома назначения для вылетающего воздушного судна.

\paragraph{} На изолированные аэродромы Авиакомпания полеты не выполняет

\subsection{Правила определения пригодности аэродрома}

\paragraph{} При определении пригодности аэродрома учитывается:
\begin{itemize}
    \item характеристики аэродрома (расположение, превышение, климатические характеристики, характеристики ВПП, РД, перронов и т.п.);
    \item несущая способность искусственного покрытия ВПП путем определения соответствия классификационного числа покрытия (англ. Pavement Classification Number (PCN)) и классификационного числа ВС (англ. Aircraft Classification Number (ACN)), если ограничения аэродрома не публикуют дополнительную информацию;
    \item наличие достаточного объема действующей аэронавигационной информации по данному аэродрому;
    \item ограничения по выполнению полетов, включая ограничения по шуму, установленные на данном аэродроме;
    \item уровень обеспечения средствами спасания и пожарной защиты;
    \item наличие требуемых видов обеспечения полетов ВС и организации воздушного движения на данном аэродроме для эксплуатируемого типа ВС и вида выполняемых полетов.
\end{itemize}


\paragraph{} Оценка состояния покрытия ВПП и показатели эффективности торможения.

Оценка состояния элементов летного поля производится по значениям величин, получаемых в процессе измерений, параметров оценки.

К параметрам оценки состояния покрытий относятся:
\begin{itemize}
    \item фрикционные свойства покрытий;
    \item вид осадков;
    \item толщина слоя осадков;
    \item доля площади, покрытая загрязнениями.
\end{itemize}


Фрикционные (тормозные) свойства покрытий оцениваются величиной коэффициента сцепления.

Вид осадков оценивается кодовыми цифрами от 1 до 9, с соответствующей каждому числу описательной характеристикой осадков. 
Толщина слоя осадков оценивается числом, соответствующим толщине слоя в миллиметрах. Доля площади, покрытая осадками, оценивается в процентах.

Коэффициент сцепления, в зависимости от применяемых средств, определяется непосредственным отсчетом результатов измерений (\hyperref[tbl:04T2]{Таблица \ref*{tbl:04T2}}), либо приведением результатов измерений к нормативным значениям с помощью корреляционных зависимостей (\hyperref[tbl:04T3]{Таблица \ref*{tbl:04T3}}).

Значения нормативного коэффициента сцепления отражают относительное улучшение или ухудшение эффективности торможения.
При отсутствии в аэропорту инструментальных средств оценки фрикционных средств дается расчетная эффективность торможения - кодовая оценка состояния покрытия ВПП (\hyperref[tbl:04T2]{Таблица \ref*{tbl:04T2}}).

\begin{table}[H]
    \begin{center}
    \caption{} \label{tbl:04T2}
    \small
    \begin{tabular}{|p{0.07\textwidth}|p{0.20\textwidth}|p{0.25\textwidth}|p{0.37\textwidth}|}
    \hline
    Код	&Измеренный коэффициент сцепления	&Расчетная эффективность торможения	&Эксплуатационные ограничения\\
    \hline
    5	&0.60 - 0.40	                    &Хорошая	                        &Отсутствие затруднений по путевому управлению\\
    4	&0.39 - 0.36	                    &Средняя хорошая	                &Отсутствие затруднений по путевому управлению\\
    3	&0.35 - 0.30	                    &Средняя	                        &Возможно ухудшение путевого управления\\
    2	&0.29 - 0.26	                    &Средняя плохая	                    &Возможно ухудшение путевого управления\\
    1	&0.25 - 0.18	                    &Плохая	                            &Путевое управление будет плохим\\
    9	&0.17 и ниже	                    &Ненадежная	                        &Путевое управление не контролируется\\
    \hline\hline
    \end{tabular}
    \end{center}
\end{table}

\begin{table}[H]
    \begin{center}
    \caption{} \label{tbl:04T3}
    \small
    \begin{tabular}{|p{0.20\textwidth}|p{0.25\textwidth}|p{0.37\textwidth}|}
    \hline
    Нормативный коэффициент сцепления	&Расчетная эффективность торможения	&Эксплуатационные ограничения\\
    \hline    
    0.57 - 0.42	                        &Хорошая	                        &Отсутствие затруднений по путевому управлению\\
    0.41 - 0.40	                        &Средняя хорошая	                &Отсутствие затруднений по путевому управлению\\
    0.39 - 0.37	                        &Средняя	                        &Возможно ухудшение путевого управления\\
    0.36 - 0.35	                        &Средняя плохая	                    &Возможно ухудшение путевого управления\\
    0.34 - 0.30	                        &Плохая	                            &Путевое управление будет плохим\\
    0.29 и ниже	                        &Ненадежная	                        &Путевое управление не контролируется\\        \hline\hline
    \end{tabular}
    \end{center}
\end{table}


\paragraph{} Оценка состояния покрытия по описательной характеристике 

Кодовая оценка состояния покрытия ВПП составляется на основании субъективного опыта лица, выполняющего оценку. Для составления кодовой оценки справочно может использоваться таблица соответствия нормативного коэффициента сцепления описательной характеристике состояния покрытия (\hyperref[tbl:04T4]{Таблица \ref*{tbl:04T4}}). 

\begin{table}[H]
    \begin{center}
    \caption{} \label{tbl:04T4}
    \small
    \begin{tabular}{|p{0.40\textwidth}|p{0.20\textwidth}|}
    \hline
    Описательная характеристика состояния поверхности	&Нормативный коэффициент сцепления\\
    \hline    
    Сухой цементобетон или асфальтобетон	            &0,6 и выше\\
    Влажный цементнобетон или асфальт	                &0,4 - 0,6\\
    Мокрый асфальтобетон	                            &0,3 - 0,6\\
    Асфальтобетон, местами лужи	                        &0,28 - 0,4\\
    Уплотненный снег при t ниже- 15°C	                &0,3 - 0,5\\
    Уплотненный снег при t выше - 14°C	                &0,2 - 0,25\\
    Лед при t выше - 10°C	                            &0,1 - 0,2\\
    Лед тающий	                                        &0,05 - 0,1\\
    \hline\hline
    \end{tabular}
    \end{center}
\end{table}

\begin{description}
    \item[Влажно] соответствует состоянию, когда поверхность изменяет цвет вследствие наличия влаги.
    \item[Мокро] поверхность пропитана водой, но стоячая вода отсутствует. 
    \item[Участки воды] видны участки стоячей воды.
    \item[Иней или изморозь] снеговидные кристаллические льдообразования на поверхности покрытия, образующиеся, как правило, в утренние часы и связанные с охлаждением поверхности. 
    \item[Сухой снег] снег, который будучи в рыхлом состоянии может сдуваться ветром или рассыпаться; плотность – до 0,35, но, не включая 0,35.
    \item[Мокрый снег] снег, который не рассыпается и образует или имеет тенденцию образовывать снежный ком; плотность - от 0,35 и до, но, не включая 0,5.
    \item[Слякоть] пропитанный водой снег, который при ударе разбрызгивается в стороны; плотность от 0,5 до 0,8.
    \item[Лед] вода в замерзшем состоянии, на аэродромных покрытиях проявляется в виде гололеда или гололедицы.
\end{description}

\textbf{Предупреждение! Посадка и взлет ВС на аэродромах Российской Федерации при нормативном Ксц менее 0,3 запрещаются.}

\paragraph{}Определение соответствия категории аэродрома по уровню обеспечения средствами спасения и пожарной защиты типу ВС

Категория аэродрома по организации аварийного обслуживания (Emergency services, Rescue and Fire Fighting) определяется в соответствии с классификацией ICAO требуемому уровню средств спасения и пожарной защиты, который обеспечивается на аэродроме для наибольшего по размерам (длине и ширине фюзеляжа) типа ВС, выполняющего полеты на данный аэродром.

Информация о категории аэродрома публикуется в сборниках аэронавигационной информации и NОТАМ.

Обеспечиваемый уровень средств спасения и пожарной защиты (ССПЗ) на аэродромах вылета и назначения в стандартных условиях должен быть не ниже уровня требуемой пожарной защиты (УТПЗ) для данного типа ВС (см. \hyperref[tbl:04T5]{таблицу \ref*{tbl:04T5}}).



\begin{table}[H]
    \begin{center}
    \caption{} \label{tbl:04T5}
    \small
    \begin{tabular}{|p{0.25\textwidth}|p{0.08\textwidth}|p{0.08\textwidth}|p{0.08\textwidth}|p{0.08\textwidth}|}
    \hline
    Категория аэродрома  &\multicolumn{4}{c|}{Типы ВС} \\
    \cline{2-5}
    по УТПЗ                     &Ан-2   &Ан-24/26   &Ан-74  &CL300\\
    \hline    
    1                           &       &  ЗОНА II  &       &       \\\hline
    2                           &       &           &ЗОНА II&ЗОНА II\\\hline
    3                           &x      &           &       &       \\\hline
    4                           &       &x          &       &       \\\hline
    5                           &Зона I &           &x      &x      \\\hline
    6                           &       &Зона I     &       &       \\\hline
    7                           &       &           &Зона I &Зона I \\
    \hline\hline
    \end{tabular}
    \end{center}
\end{table}
Х – нормальный для типа ВС УТПЗ

Категория ВПП по УТПЗ может быть понижена на одну ступень относительно величины, определенной по длине и максимальной ширине фюзеляжа, если на аэродроме количество взлетов или посадок наибольшего для данной ВПП ВС менее 700 для трех самых интенсивных по полетам месяцев года.

В случае временного уменьшения, требуемого для данного типа ВС уровня УТПЗ на аэродроме вылета и назначения, при принятии решения на полет следует руководствоваться следующими положениями:
\begin{itemize}
    \item если УТПЗ на аэродроме снизился не более чем на 2 единицы от требуемого для данного типа воздушного судна - полет может быть выполнен без ограничений;
    \item если УТПЗ на аэродроме снизился более чем на 2 единицы от требуемого для данного типа ВС - полет может быть выполнен по согласованию с администрацией аэропорта и руководством авиакомпании, принимая во внимание имеющуюся информацию о фактическом положении дел относительно обеспеченности данного аэродрома.
\end{itemize}
	
Для запасных аэродромов значение УТПЗ не может быть снижен более чем на 3 единицы от требуемого уровня для данного типа ВС. 

Если информация о снижении уровня УТПЗ на аэродроме получена в полете, для принятия решения экипаж должен руководствоваться \hyperref[tbl:04T6]{таблицей \ref*{tbl:04T6}} 

 
\begin{table}[H]
    \begin{center}
    \caption{} \label{tbl:04T6}
    \small
    \begin{tabular}{|p{0.08\textwidth}|p{0.85\textwidth}|}
    \hline
    ЗОНА I  &продолжить полет до аэродрома назначения;\\\hline
    ЗОНА II &продолжить полет до аэродрома назначения, но не начинать заход на посадку пока не будет получено подтверждение от администрации аэродрома, что все имеющиеся на аэродроме ресурсы ССПЗ подготовлены для применения в районе ВПП, планируемой для посадки ВС;\\\hline
    ЗОНА III&следовать на запасной аэродром в случае, если КВС считает, что уход на запасной более безопасен, чем посадка на а/д назначения. Если КВС принято решение о посадке на а/д назначения, то должны быть выполнены положения, определенные для ЗОНЫ II – имеющиеся ресурсы ССПЗ подготовлены для применения в районе ВПП, планируемой для посадки ВС.\\
    \hline\hline
    \end{tabular}
    \end{center}
\end{table}

	

\subsection{Принятие решения на вылет по ППП}

\paragraph{}Командир ВС принимает решение на вылет по ППП на основании анализа метеорологической обстановки, если:
\begin{itemize}
    \item на аэродроме вылета фактическая погода не ниже минимума, установленного для взлета.
    
    В том случае, если метеорологические условия на аэродроме вылета равны установленным эксплуатационным минимумам аэродрома для посадки или ниже их или не представляется возможным вернуться на аэродром вылета по другим причинам, необходимо выбрать запасной аэродром для взлета в соответствии с \hyperref[par:altrule2]{п. \ref*{par:altrule2}}; 
    \item на маршруте полета отсутствуют опасные метеоявления, обход которых невозможен;
    \item на аэродроме намеченной посадки фактическая погода и прогнозируемые условия погоды ко времени прилёта соответствуют требованиям одного из вариантов Таблицы А8.4-Т7 при принятии решения на вылет на аэродромы, где установлен государственный минимум и на которых совместно со значениями MDH(DH) дополнительно представлена информация о минимальном значении НГО в таблице минимумов с пометкой «CEILING REQUIRED» применять наибольшее из опубликованных значений MDH (DH) или НГО за параметр выбранного минимума для посадки;
    \item имеются запасные аэродромы по маршруту, выбранные в соответствии с \hyperref[par:altrule]{п. \ref*{par:altrule}}  и имеются запасные аэродромы для аэродрома назначения, в соответствии с \hyperref[sect:alt]{п. \ref*{sect:alt}}, с исключениями, указанными в \hyperref[par:altex]{п. \ref*{par:altex}}. 
    \item при вылете на аэродром, на котором отсутствуют радиотехнические средства захода на посадку, метеоусловия должны быть не ниже указанных в \hyperref[sec:visual]{п. \ref*{sec:visual}}.
\end{itemize}


\paragraph{} При принятии решения на вылет по ППП на аэродромах назначения и запасных не учитываются:
\begin{itemize}
    \item прогнозируемые ко времени прилета опасные метеоявления (кроме фронтальных гроз на запасных аэродромах);
    \item прогнозируемые ко времени прилета порывы ветра, за исключением случаев, указанных в п. 8.4.1.8;
    \item высота нижней границы облаков, если их фактическое и (или) прогнозируемое количество не более 2-х октантов;
    \item временное (ТЕМРО) ухудшение видимости и (или) понижение НГО, прогнозируемое ко времени прилета.
\end{itemize}	

Если время прилета на аэродром назначения (запасной) совпадает с прогнозируемым периодом изменения видимости или высоты НГО (BECMG), при принятии решения на вылет по ППП учитывать их наименьшее значение.

\paragraph{} \label{par:altex}В случае, если аэродром назначения имеет две независимые ВПП, пригодные для посадки ВС: 
а)	разрешается принимать решение на вылет без запасного аэродрома при продолжительности полета менее 6 часов, получена информация о фактической и прогнозируемой погоде за 1час до и через 1час после расчетного времени прибытия - НГО не ниже 600м и превышает МВС(MDA/Н) для захода на посадку с применением кругового маневрирования (circle-to-land) не менее чем на 150 м, а в случае, если такая высота не опубликована, то не ниже БВП в районе аэродрома (в секторе захода на посадку) и видимость не менее 5000м; 
б)	разрешается принимать решение на вылет, используя в качестве запасного аэродрома пункта назначения вторую непересекающуюся ВПП аэродрома назначения. В этом случае фактические и прогнозируемые метеоусловия на аэродроме назначения к расчетному времени прилета должны соответствовать требованиям к запасному аэродрому пункта назначения п. 8.4.2.14 (б).                                                               Таблица А8.4-Т7                                                                                                                                                   
Варианты	На аэродроме назначения	
Продолжительность полёта 
до аэродрома назначения 
по расчёту	Минимальное количество запасных аэродромов, полёт до которых обеспечивается с ВПР аэродрома назначения
	Соотношение фактической погоды (ВНГО, RVR, ветер, состояние ВПП) и минимума, выбранного для посадки согласно п.8.4.1.7.	Соотношение прогноза погоды (ВНГО, RVR, ветер) и минимума, выбранного для посадки согласно п.8.4.1.7.
(ко времени прилёта) 		
1	Независимо от фактической погоды	Прогнозируемые условия погоды соответствуют минимуму, выбранному для посадки или превышают их	1 час и более
	1
Согласно пункту 8.4.2.14 (б).
2	Фактическая погода 
не ниже минимума, выбранного для посадки (с учетом п. 8.4.4.7)	Прогнозируемые условия погоды соответствуют минимуму, выбранному для посадки или превышают их	До 5 часов	1
Согласно пункту 8.4.2.14 (б).
3	Фактическая погода 
не ниже минимума, выбранного для посадки (с учетом п.8.4.4.7)	Прогнозируемые условия погоды ниже минимума, выбранного для посадки	
До 5 часов	2 или 1
Согласно пункту
8.4.2.13.
4	Независимо от фактической погоды	Прогнозируемые условия погоды ниже минимума выбранного 
для посадки	Более 5 часов	
5	Информация о погоде отсутствует или прогнозируемые условия погоды ниже минимума, выбранного для посадки к расчетному времени прибытия	Независимо от продолжительности полета	не менее 2-х
Согласно пункту 8.4.2.15.
8.4.4.5. При принятии решения на вылет разрешается руководствоваться видимостью на ВПП при этом, если решение на вылет по видимости на ВПП принимается ночью, а посадка на аэродроме назначения (запасном) будет производиться в сумерках или днем, необходимо учитывать уменьшение видимости на ВПП при переходе от темного к светлому времени суток.  
8.4.4.6. В случае, когда неблагоприятная аэронавигационная или метеорологическая обстановка, или заправка топливом не позволяют выбрать запасной аэродром, уход на который возможен с ВПР/МВС аэродрома назначения, командиру ВС предоставляется право принятия решения на вылет с расчетом рубежа ухода на запасной аэродром (в том числе и на аэродром вылета), если: 
а)	при расчетной продолжительности полета с рубежа ухода до аэродрома назначения 2 часа и менее, информация о фактической погоде и прогнозе погоды на аэродроме назначения указывает на то, что ко времени прилета на аэродроме назначения и на запасном аэродроме погода будет соответствовать требованиям п. 8.4.2.14 (б); 
б)	при расчетной продолжительности полета с рубежа ухода до аэродрома назначения более 2 часов информация о фактической и прогнозируемой погоде на аэродроме назначения должна указывать на то, что в течение периода времени, за 2 часа до и через 2 часа после расчетного времени прибытия, метеорологические условия будут соответствовать требованиям п 8.4.2.14 (б), но не ниже 200 м по нижней границе облаков и не менее 2500 м по горизонтальной видимости. 
8.4.4.7. При принятии решения на вылет с продолжительностью полета до 2 ч необходимо учитывать: 
а)	соответствие фактической скорости и фактического направления ветра установленным ограничениям с учетом его порывов; 
б)	давность сведений о фактической погоде на аэродроме назначения не должна превышать 1 час с момента наблюдения. 
8.4.4.8. Аэродром категории «В» может быть выбран запасным, если экипаж имеет квалификацию в соответствии с требованиями РПП. 
8.4.4.9. Аэродром, не имеющий инструментальных схем маневрирования, может быть выбран в качестве запасного аэродрома пункта назначения при наличии соответствующей подготовки экипажа и согласно требованиям специальной инструкции по производству полетов на данный аэродром. 
8.4.4.10. Варианты принятия решения на вылет с обеспечением возможности ухода на запасной аэродром с ВПР/МВС аэродрома назначения должны рассматриваться как основные и предусматриваться при разработке планов (расписания) движения воздушных судов. 
8.4.4.11. Предельные метеоусловия на аэродроме назначения 
Прогноз погоды на аэродроме назначения предусматривает, что ко времени прилета метеоусловия будут соответствовать при планируемом заходе на посадку: 
а)	по САТ II и/или III – эксплуатационному минимуму аэродрома для посадки при САТ I или менее; 
б)	по радиомаячным системам инструментального захода на посадку (кроме САТ II и/или III) - эксплуатационному минимуму захода по схеме неточного захода на посадку или менее; 
в)	по схеме неточного захода на посадку – нижней границе облаков (вертикальной видимости) равной MDH неточного захода на посадку плюс 50м или менее, видимости (видимости на ВПП) - равной эксплуатационному минимуму для посадки при неточном заходе на посадку плюс 500 м или менее; 
г)	с применением визуального маневрирования («circle – to - land») - нижней границе облаков (вертикальной видимости) равной MDH для захода на посадку с применением визуального маневрирования плюс 100 м или менее, видимости равной эксплуатационному минимуму для захода на посадку с применением визуального маневрирования плюс 1000 м или менее. 
Если КВС принял решение на вылет при предельных метеоусловиях на аэродроме назначения и не вылетел до срока очередного перекрытия прогноза погоды, он должен запросить новый прогноз и фактическую погоду аэродрома назначения. 
8.4.5. Принятие решения на вылет по ПВП
8.4.5.1. Полет, который планируется выполнять по ПВП, не должен начинается до тех пор, пока текущие метеорологические сводки или подборка текущих сводок и прогнозов не укажут на то, что метеорологические условия на маршруте или части маршрута, по которому воздушное судно будет следовать по ПВП, обеспечат к соответствующему времени возможность соблюдать ПВП. (п.3.33. Приказ МТ РФ от 31.07.2009 №128 «Подготовка и выполнение полетов в гражданской авиации Российской Федерации»).
8.4.5.2. Для выполнения полета по ПВП командир ВС принимает решение на вылет при следующих условиях:
а)	на аэродромах вылета, назначения и запасных, фактическая погода соответствует минимуму командира ВС и она не ниже предусмотренной для полетов по ПВП;
б)	прогнозируемые видимость и высота нижней границы облаков по маршруту (в районе авиационных работ), аэродрому назначения и запасным не ниже минимума командира воздушного судна и предусмотренного для полетов по ПВП; 
в)	прогноз ветра без учета порывов в пределах установленных ограничений;
г)	по маршруту полета (в районе авиационных работ) не наблюдаются и не прогнозируются опасные метеоявления, обход которых невозможен.
Примечание. На посадочные площадки, где нет оперативных органов Роскомгидромета Российской Федерации и отсутствуют работники гражданской авиации, осуществляющие метеонаблюдения за фактической погодой, решение на вылет, для выполнения авиационных работ, принимается на основании ориентировочного прогноза по району пункта посадки.
После получения информации о погоде от экипажей ВС или гидрометеорологических станций метеорологический орган выпускает уточненный прогноз, который доводится до сведения экипажей ВС, выполняющий авиационные работы.
8.4.5.3. При отсутствии запасного аэродрома принимать решение на вылет по ПВП разрешается, если ко времени прилета на аэродроме назначения, а также по маршруту полета, прогнозируется видимость на 500 м и высота нижней границы облаков на 50 м выше установленного минимума по ПВП.
8.4.5.4. При принятии решения на вылет по ПВП по маршруту полёта, на аэродромах назначения и запасных не учитываются:
а)	высота нижней границы облаков, если их фактическое и (или) прогнозируемое количество на аэродроме назначения и запасных не более двух октантов, а по маршруту полёта прогнозируемое количество облаков не более четырёх октантов;
б)	прогнозируемые ухудшения видимости и (или) понижение нижней границы облаков, указанные в прогнозе терминами: временами (кратковременно) (ТЕМРО) и (или) местами.
Если время полёта совпадает с прогнозируемым периодом (BECMG) изменения видимости и (или) высоты нижней границы облаков, то при принятии решения на вылет по ПВП учитывается их наименьшее значение:
при прогнозировании ухудшения метеорологических условий с начала действия периода (BECMG);
при прогнозировании улучшения метеорологических условий после окончания действия периода (BECMG).
8.4.5.5. При принятии решения на вылет по ПВП, когда предполагается полет над облаками или между слоями облачности, высота нижней границы облаков по маршруту, на аэродроме вылета, назначения и запасном не учитывается, если их фактическое и прогнозируемое количество, ниже высоты полета не более четырех октантов по маршруту, а на аэродроме назначения и запасных за один час до и два часа после ожидаемого времени прилета - не более двух октантов, не прогнозируется туман, ливневые осадки и грозовая деятельность. При этом должен обеспечиваться полет с превышением над верхней границей облаков не менее 300 м, расстояние между слоями облачности не менее 1000 м, видимость по маршруту, аэродроме назначения и запасным не менее 5000 м.
Посадка воздушного судна при метеоусловиях ниже установленного минимума запрещается за исключением случаев вынужденной посадки (потеря радиосвязи, недостаток топлива, отказ авиационной техники или другое), не позволяющих продолжить полет до запасного аэродрома.
8.4.6. Принятие решения на вылет в непредвиденных обстоятельствах 
8.4.6.1. Непредвиденные обстоятельства 
Это обстоятельства, которые не предусмотрены документами регламентирующие летную деятельность на (над) территории (ей) государства, где выполняются коммерческие воздушные перевозки (стихийные бедствия, начало боевых действий, угроза теракта и т.п.). 
При выполнении полетов, связанных со спасением жизни людей при непредвиденных обстоятельствах природного, техногенного или политического характера в случаях, не терпящих отлагательств, КВС имеет право под свою личную ответственность допускать отступления от порядка и правил полетов, норм рабочего и полетного времени, изложенных в РПП и других нормативных документах в области ГА если, считает, что непринятие решения о вылете в создавшийся ситуации является более опасным, чем выполнение полета. 
Командир ВС принимает решение на вылет на основе всестороннего анализа возможностей экипажа и воздушного судна только в том случае, если он считает, что принятое решение позволит выполнить поставленную задачу без угрозы жизни и здоровья экипажа и третьих лиц. 
Решение на вылет в обстоятельствах угрозы теракта может быть принято командиром ВС только после полного устранения факторов такой угрозы для ВС, пространства, в пределах установленных границ (по площади и высоте), используемого ВС для взлета и набора безопасной высоты. 
8.4.6.2. Непредвиденные обстоятельства природного или техногенного характера 
К этим обстоятельствам относятся все виды стихийных бедствий, техногенные катастрофы, угрожающие жизни пассажиров, экипажа: 
а)	землетрясения; 
б)	наводнения; 
в)	вулканическая деятельность; 
г)	пожары; 
д)	мощные взрывы; 
е)	цунами и т.п. 
8.4.6.3. Непредвиденные обстоятельства политического характера
К ним относятся:
а)	возникающее военное противостояние между различными политическими группами; 
б)	возможные террористические акции, угрожающие жизни пассажиров, экипажа; 
в) непосредственная угроза уничтожения ВС (обстрелы, нападения на объекты, находящиеся в непосредственной близости от места стоянки ВС); 
г) принуждения к посадке на территории государства, не предусмотренного планом; 
д) внезапное начало боевых действ и т.п.
8.4.6.4. При получении информации о приближающемся стихийном бедствии, техногенной катастрофы или непредвиденных обстоятельств политического характера угрожающих жизни пассажиров и экипажа в зоне базирования, командир ВС должен:
а)	используя все возможные каналы связи, принять меры для доклада руководству и/или представителям компании, посольству РФ страны пребывания о создавшейся ситуации, в случае невозможности такого доклада приступить к самостоятельной подготовке и организации вылета; 
б)	организовать прибытие пассажиров, экипажа и ИТП на ВС, произвести предварительную и предполетную подготовку, проинформировать на рабочих частотах аэродрома вылета орган ОВД о своих действиях и принятом решении; 
в)	по согласованию с органом ОВД произвести вылет на запасной аэродром, своим расположением 
обеспечивающим безопасность пассажиров, экипажа и ВС. Руление на аэродроме и маневрирование в воздухе выполнять по запросам и командам, выдаваемым диспетчером на частотах аэродрома вылета;
г)	при полетах в горных районах и/или горных аэродромах строго выполнять схему полетов в соответствии со сборниками АНИ и / или с утвержденной инструкцией по производству полетов; 
д)	при необходимости координирует свои действия со службами МЧС. 
8.4.6.5. При организации полетов, связанных со спасением жизни людей или стихийными бедствиями, генеральный директор Авиакомпании имеет право под свою личную ответственность допускать отступления от порядка и правил полетов, принятых в Авиакомпании. При выполнении таких полетов с экипажами проводится специальный инструктаж, на котором уточняются цели полета, определяются меры безопасности и порядок действия (взаимодействия) при возникновении (угрозе возникновения) внештатных ситуаций.
8.4.7. Методы расчета эксплуатационных минимумов аэродромов
8.4.7.1. Определение эксплуатационных минимумов аэродромов при выполнении полетов по ППП самолетов АО «ЮТэйр» производится по «Методике определения эксплуатационных минимумов для взлета и посадки ВС АО «ЮТэйр» (для самолетов транспортной авиации)» утвержденной директором АО «ЮТэйр» от 12.08.2019г. и одобренной руководителем Тюменского МТУ Росавиации 20.08.2019г..
8.4.7.2. Применяемые в АО «ЮТэйр» минимумы основаны на положениях EASA Air Operations и соответствуют значениям минимумов, опубликованных в сборниках Jeppesen с надписью «STANDARD» за исключением:
а)	при выполнении неточных заходов на посадку (NPA) по системам LOK, VOR, 2NDB, NDB при отсутствии (отказе) коррекции FMS по GNSS, DME/DME, DME/DME/IRU эксплуатационный минимум аэродрома устанавливается не менее MDH = 150м(490ф) и RVR/CMV=1800м;
б)	значения минимумов публикуются в единицах измерения в зависимости от системы измерения, принятой в конкретной стране.
8.4.8. Метеорологическая информация, предоставляемая летному экипажу
8.4.8.1. Экипажи воздушных судов при выполнении полетов обеспечиваются метеорологической информацией и документацией в соответствии с требованиями Федеральных авиационных правил «Предоставление метеорологической информации для обеспечения полетов воздушных судов», утвержденных Приказом Министерства транспорта РФ от 03.03. 2014г. № 60.
8.4.8.2. Метеорологическая информация, которой обеспечивается экипаж ВС перед вылетом и во время полета, должна соответствовать времени, высоте и протяженности маршрута.
8.4.8.3. На предполетной подготовке экипаж обязан проанализировать метеоусловия аэродромов вылета, назначения, запасных аэродромах и на воздушной трассе в целях принятия обоснованного решения на вылет с учетом развития посадочных условий и возможностей обхода зон с метеорологическими явлениями, опасными для полета
8.4.8.4. Предоставление метеорологической информации экипажам ВС производится сотрудником по обеспечению полетов (полетным диспетчером), аэродромным метеорологическим органом.
Аэродромный метеорологический орган предоставляет метеорологическую информацию по заявке эксплуатанта или командира ВС, содержащую: 
а)	время вылета по расписанию; 
б)	аэродром назначения; 
в)	запасные аэродромы; 
г)	эшелон полета; 
д)	указание на правила полетов - правила визуальных полетов или правила полетов по приборам. 
Аэродромный метеорологический орган определяет место для предоставления метеорологической информации экипажам ВС на основе консультаций с эксплуатантами.
Метеорологическая информация для представления эксплуатантам и экипажам ВС включает следующую информацию: 
(1) Прогнозы: 
а)	ветра и температуры на высотах; 
б)	особых явлений погоды (SWH, SWM). 
(2) METAR, SPECI (включая прогнозы TREND) для аэродромов вылета и намеченной посадки, для запасных аэродромов вылета, на маршруте и назначения.
(3) TAF и коррективы TAF для аэродромов вылета и намеченной посадки, для запасных аэродромов вылета, на маршруте и назначения. 
(4) Информацию SIGMET и/или специальные донесения с борта ВС, касающиеся всего маршрута (к специальным донесениям с борта относятся донесения, которые не использовались при подготовке сообщений SIGMET). 
(5) Консультативную информацию о вулканическом пепле и тропических циклонах, относящуюся ко всему маршруту полета.
(6) Зональные прогнозы в формате GAMET и/или прогнозы в формате карт и информация AIRMET для полетов ниже эшелона 100 (150 или выше в горных районах), которые относятся ко всему маршруту. 
(7) Предупреждения по аэродрому для аэродрома вылета. 
(8) Данные искусственных спутников Земли (далее - ИСЗ). 
(9) Данные наземных метеорологических радиолокаторов (МРЛ, ДМРЛ).
8.4.8.5. Для полетов по маршрутам, которые по информации метеорологических органов могут быть затронуты облаками вулканического пепла, в полетную документацию включаются данные специальных наблюдений с борта ВС.
8.4.8.6. По заявкам эксплуатантов или провайдеров метеоинформации полномочным метеорологическим органом предоставляется метеорологическая информация для автоматизированных систем предполетной подготовки.
8.4.8.7. Метеорологическая информация подготавливается для экипажа воздушного судна не позднее, чем за час до запланированного времени вылета ВС.
8.4.8.8. В полетную документацию включаются прогнозы особых явлений погоды SIGWX, прогноз ветра и температуры на высотах в виде карт, масштаб и период действия которых охватывают район и время полета, включая возможный уход на запасной аэродром.
8.4.8.9. Экипажам воздушных судов предоставляются: 
(1) При полете между эшелонами полета 250 и 630 - карта особых явлений погоды SWH и прогностическая карта ветра и температуры для эшелона 340 (250 гПа). 
(2) При полете между эшелонами полета 100 и 250 - карта особых явлений погоды SWM и прогностическая карта ветра и температуры для эшелона 180 (500 гПа).
(3) При полете ниже эшелона 100 (150 или выше в горных районах) - карта особых явлений погоды и прогностические карты ветра и температуры воздуха для абсолютных высот 600м, 1500м, 3000м и 4500м в горных районах, а также на других высотах по запросу.
По требованию экипажа воздушного судна (эксплуатанта) в полетную документацию включаются дополнительные прогнозы по высотам (прогнозы особых явлений погоды и (или) ветра (температуры)). 
Если маршрут полета не укладывается полностью на прогностической карте, экипажу ВС на оставшийся участок дополнительно выдается прогностическая карта смежной зоны. При полетах ниже эшелона 100 при необходимости выдается прогноз в формате GAMET для смежного района.
8.4.8.10. Если прогнозы для полетов на эшелонах ниже FL 100 (в горной местности ниже FL 150) составляются в форме зонального прогноза GAMET, они включаются в полетную документацию вместо карт. 
8.4.8.11. При задержке вылета по запросу экипажа воздушного судна обеспечивается повторное оформление полетной документации и/или проведение консультации. 
8.4.8.12. Метеорологическая информация, необходимая экипажам ВС, выполняющих литерные, поисково-спасательные, аварийно-спасательные полеты, полеты по заказам медицинских учреждений, готовится немедленно и предоставляется в кратчайшие сроки. 
8.4.8.13. Копия информации, выданной экипажу ВС, хранится в бумажном и/или электронном виде в течение не менее 30 дней с момента ее выпуска и предоставляется эксплуатанту для выполнения анализа полета или назначенным для расследования авиационных событий лицам. При расследовании авиационных событий информация сохраняется до завершения расследования. 
8.4.8.14. При предоставлении метеорологической информации, в том числе и с использованием автоматизированных систем предполетной подготовки, обеспечивается ее целостность и полнота. 
8.4.8.15. Экипажи ВС, находящиеся в полете, обеспечиваются метеорологической информацией через орган ОВД, с которым установлена связь, посредством ATIS или радиовещательных передач VOLMET. 
Для радиовещательных передач VOLMET экипажам ВС, находящимся в полете, предоставляются: 
а)	сводки METAR (по согласованию с органом ОВД - SPECI) с прогнозами на посадку TREND (непрерывные передачи VOLMET); 
б)	сводки METAR и SPECI с прогнозами на посадку TREND, TAF, SIGMET (регулярные передачи VOLMET). 
1.Требования к метеорологической информации, предоставляемой для обеспечения полетов воздушных судов ГА опубликованы в Приложение А - 8.10.
8.4.9. Инструкция по расчету количества топлива и масла, которое необходимо иметь в баках, учитывая все условия полета, в том числе возможность разгерметизации, отказа на маршруте одного двигателя и отказа ПОС самолета в условиях обледенения
Исходными данными для определения потребного количества топлива являются: 
а)	расстояние между аэродромами вылета и аэродромом назначения; 
б)	расстояние от аэродрома назначения до выбранного запасного аэродрома пункта назначения; 
в)	распределение ветра (направление и скорость) по высотам полёта по маршруту; 
г)	ожидаемые условия полёта (обледенение, грозовая деятельность, аэронавигационная обстановка); 
д)	отклонение прогнозируемой температуры от МСА. 
Количество масла на борту воздушного судна определяется производителем ВС, контроль выполняется техническим персоналом. Дополнительно, при выполнении полётов по EDTO (ETOPS) для мониторинга расхода масла используется MAINTENANCE JOB CARD.
8.4.9.1. Правила определения минимального запаса топлива при полете по ППП (самолеты с ГТД)
Расчет потребного количества топлива на борту воздушного судна основывается на:
а)	текущих данных по расходу топлива относительно конкретного ВС, полученных от систем мониторинга, если таковые имеются, или, в случае отсутствия таких, на данных, предоставленных изготовителем самолета; 
б)	эксплуатационных условиях для выполнения запланированного полета, которые включают ожидаемую массу ВС, NOTAM, текущие метеорологические сводки или комбинацию текущих сводок и прогнозов, процедуры ОВД, ограничения и ожидаемые задержки, последствия отложенных дефектов по MEL/CDL. 

8.4.9.2. Расчет заправляемого топлива на полет для ВС отечественного производства
(1) Порядок определения минимального потребного количества топлива для самолетов с газотурбинными двигателями:
а)	определить наивыгоднейший эшелон полета в соответствии с РЛЭ ВС по известному расстоянию между аэродромами вылета и посадки;
б)	определить режим крейсерского полета (Мкр, МД и т.д.). При этом определить скорость (число М) в зависимости от полетной массы и эшелона полета;
в)	определить количество потребного топлива на полет (m потр), которое равно сумме основного запаса топлива и аэронавигационного запаса топлива. 
m потр= m т.зем + mт.наб. + mт. рейс + m т.АНЗ, где:
(2) M Т.ЗЕМ. - масса топлива на запуск, прогрев двигателей, руление к месту старта, включая работу ВСУ. Определяется из РЛЭ, а также может корректироваться в зависимости от продолжительности ожидаемого времени руления и прогрева двигателей, работы ВСУ.
Не учитывается при определении взлетной массы самолета.
(3) M Т.НАБ - масса топлива на взлет и набор высоты;
(4) m рейс - рейсовое топливо, расходуемое от момента набора высоты до посадки на аэродроме назначения. 
а)	Рейсовое топливо складывается из:
	запаса топлива, необходимого для набора высоты;
	запаса топлива, необходимого для крейсерского полета до момента начала снижения с эшелона полета;
	запаса топлива, необходимого для снижения с крейсерского эшелона полета до высоты над контрольной точкой (IAF) аэродрома;
б)	Определение m рейс производится с использованием таблиц (номограмм) РЛЭ конкретного типа ВС.
При этом должны учитываться такие факторы, как:
	отклонение прогнозируемой температуры от МСА;
	прогностический ветер по высотам;
	прогностические условия полета (обледенение, грозовая деятельность и т.д.);
	ожидаемая полетная масса ВС.
Данные для расчета расхода топлива приведены в РПП, часть В, гл. 5 типа ВС.
(5) mт. АНЗ – аэронавигационный запас топлива, включает в себя резервный запас, компенсационный и дополнительный запас топлива.
mт АНЗ = m резерв + m КЗТ + m ДОП, где:
m резерв - запас топлива для полета на запасной аэродром пункта назначения, указанного в рабочем плане полета.
а)	Состоит из: 
	запаса топлива для ухода на второй круг на аэродроме пункта назначения;
	запаса топлива для набора предполагаемой высоты крейсерского полета для следования на запасной 
аэродром пункта назначения;
	запаса топлива для полёта по предполагаемому маршруту ухода на запасной аэродром пункта назначения;
	запаса топлива для снижения до контрольной точки (IAF) запасного аэродрома пункта назначения;
	запаса топлива для захода на посадку, что соответствует расходу топлива в горизонтальном полете, не
менее 4 минут, с посадочной конфигурацией на высоте 450 м (1500ft) в стандартных температурных условиях;
	запаса топлива для выполнения захода и посадку на запасном аэродроме пункта назначения;
	запаса топлива для полета в течение 30 минут со скоростью полета в зоне ожидания на высоте 450 м(1500ft) над запасным аэродромом в стандартных температурных условиях.
б)	При наличии 2-х запасных аэродромов для пункта назначения запас топлива, который обеспечивает выполнение полета до того запасного аэродрома пункта назначения, для которого требуется большее количество топлива.
в)	При использовании в качестве запасного аэродрома второй непересекающейся ВПП аэродрома назначения, планируемый остаток топлива должен обеспечить полет после прибытия на аэродром назначения в течение не менее 60 минут на высоте 450 м.(1500ft) над аэродромом при стандартных температурных условиях.
г)	При выполнении полета с выбранным запасным аэродромом пункта назначения, уход на который возможен с рубежа ухода запас топлива mт. АНЗ позволяет:
	выполнить полет до запасного аэродрома через запланированный рубеж ухода;
	продолжить полет в течение 30 минут на высоте 450 м.(1500ft) над запасным аэродромом, либо:
	выполнить полет до аэродрома намеченной посадки и;
	продолжить полет в течение двух часов при нормальном расходе топлива в крейсерском режиме (одного часа, когда прогнозируемые метеоусловия на аэродроме назначения превышают значения, соответствующие требованиям п. 8.4.2.14. гл. А–8 РПП на 50 м по НГО и на 500 м по видимости).
д)	При выполнении полета без запасного аэродрома пункта назначения (в качестве запасного аэродрома пункта назначения может использоваться аэродром пункта назначения при наличии двух непересекающиеся ВПП, пригодные для посадки ВС, mт. АНЗ – запас топлива, обеспечивающий, после прибытия на аэродром назначения, продолжение полета в течение 30 минут со скоростью полета в зоне ожидания на высоте 450 м (1500ft) при стандартных температурных условиях. 
Аэронавигационный запас топлива не может быть меньше значения, указанного в руководстве по летной эксплуатации воздушного судна данного типа. 
(6) m КЗТ - компенсационный запас топлива для учета непредвиденных обстоятельств:
а)	ошибки в прогнозе ветра и температуры;
б)	навигационные погрешности;
в)	ограничения, налагаемые ОВД в полете, относительно высоты и маршрута следования;
г)	задержки, связанные с воздушным движением;
д)	погрешности топливоизмерительных систем;
ж)  разброс индивидуальных характеристик ВС;
и)  методические погрешности при расчете топлива;
Установленное количество m КЗТ составляет не менее 3% от запаса топлива, расходуемого на полет от аэродрома вылета до аэродрома назначения.
(7) mДОП - запас топлива сверх минимального, который приведет к экономии производственных расходов или при полетах по маршрутам, где ограничивающим фактором может быть потребный запас топлива для длительного полета в крейсерском режиме на малых высотах (отказ двигателя, отказ системы герметизации, отказ перекачивающих топливных насосов). 
(8) Общие стандартные сведения по минимальным составляющим заправки топливом по эксплуатируемым типам самолетов приведены в Части В РПП, гл. 5, по типам ВС.
Окончательное решение о количестве топлива на полет принимает командир ВС.
8.4.9.3. Правила определения минимального запаса топлива (Ан-2)
(1) Количество топлива и масла на борту самолета при полете по ПВП должно обеспечивать выполнение полета до аэродрома намеченной посадки, после чего продолжение полета до запасного аэродрома с последующим полетом в районе запасного аэродрома в течение 45 минут на скорости, оптимальной с точки зрения расхода топлива. 
(2) Количество топлива и масла на борту самолета при полете по ППП должно позволять: 
а)	при выполнении полета с выбранным запасным аэродромом пункта назначения, уход на который возможен с MDA/H аэродрома назначения, выполнить полет до аэродрома намеченной посадки и затем до наиболее критического, с точки зрения расхода топлива, запасного аэродрома, указанного в планах полета, после чего продолжить полет в течение 45 минут; 
б)	при выполнении полета с выбранным запасным аэродромом пункта назначения, уход на который возможен с рубежа ухода: 
	выполнить полет до запасного аэродрома через определенный рубеж ухода и затем продолжить полет в течение 45 минут; 
	выполнить полет до аэродрома назначения и затем продолжать его в течение 45 минут, предусмотрев дополнительный запас топлива, составляющий 15% топлива, запланированного на полет по маршруту полетного времени, но не более двух часов. 
8.4.9.4. Расчет заправляемого топлива на полет для ВС иностранного производства
(1) ПРОИЗВОДИТСЯ, КАК ПРАВИЛО, при помощи автоматизированной системы полетного планирования с учетом: 
а)	планируемой коммерческой загрузки ВС; 
б)	расчетных скоростных режимов набора высоты, горизонтального полета и снижения ВС; 
в)	варианта выбора и наличия запасных аэродромов, их удаления от аэродрома назначения; 
г)	наличия запасных аэродромов на маршруте EDTO (ETOPS); 
д)	прогнозируемого ветра на эшелонах полета; 
ж)  особенностей расхода топлива на каждом конкретном ВС (Degradation Factor); 
и)  выполнения полета на расчетном эшелоне и/или скорости (числа М) в случае разгерметизации кабин ВС или отказа двигателя во время полета на маршруте; 
к)  отклонений от маршрута, а также возможных задержек, связанных с ОВД. 
(2) Расчет минимального потребного количества топлива на полет включает в себя:
MIN REQ FUEL = Taxi Fuel + Trip Fuel + Contingency Fuel + Destination Alternate Fuel + Final Reserve
Taxi Fuel – топливо на руление, которое представляет собой количество топлива, которое ожидается использовать до взлета, а именно на запуск двигателей, непосредственно руление по РД и работу ВСУ (APU).
Величина топлива на руление определяется из РЛЭ ВС (AFM, FCOM, FPPM), а также может корректироваться в зависимости от продолжительности ожидаемого времени руления и прогрева двигателей, работы ВСУ.
Значение топлива на руление обозначается в CFP как «TAXI» и устанавливается из расчета планируемого времени на руление (от момента времени запуска двигателей до времени взлета).
Примечания:
а)	Топливо на руление после посадки ВС не входит в общее количество потребного топлива на полёт при расчете, и расходуется из запасов топлива, оставшегося после посадки;
б)	Количество топлива «TAXI» увеличивается в два раза при прогнозировании сильного ливневого снега на аэродроме вылета;
в)	Количество топлива на руление может быть увеличено в соответствии с распоряжениями и приказами Авиакомпании.
Trip Fuel - (в CFP как TRIP) рейсовое топливо, которое представляет собой количество топлива, требующееся для обеспечения полета ВС с момента взлета до посадки на аэродроме пункта назначения;
Определение рейсового топлива для ВС иностранного производства выполняется из условий расчетной величины «Cost Index», которая определяется в соответствии с топливной политикой авиакомпании. 
Примечание. Для расчёта горизонтального полета через воздушное пространство NAT HLA Северной Атлантики, кроме указанного в данной таблице режима Cost Index (Econ Cruise), используется также режим фиксированного числа М.
Contingency Fuel - (в CFP как RTE RES) запас топлива, который представляет собой количество топлива, требующегося для компенсации непредвиденных факторов. 
Непредвиденными факторами являются факторы, которые могут повлиять на расход топлива при полете до аэродрома назначения, такие как отклонение от показателей ожидаемого потребления топлива для конкретного самолета, отклонение от прогнозируемых метеорологических условий, отклонение от планируемых маршрутов и/или крейсерских эшелонов полета, увеличенные задержки на земле. 
Для расчета Contingency Fuel применяется 2 варианта расчета: 
 5% от запланированного количества топлива для полета по маршруту. 
В CFP отображается как RTE RES 5P/C. Contingency Fuel определяется как количество топлива в размере не менее 5 % от запланированного количества топлива для полета по маршруту или топлива, требующегося для полета от точки изменения плана полета, рассчитанного на основе нормы потребления топлива, используемой для планирования количества топлива для полета по маршруту. 
Примечание: Если рассчитанное значение Contingency Fuel будет меньше, чем значение топлива, требуемого для полета в течение 5 минут со скоростью полета в зоне ожидания на высоте 450 м (1500 фут) над превышением аэродрома назначения при стандартных условиях, то в расчете появится значение топлива для 5 минут полета в выше описанных условия и в CFP данное значение топлива отобразится как RTE RES MIN CONT. 
 3% от запланированного количества топлива для полета по маршруту. 
В CFP отображается как RTE RES 3P/C XXXX- Contingency Fuel определяется в размере 3% от запланированного количества топлива для полёта по маршруту, или топлива, требующегося для полета от точки изменения плана полета для ВС, на которых осуществляется постоянный мониторинг изменения индивидуальных ЛТХ по аспекту изменения расхода топлива (Degradation Factor). Для данного варианта расчета выбирается запасной аэродром по маршруту ХХХХ, (fuel alternate) с применением правил в соответствии с п. 8.4.9.7. данной главы. 
Примечание: Если рассчитанное значение Contingency Fuel для RTE RES 3P/C ХХХХ будет меньше, чем значение топлива требуемого для полета в течение 5 минут со скоростью полета в зоне ожидания на высоте 450 м (1500 фут) над превышением аэродрома назначения при стандартных условиях, то в расчете появится значение топлива для 5 минут полета в выше описанных условия и в CFP данное значение топлива отобразится как RTE RES MIN CONT ХХХХ, где XXXX - запасной аэродром по маршруту с применением правил 3% ERA в соответствии с п. 8.4.9.7. данной главы. 
В итоге в CFP возможно отображение Contingency Fuel в следующих вариантах: 
а)	RTE RES 5P/C 
б)	RTE RES MIN CONT 
в)	RTE RES 3P/C XXXX 
г)	RTE RES MIN CONT ХХХХ.
Destination Alternate Fuel - (в CFP как ALTN) запас топлива для полета до запасного аэродрома пункта назначения, который включает: 
При наличии запасного аэродрома пункта назначения: 
а)	уход на второй круг на аэродроме пункта назначения; 
б)	набор предполагаемой высоты крейсерского полета на запасной аэродром пункта назначения; 
в)	полёт по предполагаемому маршруту ухода на запасной аэродром пункта назначения; 
г)	снижение до точки начала схемы ожидаемого захода на посадку на запасном аэродроме пункта назначения; 
д)	выполнение захода и посадку на запасном аэродроме пункта назначения. 
Если в качестве запасного аэродрома пункта назначения выбран аэродром пункта назначения в соответствии с п. 8.4.2.10, то значение ALTN равно значению топлива необходимого для полета в течение 30 мин на высоте 450 м (1500 фут) над превышением аэродрома при стандартных условиях; 
Если полёт выполняется с выбранным запасным аэродромом пункта назначения, уход на который возможен с рубежа ухода, то после выполнения полета до аэродрома пункта назначения на борту требуется иметь запас топлива (ALTN), который позволит продолжить полет в течение: 
а)	1 ч. 30 мин - при нормальном расходе топлива в крейсерском режиме над аэродромом назначения, или 
б)	30 мин - при нормальном расходе топлива в крейсерском режиме над аэродромом назначения если прогнозируемые метеоусловия на аэродроме назначения превышают значения метеоусловий, указанных в п. 8.4.2.14 (б) не менее чем на 50 м по нижней границе облаков (вертикальной видимости) и на 500 м по дальности видимости. 
При наличии 2-х запасных аэродромов для пункта назначения. 
При наличии 2-х запасных аэродромов для пункта назначения запас топлива, ALTN должен обеспечить выполнение полета до того запасного аэродрома пункта назначения, для которого требуется большее количество топлива. 
При отсутствии запасного аэродрома пункта назначения. 
Если условия на аэродроме назначения соответствуют требованиям п. 8.4.2.11, то значение запаса ALTN равно значению топлива необходимого для полета в течение 30 мин со скоростью полета в зоне ожидания на высоте 450 м (1500 фут) над превышением аэродрома назначения при стандартных условиях;
Final Reserve Fuel - (в CFP как FINAL RES) окончательный (минимальный) резерв топлива- резерв топлива, который представляет собой запас топлива, рассчитанный с использованием расчетной посадочной массы при прибытии на запасной аэродром пункта назначения или на аэродром пункта назначения, когда не требуется запасной аэродром для пункта назначения. Для самолетов с ГТД он равен запасу топлива для полета в течение 30 мин со скоростью полета в зоне ожидания на высоте 450 м (1500 фут) над превышением аэродрома при стандартных условиях.
Дополнительно к рассчитанному минимальному потребному количеству топлива на полёт может быть добавлено топливо, указанное в п. 8.4.9.5.
8.4.9.5. Дополнительные запасы топлива
Extra Fuel - дополнительный запас топлива это дополнительное количество топлива, требующееся в том случае, когда запас топлива рассчитанный в соответствии в п.п. 8.4.9.4(2) недостаточен для:
а)	обеспечения возможности ВС выполнить при необходимости снижение и продолжить полет до запасного аэродрома при отказе двигателя или разгерметизации в наиболее критической точке на маршруте; 
б)	обеспечения возможности ВС, выполняющему полёт с применением правил EDTO (ETOPS), выполнить полет в соответствии с критическим сценарием полета EDTO (ETOPS) по запасу топлива; 
в)	выполнения захода на посадку и посадки; 
г)	выполнения дополнительных требований, не указанных выше.
В зависимости от причин использования этого дополнительного топлива, данное топливо определяется как: 
EXTRA EDTO – (в CFP как EXTRA-EDTO) дополнительный запас топлива, рассчитанный по критическому топливному сценарию при полётах по маршрутам с сегментами EDTO (ETOPS); 
EXTRA DEVIATION – (в CFP как EXTRA - DEV) топливо, добавленное по требованиям пунктов MEL/CDL и которое может быть использовано в полете. 
Если в CFP появляется строка с UNUSED FUEL - (в CFP как UNUSED FUEL BY MEL), то это топливо, которое находится в баках ВС, но не может быть использовано в полете. 
EXTRA TANKERING – (в CFP как EXTRA - TANKERING) топливо, предназначенное для целей: 
а)	исключения или уменьшения количества заправляемого топлива на аэродроме назначения по причине высокой стоимости; 
б)	исключения заправки топливом на аэродроме назначения по причине отсутствия топлива; 
в)	сокращения времени наземного обслуживания на аэродроме назначения. 
Топливо, определённое для тенкирования не входит в минимальный потребный запас топлива на полёт. 
 Возможность и целесообразность применения дополнительного запаса топлива «TANKERING» определяется полётным диспетчером ОПДО САОП Лётного директората на основании комплексного анализа экономических и эксплуатационных условий. «TANKERING» применяется если: 
а)	время наземного обслуживания ВС в аэропорту назначения по расписанию составляет 45-60 минут; 
б)	отсутствует топливо на аэродроме пункта назначения; 
в)	в весенний и осенний периоды года перевозка тенкерного топлива не потребует противообледенительной обработки ВС в аэропорту пункта назначения; 
г)	на ВС отсутствуют отложенные по MEL/CDL дефекты, оказывающие влияние на взлётную и посадочную массы; 
д)	расчетная взлётная масса ВС с тенкерным топливом не потребует использования максимального взлётного режима работы двигателей; 
е)	расчетный вертикальный профиль полета ВС с тенкерным топливом не будет противоречить аэронавигационной обстановке по маршруту (свободные эшелоны в OTS NAT HLA, закрытые эшелоны RVSM в ВП Афганистана и т.п.); 
ж)	посадочные массы ВС с танкерным топливом будут меньше MLW на 1% для всех типов ВС. 
При выполнении рейса с тенкерным топливом экипаж имеет право дозаправить необходимое количество топлива на аэродроме пункта назначения для выполнения полёта при превышении фактической коммерческой загрузки от величины, указанной в рабочем плане полёта. 
EXTRA OPERATIONAL – (в CFP как EXTRA – OPN) количество топлива, увеличенное в значении общей заправки на полёт на основании решения командира воздушного судна. 
EXTRA ATC DESTINATION – (в CFP как EXTRA – ATC) дополнительное топливо на полет в зоне ожидания аэродрома назначения. Определяется как топливо для полета в течение 15 мин. на высоте 1500 м. (5000f) в стандартных условиях и применяется, если наступает один или несколько из перечисленных случаев: 
а)	воздушное судно прибывает на аэродромы Московского аэроузла (МУДР) в пиковый период интенсивности движения, определенный в соответствии с распоряжениями и приказами Авиакомпании; 
б)	планируется прибытие на аэродром назначения после внеплановых закрытий (очистка ВПП и т.д.).
EXTRA ATC ALTERNATE – (в CFP как EXTRA – ATC) дополнительное топливо на полет в зоне ожидания запасного аэродрома. Определяется как топливо для полета в течение 5 минут на высоте 450 метров (1500f) в стандартных условиях и применяется для всех запасных аэродромов для исключения выхода экипажей в ситуацию, характеризующейся как «MINIMUM FUEL» (п. 8.19.3.4. данной главы) сообщением «PAN–PAN–PAN-MINIMUM FUEL). 
Данный запас топлива не входит в расчет потребного количества топлива на полет для турбовинтовых ВС.
EXTRA WEATHER – (в CFP как EXTRA – WXX) дополнительное топливо на полет в зоне ожидания аэродрома назначения. Определяется как топливо для полета в течение 15 мин. на высоте 1500 м. (5000f) в стандартных условиях и применяется, если наступает один или несколько из перечисленных случаев: 
а)	ко времени прилета прогнозируются временные изменения погоды (TEMPO) ниже минимума аэродрома пункта назначения; 
б)	прогнозируемые ко времени прилета на аэродром назначения порывы ветра, превышают эксплуатационные ограничения; 
в)	ко времени прилета прогнозируются сильные ливневые осадки, низкий коэффициент сцепления; 
г)	ко времени прилета прогнозируются фронтальные грозы. 
EXTRA DRIFTDOWN/DEPRESSURIZATION - (в CFP как EXTRA – DD/DP) дополнительное топливо для обеспечения возможности ВС выполнить при необходимости снижение и продолжить полет до запасного аэродрома при отказе двигателя или разгерметизации в наиболее критической точке на маршруте. 
EXTRA RECLEARANCE - (в CFP как EXTRA – RECL) дополнительное топливо для полета с рубежа ухода на запасной. 
EXTRA FUEL ON BOARD (в CFP как EXTRA – FOB) топливо, которое оказалось на ВС в силу производственной необходимости или нештатной ситуации (замена ВС, отмена рейса и т.д.) при выполнении суточного плана полетов. 
8.4.9.6. Во всех случаях окончательное решение о количестве топлива на борту принимает КВС на основании анализа метеорологической и аэронавигационной обстановки на маршруте, аэродромах вылета, назначения и запасных. 
8.4.9.7. Планирование и расположение запасного аэродрома по маршруту для уменьшения Contingency Fuel до 3%. 
(1) Планирование 3% ERA представляет собой основанный на эксплуатационных характеристиках метод определения Contingency Fuel как запаса топлива в значении 3% от запланированного количества топлива для полёта по маршруту, при соблюдении требований п. 4.3.6.3(c) и применения современных средств использования запасных аэродромов согласно п. 4.3.6.6 b(ii) части I Приложения 6 ИКАО, 
(2) Назначение аэродрома 3% ERA основывается на качественном и количественном допущении о том, что, даже если 3-процентный запас топлива на случай возникновения непредвиденных обстоятельств будет использован до достижения запланированного аэродрома назначения, на борту ВС будет оставаться достаточное количество топлива для выполнения посадки на этом аэродроме ERA с финальным резервом топлива.
(3) Расчет топлива с учетом местоположения аэродрома ERA не производится. Расположение аэродрома ERA позволяет выполнить безопасную посадку на нем, если уход на запасной аэродром осуществляется на крейсерском эшелоне во второй половине маршрута, но не далее точки TOD.
(4) Определение места и периода использования аэродрома 3% ERA. 
Аэродром 3% ERA расположен внутри круга с радиусом, равным 20% от общего расстояния по полетному плану, центр которого находится на запланированном маршруте на расстоянии 25% общего полетного расстояния от АД назначения или, как минимум, на расстоянии 20% от общего полетного расстояния плюс 50 nm, с учетом наибольшего из двух (см. рис 1 и рис.2); 
(5) Действия КВС в ситуации, когда полностью израсходован запас топлива на случай возникновения непредвиденных обстоятельств (плюс дополнительное количество топлива, взятое на борт по усмотрению КВС, если предусматривается) до достижения аэродрома назначения: 
а)	убедиться, что с учетом оставшихся дополнительных запасов топлива и тренда по топливу, остаток топлива над TOD будет не менее минимального количества топлива (указано в CFP), необходимого для продолжения полёта к аэродрому назначения; 
б)	если расчеты и тренд по топливу продолжает оставаться отрицательным, рассмотреть возможность ухода на ближайший по маршруту пригодный запасной аэродром на маршруте, или на аэродром 3% ERA; 
в)	в случае принятия решения ухода на аэродром 3% ERA (точка изменения плана полета – любая точка маршрута, но не далее, чем точка маршрута TOD), выполните полет на него на крейсерском эшелоне, по кратчайшему расстоянию (по возможности «прямо на») используя запасные аэродромы на маршруте, которые были ранее определены в CFP, или запасные EDTO, если полет выполнялся по правилам EDTO. 
 
 

Во всех случаях величина дополнительного топлива ограничивается значениями эксплуатационных ограничений ВС в день выполнения полёта.
8.4.10. Масса и центровка
8.4.10.1. Массовые характеристики ВС
а)	Вес пустого ВС – Basic Empty Weight (BEW). 
б)	Сухая эксплуатационная масса ВС - Dry Operating Weight (DOW, то же, что Operational Empty Weight). 
в)	Вес конструкции ВС (включает вес планера, силовых установок, жидкостей в закрытых системах и другого несъемного оборудования) - Manufacture Empty Weight (MEW).
г)	Максимальная посадочная масса ВС - Maximum Landing Weight (MLW).
д)	Максимальная коммерческая загрузка - Maximum Payload. 
е)	Максимальная взлетная масса ВС - Maximum Takeoff Weight (MTOW).
ж)	Максимальная масса, с которой разрешено выполнять руление - Maximum Taxi Weight (MTW). 
Включает массу топлива на руление.
з)	Максимальная допустимая масса ВС, не заправленного топливом - Maximum Zero Fuel Weight (MZFW).
и)	Сухая эксплуатационная масса ВС - Operational Empty Weight (OEW, то же, что Dry Operating Weight).
Масса пустого ВС (mвс) (Basic Empty Weight (BEW)) - это масса ВС после его изготовления или ремонта на заводе. Масса пустого ВС определяется взвешиванием и вписывается в формуляр ВС, в сертификат летной годности воздушного судна, а также в бортовой технический журнал. Взвешивание самолёта в период эксплуатации не требуется. Необходимость взвешивания и пересчета центровки может возникнуть в случае сложного ремонта, переоборудования в различные варианты. В случае замены или установки нового штатного оборудования производится перерасчет массы пустого ВС. Новые данные по массе приводятся в бортовом техническом журнале. 
Масса пустого ВС (Manufacture Empty Weight (MEW)) - складывается из массы: планера (mпл), массы силовых установок (mсу), массы оборудования кабины экипажа, аптечки, кислородного оборудования с дымозащитными масками, лебёдок, самолётного несъёмного бытового оборудования, массы несливаемого остатка топлива (mнот), жидкости и газов в системах.
При взвешивании не устанавливается:
а)	швартовочное и погрузочное оборудование (сетки, ремни, лямки, поддоны);
б)	все виды оборудования и приспособлений для наземного обслуживания самолёта. 
Масса пустого ВС является исходным параметром при расчете центровки и загрузки ВС.
Масса пустого снаряженного ВС (mснар.вс) (Dry Operating Weight (DOW, то же, что Operational Empty Weight)) - масса пустого ВС с основным и дополнительным снаряжением (съемным оборудованием ВС).
Стандартное снаряжение (Standard items) – оборудование и жидкости, не являющиеся несъемным оборудованием ВС и не отличающиеся для одного типа ВС.
Стандартное снаряжение может включать в себя (но не ограничено данным списком):
а)	неиспользуемое топливо;
б)	масло в двигателях;
в)	туалетные и химические жидкости;
г)	структуру кухни;
д)	дополнительное электронное оборудование.
Эксплуатационное снаряжение (Operational items) - персонал, оборудование и другие дополнения, требуемые для выполнения конкретного рейса, но не включенные в BEW.
Дополнительное снаряжение может включать в себя (но не ограничено данным списком):
а)	экипаж и его багаж;
б)	руководства и навигационную документацию;
в)	съемное сервисное оборудование пассажирского салона, кухни и бара;
г)	еду и напитки пассажиров и экипажа;
д)	спасательные жилеты, плоты, аварийный маяк;
е)	универсальные средства пакетирования.
(Смотри главу В-6 РПП типа ВС).
Масса экипажа - определяется перед полетом из расчета по 80 кг каждый член летного и кабинного экипажа с учетом ручной клади.
Масса продуктов питания (mпрод) - общая масса продуктов питания с упаковкой, посудой и контейнерами, сувениров для продажи, мягкого инвентаря и литературы (определяется взвешиванием).
Масса коммерческой загрузки (mк) - общая масса пассажиров, багажа, почты, груза.
m к. = m пасс + m бг + m пч + m гр
где mпасс - суммарная масса взрослых пассажиров, детей и 5 кг ручной клади на каждого пассажира;
mбг - суммарная масса багажа (определяется взвешиванием при регистрации);
mпч - суммарная масса почты (определяется на грузовом складе);
mгр - суммарная масса груза (определяется на грузовом складе);
Максимальная масса коммерческой загрузки (Maximum Payload) (mк.макс) - наибольшая коммерческая загрузка, ограниченная количеством пассажирских мест, вместимостью багажно-грузовых помещений и прочностью планера. 
Предельная масса коммерческой загрузки (mпред.к) - наибольшая коммерческая загрузка, определяемая требованиями безопасности полета в ожидаемых условиях предстоящего рейса.
Масса балласта (mбалл) - балансировочная масса, обеспечивающая полетную центровку ВС при отсутствии достаточной коммерческой загрузки.
Загрузка ВС - размещение (наличие) пассажиров в салонах; багажа, почты, груза, балласта в багажно-грузовых помещениях; в соответствии с центровочным графиком и схемой загрузки, сводной загрузочной ведомостью.
Масса ВС без топлива (m без т) - суммарная масса ВС, подготовленного в рейс, но не заправленного топливом.
m без т = m снар.вс + m эк + m бпр + m прод + m кз
Масса ВС без топлива используется для упрощенного расчета коммерческой загрузки.
Эксплуатационная масса ВС (m экспл) - взлетная масса ВС, но без коммерческой загрузки:
m эксп = m снар.вс + m эк + m бпр + m прод + m т
Эксплуатационная масса ВС используется при расчете предельной коммерческой загрузки, взлетной и посадочной массы ВС.
Допустимая взлетная масса ВС (mдоп взл) - наибольшая масса ВС на старте, определяемая требованиями безопасности в условиях предстоящего взлета, полета и посадки.
Величина m доп взл определяется расчетом по методике, изложенной в РЛЭ или с помощью специальной программы.
Определяется m доп взл с учетом полученных результатов характеристик и метеоусловий аэродрома вылета. Подсчитанная величина m доп взл обеспечивает безопасность на всех режимах полета. По ней ответственное лицо по загрузке производит предварительный расчет величины m пред к. 
M ПРЕД. К. = M ДОП.ВЗЛ – M ЭКСПЛ
и предварительный расчет mк.
В процессе предварительной подготовки экипаж уточняет запас топлива, допустимые посадочную, полетную и взлетную массу ВС. Ответственное лицо по загрузке производит окончательный расчет предельной коммерческой загрузки и корректирует весь расчет коммерческой загрузки.
Максимальная взлетная масса ВС (Maximum Takeoff Weight (MTOW) (mвзл мах) - наибольшая масса ВС на старте, ограниченная прочностью конструкции планера.
Безопасность полета по условию прочности конструкции ВС обеспечивается в течение срока службы ВС, только при условии, когда вышеуказанные нагрузки, в основном массовые силы, на которые рассчитана прочность конструкции, не превышает величины m взл мах.
Полетная масса ВС (m пол) - масса ВС в данный момент полета.
Полетная масса ВС непрерывно уменьшается от m взл до m пос. 
Максимально допустимая полетная масса ВС (mдоп пол) - наибольшая масса ВС, определяемая требованиями безопасности в условиях предстоящего полета.
Максимально допустимая посадочная масса ВС (mдоп пос) - наибольшая масса ВС, определяемая требованиями безопасности в условиях предстоящей посадки.
Находится максимальная допустимая посадочная масса ВС mдоп.пос с учетом характеристик основного и запасного аэродромов и ожидаемых метеоусловий. 
Максимальная посадочная масса ВС (Maximum Landing Weight (MLW) (m пос мах) - наибольшая масса ВС на посадке, ограниченная прочностью конструкции планера.
8.4.10.2. Центровочные характеристики ВС
Центровка пустого ВС - центровка, полученная по результатам взвешивания ВС после его изготовления или ремонта на заводе, фиксируется в формуляре ВС и заносится в сертификат летной годности воздушного судна и в бортовой технический журнал.
Является исходным параметром при расчете центровки ВС с помощью центровочного графика и ЭВМ.
Центровка пустого снаряженного ВС - центровка пустого ВС с основным и дополнительным снаряжением (из центровки, указанной в формуляре, бортовом журнале вычитается 0,6% САХ).
Взлетная центровка ВС - центровка ВС на старте при взлетной массе ВС и выпущенных шасси.
Посадочная центровка ВС - центровка ВС на посадке при посадочной массе ВС и выпущенном шасси.
Полетная центровка ВС - центровка ВС в данный момент полета при убранном шасси.
Предельно допустимые полетные центровки ВС - это крайние значения центровки: предельно передняя и предельно задняя, которые допускаются на взлете, в полете и на посадке ВС данного типа.
Диапазон предельно допустимых полетных центровок ВС - это разность между предельно допустимой передней и задней полетными центровками. Диапазон предельно допустимых центровок оказывает влияние на экономическую эффективность ВС.
Центровка опрокидывания ВС – центровка, при которой возможно опрокидывание ВС на хвост на земле.
Рекомендуемая центровка самолета - центровка, способствующая наилучшей устойчивости и управляемости ВС в полете при обеспечении приемлемой экономической эффективности полета.
Центровочный график (ЦГ) – официальный рабочий документ, в котором зафиксированы данные рейса, расчет предельной коммерческой загрузки, распределение фактической загрузки в соответствии с заданным диапазоном предельно допустимых полетных центровок ВС и полученные при этом взлетно-посадочные массы и центровки. 
Схема загрузки самолета (СЗ) – официальный рабочий документ, в котором по данным ЦГ зафиксировано требуемое размещение багажа, почты и грузов в багажно-грузовых помещениях ВС.
8.4.10.3. Основные документы, связанные с загрузкой и центровкой ВС
Сводная загрузочная ведомость (Loadsheet) - один из основных полетных документов; содержит следующую информацию:
а)	основные данные рейса и самолета;
б)	коммерческая загрузка самолета (расчет предельной и фактической коммерческой загрузки, распределение коммерческой загрузки);
в)	масса самолета;
г)	центровка самолета и распределение пассажиров по зонам салонов;
д)	изменения в последнюю минуту.
Схема загрузки самолета (СЗ) - официальный рабочий документ, в котором по данным центровочного графика зафиксировано требуемое размещение багажа, почты и грузов в багажно - грузовых помещениях ВС.
Центровочный график (Trim sheet, Trim chart, Weight and balance sheet, Balance chart)-официальный рабочий документ, в который внесены данные рейса, данные расчета предельной коммерческой загрузки , порядок распределения фактической загрузки в соответствии с заданным диапазоном предельно-допустимых полетных центровок самолета и полученные при этом взлетно-посадочные массы и центровки.
Бланки центровочных графиков составляются разработчиком ВС или квалифицированными специалистами Авиакомпании, если допустимо.
8.4.10.4. Методы, процедуры и должностные лица, ответственные за расчет масс и центровок ВС
Службы организации перевозок (СОП) аэропортов определяют методы и процедуры определения массы и центровки ВС с оформлением сопроводительной документации и несут за это ответственность. В случае отсутствия технической возможности в аэропорту отправления, центровочный график составляется экипажем ВС.
Если у экипажа возникли сомнения относительно достоверности данных о коммерческой загрузке,  
 
указанных в сопроводительных документах, командир ВС вправе потребовать от СОП аэропорта вылета повторного взвешивания багажа или груза. При этом командир ВС ответственности за задержку вылета по этой причине не несет. Командир ВС имеет право потребовать контрольного взвешивания загрузки и после посадки (в аэропорту назначения или промежуточном), если у него возникли сомнения в достоверности сведений о загрузке в процессе полета.
В аэропортах, не укомплектованных группами центровки, обязанности ответственного лица по центровке выполняет второй пилот, а обязанности ответственного лица по загрузке - второй пилот или бортоператор.
Перед вылетом командир ВС обязан проверить правильность расчета массы, центровки по центровочному графику и подписать его. В случае обнаружения ошибок необходимо исправить их с помощью ответственного лица СОП или лично.
Командир ВС несет ответственность за соответствие фактического количества пассажиров на борту судна количеству, указанному в перевозочных документах (при наличии в составе экипажа бортпроводника – по его докладу).
В Авиакомпании ответственность за формирование сводной информации о весовых и центровочных данных ВС несет инженерно-авиационная служба. Ответственность за предоставление актуальной информации в аэропорты полетов несет отдел наземного обслуживания.
8.4.10.5. Изменения в последнюю минуту
Возможные изменения в последнюю минуту. Если фактическая коммерческая загрузка не соответствует заявленной в сводной загрузочной ведомости (Loadsheet), представитель СОП вносит изменения в разделе «Изменения в последнюю минуту» (Last minute changes) с указанием ответственного за загрузку лица. Добавление (снятие) коммерческой загрузки согласовывается с командиром ВС. При этом контролируются допустимые ограничения по весовым и центровочным характеристикам ВС, изменяющимся в связи с изменением фактической загрузки от плановой.
Допускается максимальное изменение коммерческой загрузки без выпуска новой сводной загрузочной ведомости в пределах:
                                                                                                                                   Таблица А8.4-Т8
Самолеты с максимальной взлетной массой (т)	Ан-2 с максимальной взлетной массой (т)
до 30	300 кг	До 5	Не допускается
от 30 до 75	400 кг	Более 5	150 кг
Информация для обслуживающей организации аэропорта, необходимая для выполнения расчетов весов и центровок, предоставляется экипажем в устной форме (в том числе по телефону), РД по радио для ПДСП транзитного аэропорта. Ответственность за предоставляемую информацию несет командир ВС. 
8.4.10.6. Порядок использования каждой нормативной и/или фактической массы
Фактические массы изменяются по величине, к ним относятся: m т; m доп взл; m пасс; m бг; m пч; m гр; m прод.; m эк.; m вс . Информация о массах содержится в следующих источниках:
а)	нормированные данные сосредоточены в плане-сводке движения ВС аэропорта в течение суток; 
б)	в приложении РЦЗ - 83;
в)	в справочных таблицах о базовых и транзитных ВС;
г)	РЗЦ – типа ВС;
д)	в сводно-загрузочных ведомостях (Load sheet) транзитных самолетов.
m т и m доп взл заранее подсчитаны для простых метеоусловий по маршрутам полетов. 
Остальные фактические данные (m пасс, m бг, m прод, m пч, m гр) определяются на основании предварительной информации о количестве проданных билетов и взвешивания.
m доп взл определяется командиром ВС.
Ответственное лицо по центровке вписывает исходные данные в верхнюю часть центровочного графика, определяет эксплуатационную массу ВС и предельную коммерческую загрузку.
m экспл = m снар сам + m эк + m бпров + m прод + m т.
m пред комм = m доп взл - m экспл.
В случае изменения условий экипаж вносит коррективы в предварительный расчет коммерческой загрузки.
Ответственное лицо по центровке производит расчет размещения загрузки на ВС с помощью ЦГ. Исходными данными для расчета являются: центровка пустого снаряженного ВС и рекомендуемая центровка.
Если коммерческой загрузки недостаточно для обеспечения центровки в допустимых пределах, то следует использовать балласт. Ответственное лицо по центровке рассчитывает массу и местоположение балласта. Балласт учитывается в фактической коммерческой загрузке ВС.
8.4.10.7. Способы установления соответствующих масс пассажиров
При расчете массы пассажиров на самолетах внутренних и международных линий руководствоваться следующими нормативами:
(1) Внутрироссийские и международные линии:
(1.1) Масса взрослого пассажира за исключением вещей, находящиеся при нем (ручной клади):
а) с последнего воскресенья октября по последней субботе марта - m пасс = 80 кг;
б) с последнего воскресения марта по последнюю субботу октября - m пасс = 75 кг.(1.2) Масса детей от 2 до 12 лет – m рб = 30 кг;
(1.3.) Масса детей до 2 лет - m рм = 15 кг.
(2) При выполнении полетов по контрактам с миссиями ООН на весь период контракта, независимо от времени года:
а) Масса одного пассажира (m пасс) = 85 кг.	
(3) Нормативная масса одного члена летного экипажа = 80 кг. 
(4) Нормативная масса одного члена кабинного экипажа = 75 кг (с ручной кладью). 
При расчете загрузки и центровки должен учитываться фактический вес ручной клади пассажиров (по данным системы регистрации).
В случае, отсутствия данных о фактическом весе ручной клади пассажиров, а также при отсутствии технической возможности внесения данных в программу автоматизированного расчета загрузки и центровки, необходимо учитывать среднестатистические значения веса ручной клади – 3 кг на каждого взрослого пассажира. 
8.4.10.8. Учет влияния пассажиров на центровку 
Рекомендуемым и более точным методом учета влияния пассажиров на центровку самолета является индивидуальный учет расположения каждого пассажира. 
8.4.10.9. Определение эксплуатационных (операционных) ограничений по массам и центровкам 
С целью обеспечения соответствия центровки загруженного самолета сертифицированным ограничениям, приведенным в Airplane Flight Manual и Weight and Balance Manual (для CL300) или РЛЭ (для остальных типов ВС), при разработке центровочного графика определяются эксплуатационные (операционные) ограничения (operational limits) по массам и центровкам, включающие в себя факторы влияния на центр тяжести самолета: 
а)	перемещения шасси и закрылок; 
б)	неравномерности размещения пассажиров; 
в)	неравномерности размещения загрузки багажных отсеков; 
г)	перемещения пассажиров, персонала и оборудования в полете; 
д)	изменения центровки при заправке/выработке топлива с учетом различных плотностей топлива. 
Рассчитанные эксплуатационные (операционные) границы центровки указываются во всей массовой и центровочной документации (центровочные графики) и, соответственно, заносятся во все автоматизированные системы расчета загрузки и центровки.
8.4.11. Составление и подача планов полетов
8.4.11.1. Планирование полетов
Планирование полетов осуществляется с целью согласования работы служб Авиакомпании, повышения экономической эффективности производства.
Планирование полетов включает в себя комплекс мероприятий, обеспечивающих наиболее эффективное использование экипажей и воздушных судов авиакомпании для выполнения коммерческого плана, обеспечения графика труда и отдыха экипажей и безопасности полетов.
Объем процедур предварительного и непосредственного планирования определяется в зависимости от вида предполагаемого полета и вида авиационных работ:
а)	регулярные полеты;
б)	полеты по выполнению авиационных работ;
в)	чартерные полеты;
г)	чартерные международные полеты.
Оперативное планирование осуществляется диспетчером ОКВР (отдел контроля выполнения рейсов).
8.4.11.2. Составление повторяющегося плана полета (РПЛ)
При составлении повторяющегося плана полета (РПЛ), флайт-плана (ФПЛ) руководствоваться требованиями «Табеля сообщений о движении воздушных судов в Российской Федерации», утверждённого приказом Минтранса РФ от 24.01.2013г. № 13, а также Федеральными авиационными правилами «Организация планирования использования воздушного пространства Российской Федерации», утвержденными приказом Минтранса РФ от 16.01.2012г. №6.
Подача РПЛ производится на все регулярные рейсы по расписанию. Сообщение о повторяющемся плане полета воздушного судна передается не менее чем за 14 суток, а изменения, вносимые в этот план, представляются не менее чем за 7 суток.
При составлении представленного плана полета (FPL) руководствоваться требованиями «Табеля сообщений о движении воздушных судов в Российской Федерации», утверждённого приказом Минтранса 
РФ от 24.01.2013г. № 13, а также Правилами полетов и обслуживания Воздушного движения (Doc4444 RAC/501ICAO). Подача FPL производится на каждый полет.
Все виды полетов должны иметь:
а)	страховое обеспечение, которое включает: страхование ВС (по решению Авиакомпании или по требованию страны пребывания), страхование пассажиров, экипажа, ответственности перед третьими лицами;
б)	финансовое обеспечение, которое включает оплату аэронавигации, аэропортовых сборов, ГСМ и т.д.
в)	КОНТРОЛЬ ЗА ПОВТОРЯЮЩИМИСЯ ПЛАНАМИ ПОЛЕТОВ (РПЛ) ВОЗЛОЖЕН НА КОММЕРЧЕСКИЙ ОТДЕЛ.
8.4.11.3. Планирование полетов воздушных судов, допущенных к RVSM
Если ВС допущено к RVSM, то, независимо от маршрута и эшелона планируемого полета, в п. 10 FPL (оборудование) в дополнение к остальным индексам вносится индекс «W», а при составлении RPL в п. Q вносится обозначение «EQPT/W».
Если маршрут планируемого полета пересекает географические границы района действия EUR RVSM, то в п.15 FPL (ROUTE) и п.Q RPL (Enroute) дополнительно включаются:
а)	точка входа (Entry point) и запрашиваемый эшелон полета (Requested Flight Level–RFL) в пределах EUR RVSM;
б)	точка выхода (Exit point) и запрашиваемый эшелон полета RFL за пределами EUR RVSM.
в)	Запрашиваемые эшелоны RFL в пределах EUR RVSM выбираются:
г)	в общем случае – в зависимости от направления полета;
д)	в районах и на маршрутах, где действуют схемы распределения эшелонов – в соответствии с опубликованными FLAS, но не выше FL390;
е)	в районах и на маршрутах, где действуют соглашения между центрами ОВД - в соответствии с информацией, опубликованной в документах аэронавигационной информации (AIP), но не выше FL390.
Запрашиваемые эшелоны RFL в пределах воздушного пространства Китая выбираются в соответствии с Приложением С-1.3.
Запрашиваемые эшелоны (RFL) за пределами EUR RVSM выбираются:
а)	в общем случае при выходе из воздушного пространства EUR RVSM (сплошного применения и транзитных зон) - эшелоны CVSM в зависимости от направления полета согласно Tables of Cruising Levels ICAO Annex 2, Appendix 3, а), но не выше FL390; 
б)	а при вхождении в воздушное пространство России - согласно правил эшелонированияР в воздушном пространстве РФ;
в)	в районах и на маршрутах, где действуют схемы распределения эшелонов — в соответствии с опубликованными FLAS, но не выше FL390;
г)	в районах и на маршрутах, где действуют соглашения между центрами ОВД -в соответствии с информацией, опубликованной в документах аэронавигационной информации (AIP), но не выше FL390 или 12100. 
8.4.11.4. Планирование полетов ВС, не допущенных к RVSM
При составлении FPL для международного рейса в пределах региона EUR, который планируется выполнять на ВС, не допущенном к RVSM, в п.10 FPL обозначение «W» не вносится, в п.15 FPL (ROUTE) дополнительно вносится: 
а)	точка входа в EUR RVSM и RFL ниже FL 290 (FL280, FL270);
б)	точка выхода из EUR RVSM и RFL за пределами EUR RVSM.
Если планируемый полет ранее был обеспечен FPL с внесенным в п.10 FPL обозначением «W», а для выполнения полета выделено не допущенное к RVSM ВС, то специалистами в органы ОВД представляется: 
а)	сообщение «CHG» с измененным статусом RVSM - если маршрут планируемого полета не пересекает боковых границ EUR RVSM; 
б)	сообщение об отмене ранее поданного FPL и представляется новый FPL, составленный с учетом требований, если маршрут планируемого полета пересекает боковые границы EUR RVSM. 
8.4.11.5. Особенности планирования полетов в регионах применения зональной навигации
Особенности планирования полётов в условиях зональной навигации обусловлены выполнением требований навигационных спецификаций RNAV, RNP и необходимостью применения специальных процедур обслуживания ВС, не сертифицированных или потерявших возможность применения необходимых спецификаций.
Если ВС допущено к полетам в условиях зональной навигации, то независимо от маршрута планируемого полета в поле 10 FPL (EQUIPMENT AND CAPABILITY) вносится индекс «R». В поле 18 FPL после группы знаков PBN/ указываются достижимые уровни основанной на характеристиках навигации.
В том случае, когда при подготовке к полету определена неисправность или снижение характеристик точности бортового оборудования и требования RNAV, RNP не могут быть выполнены, необходимо подать в орган ОрВД новый FPL. В поле 18 FPL соответствующий индекс навигационных спецификаций RNAV и/или RNP не вносится, предыдущий FPL должен быть отменен в установленном порядке.
8.4.11.6. Особенности планирования полетов в регионах применения радиостанций диапазона VHF с интервалом сетки частоты 8.33 Кгц.
Особенности планирования полетов в регионах применения радиостанций диапазона VHF с интервалом 
сетки частоты 8,33 Кгц продиктованы алгоритмами одобрения плана полета в автоматизированных системах. 
Если это установлено требованиями AIP государства, в пределах которого выполняется полет, воздушное судно должно быть оборудовано двумя независимыми радиостанциями диапазона VHF с сеткой 8,33кГц. В поле 10 FPL (EQUIPMENT) должна быть указана буква «Y». 
Для ВС, не оборудованных для ведения радиосвязи с разносом частот 8,33кГц, в п.10 FPL буква «Y» не указывается, а при наличии имеющегося освобождения, в п.18 FPL должно быть указано «STS/ EXM 8.33». 
При отсутствии или неисправности связного оборудования с разделением сетки частот с интервалом 8,33 кГц, эшелон полета в поле 15 (ROUTE) планировать согласно требованиям государства, в пределах которого выполняется полет. В пределах Европейского региона полеты выполнять не выше FL245, в ВП Франции - FL195.
8.4.11.7. Особенности планирования через/в зону свободного полета (Free Route Airspace (FRA)) 
В настоящее время во многих европейских государствах, для увеличения пропускной способности, повышения эффективности и экономичности воздушного пространства применяется система Free Route Airspace (FRA). 
FRA является специфическим воздушным пространством, в пределах которого пользователи должны свободно планировать свои маршруты между точкой входа и точкой выхода без привязки к сети маршрутов ОВД. 
Введение FRA может вводить и существенные ограничения: 
а)	использование только в ночное время; 
б)	использование определенных эшелонов полета; 
в)	использование только для транзитных рейсов без изменения эшелона полета и истинной воздушной скорости (TAS). 
При планировании и выполнении международных полетов через воздушное пространство, в котором используется FRA, необходимо применять положение раздела AIR TRAFFIC CONTROL сборника JEPPESEN, о правилах полетов в воздушном пространстве, пролетаемых государств, в котором применяется система Free Route Airspace(FRA).
8.4.11.8. Порядок составления стандартных телеграфных сообщений по ИВП и ОВД
(1) Телеграфные сообщения - заявки на ИВП составляются на стандартных бланках, другие сообщения - на бланках обычных телеграмм.
Бланки ППЛ, ФПЛ и телеграмм с другими видами стандартных сообщений по ИВП и ОВД заполняются с обязательным соблюдением следующих правил:
а)	Бланки планов полетов по ИВП и других стандартных сообщений по ОВД на внутренние полеты заполняются печатными буквами русского алфавита.
б)	Бланки ФПЛ и других стандартных сообщений по ОВД на международные полеты заполняются печатными буквами латинского алфавита.
в)	Время в планах полетов по ИВП и других стандартных сообщениях по ОВД указывается Всемирное координационное (UTC).
г)	При заполнении бланков перенос текста на другую строку допускается только целыми группами, без их разрывов. Общее количество знаков в одной строке, включая промежутки между полями, не должно превышать 69.
д)	Каждое сообщение по ИВП и ОВД оформляется в виде телеграммы и передается установленным порядком в органы, обеспечивающие полет ВС (деятельность по ИВП и ОВД).
е)	Первая (адресная) часть сообщения заполняется диспетчером АДП, ЗЦ (получаемого от своего военного сектора) в соответствии с правилами адресования и передачи телеграфных сообщений.
ж)	Вторая (информационная) часть сообщения заполняется КВС (эксплуатантом ВС) и включает:
	данные, предусмотренные полями 3 – 18, они заключены в круглые скобки и предназначены для автоматической обработки в вычислительных комплексах АС УВД и передачи данных на пункты ОВД этих систем;
	дополнительную информацию о полете; она помещена за пределами круглых скобок в сообщениях (кроме планов полетов) и предназначена для других служб аэропортов, заполняется диспетчером АДП.
з)	Третья часть планов по ИВП (поле 19 ФПЛ и оборотная сторона ППЛ) заполняется КВС установленными данными во всех случаях и в телеграфных сообщениях не передается, за исключением подписных данных некоторых сообщений.
и)	Сообщения, касающиеся ППЛ и информация за круглыми скобками других сообщений по ИВП и ОВД, передаются только в адреса на территории РФ.
к)	Данные, заключенные в круглые скобки, в сообщениях всех видов по ИВП и ОВД должны быть составлены в порядке и последовательности, определенных Таблицей стандартных сообщений по ОВД. Искажение порядка и формата сообщений недопустимо в связи с тем, что автоматизированная обработка этих данных при искажении невозможна.
(2) Анализ маршрута.
До начала выполнения полетов по любому маршруту выполняется анализ маршрута, аэропорта назначения и запасных аэропортов включая следующее:
а)	навигационных средств по маршруту и аэропортов;
б)	ОВД маршрута и аэродромов;
в)	климатических условий на аэродромах или регионов выполнения авиаработ;
г)	пригодность ВПП для данного типа ВС и светотехническое обеспечение;
д)	местоположение центров спасения;
е)	возможности спасательных и пожарных команд, включая временные периоды снижения их возможностей; 
ж)	условий, что время полета с крейсерской скоростью при одной неработающей силовой установке ВС с двумя газотурбинными двигателями от какой-либо точки маршрута до соответствующего требованиям запасного аэродрома не превышает 60 минут полета;
з)	определить политику при отказе двигателя применительно для данного маршрута (требования ETOPS); 
и)	рассчитать (определить) значения минимальных безопасных высот для всех фаз полета по данному маршруту;
к)	направления снижения в случае разгерметизации над критическими областями (зонами с высокой местностью);
л)	определить высоты полета ВС при возникновении внештатной ситуации с отказом наиболее критического двигателя самолета, или, когда ВС с двумя двигателями теряет свою способность к продолжению полета по заданному маршруту. При этом должна предусматриваться возможность продолжения полета до аэродрома совершения безопасной посадки, не снижаясь ниже минимальной высоты в любой фазе полета. Если это требование не выполняется, изменить маршрут полета;
м)	выполнить инженерно-штурманский расчет по маршруту;
н)	определить наличие зон, в которых действуют требования RVSM, RNAV;
о)	произвести расчет эксплуатационных минимумов основных и запасных аэродромов;
п)	определить категорию сложности аэродромов и внести их в списки с соответствующей классификацией (категории А, В, С). 
(3) При полетах над удаленными или малонаселенными районами должна быть предоставлена экипажам информация по аэропортам, которые экипаж может использовать в случае экстренной ситуации на маршруте. Выбор конкретных аэродромов перед вылетом производится по данным сборников АНИ для конкретного типа ВС и регламентам работы аэродромов. 
(4) Необходимые данные и вид заявки, подаваемой для получения разрешения, определяются различными факторами (требованиями страны предполагаемого полета и др.) ответственное лицо запрашивает у соответствующих отделов авиакомпании необходимые данные в соответствии с этими требованиями, и затем, составляет и подает заявку.
(5) При разработке маршрута и выполнении заявки ответственное лицо авиакомпании изучает и учитывает ограничения по трассам и аэродромам в соответствии с извещениями САИ и NOTAM.
(6) В результате анализа полученная информация предоставляется службам подготовки полета и летным экипажам. Данная информация предоставляется экипажу перед полетом.
(7) При получении извещений об изменениях, влияющих на условия разработки маршрута, производятся соответствующие корректировки.
(8) Проверки используемых маршрутов на соответствие установленным требованиям производится при изменениях в документах АНИ.
(9) Дополнительные условия при анализе маршрута полета и выборе запасных аэродромов по маршруту: 
а)	Маршрут полета ВС должен быть запланирован таким образом, чтобы в любой точке маршрута имелся запасной аэродром на случай отказа двигателя (двигателей). Максимальная взлетная масса ВС дополнительно ограничивается возможностью выполнения процедуры снижения «drift-down», при этом траектория полета с отказавшим двигателем (двигателями) должна иметь положительный градиент набора на высоте 450 м (1500 ft) над уровнем аэродрома, на котором предполагается посадка после отказа (выключения) двигателей.
б)	Расчетная масса ВС над точкой отказа двигателя (двигателей) должна включать достаточное количество топлива для полета до аэродрома предполагаемой посадки, выполнения захода на посадку на высоте 450 м (1500 ft) и полета в течение не менее 30 минут.
в)	При метеорологических условиях, в которых требуется включение противообледенительной системы ВС, необходимо учитывать ее влияние.
г)	Градиент траектории полета ВС с одним отказавшим двигателем должен быть положительным на высоте, обеспечивающей запас пролета над препятствиями не менее 300 м (1000ft) в пределах 9,3 км (5 NM) по обе стороны запланированного маршрута.
д)	Траектория полета ВС с отказавшим двигателем (двумя отказавшими двигателями на ВС с тремя и более двигателями) с эшелона полета до запасного аэродрома должна обеспечивать запас высоты пролета препятствий не менее 600 м (2000 ft) в пределах 9,3 км (5 NM) по обе стороны запланированного маршрута ухода на запасной аэродром.
е)	Если навигационная точность не отвечает уровню точности 95%, то расстояние учета препятствий по обе стороны от оси запланированного маршрута должно быть увеличено до 18,5 км (10 NM).
Вышеуказанные требования применимы для ВС с тремя и более двигателями, когда время полета ВС до запасного аэродрома превышает 90 минут на крейсерской скорости, соответствующей скорости полета со всеми работающими двигателями при стандартной атмосфере в штилевых условиях.
8.4.12. Рабочий план полета
8.4.12.1. Рабочий план полёта (operational flight plan (OFP)) является внутренним документом авиакомпании и служит для повышения качества подготовки к полётам и их выполнения.
Рабочий план полета является частью комплекта полетной документации, составляемого эксплуатантом для безопасного выполнения полета с учетом летно-технических характеристик ВС, эксплуатационных ограничений и ожидаемых условий на заданном маршруте и соответствующих аэродромах, содержит информацию, касающуюся наиболее важных, с точки зрения безопасности, вопросов планирования и подготовки рейса.
8.4.12.2. Полеты воздушных судов авиакомпании обеспечиваются Рабочим планом полёта (навигационным планом полета), выполненным ОАО. Давность выпуска навигационного плана (Computer Flight Plan (CFP)) не должна превышать трех часов до времени вылета. Рабочий план предоставляются экипажу группой АНШОП. 
8.4.12.3. Образец Рабочего плана полета, определяющий форму, содержание и правила внесения записей, приведены в главе А-13 РПП. Рабочий план полёта утверждается и подписывается командиром ВС. 
Первая страница рабочего плана полётов с подписью командира ВС и Чек-лист готовности к полету передаются представителю Авиакомпании или, если это невозможно, сдаются на хранение в пункте вылета и хранятся в течение трех месяцев. Второй экземпляр чек – листа прикладывается к заданию на полет и хранится в авиакомпании в течение 90 дней. С момента подписания Рабочий план полета приобретает статус официального документа, на основании которого производится контроль выполнения полета. Рабочий план полета располагается в кабине экипажа в доступном для использования летным экипажем месте.
8.4.12.4. Если в процессе предполётной подготовки нет возможности получить новый навигационный расчет полёта (СFP), но возникла необходимость выбрать другой запасной аэродром или коммерческая загрузка значительно отличается от загрузки, указанной в расчете, значение количества топлива может быть изменено КВС в том числе и в сторону уменьшения. В таком случае экипаж производит ручную коррекцию всего расчета топлива (в том числе и в маршрутной части), используя РЛЭ (FCOM, AFM). 
Расчёт производится на действующем бланке расчёта CFP и заверяется подписью КВС.
8.4.12.5. Навигационный план полета располагается в кабине экипажа в месте доступном для использования экипажем. При вылете экипаж использует расчетные взлётные данные для контроля выдерживания режима на взлёте.
8.4.12.6. В процессе выполнения полёта по маршруту экипаж использует Рабочий план полета для контроля выдерживания заданного маршрута полёта. Контролирует расход топлива, записывает в соответствующие графы остаток топлива с интервалом не мене 30 минут, уточняет расчётное время пролёта контрольных рубежей и расчетное время прибытия в аэропорт назначения.
В соответствующие графы записываются определённые в полёте направление и скорость ветра, а также записываются данные произведённого осреднения показаний высотомеров после занятия эшелона полёта.
При подходе к аэродрому посадки экипаж уточняет расчётные посадочные данные, заполняет данные по схемам захода (посадочный минимум, МПУ пос., превышение аэродрома и т.д.) и контролирует выдерживание режима захода на посадку. 
После окончания полета записывается остаток топлива и фактическое время полёта – общее и ночью. Заполненный рабочий экземпляр навигационного расчета полёта подписывается штурманом (вторым пилотом), утверждается командиром ВС и прилагается к Заданию на полет.
8.4.12.7. Рабочий план полёта включает в себя: 
а)	навигационный расчет полёта. (содержание изложено в главе А-13); 
б)	метеорологическую информацию о фактической и прогнозируемой погоде по маршруту полета, на аэродромах назначения и запасных, синоптические карты;
в)	бюллетень полетной информации (лист предупреждений NОТАМ); 
г)	эксплуатационное уведомление с приложениями, если таковые имеются;
д)	NОТАМ Авиакомпании; 
ж)  бланки «Чек-лист готовности к полету» (Приложение А-8.4. РПП); 
и)  бланки палеток «Взлет» - «Посадка» соответствующего типа ВС; 
к)  паспорт ВС (Статус ВС), с указанием перечня отложенных неисправностей (MEL/CDL) и информации об индивидуальных особенностях ВС; 
л) другая оперативная информация, необходимая для предполетной подготовки и выполнения полета. 
8.4.12.8. Запрос на новый навигационный расчет (CFP) или его перевыпуск производится в следующих случаях: 
а)	произошла замена ВС; 
б)	в качестве первого запасного аэродрома выбран другой аэродром; 
в)	рейс с вылетом задерживается более чем на 3 часа; 
г)	экипаж обнаружил ошибки в маршруте, или маршрут отличается от указанного в разрешении на полёт; 
д)	при планировании полётов через Северную Атлантику выбранный профиль полёта не учитывает структуру организованных треков NAT HLA; 
ж)  в любом случае по требованию КВС. 
8.5. Бортовой журнал технического состояния воздушного судна
В АО «ЮТэйр» используются бортовые журналы ВС отечественного производства (ОП) и бортовые журналы ВС иностранного производства (ИП) CL 300.

8.5.1. Бортовой журнал технического состояния ВС отечественного производства
Бортовой журнал самолёта предназначен для контроля за техническим состоянием и оформлением приёма-передачи воздушного судна. При выполнении полета бортовой журнал должен находиться на борту ВС.
В авиакомпании используется типовой бортовой журнал, введенный указанием МГА от 01.11.1975г. №161, включающий в себя XI разделов.
8.5.2. Бортовой журнал технического состояния CL 300 
Бортовой журнал ВС CL 300 предназначен для контроля за техническим состоянием ВС и полнотой выполнения работ, предусмотренных программой технического обслуживания данного ВС, оформления процедуры приема-передачи имущества и документации ВС между ИТП и ЛС или между экипажами (когда ВС не передается ИАС).
В бортовом журнале (далее – БЖ) вносятся также отказы и неисправности АТ, обнаруженные в полете экипажем (далее - ЛС) и техническим составом (далее – ИТП) во время технического обслуживания ВС (далее - ТО) в (вне) аэропорта с базовой станцией ТО, и информация ИТП о способе их устранения.
Примечание. Порядок использования бортовых журналов ОП и ИП описан в Приложении А 8.11.
8.6. Документы находящиеся на ВС во время полёта
8.6.1. Общие положения 
На воздушном судне должны находиться документы, которые члены экипажа воздушного судна предъявляют по требованию уполномоченных должностных лиц. Эти документы условно разделяются на судовые и полётные.
К группе судовых относятся документы:
	подтверждающие соответствие Авиакомпании и ВС требованиям авиационного законодательства;
	необходимые экипажу в нормальных, аварийных и нештатных ситуациях.
Отличительными особенностями судовых документов являются длительный срок действия и постоянное нахождение на борту ВС.
К группе полётных относятся документы:
	подтверждающие квалификацию членов экипажа;
	используемые экипажем для подготовки и выполнения полёта.
Отличительными особенностями полётных документов являются:
	кратковременный (только на предстоящий пролёт, либо серию полётов) срок действия;
	нахождение на ВС только на период выполнения полёта так как на ВС их доставляет экипаж;
	члены экипажа несут ответственность за актуализацию документов. 
Подготовка судовой документации - Оформление (проверка) документа перед размещением на ВС в соответствие с действующими правилами.
Комплектация судовой документации - Доставка документов на борт ВС, размещение их в установленном месте.
Ведение судовой документации – поддержание документации в состоянии пригодном для пользования, замена листов, обложек, пришедших в негодность, а также внесение изменений и дополнений в экземпляры документов.
Полеты ВС без действующих судовых и полетных документов запрещаются.
Перечень судовой и полетной документации определяется документами уполномоченного органа в области ГА и приказами (указаниями) директора авиакомпании. (Приложение А-8.7 к настоящей главе и Часть А-13 «Отчетная документация о полете»). 
Судовые и полётные документы могут находиться на ВС на бумажных или электронных носителях.
8.6.2. Судовые документы
8.6.2.1. Учет и хранение судовой документации, организация ее приема-передачи 
(1) Комплектация судовых документов ВС осуществляется согласно Приложению А-8.7.
(2) Комплектация портфеля судовыми документами осуществляется ведущим инженером по судовой документации Управления поддержания лётной годности ВС ИАС Авиакомпании (далее – УПЛГ ВС ИАС Авиакомпании). 
(3) Сотрудники Авиакомпании, ответственные за первоначальную подготовку (замену) судовых документов (Приложение А-8.7 к настоящей главе) передают их в УПЛГ ВС ИАС Авиакомпании в готовом для размещения на ВС виде и в нужном количестве, при необходимости с приложением инструкции по внесению и с указанием сроков размещения на ВС.
Копии документов заверяются лицом, ответственным за подготовку документа, при этом ответственный за подготовку документа должен быть наделен данным полномочием в соответствии с приказом Авиакомпании. 
(4) Должностные лица УПЛГ ВС ИАС Авиакомпании, маркируют судовые документы государственным и регистрационным знаками экземпляра ВС. 
(5) Ведущий инженер по судовой документации организовывает доставку портфеля с судовой документацией к месту базирования ВС и осуществляет контроль по размещению портфеля на ВС.
(6) Ответственность за сохранность судовой документации и бортового журнала несет работник, принявший бортовой журнал и портфель с судовой документацией:
	бортинженер (бортмеханик), 2й пилот – во время выполнения рейса;
	начальник участка ТО ВС (руководитель ТО ВС) – при выполнении оперативного технического обслуживания;
	должностное лицо УПЛГ ВС ИАС Авиакомпании – при передаче судовой документации и бортового журнала в ИАС, при длительных перерывах в полетах, выполнении периодического технического обслуживания и хранении ВС.
Ответственным за сохранность судовой документации, находящейся на борту ВС, является ИТП или член экипажа, принявший самолет под роспись в бортовом журнале. 
(7) Прием-передача судовой документации от ИТП экипажу, между членами экипажа и ИТП, осуществляется при приеме-передаче ВС и подтверждается подписями в разделе X бортового журнала «Прием-передача ВС».
При приеме специалист проверяет наличие документов, согласно Перечню судовой документации, расписывается за прием, а сдающий расписывается за сдачу. 
(8) Ответственный за сохранность судовой документации обязан:
	при приеме судовой документации проверять ее комплектность по Перечню;
	не оставлять судовую документацию без присмотра.
(9) В случае обнаружения при приеме-передаче ВС некомплектности, порчи судовой документации, должностное лицо, обнаружившее некомплектность, сообщает об этом в УПЛГ ВС ведущему инженеру по судовой документации для организации оперативного доукомплектования судовой документации. 
(10) При нахождении ВС на периодическом ТО, начальник участка ТО ВС (руководитель ТО ВС) передает портфель с судовыми документами в УПЛГ ВС ИАС Авиакомпании для проверки комплектности и актуальности судовой документации ведущим инженером по судовой документации.
(11) В период ремонта ВС судовая документация хранится в УПЛГ ВС ИАС Авиакомпании. При необходимости, до поступления ВС из ремонта, судовая документация данного ВС может быть передана другому ВС и маркирована его государственным и регистрационным знаками.  
 (12)	Экипаж ВС и ИТП обязаны обеспечить надлежащее использование, хранение и прием-передачу судовых документов. 
В промежуточных аэропортах (аэропортах временного базирования) судовая документация предъявляется для проверки только по требованию инспекторского состава уполномоченного органа воздушного транспорта или Авиакомпании в присутствии экипажа. 
Примечание: Допускается хранение судовой документации на воздушном судне, если оно в установленном порядке сдается под охрану САБ авиапредприятия, а на оперативной точке (временном аэродроме) – должностному лицу службы охраны Заказчика.
8.6.2.2. Особенности ведения судовой документации
(1) Документы, которые УПЛГ ВС получает от других подразделений Авиакомпании
Для внесения и обновления судовой документации на самолетах, отдел летных стандартов летной службы Авиакомпании предоставляет в УПЛГ ВС документы и ревизии к следующим документам:
	Руководство по летной эксплуатации ВС;
	Сборник рекомендаций для экипажа по действиям в особых случаях полета;
	Карты контрольных проверок; 
	Листы осмотра ВС экипажем;
	Сертификат эксплуатанта коммерческих воздушных перевозок и эксплуатационные спецификации;
	Сертификат эксплуатанта авиационных работ и эксплуатационная спецификация;
	Свидетельство АОН с приложениями;
	Формы донесений;
	Документы по авиационной безопасности;
Документ, поправка к документу или ревизия передаются в УПЛГ ВС ведущему инженеру по судовой документации в готовом для размещения на ВС виде, необходимом количестве бумажных экземпляров и при необходимости с краткой инструкцией по внесению в бортовой экземпляр документа. Срок размещения на ВС 15 календарных дней с момента поступления документа в УПЛГ ВС.
Руководство по производству полетов. 
РПП разрабатывается летной службой Авиакомпании и является основным нормативным документом авиакомпании, регламентирующим организацию летной работы и правил выполнения полетов.
Бортовой экземпляр РПП хранится в электронной системе бортовой документации Electronic Flight Bag (EFB). Членам экипажа организован доступ к EFB через персональные электронные устройства (IPAD). 
Ответственным за размещение актуальной версии бортовых экземпляров РПП в EFB является начальник отдела аэронавигационного обеспечения лётной службы Авиакомпании.
Порядок внесения поправок в бортовые экземпляры РПП и их плановых поверок изложены в Руководстве по производству полетов Авиакомпании (глава А-0).
Руководство по летной эксплуатации (РЛЭ).
Держателем контрольного экземпляра РЛЭ и ответственным за его ведение является отдел летных стандартов, который вносит изменения (дополнения) в контрольный экземпляр и обеспечивает тиражирование в необходимом количестве и передачу на бумажных носителях изменений (дополнений) в УПЛГ ВС для дальнейшего размещения в рабочих экземплярах на ВС.
Внесение изменений в бортовые экземпляры РЛЭ выполняет ведущий инженер по судовой документации или ответственный сотрудник из числа сотрудников летной службы либо ИАС, в течение 15 календарных дней с момента поступления изменений в УПЛГ ВС. 
Сверка рабочих экземпляров РЛЭ с контрольным экземпляром проводится 2 раза в год при подготовке ВС к ОЗП и ВЛП. 
Сборник рекомендаций по действиям экипажа в аварийных ситуациях. 
Держателем контрольного экземпляра Сборника рекомендаций по действиям экипажа в аварийных ситуациях (далее – Сборник) и ответственным за его ведение в авиакомпании является специалист отдела летных стандартов, который вносит изменения (дополнения) в контрольный экземпляр и обеспечивает тиражирование бумажных экземпляров и передачу в необходимом количестве изменений (дополнений) в УПЛГ ВС ИАС для размещения на ВС.  
Сертификат эксплуатанта коммерческих воздушных перевозок и авиационных работ с эксплуатационными спецификациями, а также свидетельство АОН с приложениями.
Копии данных документов, заверенные печатью Авиакомпании, передают в УПЛ ВС. Ведущий инженер по судовой документации в течение 5 дней обеспечивает их размещение на борту ВС и в течение 15 календарных суток при нахождении ВС на аэродромах оперативного базирования. В любом случае Эксплуатационные спецификации должны быть размещены на борту ВС до начала эксплуатация ВС в новых условиях.
 Страховые полисы.
В состав судовой документации входят страховые полисы обязательного страхования ответственности Авиакомпании перед третьими лицами, пассажирами и груза на каждое ВС, а также страхование членов экипажа от несчастного случая. Заказ и продление страховых полисов осуществляет коммерческий отдел Авиакомпании. Специалисты отдела передают заверенные копии страховых полисов, в необходимом количестве в УПЛГ ВС, не позднее 7 рабочих дней до начала их срока действия, для своевременного размещения на борту самолета. 
Свидетельство о регистрации гражданского ВС.
Выдается на весь календарный срок службы гражданского воздушного судна уполномоченным органом гражданской авиации, после регистрации его в Государственном реестре гражданских ВС.
Замена Свидетельства, выдача дубликата или внесение изменений производится в следующих случаях:
	изменение собственника ВС;
	изменение назначения ВС после его переоборудования;
	порчи или утери Свидетельства.
На ВС находится оригинал Свидетельства. Допускается размещение на ВС копии Свидетельства в течении 30 дней с даты его выдачи.
Информация по инженерно-авиационному обеспечению полетов воздушных судов (выписка из РОТО).
Выписка из РОТО разрабатывается ИАС Авиакомпании и является основным нормативным документом авиакомпании, регламентирующим организацию технического обслуживания и ремонта авиационной техники.     Сверка данного документа производится ведущим инженером по судовой документации УПЛГ ВС на основании информации, предоставленной ОТК ИАС. 
Карта контрольных проверок самолета и листы осмотра ВС экипажем.
Управляются отделом летных стандартов на основании РЛЭ ВС.
Количество экземпляров карт контрольных проверок и листов осмотра ВС экипажем на борту ВС, должно соответствовать количеству членов экипажа типа ВС.
Документы по авиационной безопасности.
Документы по авиационной безопасности, для размещения их на ВС, поступают в УПЛГ ВС ИАС Авиакомпании от специалиста по авиационной безопасности Авиакомпании. Об изменении документов, выходе новой ревизии, специалист по авиационной безопасности оповещает всех заинтересованных лиц посредством электронной почты и предоставляет на бумажном носителе необходимое количество экземпляров для размещения/замены на ВС, с указанием сроков размещения. Документы по авиационной безопасности размещаются на ВС в виде копий.
Санитарный журнал.
Контроль санитарного состояния воздушных судов осуществляется в соответствии с требованиями РПП Авиакомпании, РОТО и документа КД-ДП-В5.015 «Организация санитарно-противоэпидемических мероприятий на воздушных судах Группы «ЮТэйр».
 Проверка санитарного состояния воздушных судов проводится:
	постоянно - перед каждым вылетом;
	периодически с записью в Санитарном журнале воздушного судна – не реже одного раза в 3 месяца.       
Дополнительно периодическая проверка проводится после трудоемких регламентных работ, при поступлении воздушного судна с завода, при подготовке к полетам в осенне-зимний и весеннее-летний периоды.
При выполнении полетов за пределами Российской Федерации:
	ответственным за организацию и контроль проверки санитарного состояния воздушного судна назначается командир авиагруппы;
	лицом, уполномоченным проводить контроль санитарного состояния воздушного судна с записью в санитарном журнале воздушного судна, назначается бортпроводник (бортоператор). 
Организация оперативной замены санитарного журнала ВС в случае окончания, утраты или порчи, обеспечивается инженерно-авиационной службой.
           (2)  Документы, выданные авиационными властями Сертификат летной годности 
                                                                гражданского ВС.
Выдается (продлевается) уполномоченным органом гражданской авиации на период действующих сроков службы (ресурсов) экземпляра ВС его технического состояния, но на срок не более двух лет.
Порядок выдачи, замены и продления срока действия Сертификата летной годности ВС регламентируется Федеральными авиационными правилами. 
 На ВС находится оригинал СЛГ. Допускается размещение на ВС копии СЛГ в течение 30 дней от даты его выдачи.
Удостоверение о годности ГВС по шуму на местности.
Выдается и меняется уполномоченным органом в области гражданской авиации сроком на 2 года.
Продление срока действия Удостоверения в процессе эксплуатации производит межрегиональное территориальное управление Росавиации.
На ВС находится оригинал Удостоверения. Допускается размещение на ВС копии Удостоверения в течение 30 дней от даты его выдачи.
Разрешение на бортовую радиостанцию.
Выдается уполномоченным органом в области гражданской авиации. Срок действия документа – бессрочный. 
На ВС находится оригинал Разрешения. Допускается размещение на ВС Разрешения в течение 30 дней от даты его выдачи.
8.6.3. Полётные документы
(1)	Комплектация пакета полётных документов осуществляется членами экипажа ВС. Командир ВС осуществляет контроль полноты и качества полётных документов. 
(2)	Лица, ответственные за первоначальную подготовку (замену) полётных документов (Таблица А8.6-Т1) передают их соответствующему члену экипажа ВС в готовом для размещения на ВС виде и в нужном количестве. 
(3)	Комплектация полётных документов ВС осуществляется согласно Таблицы А8.6-Т1. 
(4)	Доставку документов на ВС осуществляют лица, указанные в столбце «Ответственный за комплектацию ВС» Таблицы А8.6-Т2.
(5)	Ответственность за сохранность полётных документов несу лица, указанные в столбце «Ответственный за комплектацию ВС» Таблицы А8.6-Т1.
(6)	Ответственный за сохранность полётных документов обязан при получении документов проверить их актуальность.
(7)	В случае обнаружения неактуальности документов член экипажа сообщает об этом командиру ВС, который принимает меры к получению нового документа, или подтверждения его актуальности. 
Примечание: запрещается выполнение полёта с каким-либо неактуальным полётным документом.
(8)	Экипаж ВС обязан обеспечить надлежащее использование, хранение и передачу полётных документов после завершения выполнения полёта. 
                                                                                                                                                 Таблица А8.6-Т1
№
п/п	Наименование документа	Ответственный за первоначальную подготовку (замену)	Ответственный за комплектацию ВС	Места размещения документа
1	Свидетельства членов экипажа ВС	Каждый член экипажа	Каждый член экипажа	Личные вещи члена экипажа
2	Приложения к свидетельству членов экипажа ВС	Каждый член экипажа	Каждый член экипажа	Личные вещи члена экипажа
3	Медицинские заключения, подтверждающее соответствие членов экипажа требованиям к состоянию их здоровья	Каждый член экипажа	Каждый член экипажа	Личные вещи члена экипажа
4	Задание на полет	Командир лётного подразделения	Командир ВС	Кабина экипажа
*5	Генеральная декларация (при международных полетах)	Подразделение государственной пограничной охраны в аэропорту вылета	Командир ВС	Кабина экипажа
*6	Сводная загрузочная ведомость	Провайдер наземного обслуживания ВС в аэропорту вылета	Второй пилот	Кабина экипажа
*7	Пассажирская ведомость (манифест)	Провайдер обслуживания пассажиров в аэропорту вылета	Второй пилот	Кабина экипажа
*8	Грузовая ведомость (манифест)	Провайдер обработки грузов в аэропорту вылета	Второй пилот	Кабина экипажа
*9	Документ, содержащий информацию об опасном грузе, предусмотренный ФАП 141	Провайдер обработки грузов в аэропорту вылета	Второй пилот	Кабина экипажа
10	Метеорологическая информация, необходимая для выполнения полета	Провайдер метеорологического обеспечения в аэропорту вылета	Командир ВС	Кабина экипажа
11	Актуализированная аэронавигационная информация, касающаяся запланированного полета	Начальник отдела аэронавигационного обеспечения	Начальник отдела аэронавигационного обеспечения	EFB (IPAD)
12	Аэронавигационные (полетные) карты	Начальник отдела аэронавигационного обеспечения	Начальник отдела аэронавигационного обеспечения	EFB (IPAD)
13	Рабочий план полета	Начальник отдела аэронавигационного обеспечения	Второй пилот (штурман)	Кабина экипажа
Примечания:
*	документы присутствуют на ВС при необходимости, продиктованной особенностью выполнения рейсов, типом ВС, иными причинами.
8.7. Послеполетные разборы
8.7.1. Общие положения
8.7.1.1. Основной целью послеполетного разбора является анализ и оценка выполненного полета на основе докладов членов экипажа и изучения полетной документации. 
Необходимость проведения послеполетного разбора обусловлено спецификой летной работы. 
Послеполетный разбор в экипаже должен быть местом подробного обсуждения членами экипажа допущенных отклонений, ошибок и нарушений. В процессе разбора должны быть определены причины отклонений и ошибок, выработаны рекомендации по их предупреждению в последующих полетах. 
Послеполетный разбор в экипаже проводит командир ВС (проверяющий) после каждого выполненного полетного задания в кабине экипажа. Послеполетный разбор может быть проведен позднее, но обязательно до начала следующего полета. 
Послеполетный разбор проводится в порядке и объеме, определенным технологией работы членов экипажа, и складывается из: 
а)	анализа каждым членом экипажа своей работы при подготовке к полету и его выполнении;
б)	разбора командиром ВС ошибок и нарушений, допущенных членами экипажа;
в)	оценки командиром ВС работы каждого члена экипажа и выполненного полета в целом;
г)	указаний и рекомендаций командира ВС по предупреждению повторяемости ошибок и нарушений.
Анализ деятельности летного экипажа должен строиться по схеме: отклонение - ошибка – причина.
8.7.2. Процедура послеполетного разбора
8.7.2.1. Командир ВС объявляет о начале разбора и предоставляет членам экипажа слово для доклада в следующей последовательности:
а)	Бортпроводник/бортоператор
б)	Бортмеханик;
в)	Штурман;
г)	Второй пилот.
8.7.2.2. После доклада бортпроводника (бортоператора) о замечаниях и пожеланиях пассажиров, (грузоотправителей), о замечаниях по работе бытового, бортового погрузочного оборудования и оценки работы - дальнейший разбор проводится в составе летного экипажа.
Бортмеханик докладывает КВС о работе материальной части в полёте, о замеченных недостатках по её лётно-технической эксплуатации (какие допущены отклонения или нарушения с анализом их причин), замечания к себе и членам экипажа, об остатке и расходе топлива, записывает в бортжурнал замечания по работе ВС и остаток топлива. Также оформляет справку о работе АТ в полёте, информирует ИТП о состоянии авиационно й техники, осматривает самолёт, докладывает КВС о результатах осмотра, сдаёт ВС в ИАС. Штурман докладывает о налёте за рейс, в том числе ночью, о работе пилотажно-навигационного оборудования и замеченных недостатках по его использованию, об отклонениях от установленных правил полёта и замечаниях по взаимодействию в полёте, осматривает ВС согласно РЛЭ и сдаёт документацию и штурманское снаряжение.
Второй пилот докладывает КВС о выполнении своих обязанностей, замечаниях и недостатках в технике пилотирования, ведения радиосвязи, о взаимодействии в полёте, замечания по лётной эксплуатации самолёта, осматривает самолёт, докладывает о результате осмотра, заполняет задание на полёт и служебную документацию.
Доклад должен содержать оценку работы оборудования и систем ВС, замечания, полученные со стороны служб, оценку собственной работы и техники пилотирования, свои замечания и предложения по улучшению работы, замечания к другим членам летного экипажа.
8.7.2.3. Командир ВС анализирует доклады членов экипажа, указывает на допущенные ошибки, помогает вскрывать причины отклонений и ошибок, оценивает работу членов экипажа, отмечает положительные и отрицательные моменты в работе, ставит задачу на предстоящий период, проверяет правильность оформления полетной документации и производит запись в задании на полет в раздел «Послеполётный разбор» и в задание кабинного экипажа.
8.8. Летные процедуры
8.8.1. Общие требования
Перед полетом члены экипажа ВС проходят предполетную подготовку в установленном объеме.
Командир ВС, член экипажа имеет право отказаться от выполнения задания на полет, если он не в состоянии его выполнить.
8.8.1.1. Основные принципы управления ресурсами экипажа:
(1) Каждый член экипажа считает: "Безопасность начинается с меня". Поэтому каждый член экипажа без колебаний высказывает свое мнение, когда возникшая ситуация или чьи-то действия представляются ему угрожающими безопасности полетов. Члены экипажа считают категорически неприемлемыми пассивность и равнодушие по отношению к любым нарушениям или недобросовестным действиям. Каждый член экипажа должен иметь мужество сказать - «Нет», если другие члены экипажа толкают его на осознанное нарушение уста-новленных правил и законов летной деятельности.
(2) Каждый член экипажа принимает ответственность за свои решения и действия. Все члены экипажа берут на себя ответственность за преодоление своих недостатков: лени, трусости, пассивности, ложного чувства товарищества. Они также берут на себя ответственность за совершенствование коллективной

деятельности в кабине, что поможет избежать 'многих проблем в процессе летной работы.
(3) Центральной фигурой в обеспечении безопасности полетов является командир воздушного судна. Поэтому именно он осознает всю полноту своей ответственности. Именно он противостоит неблагоприятным обстоятельствам, соблазну нарушать правила полетов. Главная задача командира так организовать взаимодействие в своем экипаже, чтобы максимально мобилизовать его ресурс, чтобы найти то единственно правильное решение, которое практически всегда есть в экипаже.
(4) Командир поощряет высказывание мнений другими членами экипажа, особенно, если возникли сомнения и разногласия, он активно анализирует свои действия и действия подчиненных. Побуждает других членов экипажа к конструктивной дискуссии, в том числе и к анализу собственных действий. Лучший способ реализации власти - мобилизовать наличные ресурсы членов экипажа.
(5) Обязанность всех членов экипажа и, в первую очередь, командира - создать в экипаже обстановку взаимного уважения. Постановка вопросов, обсуждение проблем и поиск коллективного решения не считаются ни чрезмерной осторожностью, ни пустой тратой времени. Если один из членов экипажа, вне зависимости от возраста и опыта, чувствует, что какие-либо действия не безопасны, остальные члены экипажа считают своим долгом разобраться в причинах его опасении и лишь тогда эти действия продолжить. Такой подход создает уверенность в информированности и владении ситуацией всеми членами экипажа и а готовности каждого к любой экстремальной ситуации.
(6) Каждый член экипажа, и в особенности командир, старается услышать другого. Каждый ищет в заданном ему вопросе не подвох или признаки подрыва авторитета, а искреннюю заинтересованность в обеспечении безопасности полета. В полете не должно быть места раздраженному тону, взаимным обвинениям, насмешкам и тем более оскорблениям. Каждый член экипажа должен помнить, что его амбиции - прямая угроза жизни пассажиров. Святая обязанность командира - немедленно погасить любые возникающие в полете трения, переведя их в конструктивную дискуссию.
(7) Каждый член экипажа твердо уверен: никакие возможные экономические потери не оправдывают риска неблагополучного завершения полета. Поэтому задержка взлета или уход на запасной аэродром не воспринимается им как досадная неприятность или недостаток.
8.8.2. Распределение обязанностей летного экипажа
8.8.2.1. Распределение обязанностей членов летного экипажа во время взлета, набора высоты, горизонтального полета, снижения, захода на посадку, посадки указаны в РЛЭ (AFM, FCOM) и в Инструкции по взаимодействию и технологии работы экипажа самолета данного типа (SOP).
8.8.2.2. Члены летного экипажа должны осуществлять взаимную перекрестную проверку действий друг друга и подтверждение правильности выполнения важнейших технологических операций, включая:
а)	изменения конфигурации ВС (положения шасси, закрылков, интерцепторов, спойлеров);
б)	установки задатчиков на высотомерах, указателях скорости;
в)	установки значений барометрического давления на высотомерах;
г)	установки задатчиков заданного эшелона полета на высотомерах;
д)	передачу управления воздушным судном от одного пилота другому;
ж) проверку правильности данных, вводимых в AFS/FMS и радиотехнические системы ВС на этапах перед взлетом и перед заходом на посадку;
и) проверку правильности расчетов масс и центровок ВС и данных, вводимых в AFS/ FMS;
к) проверку правильности расчета ЛТХ и правильности ввода их в AFS / FMS.
Процедуры проведения перекрестного контроля (crosscheck) опубликованы в Инструкциях по взаимодействию и технологиях работы членов экипажей (SOP) типов ВС (РПП, Часть В, типа ВС, гл.2).
8.8.2.3. Не пилотирующий пилот ведет инструментальный и визуальный контроль за выполнением полета. В случае потери работоспособности пилотирующего пилота, не пилотирующий пилот должен взять на себя управление ВС. 
8.8.2.4. В процессе руления и в полете, на высотах ниже 3000 м (10000ft), а также в других критических фазах полета, летный экипаж должен выполнять только те обязанности, которые связаны с непосредственным управлением ВС. 
При этом, разговоры в кабине летного экипажа, не связанные с выполнением технологических операций, запрещаются. 
Под критическими фазами полета подразумеваются этапы:
а)	управление ВС на земле, включая руление и буксировку – от начала подготовки кабины к вылету до выключения двигателей после посадки;
б)	взлет и набор высоты;
в)	заход на посадку и посадка;
г)	полет ниже 3000 м (10000 ft), исключая полет на эшелоне;
д)	смена эшелона полета;
ж)	рубежи приема диспетчерского разрешения. 
8.8.2.5. На установленных рубежах (на земле и в воздухе), члены летного экипажа проверяют готовность к выполнению очередного этапа полета по контрольным листам, разработанным для каждого типа ВС, и содержат стандартные и аварийные процедуры. 
Основным средством организации дополнительного контроля за выполнением наиболее ответственных операций, определяющих готовность ВС и экипажа к очередному рубежу или этапу полета и непосредственно влияющих на безопасность полета, является карта контрольной проверки.
Контроль по карте контрольной проверки проводится только после того, как каждый член экипажа доложит о завершении подготовки в соответствии с листом контрольного осмотра.
Контроль по карте контрольной проверки является обязательным комплексом операций, проводимых экипажем под руководством и ответственностью командира ВС на предписанных рубежах при выполнении полетов любого назначения. Процедура контроля по карте контрольной проверки должна обязательно обеспечиваться перекрестным контролем, когда выполняющий очередную операцию член экипажа должен получить подтверждение правильности выполнения от другого члена экипажа.
Чтение вслух соответствующего раздела карты производится членом экипажа в соответствии с Инструкцией по взаимодействию и технологией работы экипажа, который после поступления последнего доклада по последнему пункту зачитываемого раздела карты докладывает командиру ВС о завершении контроля по соответствующему разделу карты.
Пункты контрольной карты зачитываются раздельно, громким голосом, исключая случаи, когда инструкции для данного типа ВС предусматривают выполнение определенной части контрольного перечня молча. 
Последующий пункт не должен зачитываться, пока не будет должным образом проверен предшествующий ему пункт. Подлежит соблюдению точная терминология контрольного перечня, используемого в кабине экипажа.
8.8.2.6. На пассажирских воздушных судах закрытие дверей и грузовых люков после посадки пассажиров осуществляет бортмеханик, а их открытие после заруливания на стоянку осуществляют члены экипажа в соответствии с требованиями РЛЭ ВС (AFM, FCOM) и Инструкции по взаимодействию и технологии работы экипажа.
8.8.2.7. На грузовых воздушных судах закрытие дверей осуществляется бортмехаником, закрытие грузового люка, а также эксплуатацию специального погрузочно-разгрузочного оборудования осуществляет бортоператор, а открытие дверей и установку трапа после заруливания на стоянку осуществляют члены экипажа в соответствии с требованиями РЛЭ ВС и Инструкции по взаимодействию и технологии работы экипажа. Грузовой люк (рампу) открывает бортоператор.
8.8.2.8. Время и очередность приема пищи членами летного экипажа в полете определяет командир ВС. Не допускается одновременный прием пищи двумя пилотами.
8.8.2.9. Использование посадочных фар ВС определяется командиром ВС в зависимости от условий полета, орнитологической обстановки, учебно-тренировочного задания на полет и в соответствии с РЛЭ ВС (AFM, FCOM).
8.8.2.10. Распределение обязанностей членов летного экипажа во время взлета, набора высоты, горизонтального полета, снижения, захода на посадку и посадки указаны в РЛЭ ВС (AFM, FCOM) и в Инструкции по взаимодействию и технологии работы экипажа самолета данного типа. 
8.8.2.11. Командир воздушного судна является старшим на борту ВС, осуществляет общее руководство работой экипажа, обеспечивает выполнение полетных процедур в соответствии с Инструкцией по взаимодействию и технологией работы экипажа, РЛЭ ВС (AFM, FCOM) и РПП авиакомпании (доп.см.п.4.3, А-4 РПП). Экипаж должен в течение всего полета иметь достаточно ясное представление об окружающей его воздушной обстановке, имея в виду не только те ВС, которые находятся в пределах видимости, но и те, которые могут находиться в данном районе или месте пересечения воздушных трасс.
8.8.2.12. Основными обязанностями пилотирующего пилота являются пилотирование, контроль за управлением и осуществление навигации. Он должен контролировать полет, работу систем ВС и двигателей и быть в любой момент готовым к переходу на ручной режим управления. 
Непилотирующий пилот ведет инструментальный и визуальный контроль за выполнением полета. В случае потери работоспособности пилотирующего пилота, непилотирующий должен взять управление ВС на себя.
8.8.2.13. В зависимости от конкретных условий полета пилоты должны, как правило, чередовать функции пилотирующего и непилотирующего пилота в равной пропорции.
8.8.2.14. Для регулирования рабочей нагрузки на экипаж ВС при заходе на посадку в приборных метеоусловиях ночью рекомендуется использовать распределение обязанностей между пилотами, при котором: 
а)	максимально используются автоматизированные режимы работы систем ВС;
б)	при невозможности использования в полном объеме автоматизированных режимов работы систем ВС обязанности пилотирующего пилота выполняет командир ВС;
в)	при заходе на посадку в условиях ниже минимума САТ I обязанности пилотирующего пилота выполняет командир ВС, второй пилот выполняет обязанности непилотирующего, если иное не предусмотрено SOP.
Ответственность за равномерное распределение рабочей нагрузки между членами летного экипажа несет командир ВС.
8.8.3. Смена члена летного экипажа в полете
Общие положения
8.8.3.1. В том случае, когда плановая продолжительность полётной смены превышает максимально допустимую, установленную Главой А -7 РПП, для выполнения функций члена экипажа в полёте на время регламентированного технологического перерыва вводятся дополнительные члены экипажа (далее – увеличенный состав экипажа).
8.8.3.2. Продолжительность полетной смены увеличенного состава летного экипажа устанавливается в зависимости от числа дополнительных членов летного экипажа и количества посадок, запланированных заданием на полет согласно Таблицы А7.7 – Т1 Главы А7 РПП.
8.8.3.3. В увеличенном экипаже командиром является пилот, имеющий квалификацию КВС (инструктор), на которого оформлено задание на полет. Он занимает левое пилотское сиденье, несет ответственность за подготовку и выполнение полёта.
8.8.3.4. Замена члена лётного экипажа в полёте производится на высоте полёта не менее 3000м (10000ф), за исключением случая замены члена экипажа, потерявшего работоспособность.
8.8.3.5. Порядок смены членов экипажа и смены экипажей в длительном полёте
В длительном полёте смена членов экипажа, в том числе КВС, производится по указанию КВС только в прямолинейном горизонтальном полёте и при условиях, что:
а)	смена предусмотрена заданием на полёт;
б)	заменяющий член экипажа участвовал в предполётной подготовке;
в)	оставшееся время до пролёта очередного пункта ОД или пересечения трассы – не менее 5 минут;
г)	оставшееся время до начала снижения – не менее 10 минут.
8.8.3.6. Перед заменой в полёте член экипажа знакомится с навигационными особенностями полёта, работой оборудования на рабочем месте по информации заменяемого, и после разрешения командира ВС занимает рабочее место и докладывает: «Рабочее место занял, параметры полёта контролирую».
8.8.3.7. После доклада, ответственность за дальнейшее безопасное выполнение этого полёта ложится на члена экипажа, занявшего рабочее место.
8.8.3.8. В процессе смены остальные члены экипажа должны находиться на своих рабочих местах. Смена членов экипажа в полёте производится поочередно с интервалом не менее 5 мин. с соблюдением указанных условий.
8.8.3.9. В двойном экипаже смена полного состава производится, как указано, в пунктах 8.8.3.6, 8.8.3.7, 8.8.3.8.
           8.8.3.10. Смена командира ВС или второго пилота на крейсерском этапе полета
В состав усиленного экипажа для смены пилотов может быть включён дополнительный пилот, имеющий квалификацию командира ВС, допущенный к полётам с правого пилотского сиденья, или квалификацию «инструктор» на данном типе ВС.
8.8.3.11. Дополнительный пилот, указанный в п. 8.8.3.10, в полёте может занимать левое или правое места пилотов, по указанию КВС. При этом он несет ответственность за выполнение функций соответственно командира ВС или второго пилота, согласно РЛЭ на период подмены основного члена экипажа.
8.8.3.12. Порядок смены пилотов в длительном полёте
В длительном полёте смена командира ВС или второго пилота производится по указанию КВС при условиях и в порядке, предусмотренном подразделом 8.8.3. настоящего раздела РПП. В процессе смены пилота управление ВС должно осуществляться от автопилота.
8.8.3.13. В процессе смены пилота, другой пилот сохраняет полный доступ к органам управления ВС, контролирует режим полёта (находится в активной позе в готовности к управлению в штурвальном режиме).
8.8.4. Распределение обязанностей членов кабинного экипажа
8.8.4.1. Распределение обязанностей членов кабинного экипажа дает необходимое понимание выполнения процедур для исполнения обязанностей в штатных и аварийных ситуациях и устанавливается в зависимости от назначенного по заданию на полет порядкового номера в экипаже.
8.8.4.2. Распределение обязанностей бортпроводников (бортоператоров) в зависимости от номера по полетному заданию по выполнению процедур при подготовке и выполнении полета, а также при действиях в аварийных ситуациях описаны в Технологии работы бортпроводников (бортоператоров). 
8.8.4.3. Распределение обязанностей членов кабинного экипажа при аварийных ситуациях основаны на положениях РЛЭ конкретного типа ВС «Аварийное расписание» и РПП авиакомпании глава В -11по типу ВС.
8.8.4.4. В случае потери работоспособности одного из членов кабинного экипажа КВС производит перераспределение обязанностей по обслуживанию пассажиров и на случай возникновения угрозы безопасности полета в аварийных ситуациях.
Перераспределение обязанностей производится с учетом квалификации конкретных членов экипажа.
8.8.5. Взаимодействие летного и кабинного экипажей
8.8.5.1. Порядок взаимодействия летного и кабинного экипажа в нормальном полете
На период выполнения рейса бортпроводник (бортоператор) подчиняется КВС и несет ответственность за подготовку и работу в рейсе, обеспечение безопасности пассажиров и пассажирской кабины.
Перед вылетом, в период предполетной подготовки командир ВС:
а)	принимает доклад СБ о составе кабинного экипажа и его готовности к рейсу;
б)	представляется и информирует о составе летного экипажа;
в)	сообщает расчетное время полета и коммерческую загрузку;
г)	информирует о наличии опасных грузов на борту ВС и определяет порядок действий кабинного экипажа при возникновении аварийной ситуации;
д)	согласовывает порядок доведения информации пассажирам из кабины экипажа;
е)	определяет условные сигналы связи бортпроводника (бортоператора) с кабиной летного экипажа;
ж)	проводит брифинг с кабинным экипажем, информируя об особенностях предстоящего полета: предполагаемых зонах турбулентности, дополнительных мерах по обеспечению авиационной безопасности, особенностях полета над большими водными пространствами, в полярных широтах, горными массивами и т.д.
Примечание. Брифинг с кабинным экипажем может проводиться до прибытия на ВС или на борту ВС. 
Бортпроводник (далее БП) обязан, по прибытию командира ВС на самолет: 
а)	доложить:
	о ходе подготовки к полету;
	об отклонениях от технологического графика;
	обо всех обнаруженных недостатках в ходе подготовки ВС;
	о наличии и исправности БАСО;
	о готовности кабинного экипажа и пассажирского салона к посадке пассажиров.
б)	согласовать с командиром ВС:
	размещение в салонах ВС больных, инвалидов, несопровождаемых детей, депортированных пассажиров и сотрудников безопасности;
	порядок загрузки багажных помещений и размещение опасных грузов (при необходимости);
	порядок входа в кабину летного экипажа, условные сигналы;
	время и очередность приема пищи членами летного экипажа.
После окончания посадки пассажиров БП докладывает командиру ВС:
	наличие перевозочных документов;
	точное количество пассажиров на борту;
	коммерческую загрузку и ее размещение в багажных отсеках;
	наличие служебной корреспонденции, опасных грузов, оружия и/или боеприпасов;
	наличие депортированных, больных пассажиров, инвалидов и сопровождающих их лиц;
	наличие несопровождаемых детей.
БП должен получить от КВС разрешение на закрытие дверей (при выполнении этих функций).
Перед взлетом, до занятия исполнительного старта и перед посадкой на установленном рубеже, командир ВС должен:
а)	установленным сигналом проинформировать кабинный экипаж о необходимости занять свои места и пристегнуться привязными ремнями;
б)	получить доклад БП о готовности пассажирской кабины к взлету/посадке.
Проводить переговоры между летным и кабинным экипажем на высоте ниже 3000м. (10000ф.) запрещается, кроме случаев возникновении на борту нестандартных ситуаций, влияющих на безопасность полета.
В полете БП немедленно докладывает КВС о возникновении на борту любых нестандартных ситуаций, влияющих на безопасность полета, а также об использовании аварийно-спасательного оборудования в полете.
При возникновении признаков отказа или неисправности электрооборудования пассажирского салона ВС, бортпроводник обязан отключить электрооборудование с немедленным докладом командиру ВС.
При подходе к зоне турбулентности командир ВС оповещает кабинный экипаж включением табло «ЗАСТЕГНУТЬ РЕМНИ» (FASTEN SEAT BELTS), при подходе кзоне умеренной или сильной турбулентности, дополнительно дает информацию по внутрисамолетной связи: «Бортпроводникам занять свои места».
По этому сигналу бортпроводники должны:
а)	прекратить обслуживание пассажиров;
б)	занять свои (или ближайшие свободные) места;
в)	застегнуть привязные ремни.
После заруливания на стоянку и выключения двигателей, командир ВС дает команду на открытие дверей самолета отключением табло «ЗАСТЕГНУТЬ РЕМНИ» («FASTEN SEAT BELTS»).
При любых задержках или отклонениях от плана полета (вынужденная посадка, полет в зоне ожидания, уход на запасной аэродром и т.д.) командир ВС информирует БП и пассажиров о причинах, времени задержки и дальнейших действиях.
8.8.6. Распределение обязанностей в аварийных ситуациях
8.8.6.1. Пилотирующий пилот должен нести ответственность, главным образом, за управление ВС и осуществление контроля за выполнением полета в то время, когда другие члены экипажа выполняют свои обязанности, связанные с локализацией аварийной ситуации. 
Когда второй пилот пилотирует ВС, он выполняет все функции, обозначенные ПАУ (пилот активно управляющий).
Передача управления воздушным судном должна производиться установленным, исключающим ошибки, образом в соответствии со стандартными эксплуатационными процедурами. 
8.8.6.2. Командир ВС, исходя из ситуации, сохраняет решающее право на перераспределение функций в экипаже на любом этапе полета. Такое перераспределение может носить временный характер или на весь период полета.
В зависимости от обстоятельств командир ВС может назначить дополнительные обязанности всем членам летного и кабинного экипажа. 
8.8.6.3. Действия члена экипажа вне зоны его ответственности разрешаются только по команде командира ВС.
8.8.6.4. Основным средством организации контроля за выполнением наиболее ответственных операций, необходимых для выполнения экипажем в нормальных условиях эксплуатации для обеспечения максимальной безопасности полета, готовности ВС и экипажа к очередному рубежу или этапу полета является Карта контрольных проверок.
Пункты Карты контрольных проверок зачитываются раздельно, громким голосом.
Последующий пункт не должен зачитываться, пока не будет должным образом проверен предшествующий ему пункт. 
Процедура контроля по карте контрольной проверки должна обязательно обеспечиваться перекрестным контролем, когда выполняющий очередную операцию член экипажа должен получить подтверждение правильности выполнения от другого члена экипажа.
В случае отсутствия инструкций в контрольных листах и РЛЭ типа ВС (AFM, FCOM), окончательное решение о выполнении процедур по управлению ВС принимает командир ВС.
В любом случае окончательное решение по действиям в аварийной ситуации принимает командир ВС.
8.8.6.5. Экипаж обязан осуществлять перекрестные проверки «Сrosscheck» (при помощи ответов) перед приведением в действие критических средств управления ВС, включая:
а)	снижение уровня тяги у отказавшего двигателя с помощью рычага управления двигателем;
б)	изменения конфигурации;
в)	установка курса, высоты, высотомера и воздушной скорости;
г)	передача управления ВС;
д)	изменения в AFS/FMS и средствах радио навигации во время вылета или фазы захода на посадку;
е)	расчеты рабочих характеристик, включая данные AFS/FMS;
ж)	включение переключателей управляющими перекрывными топливными кранами и переключателями, управляющими Расходом топлива;
з)	включение пожарной системы и огнетушителей;
и)	выключение генераторов.
8.8.6.6. На воздушных судах, оборудованных самолетным переговорным устройством, переговоры в кабине летного экипажа ведутся с его использованием, за исключением, когда ВС оборудовано микрофонами постоянной записи. Устройства записи переговоров в пилотской кабине и системы регистрации полетных данных не должны намеренно выключаться, за исключением случаев, когда это необходимо в целях сохранения данных, связанных с авиационными происшествиями или инцидентами.
8.8.7. Стандартные выражения и терминология
Для исключения неправильного понимания в кабине экипажа или при ведении связи с наземным персоналом, используются следующие правила:
а)	при выполнении стандартных процедур, таких как выполнение карты контрольных проверок, должен использоваться русский или английский язык в соответствии с требованиями РЛЭ (AFM, FCOM) воздушного судна;
б)	должна использоваться стандартная терминология и сигналы;
в)	при возникновении трудностей во взаимодействии между летным и наземным персоналом, используется любой удобный для понимания языков.
Для стандартизации процедур «Сrosscheck» - перекрестных проверок, повышения уровня координации и осведомленности членов экипажа при выполнении операционных процедур в полете в Авиакомпании используются стандартные отклики (ответы) - «Standard callouts», которые употребляются для того, чтобы:
а)	подать команду, поставить задачу;
б)	подтвердить получение команды, задачи;
в)	подтвердить выполнение операций по карте контрольной проверки (Check list);
г)	оповещать об изменениях показаний приборов;
д)	информировать о приближении параметров полета к предельно - допустимым и выходе их за пределы;
е)	идентифицировать конкретные события.
Порядок применения стандартных ответов (откликов) (Standard callouts) при выполнении полетов на конкретном типе ВС описан в Инструкциях и технологиях работы экипажей (SOP) типов ВС (РПП, Часть В, типа ВС, гл.2).
8.8.8. Ведение радиосвязи
8.8.8.1. Радиообмен с диспетчером ОВД ведет командир ВС или, по его поручению, другой член летного экипажа, в соответствии с установленными правилами и фразеологией. Радиообмен должен внимательно прослушиваться, как минимум двумя членами экипажа в течение всего полета. По возможности производить записи полученных диспетчерских разрешений.
Отступление от установленных правил допускается при ситуации, угрожающей безопасности полета. Ведущий радиосвязь должен убедиться в правильности принятой информации, сверив ее восприятие другими членами экипажа. При выявлении разного понимания информации запросить повторения информации.
Перед началом передачи прослушать наличие радиообмена на подлежащей использованию частоте, убедиться в отсутствии сигнала занятости наземного канала связи, чтобы исключить возможность возникновения помех уже ведущейся передаче.
8.8.8.2. Экипаж воздушного судна обязан немедленно сообщить органу ОВД о наблюдаемых опасных метеорологических явлениях, опасных сближениях с воздушными судами и другими материальными объектами и других опасных для полета обстоятельствах.
Летный экипаж докладывает диспетчеру ОВД:
а)	об условиях руления или полета, препятствующих выполнению Правил выполнения полетов и требований руководства по летной эксплуатации ВС;
б)	о выполняемых вертикальных маневрах по рекомендациям БСПС, а также о восстановлении условий, заданных в указании или диспетчерском разрешении ОВД, после разрешения конфликтной ситуации;
в)	о пролете пунктов обязательного донесения доклад может не производиться при получении экипажем информации от органа ОВД о радиолокационном опознавании ВС;
г)	о времени начала снижения с эшелона крейсерского полета;
д)	об условиях полета по запросу диспетчера ОВД;
е)	о входе в аэродромный круг движения при полетах по правилам полетов по приборам, если до этого не было выдано разрешение на посадку;
ж)	о выполнении иных указаний, установленных органом ОВД.
Особое внимание при ведении радиосвязи обращается при получении инструкций на предварительном старте, полетах в горной местности, при изменении курса, изменении радиочастот, изменении эшелона или высоты полета.
Экипаж ВС при получении диспетчерских указаний обязан повторить:
а)	сообщения, отличающиеся от типовых или требующие изменения принятого решения (или задания на полет);
б)	разрешения или запрещения на пересечение ВПП, взлёта, занятие исполнительного старта, захода на посадку, посадки, изменения эшелона высоты) полета и т.д.;
в)	значение принятого и установленного на высотомере давления;
г)	значение контрольной высоты;
д)	значение заданного времени;
е)	заданный эшелон (высоту) полёта;
ж)	заданный курс полета;
з)	значение МПУ ВПП, номера ВПП;
и)	заданную скорость полета или число «М»;
к)	значение заданной частоты (номера) канала связи.
8.8.9. Политика применения автоматических систем управления ВС
8.8.9.1. Автоматические системы управления ВС применяются в целях: 
а) повышения безопасности полета; 
б) уменьшения рабочей нагрузки на экипаж; 
в) увеличения операционных возможностей членов летного экипажа; 
г) увеличения ситуационной осведомленности; 
д) улучшения условий для принятия решений; 
ж) оказания помощи экипажу в действиях по управлению рисками. 
8.8.9.2. Пилотирование ВС и эксплуатация автопилота осуществляется в строгом соответствии с РЛЭ (AFM, FCOM).
Как правило, автоматическое управление полетом следует применять на максимально возможном уровне. 
Пилоты (члены летного экипажа) должны быть подготовлены для использования всех уровней автоматизации, определения условий понижения уровня автоматизации полета и иметь навыки для перехода с одного уровня на другой. 
8.8.9.3. Вход в зону RVSM и полёт в ней выполняются только с включённым автопилотом. При отказе автопилота или появлении неисправности, не позволяющей выдерживать высоту с установленной точностью, экипаж сообщает органу ОВД, который вправе потребовать выхода ВС из зоны RVSM.	
Вне зоны RVSM автопилот может выключаться для балансировки ВС или тренировки пилота в ручном пилотировании ВС, в том числе на этапе захода на посадку.
8.8.9.4. В условиях слабого или умеренного сдвига ветра или турбулентности на всех этапах полёта рекомендуется выполнять полёт в режиме автоматического управления с включённым автопилотом. При наличии информации о возможном сдвиге ветра или турбулентности (болтанке) на этапах взлёта или захода на посадку рекомендуется увеличить скорость полёта в соответствии с требованиями РЛЭ (AFM, FCOM).
8.8.9.5. Как правило, автоматическое управление полетом следует применять на максимально возможном уровне. Выбранный уровень автоматизации для конкретных условий полета должен обеспечивать оптимальное распределение рабочей нагрузки между членами летного экипажа (пилотами), постоянный контроль за профилем полета и положением ВС в пространстве. 
8.8.9.6. В любой нештатной ситуации, требующей отклонения от стандартных эксплуатационных процедур, автопилот рекомендуется включать для снижения операционной нагрузки при выполнении процедур по локализации отказов и неисправностей. 
8.8.9.7. Основные принципы использования автоматических систем управления ВС: 
а)	пилотирующий пилот должен постоянно контролировать соответствие работы автоматической системы с траекторией движения ВС; 
б)	если системы автопилота не работают, как ожидается, изменить уровень автоматизации или отключить эту функцию; 
в)	члены экипажа должны быть информированы о любых изменениях установок в автоматической системе. Если пилотирующий пилот самостоятельно определил изменение режима, он обязан оповестить об этом экипаж; 
г)	оперативные действия по управлению профилем полета и режимами работы автоматической системы осуществляет пилотирующий пилот с немедленным докладом экипажу; 
д)	после включения какого-либо автоматического режима члены экипажа должны убедится, что выбранный режим включился и его индикация соответствует заданной; 
е)	все команды и информация должны быть немедленно подтверждены членом экипажа, к которому они адресованы; 
ж)	любой выбранный уровень автоматизации полета не может исключать ведения визуальной осмотрительности; 
з)	автоматические системы управления ВС рекомендуется применять в районе аэродромов с высокой интенсивностью, предполетный и предпосадочный брифинг должен включать все особенности применения автоматических систем, распределение обязанностей и ответственности членов экипажа ВС; 
и)	пилоты должны контролировать минимальную высоту включения/отключения автоматических режимов управления в соответствии с РЛЭ типа ВС; 
к)	автопилот и директорное управление полетом рекомендуется использовать для захода на посадку при видимости на ВПП менее 1200 м (4000 ft); 
8.8.9.8. При выполнении полёта в автоматическом режиме на ВС ИП члены экипажа осуществляют перекрёстную проверку («Сrosscheck») состояния навигационных систем и изменяемых навигационных элементов полёта. В процедуре перекрестного контроля должны участвовать два члена экипажа. Один вводит данные в навигационную систему, другой контролирует вводимые данные в соответствии с CFP, планом полета, таблицами установочных данных, полетной картой, схемой и т.п. и соответствующий отклик ВС на ввод команд и данных:
а)	при включении автопилота от бортовой навигационной системы (FMS) проверяется соответствие курса следования заданному путевому углу с учётом угла сноса и сохранения заданных высоты и скорости полёта (сохранения поступательной и вертикальной скорости при наборе высоты или снижении);
б)	при пролёте ПОД проверяется точность разворота на расчётный курс следования и соответствие отображаемых на индикаторе данных навигационному расчёту полёта;
в)	на этапах схемы прибытия (STAR) и выхода (SID) оба пилота непрерывно контролируют соответствие навигационных параметров полёта установленным схемам и своевременность занятия высот (эшелонов) к заданным рубежам;
г)	при заходе на посадку один пилот контролирует параметры полёта, находится в готовности отключить автопилот и перейти к ручному режиму пилотирования при появлении значительных (более, чем на оценку удовлетворительно) отклонений от заданного режима полёта, а другой пилот контролирует соответствие навигационных параметров полёта установленной схеме.
Перекрестный контроль заключается в коллективном контроле местоположения самолета с использованием основных и дополнительных средств навигации. При этом каждый член экипажа (пилот) обязан своевременно информировать о замеченных отклонениях от заданных траекторий полета, удалениях до очередных пунктов маршрута, расчетном времени пролета.

8.9. Выполнение полета
8.9.1. Буксировка воздушного судна 
8.9.1.1. Буксировка на место запуска производится: 
а)	по разрешению органа ОВД и руководителя буксировкой; 
б)	в соответствии с установленной на данном аэродроме схемой наземного движения; 
в)	при наличии непрерывной двухсторонней связи между руководящим буксировкой лицом и экипажем воздушного судна по переговорному устройству, по радио или визуально с помощью установленных сигналов воздушного судна и с органом ОВД. 
8.9.1.2. Пересечение и занятие ВПП или РД при буксировке ВС производится по разрешению органа ОВД. 
8.9.1.3. При пересечении, занятии ВПП или РД летный экипаж и/или лица, осуществляющие буксировку: 
а)	соблюдают визуальную и радиоосмотрительность; 
б)	докладывают органу ОВД об освобождении ВПП или РД. 
Безопасность буксировки обеспечивается лицом, руководящим буксировкой. 
8.9.1.4. Буксировка воздушного судна выполняется с включенными аэронавигационными огнями и проблесковыми маяками (если предусмотрено РЛЭ (AFM, FCOM) типа ВС). 
Примечание: Положение ног пилотов на органах управления в процессе руления должно исключить возможность непреднамеренного торможения ВС.
8.9.2. Запуск двигателя (двигателей) воздушных судов
8.9.2.1. Запуск двигателей, прогрев на повышенных оборотах, опробование двигателей ВС выполняется по разрешению органа ОВД с докладом получения последней информации ATIS и производится под контролем технического персонала: 
а)	на стоянке (в том числе у аэровокзала); 
б)	на участках РД, отведенных для этой цели; 
в)	на специально оборудованной площадке, определенной в установленном порядке для данного аэродрома; 
г)	в процессе буксировки, если это не противоречит требованиям инструкции по эксплуатации ВС (РЛЭ) и аэронавигационного паспорта или инструкции по буксировке воздушных судов аэродрома. 
8.9.2.2. Перед началом запуска двигателя (двигателей) на воздушном судне: 
а)	удостовериться в безопасности людей и отсутствии посторонних предметов, которые могут быть повреждены или представлять опасность при запуске. При невозможности лично убедиться в безопасности запуска, получить необходимую информацию от лица, руководящего с земли запуском двигателей по переговорному устройству, по радио или визуально с помощью установленных сигналов;
б)	технологические процедуры согласно РЛЭ должны быть выполнены с обязательной проверкой по карте контрольных проверок; 
в)	включить проблесковые маяки (если предусмотрено РЛЭ типа ВС).
8.9.2.3. При запуске двигателей воздушного судна летным экипажем ВС поддерживается двухсторонняя связь (проводная, радио, визуальные сигналы) с лицом наземного персонала, обеспечивающим выпуск воздушного судна.
8.9.2.4. Запрос члена летного экипажа на запуск двигателя на контролируемом аэродроме или запуск двигателя на неконтролируемом аэродроме с целью производства полета свидетельствует о принятии решения КВС о начале полета.
8.9.3. Руление
8.9.3.1. Перед началом руления экипаж должен ознакомиться со схемой руления на аэродроме, маркировкой маршрута руления, должны быть выполнены все технологические процедуры согласно РЛЭ ВС (AFM, FCOM) с проверкой по карте контрольных проверок. 
8.9.3.2. На контролируемом аэродроме руление выполняется пилотом после получения от органа ОВД соответствующего разрешения на руление и информации о схеме руления по аэродрому. Орган ОВД, управляющий движением воздушного судна по аэродрому, информирует экипажи воздушных судов о взаимном расположении воздушных судов, в том числе и следующих по одному маршруту при рулении в условиях видимости менее 400 м. Пилоту органом ОВД может передаваться другая информация, необходимая для обеспечения безопасности руления или буксировки.
8.9.3.3. Осуществляя маневрирование на земле перед взлетом ВС и после посадки во избежание столкновения с другими ВС или препятствиями на земле, КВС и члены экипажа должны использовать все имеющиеся ресурсы для понимания в любой момент времени места нахождения ВС на рабочей площади аэродрома. 
Такими ресурсами могут быть: 
а)	различные аэродромные знаки, курсовые указатели, указатели РД и ВПП; 
б)	маркировки поверхности аэродрома; 
в)	светосигнальное оборудование маршрутов руления; 
г)	освещение отдельных зон территории аэродрома. 
При подготовке к полету экипаж должен в достаточной степени изучить схемы руления на аэродромах вылета, назначения, запасных, предварительно наметить возможные варианты маршрута руления.
При получении разрешения (указания) по порядку руления, если маршрут руления достаточно сложен, произвести запись основных пунктов информации во избежание ошибок при рулении. 
В процессе руления отслеживать прохождение маршрута руления, контролировать его соответствие полученному указанию от диспетчера руления. При возникновении каких-либо затруднений при рулении запрашивать уточняющую информацию у диспетчера. 
Руление ВС осуществлять с включенными проблесковыми маяками, рулежным светом фар, ночью с включенными аэронавигационными огнями (АНО). 
Для обеспечения руления в условиях пониженной видимости, особенно ночью, ориентироваться по огням светооборудования РД, ВПП, осевым огням РД, ВПП. При затруднениях в правильности выполнения маршрута руления запросить машину сопровождения. 
Пересечения ВПП, занятие ВПП для взлета производить только по разрешению диспетчера. При занятии ВПП убедиться визуально, прослушиванием радиообмена, путем анализа информации БСПС (TCAS) в отсутствии какого-либо ВС на предпосадочной прямой. 
Взлет производить с включенными на максимальный режим взлетно-посадочными фарами, проблесковыми маяками, ночью – с включенными АНО. 
После посадки и освобождения ВПП запросить указания по рулению. 
На неконтролируемых аэродромах и площадках перед началом руления воздушного судна КВС осуществляет осмотр летного поля и выбирает маршрут буксировки, руления. 
8.9.3.4. Выруливание с места запуска двигателя (двигателей) выполняется с разрешения: 
а)	диспетчера ОВД; 
б)	ответственного лица, обеспечивающего выпуск воздушного судна, а при его отсутствии по решению командира ВС.
В начале руления экипаж воздушного судна проверяет работоспособность тормозной системы. 
8.9.3.5. Экипажу воздушного судна запрещается начинать и продолжать руление, если: 
а)	давление в тормозных системах не соответствует эксплуатационным ограничениям или имеются другие признаки неисправности тормозов; 
б)	на контролируемом аэродроме не получено разрешение органа ОВД или органа управления движением на перроне; 
в)	безопасность руления не обеспечивается из-за наличия препятствий, неудовлетворительного состояния места стоянки или рулежных дорожек. 
г)	отсутствует автомашина сопровождения, если сопровождение при рулении обязательно или выполняется по запросу экипажа. 
8.9.3.6. Руление производится на минимально возможной тяге двигателей для уменьшения воздействия шума и реактивных струй двигателей. 
Руление выполняется по маршруту, указанному диспетчером ОВД. В процессе руления наличие непрерывной двухсторонней связи с органом ОВД обязательно. О невозможности выполнить заданный маневр командир ВС должен доложить органу ОВД. 
Пересечение критических зон посадочных маяков либо ВПП производится по отдельному разрешению органа ОВД. После пересечения (освобождения) ВПП по маршруту руления командир ВС обязан сообщить об этом органу ОВД.
Независимо от полученного указания органа ОВД, перед пересечением, занятием ВПП или рулежной дорожки летный экипаж воздушного судна и (или) лица, осуществляющие буксировку воздушного судна, обязаны убедиться в безопасности маневра.
Если на аэродроме предусмотрена система управления рулением ВС с применением огней на РД, экипаж обязан строго выдерживать заданный огнями маршрут и быть готовым прекратить руление при загорании сигналов остановки. 
8.9.3.7. Скорость руления
Выбирается командиром ВС в зависимости от: 
а)	ограничений РЛЭ воздушного судна; 
б)	состояния перрона, РД и ВПП, по которым проходит маршрут руления; 
в)	наличия препятствий по маршруту руления; 
г)	видимости и степени освещенности на маршруте руления; 
д)	других условий, по усмотрению КВС, определяющих безопасность руления. 
Не рекомендуется руление на скорости более 40 км/ч (20 КТ) на прямых участках и 10 км/час (5 КТ) на разворотах, на скользких поверхностях – более 5-10 км/час (3-6 КТ). 
При наличии в аэропорту системы позиционирования на места стоянок (SAFEDOCK) скорость заруливания на стоянку ограничивается согласно документам АНИ. 
В нормальных условиях не рекомендуется применять раздельное торможение и несимметричную тягу двигателей на разворотах. 
Скорость руления в любом случае должна обеспечивать безопасную остановку ВС. 
Командир ВС несет ответственность за обоснованность выбора скорости руления. 
При рулении необходимо следить, чтобы воздушные суда и обслуживающий персонал на перроне не подвергались опасности воздействия реактивной струи или воздушного потока от воздушного винта. 

8.9.3.8.	Предотвращение столкновений
Пилоты при рулении должны вести постоянную визуальную и радио осмотрительность. Остальные члены летного экипажа при рулении также должны вести осмотрительность, если это не затрудняет выполнение их основных обязанностей, и предупреждать КВС о препятствиях по маршруту руления.
При рулении воздушных судов навстречу друг другу их КВС обязаны уменьшить скорость руления до безопасной и, держась правой стороны, разойтись левыми бортами. 
При сближении воздушного судна на пересекающихся направлениях КВС обязан пропустить воздушное судно, двигающееся справа. 
Запрещено обгонять рулящее воздушное судно. 
При рулении под контролем органа ОВД порядок взаимного расхождения воздушных судов на пересекающихся маршрутах определяет орган ОВД. 
Командир ВС несет ответственность за соблюдение правил руления. 
При обнаружении препятствия на маршруте руления командир ВС: 
а)	принимает меры по предотвращению столкновения, вплоть до полной остановки; 
б)	докладывает органу ОВД о наличии препятствия; 
в)	после устранения препятствия продолжает руление с разрешения органа ОВД. 
8.9.3.9.	Применение внешнего светового оборудования ВС 
Руление ночью, а также днем при видимости менее 2000 м осуществляется с включенными аэронавигационными огнями и фарами.
Проблесковые маяки днем и ночью должны быть включены с момента запуска двигателей до момента их останова. 
8.9.3.10.	Сопровождение ВС на рулении
Воздушные суда при рулении в обязательном порядке сопровождаются радиофицированной автомашиной сопровождения: 
а)	днем и ночью – при метеорологической видимости менее 400 м; 
б)	ночью – воздушные суда 1 и 2 класса независимо от метеоусловий; 
в)	если маркировочная разметка рулежных дорожек по маршруту руления или места стоянки хотя бы частично не просматривается из-за наличия снега, льда или по другим причинам; 
г)	по запросу экипажа независимо от времени суток, метеоусловий и класса воздушного судна; 
д)	в зарубежных аэропортах - согласно инструкции данного аэродрома. 
При лидировании воздушных судов безопасную дистанцию между автомашиной сопровождения и лидируемым воздушным судном поддерживает командир ВС. 
Ответственность за предотвращение столкновений ВС с препятствиями при лидировании ВС машиной сопровождения возлагается на командира ВС. 
Заруливание на место стоянки выполняется по:
а)	указаниям диспетчера ОВД;
б)	маркировочной разметке места стоянки;
в)	сигналам и указаниям дежурного по сопровождению;
г)	сигналам встречающего лица ИАС, при его отсутствии по решению командира ВС;
д)	сигналам системы позиционирования и остановки ВС.
При установке воздушного судна не по маркировочной разметке места стоянки экипаж воздушного судна:
а)	информирует об этом диспетчера ОВД;
б)	по согласованию с диспетчером ОВД повторяет маневр заруливания на стоянку;
в)	вызывает буксир для постановки воздушного судна на место стоянки.
8.9.3.11.	Перегрев тормозов
Во избежание перегрева тормозов после интенсивного использования при посадке, экипаж ВС может запросить диспетчера органа ОВД о необходимости сокращения маршрута руления, если есть такая возможность. Экипаж ВС учитывает возможный перегрев тормозных устройств, колес шасси в процессе руления после посадки, при необходимости прекращает руление, запрашивает облив тормозов водой согласно РЛЭ ВС. 
На стоянке, после установки упорных колодок, по согласованию с наземным персоналом, экипаж ВС выключает стояночный тормоз для ускоренного охлаждения тормозов.
8.9.3.12.	Выключение двигателей
Производится после заруливания на стоянку, с соблюдением технологических процедур согласно РЛЭ воздушного судна. 
Перед выключением двигателей устанавливается стояночный тормоз и проверяются параметры электросети от наземного источника или ВСУ. Момент выключения двигателей определяет командир ВС. 
В случае если от наземного персонала поступила информация о нештатной ситуации, командир ВС принимает меры к экстренному выключению двигателей. 

8.9.3.13.	Использование стояночного тормоза 
Командир ВС должен использовать во всех случаях, указанных в РЛЭ воздушного судна. Установка (снятие) стояночного тормоза согласовывается с наземным персоналом. 
Особое внимание уделяется установке стояночного тормоза перед запуском и выключением двигателей для предотвращения неконтролируемых перемещений ВС. Порядок использования стояночного тормоза при запуске двигателя в процессе буксировки координируется с наземным персоналом. 
Снятие со стояночного тормоза производится только после выключения двигателей и получения информации от наземного персонала об установке упорных колодок. 
При сильном ветре или неудовлетворительном состоянии поверхности стоянки (снег, лед, значительный уклон и т.п.) стояночный тормоз оставляется включенным, а при необходимости командир ВС требует установку дополнительных упорных колодок под все колеса. 
 8.9.3.16.	Световые сигналы, подаваемые с автомашины сопровождения
Рулите за мной – зеленый свет.
Прекратить руление – красный свет.
8.9.4.	Взлет
8.9.4.1. На контролируемых аэродромах взлет осуществляются по разрешению органа ОВД. Экипаж должен подтвердить разрешение на взлёт. В случае ожидания взлёта на ВПП, но неполучения разрешения в установленный период времени экипаж должен повторно запросить разрешение на взлёт.
8.9.4.2. На аэродроме и посадочных площадках, где нет органов ОВД, взлет выполняется по решению командира ВС. Место начала взлета и его направление определяет командир ВС. Информацию о времени, месте 
и направлении взлета командир ВС передает на частоте органа ОВД, в районе ответственности которого находится ВС. В случаях необходимости длительного занятия ВПП (более 1 мин.), экипаж ВС до ее занятия необходимом времени для подготовки к взлету. Если после выдачи разрешения на взлет прошло более 1 минуты, то экипаж воздушного судна обязан запросить повторное разрешение на взлет.
8.9.4.3. Перед взлетом: 
а)	летный экипаж воздушного судна проверяет установку высотомеров; 
б)	КВС убеждается в готовности воздушного судна и членов экипажа воздушного судна к взлету; 
в)	КВС убеждается в отсутствии наблюдаемых препятствий впереди на ВПП и по траектории взлета; 
г)	КВС убеждается в соответствии фактической погоды минимуму для взлета и фактической погоды, состояния ВПП ограничениям летно-технических характеристик ВС с учетом фактической погоды; 
д)	КВС убеждается в отсутствии по траектории полета зон опасных метеорологических явлений; 
е)	на контролируемом аэродроме КВС получает разрешение на взлет от органов ОВД. 
Запрещается выполнение взлета при наличии информации о сильном дожде, с интенсивностью, ухудшающей метеорологическую видимость менее 600 м., без использования бортового радиолокатора и системы заблаговременного предупреждения о сдвиге ветра. 
При наличии информации о видимости в трех частях ВПП видимость на ВПП (далее - RVR) оценивается командиром ВС в начале разбега, в средней точке и в конце ВПП - по сообщенной органом ОВД или АТИС). 
8.9.4.4. При выполнении взлета располагаемая длина разбега и взлетная дистанция от места начала разбега должны соответствовать потребной дистанции продолженного взлета и длине разбега для фактической взлетной массы воздушного судна и условий взлета. 
Взлет воздушных судов производится, как правило, от начала ВПП. 
Разрешается выполнять взлет не от начала ВПП при условии, если:
а)	это предусмотрено инструкцией по производству полетов на данном аэродроме (аэронавигационным паспортом аэродрома);
б)	располагаемые характеристики летной полосы от места начала разбега соответствуют потребным для фактической взлетной массы воздушного судна и условиям взлета. 
На неконтролируемых аэродромах место начала взлета и его направление определяет КВС. На неконтролируемых аэродромах перед взлетом КВС обязан передать на частоте органа ОВД, в районе ответственности которого он находится, место и магнитный курс взлета. 
Взлет с кратковременной остановкой на ВПП рекомендуется выполнять на мокрых, обледенелых, заснеженных и покрытых слякотью ВПП, а также в условиях интенсивности воздушного движения на аэродроме по согласованию с органом ОВД. 
Взлет без остановки на ВПП (немедленный взлет) выполняется при дальности видимости на ВПП (RVR) не менее 400м. При этом значения РВД, РДПВ и РДР для расчета взлетных характеристик должны быть уменьшены на 150 м. 
Если к моменту достижения контрольной скорости двигатели не вышли на режим взлетной тяги (не поступил доклад «Режим взлетный»), взлет должен быть прекращен в соответствии с РЛЭ ВС (AFM, FCOM). 
Если фактическая взлетная масса самолета ограничена по условиям располагаемой длины ВПП, выполняется нормальный взлет с выводом двигателей на взлетную тягу до страгивания ВС на исполнительном старте. В этом случае применение взлетного режима обязательно.
Первый разворот после взлета (отворот с курса взлета) выполняется на удалении и высоте, установленной схемой выхода из района аэродрома (SID).
Экипажу воздушного судна с момента начала разбега воздушного судна и до набора высоты 200 метров запрещено вести радиосвязь, а органу ОВД вызывать экипаж воздушного судна, за исключением случаев, когда это необходимо для обеспечения безопасности. 
8.9.4.5.  Экипажу взлетать запрещается, если:
а)	экипаж получил информацию, что взлет создаст помеху воздушному судну, которое выполняет прерванный заход на посадку (уход на второй круг); 
б)	впереди на ВПП (летной полосе) имеются препятствия; 
в)	по курсу взлета имеются опасные метеоявления, скопления птиц, угрожающие безопасности взлета; 
г)	взлетная масса ВС превышает максимальное значение, предусмотренное РЛЭ или эквивалентным ему документом (рассчитанное в соответствии с РЛЭ) для фактических располагаемых дистанций, метеорологических условий и требуемого градиента набора на начальном участке схемы вылета; 
д)	при полетах ночью отсутствует действующее светосигнальное оборудование; 
ж)  фактическая погода ниже установленного для взлета эксплуатационного минимума; 
и)  имеется информация о сильном дожде с интенсивностью, ухудшающей метеорологическую видимость до значения менее 600 м., без использования, при этом, бортового радиолокатора и системы заблаговременного предупреждения о сдвиге ветра; 
к)  скорость ветра у земли с учетом его направления и порывов, а также состояние поверхности ВПП и значение коэффициента сцепления не соответствует установленным РЛЭ данного типа ВС ограничениям; 
л) поверхность ВС покрыта льдом, инеем, мокрым снегом или истек срок действия противообледенительной обработки. 
8.9.4.6. Взлет выполняется КВС или, по указанию КВС, вторым пилотом.
Выполнение взлета вторым пилотом в целях тренировки разрешается:
а)	под контролем пилота – инструктора при соблюдении следующих условий:
	боковая составляющая ветра не более 80% от предельно-допустимой для фактического состояния поверхности ВПП;
	при видимости (видимости на ВПП) не менее 400 м;
б)	под контролем командира ВС:
	при боковой составляющей ветра не более 50% от предельно-допустимой для фактического состояния поверхности ВПП;
	при видимости (видимости на ВПП) не менее 400 м.
8.9.4.7. Взлет воздушного судна производится с включенными фарами, в зависимости от внешних условий и в соответствии с РЛЭ ВС по решению КВС, и до высоты не менее 50 м. 
8.19.4.8. Разрешается взлет при попутном ветре, если это предусмотрено инструкцией по производству полетов на данном аэродроме (аэронавигационным паспортом аэродрома) при величине попутной составляющей ветра, не превышающей требования руководства по летной эксплуатации воздушного судна данного типа.
8.9.4.9.	Взлет в условиях ограниченной видимости (LVTO)
Взлет в условиях ограниченной видимости (LVTO) – взлет при значениях RVR менее 400 м. (на аэродромах Российской Федерации при значениях RVR менее 550 м).
Взлет выполняется командиром ВС при соблюдении следующих условий:
а)	минимум для взлета 200м. (днем и ночью) применяется, если ВПП имеет огни осевой линии;
б)	без огней осевой линии с маркировкой осевой линии (RCLM), включены и работают огни высокой или средней интенсивности (HIRL, MIRL) –минимум для взлета составляет: днем – 300м., ночью – 400м. 
в)	при отсутствии огней ВПП (HIRL, MIRL) или их отказе в работе величина видимости на ВПП применяется не менее:
	для категории ВС А и В – 300 м днем и ночью;
	для категории ВС С – 500 м днем и 700 ночью. 
В этих условиях взлет должен обеспечиваться другой маркировкой или огнями ВПП, достаточными для визуальной ориентировки, беспрерывного наблюдения поверхности ВПП и выдерживания направления в процессе всего разбега ВС при взлете.
г)	Если процедуры при низкой видимости (LVР) не действуют, минимум для взлета всех категорий ВС устанавливается 400 м днем – при наличии маркировки осевой линии ВПП, ночью – при наличии маркировки осевой линии ВПП и работе огней ВПП. Если нет маркировки центральной линии ВПП и не работают огни ВПП, минимум устанавливается не менее 400 м для категории ВС А и В, 700 м для категорий С и только днем.
д)	Минимум для взлета 200 м применяется при коэффициенте сцепления на ВПП не менее 0,5 и боковой составляющей скорости ветра не более половины предельного допустимого значения для взлета данного типа.
е)	Решение на вылет по минимумам, указанным в графах 2,3 и 4, таблицы СП1-Т1 Приложения СП1 принимается по наименьшему значению видимости на ВПП, измеренной вдоль ВПП дистанционными измерителями (регистраторами) дальности видимости. При неработающих дистанционных измерителях или их отсутствии на аэродроме следует руководствоваться значениями видимости, указанными в графах 5 и 6 Таблицы СП1-Т1.
ж)	В тех случаях, когда днем не обеспечено обозначение осевой линии маркировочными знаками по всей длине ВПП, руководствоваться значениями, указанными в графах 4 и 6 Таблицы СП1-Т1 Приложения СП1.
з)	Минимумы для взлета применяются при наличии запасного аэродрома для взлета. Запасным аэродромом для взлета может быть выбран аэродром, на котором фактическая или прогнозируемая погода соответствует пункту 8.4.2.3, а время полета от аэродрома вылета до запасного аэродрома определяется в соответствии с рекомендациями РЛЭ, но во всех случаях не превышает:
	одного часа полета с одним отказавшим (критическим) двигателем для двухдвигательных самолетов;
Решение на вылет без запасного аэродрома для взлета может быть принято при метеоусловиях на аэродроме вылета выше минимума для посадки на нем, при условии, что нет других причин, препятствующих возврату на аэродром вылета.
и)	Заход на посадку и посадка по приборам по категории II и III не разрешается, если не предоставляется информация о RVR. Контрольная RVR определяется по сообщенным значениям RVR в одной или нескольких точках наблюдения за RVR (точка приземления, средняя точка и дальний конец ВПП), используемые в целях определения соблюдения установленных эксплуатационных минимумов. В случае, если используется информация о RVR в разных точках, контрольная RVR представляет собой RVR в точке приземления, при этом RVR в средней точке и в дальнем конце ВПП не менее RVR установленного минимума для взлета.
к)	Значение RVR указывается только тогда, когда это значение не совпадает со значением VIS. В этом случае перед цифровым значением минимума указывается аббревиатура RVR.
8.9.4.10.	Действия экипажа при прерванном и продолженном взлете 
Решение о прекращении или продолжении взлета принимает командир ВС. 
Скорость принятия решения V1 – наибольшая скорость разбега, при которой в случае отказа критического двигателя возможно, как безопасное прекращение, так и безопасное продолжение взлета. 
Расчет скорости принятия решения V1, скорости подъема передней ноги VR и безопасной скорости на взлете V2 производится (контролируется расчет) экипажем ВС перед каждым взлетом согласно РЛЭ воздушного судна (AFM, FCOM) с учетом конкретных условий предстоящего взлета и состояния ВПП.
(1) Прерванный взлет
Решение на прекращение взлета и выполнение всех операций по его прекращению (подача соответствующих команд на использование необходимых для торможения систем ВС) является прерогативой командира ВС в любом случае, независимо от того, кто пилотирует воздушное судно. 
Командир ВС (или по его команде бортмеханик) в процессе разбега держит руку на рычагах управления двигателями до скорости V1, а при принятии решения на прекращение взлета дает команду «РУД О» и предпринимает все необходимые действия по остановке воздушного судна в соответствии с РЛЭ. 
Решение на прекращение взлета может быть принято на усмотрение командира ВС при отказе двигателя или при появлении других неисправностей, угрожающих безопасности полета, если не достигнута скорость принятия решения на продолжение взлёта или если ВС отклонилось от заданного направления настолько, что продолжение разбега не обеспечивает безопасности. Запрещается отрыв воздушного судна от земли на скорости, менее установленной РЛЭ.
В случае прекращения взлёта по причине отказа или неисправности воздушного судна повторный взлет запрещается до выяснения и устранения причин, вызвавших прекращение взлета. 
Если прекращение взлёта не связано с отказом или неисправностью воздушного судна, решение о выполнении повторного взлёта может быть принято КВС, после проведения работ, если они предусмотрены в эксплуатационной документации воздушного судна.
(2) Продолженный взлет
При любых отказах на взлете на скорости выше скорости принятия решения V1 взлет должен быть продолжен. 
Никакие действия не предпринимаются до надежной стабилизации полета, кроме отключения звуковой сигнализации, если она мешает нормальному взаимодействию, до тех пор, пока: 
а)	не закончены нормальные процедуры;
б)	не закончены действия, предусмотренные РЛЭ типа ВС (AFM, FCOM); 
в)	не достигнута высота 400 ft (120 м) AGL в случаях отказа в процессе взлета;
г)	не достигнута высота 1200 ft (400 м) AGL в случае отказа двигателя в процессе взлета на Ан-74;
Ниже высоты 400 ft (120 м) AGL допускается только: 
д)	увеличение тяги двигателей; 
е)	уборка/выпуск шасси, если это не приведет к опасному нарушению балансировки воздушного судна.
Примечание. Для Ан-74ниже высоты 700 ft (200 м) (при отказе двигателя ниже 1200 ft (400 м)) AGL.
8.9.4.11.	Процедуры маневрирования при отказе двигателя на взлете на скорости более V1
При отказе двигателя на взлете после скорости V1 существует несколько видов процедур (ENGINE FAILURE PROCEDURE (EFP)), опубликованных в программе расчета ВПХ, для обеспечения безопасного и эффективного выполнения полета, которые должны быть выполнены экипажем:
а)	ENGINE OUT SID (SPECIAL ENGINE FAILURE PROCEDURE); 
б)	STANDARТ ENGINE OUT PROCEDURE; 
в)	STANDART INSTRUMENT DEPARTURE; 
г)	IMMEDIATE VISUAL RETURN 
а) ENGINE OUT SID (SPECIAL ENGINE FAILURE PROCEDURE) выполняется, если опубликована, при отказе двигателя до пролета точки отворота на начальный пункт схемы ENGINE OUT SID. 
Полет выполняется в соответствии с опубликованной процедурой ENGINE OUT SID. На engine out acceleration altitude (height) переведите ВС в горизонтальный полет для полной уборки механизации крыла в соответствии с FCOM. После уборки механизации крыла продолжите набор требуемой высоты или MSA. Выполните процедуры согласно QRH и информируйте ОВД о принятом решении. 
Примечание. При отказе двигателя после пролета точки отворота на ЕО SID выполняюm предписанный SID. 
б) STANDARТ ENGINE OUT PROCEDURE (выполняется при отсутствии препятствий по курсу взлета по решению КВС, при отсутствии ENGINE OUT SID). 
На курсе взлета набирайте 800 ft над уровнем аэродрома, информируйте ОВД о принятом решении. Переведите ВС в горизонтальный полет для полной уборки механизации крыла в соответствии с FCOM во время разворота на выбранную точку. После уборки механизации крыла продолжите набор требуемой высоты или MSA. Выполните процедуры согласно QRH. При необходимости запросите «векторение» для захода на посадку или полета в зону ожидания. 
в) STANDART INSTRUMENT DEPARTURE (выполняется по решению КВС при отсутствии значительных разворотов после взлета). 
Выполняйте полет в соответствии с опубликованным маршрутом выхода (SID). Переведите ВС в горизонтальный полет на engine out acceleration altitude (height) для полной уборки механизации крыла в соответствии с FCOM. После уборки механизации продолжайте набор требуемой высоты или MSA. Выполните процедуры согласно QRH и информируйте ОВД. 
г) IMMEDIATE VISUAL RETURN (в визуальных условиях при наличии негасимого пожара или бомбы на ВС). 
Если требуется выполнить немедленный возврат на аэродром вылета при полете в визуальных условиях, выполнить процедуры ENGINE FAILURE AFTER V1» до достижения acceleration altitude (height). Перевести ВС в ГП и сохранять механизацию крыла в положении FLAPS 1 до выполнения третьего разворота (начала доворота на посадочный курс).
8.9.4.12.	Использование минимальных безопасных высот в аварийных случаях после взлета
При аварийной обстановке после взлета при маневрировании ВС в пределах зоны ВЗП могут использоваться высоты Нмс (МВС) для визуального захода на посадку соответствующей категории ВС. При этом запас высоты над препятствиями составит:
а)	для ВС категории A и B - 90 м;
б)	для ВС категории С и D-120 м;
При маневрировании ВС за пределами зоны ВЗП, но в пределах 50 км от аэродрома минимальной используемой высотой является высота БВП для горных аэродромов и БВП, уменьшенная на 100 м, для остальных аэродромов. 
Примечание. Под термином «маневрирование» понимается выполнение полета в зоне ВЗП с учетом ограничений данного аэродрома, либо выполнение полета по установленной схеме выхода или входа при принятии решения о возвращения на аэродром вылета.
8.9.4.13.	Маневрирование после взлета в случаях негасимого пожара или бомбы на борту ВС (IMMEDIATE VISUAL RETURN)
При выполнении аварийной посадки после взлета, при наличии негасимого пожара или бомбы на борту ВС, экипажу ВС предоставляется право, исходя из конкретных условий полета, использовать, с учетом положений п. 8.9.4.12 следующие варианты схем:




Вариант 1.

                                                 R взп
	
                                                                                   
                                                                  ВПП          S1
                                       

                                                                                              S2
                                                                  



Рис.1


При принятии решения о выполнении экстренной посадки после взлета и до достижения удаления S1 выполняется разворот на 80º, а затем разворот на 260º в противоположном направлении с выходом на посадочную прямую с курсом, обратным посадочному, на удалении S2 (Таблица А-8.9-Т1).








Вариант 2.                     
                                                    R взп
                                                                                  S1
                                                               

                                                                ВПП
                                                      
                                                                                
                                                                                       2R
                                                                                     
                                                                                               S 2
                                                                                      


Рис.2
При выполнении аварийной посадки на участке от удаления 2R до S2 выполняется разворот на 260º, а затем на 80º в противоположном направлении с выходом на посадочную прямую с курсом обратным посадочному на удалении S1 (Таблица А-8.9-Т1).
Вариант 3 




                      


                                                               ВПП

                                                                                                             
                                                                                                                                                                     
                             R взп
                                                                                  S 2
                                                                                      
                                            
Рис.3
При принятии решения о выполнении аварийной посадки на ВПП с рабочим курсом до удаления S2 выполняется разворот на 180º с дальнейшим выполнением визуального захода на посадку (Табл. А8.9-Т1).
Примечания:
а)	При выполнении аварийной посадки выполнение первого разворота на удалениях более S1 для варианта №1 и S2 для варианта №2 приведет к выходу за пределы зоны ВЗП соответствующей категории ВС. В этом случае необходимо использовать НМС ВЗП для более высокой категории ВС.
б)	Использование варианта №2 на удалении менее 2R не позволит выйти на посадочную прямую до порога ВПП.                                                                                                                                     Таблица А-8.9 -Т1
Катего
рия
ВС	Скорость
выполнения
маневра
(км / час / узлы)	R
При крене
15º/30º
(км)	S1
(км)	S2
(км)	2R
(км)	R
ВЗП
(км)	Запас высоты
над
препятствиями
(км)	НМС ВЗП
(минимальная)
(м)		
										
										
										
										
		1,3	1.0	3,6	2,6					
В	210/114	0,6	3.1	4.3	1.2	4,90	90	150		
		1,6	0,1	3,3	3.2					
	230/124	0,7	2.8	4,2	1.4					
		1, 8	2,4	6,0	3,6					
С	250/135	0,8	5,4	7,0	1,6	7,85	120	180		
		2,2	1,2	5,6	4,4					
	270/146	1,0	4,8	6,8	2.0					
		2,5	2,3	7,3	5,0					
	290/157	1,2	6,2	8,6	2,4					
		2,8	1,1	6,7	5,6					
D	310/167	1,3	1,9	8,5	2,6	9,79	120	210		
		3,2	0,2	6,6	6,4					
	330/178	1,5	5,3	8,3	3,0					
		3,6*	-1,0	6,2	7,2					
	350/189	1,7	4,7	8,1	3,4					
		4,0*	-2,2	5,8	8,0					
	370/200	1,9	4,1	7,9	3,8					
Примечание: При выполнении маневра при данных скоростях и кренах необходимо использовать данные по выполнению захода с круга для ВС категории Е, при этом:
Rвзп = 12,82 км, запас над препятствиями = 150 м, Нмс (миним) = 240 м.
8.9.5.  Набор высоты
8.9.5.1. Набор высоты после взлета производится с курсом взлета до высоты над аэродромом не менее: 
а)	50 м на воздушном судне при выполнении авиационных работ, если работы выполняются на высоте 50 м и менее; 
б)	установленной схемой выхода из района аэродрома, но не менее высоты установленной схемой вылета или РЛЭ типа ВС.
Выход воздушного судна из района контролируемого аэродрома осуществляется по установленной схеме или по указаниям органа ОВД. 
При наличии нескольких опубликованных схем выхода орган ОВД заблаговременно информирует экипаж воздушного судна о схеме выхода и первоначально заданной высоте, если она не установлена в аэронавигационной информации.
Дальнейший набор согласованной с органом ОВД высоты производится при полете по установленной схеме выхода из района аэродрома.
8.9.5.2. С момента начала разбега воздушного судна и до набора высоты 200 м экипажу и диспетчеру пункта ОВД не допускается вступать в радиосвязь, за исключением случаев, когда возникает угроза безопасности полета.
8.9.5.3. Разрешение экипажу ВС на выполнение взлета является одновременно разрешением для перехода на связь с диспетчером круга на высоте 200 метров (безопасной или заданной). До набора этой высоты экипаж ВС обязан прослушивать частоту диспетчера старта. Если после выдачи разрешения на взлет прошло более одной минуты, то экипаж ВС обязан запросить повторное разрешение на взлет. 
С момента начала разбега воздушного судна и до набора высоты 200 м экипажу и диспетчеру пункта ОВД не допускается вступать в радиосвязь, за исключением случаев, когда возникает угроза безопасности полета.
8.9.5.4. При взлете на контролируемом аэродроме и получении разрешения бесступенчатого набора заданного эшелона полета доклад диспетчеру ОВД о взлете может не производиться. Экипаж ВС обязан прослушивать частоту ДПК до пересечения заданного эшелона (рубежа передачи ОВД диспетчеру ДПП).
После взлета на контролируемом аэродроме и при невозможности занятия заданного эшелона (высоты) полета к установленному или заданному органом ОВД рубежу, командир ВС информирует об этом соответствующий орган ОВД.
8.9.5.5. В процессе выполнения набора высоты члены летного экипажа должны выполнять все процедуры, предусмотренные РЛЭ ВС (AFM, FCOM), технологией работы и настоящим РПП. 
8.9.5.6. Выполнение установленных стандартных процедур выхода (SID) обязательно. Указания диспетчера органа ОВД выполняются экипажем ВС в том случае, если они не противоречат ограничениям РЛЭ воздушного судна (AFM, FCOM). 
В наборе высоты все члены летного экипажа должны соблюдать визуальную и радиоосмотрительность. 
8.9.5.7. При пересечении высоты перехода в наборе экипажем ВС по команде командира производится установка стандартного давления на высотомерах в соответствии со стандартными операционными процедурами. 
8.9.5.8. При пересечении эшелона FL100 (3000м) экипажем ВС производится контроль работы высотной системы воздушного судна в соответствии со стандартными операционными процедурами. 
8.9.5.9. В наборе высоты, во избежание срабатывания бортовой системы предупреждения столкновений (далее - БСПС (TCAS)), за 1000 ft (300м) до заданного эшелона (высоты) полета экипажем ВС устанавливается вертикальная скорость набора не более 1379 ft в минуту (7м/сек). 
Командир ВС должен контролировать и корректировать вертикальную скорость для поддержания заданного градиента набора высоты.
8.9.5.10. Если воздушное судно не может занять заданный органом ОВД эшелон (высоту) к установленному или заданному органом ОВД месту, экипаж воздушного судна обязан своевременно проинформировать об этом орган ОВД. 
8.9.5.11. В процессе набора высоты при получении команды диспетчера ОВД на занятие (сохранение) высоты (эшелона) полета экипаж воздушного судна подтверждает получение команды на занятие (сохранение) заданной высоты и контролирует правильность установки задатчика высоты.
8.9.5.12. По окончании набора заданного эшелона летный экипаж воздушного судна должен сличить показания барометрических высотомеров.
8.9.5.13. Для выполнения полетов в воздушном пространстве, в котором установлен порядок измерения высот полета в футах, ВС должно быть оборудовано соответствующими высотомерами (футомерами).
8.9.6.	Полеты по воздушным трассам и маршрутам
8.9.6.1. При выполнении полета по маршруту в контролируемом воздушном пространстве выдерживаются требования по точности аэронавигации, установленные для данного района полета и заданные органом ОВД высоты (эшелоны) полета.
Контроль курса, других навигационных элементов полета и ветра выполняется с периодичностью, позволяющей исключить отклонение ВС от заданной траектории на величину, превышающую допустимое значение для данного района полетов. 
В случае непреднамеренных отклонений от текущего плана полета, экипажем воздушного судна предпринимаются следующие действия: 
а)	если воздушное судно отклонилось от линии пути, экипаж воздушного судна корректирует курс воздушного судна в целях быстрейшего возвращения на линию заданного пути; 
б)	если среднее значение истинной воздушной скорости на крейсерском эшелоне между двумя контрольными пунктами не является неизменным или ожидается, что оно изменится на плюс-минус 5% от истинной воздушной скорости, указанной в плане полета, экипаж информирует об этом орган ОВД; 
в)	если обнаружится, что уточненный расчет времени пролета очередного запланированного контрольного пункта отличается более чем на 2 минуты от времени, о котором была уведомлен орган ОВД, экипаж воздушного судна информирует орган ОВД об уточненном времени. 
8.9.6.2. Изменение в полете плана полета в целях изменения маршрута или следования на другой аэродром производится при условии, что, начиная с места, где было произведено изменение маршрута полета, соблюдаются требования по запасу топлива и масла для обеспечения полета (по прибытии на другой аэродром) в течение не менее 30 минут со скоростью полета в зоне ожидания на высоте 450 метров при стандартных температурных условиях. 
8.9.6.3. Полеты по воздушным трассам, МВЛ и маршрутам полета в зависимости от метеорологических условий, типов ВС и их оборудования выполняются по ППП или ПВП на заданных высотах (эшелонах) полета в пределах установленной ширины трассы (МВЛ, маршрута полета).
8.9.6.4. В случае угрозы безопасности полета допускается изменение заданной высоты (эшелона) полета и уклонение от линии заданного пути, при этом экипаж ВС немедленно информирует об этом орган ОВД.
8.9.6.5. Экипаж ВС не позднее, чем за 5 минут до входа в воздушную трассу (МВЛ, маршрут полета) запрашивает разрешение и уточняет условия входа у органа ОВД, осуществляющего обслуживание воздушного движения на воздушной трассе (МВЛ, маршруте полета).
8.9.6.6. Разрешение и условия на вход в воздушную трассу (МВЛ, маршрут полета), при взлете с близко расположенного аэродрома, экипаж ВС запрашивает сразу после взлета.
8.9.6.7. Экипаж ВС не позднее, чем за 5 минут до выхода из воздушной трассы (МВЛ, маршрута полета) получает разрешение и условия выхода от органа ОВД, который будет осуществлять обслуживание воздушного движения после выхода ВС из воздушной трассы (МВЛ, маршрута полета).
8.9.6.8. Вход в воздушную трассу (МВЛ, маршрут полета) и выход из нее (его) производятся в режиме горизонтального полета на предварительно согласованных с органами ОВД эшелонах (высотах).
8.9.6.9. Занятие заданной высоты (эшелона) входа в воздушную трассу (МВЛ, маршрут полета) производится не менее чем за 10 км до ее границы. Изменение высоты (эшелона) полета после выхода из воздушной трассы (МВЛ, маршрута полета) производится на удалении не менее 10 км от границы воздушной трассы или по указанию органа ОВД.
8.9.6.10. Экипаж ВС непрерывно прослушивает канал (частоту) радиосвязи диспетчера того диспетчерского пункта (ДП), на ОВД которого он находится. Переход на радиосвязь с диспетчером другого ДП осуществляется только после получения разрешения на это от диспетчера ДП, в зоне (районе) которого ВС находилось. 
8.9.6.11. При наличии информации о воздушной обстановке на пунктах управления воздушным движением от автоматизированных систем управления воздушным движением или вторичного радиолокатора по указанию органа ОВД, экипаж ВС может быть освобожден от докладов о пролете пунктов обязательного донесения и эшелоне полета. 
8.9.6.12. Запрещено выполнять полет ВС над территориями населенных пунктов и над местами скоплений людей при проведении массовых мероприятий ниже высоты, допускающей, в случае отказа двигателя, аварийную посадку без создания чрезмерной опасности для людей и имущества на земле. 
В случае, когда метеоусловия не позволяют выдерживать установленную высоту полета, обход населенного пункта производится, как правило, с правой стороны, если не установлен иной порядок обхода.
8.9.6.13. Орган ОВД, по запросу экипажа или в случае отклонения ВС от линии заданного пути, при наличии радиолокационного контроля предоставляет экипажу имеющуюся информацию о его местоположении.
8.9.6.14. При возникновении в полёте признаков приближения к зоне опасных метеорологических явлений или получении соответствующей информации КВС обязан принять меры для обхода опасной зоны, если полёт в ожидаемых условиях не разрешён РЛЭ. При невозможности продолжать полёт до пункта назначения из-за опасных метеорологических явлений КВС может произвести посадку на запасном аэродроме или вернуться в пункт вылета. О принятом решении и своих действиях КВС должен, при наличии связи, сообщить органу ОВД, который обязан принять необходимые меры по обеспечению безопасности дальнейшего полёта. 
8.9.6.15. В полете экипаж должен постоянно анализировать аэронавигационную и метеорологическую обстановку по маршруту полета (в районе выполнения авиационных работ), на запасных аэродромах, на аэродроме назначения и запасных аэродромах пункта назначения и вести контроль расхода топлива.
При получении информации об ухудшении метеоусловий ниже установленного минимума или технической неготовности на аэродроме назначения (запасном аэродроме), делающих невозможным совершение безопасной посадки, орган ОВД немедленно сообщает об этом экипажу ВС.
Полет на запасной аэродром может выполняться с оптимальным профилем полета, по кратчайшему расстоянию вне воздушных трасс (по согласованию с органом ОВД). 
На основании анализа аэронавигационной и метеорологической обстановки КВС может выбрать запасной аэродром в полете. 
Полет по ППП продолжается в направлении аэродрома намеченной посадки только в том случае, если самая последняя имеющаяся информация указывает на то, что к расчетному времени прилета посадка на указанном аэродроме или на одном запасном аэродроме пункта назначения может быть выполнена с соблюдением эксплуатационных минимумов для посадки.
При входе в район ОВД, где находится рубеж ухода на запасной аэродром, экипаж ВС обязан информировать орган ОВД о расчетном времени пролета рубежа ухода и выбранном запасном аэродроме. При получении указанной информации, в случае если воздушное судно находится вне зоны вещания автоматизированной системы ВОЛМЕТ, орган ОВД немедленно запрашивает данные о фактической и прогнозируемой погоде, а также подтверждение технической готовности запасного аэродрома и аэродрома назначения к приёму воздушного судна и передаёт эти сведения экипажу воздушного судна;
Решение на продолжение полета до аэродрома назначения с рубежа ухода может быть принято КВС, если последняя информация указывает на то, что: 
а)	прогнозом погоды на аэродроме назначения ко времени прилета предусматриваются метеоусловия, соответствующие требованиям для запасного аэродрома 8.4.2.14; 
б)	есть информация о технической готовности аэродрома назначения к приему ВС.
8.9.6.16. В контролируемом воздушном пространстве при входе в район ОВД на установленном рубеже передачи связи командир ВС сообщает органу ОВД свое местонахождение, высоту полета и получает от него дальнейшие условия для полета.
8.9.6.17. За 30-40 мин. до расчетного времени прибытия экипаж должен установить связь по специальному каналу связи с соответствующей службой аэропорта посадки и (или) авиакомпании и передать информацию об неисправностях или отказах в работе авиатехники выявленных в полете. Передается также информация коммерческого содержания для оптимизации процедур подготовки ВС к дальнейшему полету.
8.9.7.	Снижение, заход на посадку и посадка
8.9.7.1.	Предпосадочная подготовка
Перед снижением для захода на посадку под руководством КВС проводится предпосадочная подготовка экипажа и воздушного судна. При продолжительности полета менее 1 чaca часть предпосадочной подготовки по решению командира ВС может быть проведена перед вылетом.
(1) В процессе предпосадочной подготовки экипаж должен:
а)	проверить правильность работы бортовых навигационных средств самолетовождения, включая:
	проверку RNP/ANP;
	проверку правильности работы СНС, НК, FMS и т.п.;
	проверку точности самолетовождения помощью радионавигационных средств (VOR, DME, АРК, РЛС и т.п.);
б)	настроить радионавигационные средства самолетовождения (ILS, VOR, DME, АРК и т.п.);
в)	проверить соответствие фактического местоположения ВС определенному навигационной системой.
(2) Рубеж начала снижения (TOD) рассчитывается с учетом маршрута снижения, фактической высоты полета, ограничений по высотам и скоростям, необходимости применения противообледенительной системы, направления и скорости ветра по высотам, массы воздушного судна. Снижение планировать заблаговременно для исключения чрезмерно крутой траектории. 
(3) Принятая информация АТIS записывается в палетку "Взлет-Посадка" установленного в авиакомпании образца. 
Давление аэродрома по QNH, при необходимости, пересчитывается в давление QFE, как минимум, с обязательным сравнением результатов. На основе полученных данных о фактических метеоусловиях и уточненной массе ВС рассчитываются посадочные характеристики.
В зависимости от переданной информации АТИС экипаж при проведении предпосадочной подготовки по таблице в главе В.0 производит пересчет для установки давления на основных барометрических высотомерах значения давления из миллибар в миллиметры. Производит перерасчет контрольных высот захода на посадку (высота 4го разворота, точка входа в глиссаду, высота пролета дальнего/ближнего привода, высота принятия решения) из значений в футах в значения в метрах. 
На футомере КВС на эшелоне перехода устанавливается давление аэродрома (QNH или QFE), на высотомерах остальных членов экипажей устанавливается пересчитанное давление аэродрома (QNH или QFE). Заход на посадку выполняется в соответствии со схемой захода на посадку, опубликованной в сборнике Jeppesen.
При перерасчете значений давлений из миллибар в миллиметры, контрольных высот из футов в метры, перерасчет производится вторым пилотом и штурманом с последующим контролем КВС. 
После установления полученных данных до выполнения карты контрольных проверок на эшелоне перехода производится перекрестный контроль рассчитанных данных.
(4) Схема захода на посадку располагается в кабине экипажа таким образом, чтобы каждый пилот мог видеть всю необходимую информацию. Маневр захода на посадку выполняется в соответствии с опубликованной схемой и указаниями диспетчера органа ОВД.
(5) При смене ВПП (курса посадки) или возникновении условий, требующих изменения ранее принятых решений, экипажем ВС должна быть проведена дополнительная подготовка и повторная проверка выполненных операций по карте контрольной проверок. 
(6) При выполнении неточного захода на посадку, за исключением заходов с применением метода CDFA, экипаж в процессе предпосадочной подготовки рассчитывает удаление VDP* от торца ВПП и способы определения:
а)	временем полета от FAF;
б)	временем полета от ДПРМ (locator outer with radio marker(LOM));
в)	удалением по DME;
г)	с использованием FMS.
*VDP (точка начала визуального снижения) – точка, расположенная на посадочной прямой, в которой ВС находится на стандартной траектории снижения(УНГ=3°) и высоте, равной минимальной высоте снижения (МВС).
VDP следует рассматривать как последнюю точку на траектории захода на посадку по приборам, с которой может быть выполнено стабилизированное визуальное снижение и посадка на данную ВПП.
В зависимости от МВС при УНГ=3° VDP определяется по Таблицам А 8.9-Т2 и А - 8.9-Т3.
                                                                                                                                               Таблица А 8.9-Т2
МВС	80	90		100		110	120	130		140		150	160	170	180	190	200	210	220	230	240	250
(м)																						
																											
S до																											
ВПП	1200	1400		1600		1800	2000	2200		2400		2600	2800	3000	3100	3300	3500	3700	3900	4100	4300	4500
(м)																											
                                                                                                                     
                                                                                                                                                 Таблица А 8.9-Т3
МВС
(фут)	260	300	330		360		390		430		460		490	520	560	590		620	660	690		720	750	790	820
																									
																											
S до
ВПП
(n. m)																											
	0.7	0.8	0.9		1.0		1.1		1.2		1.3		1.4	1.5	1.6	1.7		1.8	1.9	2.0		2.1	2.2	2.3	2.4
																											
(7) Для реализации процедуры непрерывного снижения при неточном заходе на посадку (CDFA) с неавтоматизированным расчетом траектории снижения (без применения консультативного наведения VNAV) экипаж должен рассчитать вертикальную скорость снижения на конечном участке захода на посадку.
8.9.7.2.	Предпосадочный брифинг
Предпосадочный брифинг командир ВС проводит, как правило, до начала снижения, после завершения всех операций предпосадочной подготовки и получения докладов от членов летного экипажа о готовности к снижению.
При предпосадочном брифинге обсуждаются (но не ограничиваются) следующие элементы:
а)	предполагаемая для посадки ВПП, ее состояние, коэффициент сцепления; тип и состав светотехнического оборудования, огни «визуальной» глиссады, маркировка ВПП, ее длина и ширина, наличие смещенных порогов, расположение РД и карманов для разворота; наличие параллельных (близких по направлению) ВПП, РД, ГВПП, автодороги, мосты и прочие объекты, которые могут быть ошибочно восприняты за назначенную для посадки ВПП;
б)	метеорологические фактические и прогнозируемые условия на маршруте снижения и при заходе на посадку, информация ATIS, наличие опасных метеоявлений, порядок использования ПОС, локатора и других систем ВС;
в)	планируемое положение закрылков в зависимости от внешних условий полета, состояния ВПП, технического состояния ВС
г)	порядок использования систем торможения, в том числе особенности применения автоматического и неавтоматического режима торможения, особенности применения реверса тяги с учетом обстоятельств, ухудшающих условия торможения к моменту расчетного времени посадки;
д)	маневр входа в зону аэродрома (STAR), включая маршрут и схему захода на посадку, ограничения по высотам и скоростям, противошумовые процедуры, безопасные высоты в районе аэродрома и возможные маршруты векторения;
е)	порядок использования высотомеров (QNH / QFE), эшелон перехода;
ж)	система захода на посадку – основная, резервная, эксплуатационный минимум захода на посадку по основной и резервной системам;
з)	действия при отказе основной системы захода на посадку, принятие решения о применении резервной системы; 
и)	настройка радиосредств (частоты, курсы, режимы, положение переключателей и т.д.);
к)	маневр ухода на второй круг, включая маршрут, ограничения по высотам и скоростям, порядок взаимодействия;
л)	запасной аэродром, остаток топлива, максимальное время ожидания, маршрут ухода на запасной аэродром и порядок взаимодействия;
м)	порядок использования автоматизированной системы управления ВС, особенности эксплуатации ВС при заходе на посадку;
н)	предполагаемый маршрут руления после освобождения ВПП;
о)	техническое состояние ВС и его систем, влияние имеющихся неисправностей на предстоящий этап захода на посадку и производство посадки, 
п)	особенности захода на посадку и производство посадки при наличии отложенных неисправностях по ПМО/MEL/CDL;
р)	при наличии в кабине экипажа специалиста (обзёрвер, стажер, проверяющий, контролирующий) дополнительно проводится его инструктаж. 
Уделяется внимание порядку использования откидного сидения, использования ремней безопасности, пользования кислородом, покидания кабины в штатной и аварийной ситуациях.
В процессе проведения командиром ВС предпосадочного брифинга для захода на посадку в условиях минимума САТ II и III дополнительно обсуждаются: 
а)	наличие допуска членов летного экипажа к выполнению захода на посадку САТ II или III; 
б)	состояние аэродрома посадки и его оборудования на соответствие требованиям, установленным для данной категории захода на посадку; 
в)	состояние систем ВС и его оборудования на соответствие требованиям, установленным для данной категории захода на посадку; 
г)	порядок взаимодействия членов летного экипажа при заходе на посадку по указанному минимуму; 
д)	порядок выполнения процедуры ухода на второй круг; 
е)	действия экипажа (процедуры) при отказах систем (оборудования) ВС или оборудования аэродрома посадки на различных этапах захода, определяемых РЛЭ ВС (AFM, FCOM); 
ж)	дополнительные процедуры, стандартная фразеология и отклики, характерные для захода по САТ II или III; 
з)	использование светотехнического оборудования ВС; 
и)	регулировка положения пилотских кресел по высоте и удалению от органов управления. 
При изменении условий захода на посадку (ВПП, STAR, метеорологическая обстановка, система захода и т.п.) командир ВС должен провести дополнительный предпосадочный брифинг
8.9.7.3.	Снижение ВС с эшелона полета
При входе в район ОВД, в котором расположен аэродром посадки, командир ВС (член экипажа по указанию КВС) информирует орган ОВД о выбранном запасном аэродроме. 
Снижение воздушного судна с заданного эшелона (высоты) полета производится по разрешению диспетчера органа ОВД с докладом экипажа о начале снижения. Если указания диспетчера, по мнению командира ВС, не обеспечивают безопасности, следует немедленно запросить изменение полученного указания. 
Вход воздушного судна в район контролируемого аэродрома производится по схеме опубликованной аэронавигационной информации или по указаниям органа ОВД. При наличии нескольких опубликованных схем захода орган ОВД заблаговременно информирует экипаж воздушного судна о схеме захода, по которой следует выполнять полет. 
В процессе снижения экипажи воздушных судов во избежание срабатывания БСПС (TCAS)
выдерживают рекомендованные ограничения по вертикальной скорости не более 7 м/с (1379 ft в минуту) за 300 м (1000 ft) до заданного эшелона (высоты).
(5) В зонах с интенсивным воздушным движением в процессе снижения с эшелона 3000 м (10000 ft) для захода на посадку устанавливается приборная скорость не более 450 км/ч. (250 kt).
При снижении ниже высоты 3000 м (10000 ft) вертикальная скорость снижения не должна превышать значения 15 м/с (3000 ft/м), ниже высоты Н эш.перех. (Н IAF) - не должна превышать 10 м/с (2000 ft/ м), ниже Н твг (Н FAF) – не должна превышать 5 м/с (1000 ft/ м). При этом, если расчетная вертикальная скорость на участке конечного этапа захода на посадку на конкретном аэродроме более 5 м/с, это должно оговариваться на предпосадочном брифинге.
Информация органа ОВД «Снижение без ограничений» является основанием для выдерживания скоростей на усмотрение командира ВС.
Сведения о введении ограничений публикуются в документах аэронавигационной информации. 
Наиболее предпочтительным распределением обязанностей между пилотами при заходе на посадку в сложных метеорологических условиях и невозможности использования автоматизированной системы захода на посадку является вариант активного пилотирования вторым пилотом до ВПР(DA/H). 
На контролируемом аэродроме при невозможности выдерживания параметров полета заданных диспетчером органа ОВД, невозможности занятия заданного эшелона (высоты) к установленному или заданному рубежу командир ВС своевременно информирует об этом орган ОВД.
При входе в район контролируемого аэродрома командир ВС, выполняющий полет по ПВП, сообщает органу ОВД свое местонахождение, высоту полета и получает от него условия для захода на посадку.
В целях регулирования интервалов между воздушными судами, оказания содействия экипажам по обходу районов с неблагоприятными метеорологическими условиями, орган ОВД может производить векторение, а также задавать режимы поступательных и вертикальных скоростей в допустимых для данного ВС пределах, при этом:
а)	при осуществлении векторения точность выдерживания параметров, задаваемых органом ОВД, обеспечивает летный экипаж воздушного судна с учетом летно-технических характеристик ВС;
б)	векторение обеспечивается посредством указания пилоту конкретных курсов, позволяющих экипажам воздушным судов выдерживать необходимую линию пути; 
в)	в случае, если воздушное судно начинает наводиться с отклонением от ранее заданного маршрута, пилоту сообщается органом ОВД о целях такого наведения и не даются указания на снижение ниже высоты, обеспечивающей предписанный запас высоты над препятствием, в том числе с учетом влияния низких температур;
г)	векторение ВС прекращается органом ОВД после возобновления пилотом самостоятельного самолетовождения на основании выданного диспетчером ОВД указания, содержащего информацию о местоположении воздушного судна, точке выхода на заданный маршрут, магнитном путевом угле и расстоянии до неё; 
д)	момент доворота воздушного судна для выхода на траекторию конечного этапа захода на посадку является окончанием векторения. Разрешение на заход выдается органом ОВД одновременно с последним заданным курсом, сообщения о местонахождении ВС; 
е)	при заходе на посадку по приборам начатое векторение продолжается до выхода самолета на конечный этап захода на посадку по приборам или до начала визуального захода на посадку по разрешению органа ОВД; 
ж)	после получения разрешения на заход лётный экипаж ВС выдерживает последний заданный курс до входа в зону действия средств наведения на конечном этапе захода на посадку, затем без дополнительных указаний органа ОВД выполняет доворот и стабилизацию воздушного судна на линии, заданной средством наведения на продолженном конечном этапе захода на посадку.
Снижение ВС для посадки на горном аэродроме производится:
а)	при полетах по ППП - при наличии радиолокационного контроля или при применении: угломерно-дальномерных систем, или стационарного спутникового приемоиндикатора (при наличии схемы захода на посадку по СНС), при устойчивой работе бортового навигационного оборудования и знания летным экипажем местоположения ВС после пролета маркированного рубежа с соблюдением схемы захода на посадку;
б)	при полетах по ПВП - в соответствии с данными правилами с обязательным применением радиотехнических средств захода на посадку.
Внимание! На горных контролируемых аэродромах полеты по траекториям, задаваемым органом ОВД, не допускаются.
8.9.7.4.	Полет в зоне ожидания
(1) Вход в зону ожидания осуществляется по установленному маршруту, а при его отсутствии – по стандартным правилам: параллельный, смещенный или прямой вход.
(2) Правила входа в зону ожидания(ЗО).     
 
1. «parallel entry» - выход на fix ЗО, полет с обратным курсом на предписанной скорости (табл.А8.9-Т4) в течении времени, указанного на схеме или 1мин. на FL до 140 включительно, или 1.5мин.на FL более 140, разворот на fix, при пролете fix вписывание в схему ЗО;
2. «offset entry» - выход на fix ЗО, полет с курсом, отличающимся от обратного на 30º, на предписанной скорости (табл.А8.9-Т4) в течении времени, указанного на схеме или 1мин. на FL до 140 включительно, или 1.5мин.на FL более 140, разворот на fix, при пролете fix вписывание в схему ЗО;
3. «direct entry» - выход на fix ЗО, при пролете fix вписывание в схему ЗО.

       (3) Полет в зоне ожидания производится по установленной схеме или в соответствии с указанием органа ОВД.

 
В контролируемом воздушном пространстве изменение высоты (эшелона) полета в зоне ожидания производится с разрешения органа ОВД, под управлением которого находится воздушное судно. 
Полет в зоне ожидания выполняется на скорости не более опубликованной для зоны ожидания, а если она не опубликована, то на скорости, не превышающей значений, указанных в таблице А 8.9-Т4
Вход в зону ожидания и выполнение процедуры ожидания выполняется с креном 25 градусов или с угловой скоростью разворота не менее 3 град/с, если на схеме не опубликованы другие данные.
Скорость полета в зоне ожидания выдерживается с точностью не ниже 5 км/ч. (см. табл.А 8.9-Т4).
Скорости полета по прибору в зоне ожидания
                                                                                                                                          Таблица А 8.9-Т4
ВЫСОТА/ЭШЕЛОН	ОБЫЧНЫЕ УСЛОВИЯ	УСЛОВИЯ ТУРБУЛЕНТНОСТИ
ДО FL 140 (4250М) 
ВКЛЮЧИТЕЛЬНО	170 KT (315КМ/Ч) ВС КАТ. А, В
230 KT (425КМ/Ч) ВС КАТ. С, D	170 KT (315 КМ/Ч) ВС КАТ. А И В
280 KT (520*КМ/Ч)
ВЫШЕ FL 140 ДО FL 200 (6100М) ВКЛЮЧИТЕЛЬНО	240 KT (445**КМ/Ч)	
280 KT (520КМ/Ч) ИЛИ 0.8 М
В ЗАВИСИМОСТИ ОТ ТОГО, ЧТО МЕНЬШЕ*
ВЫШЕ FL 200 ДО FL 340 (10350М) ВКЛЮЧИТЕЛЬНО	265 KT (490**КМ/Ч)	
ВЫШЕ FL 340	0.83 М
 Для схем ожидания, связанных со структурой маршрутов, используется скорость полета 450 км/ч.
Скорость 520 км/ч или 0.8 М, рассчитанная из условий турбулентности, используется для полета в зоне ожидания только после предварительного разрешения органа ОВД.
Для схем ожидания, связанных со структурой маршрутов, используется скорость полета 520 км/ч.
(4) Разворот на линию пути удаления начинается в момент выхода ВС в контрольную точку ожидания. 
Разворот на линию пути приближения начинается без упреждения в случаях, когда начало разворота задается моментом достижения заданного значения навигационного параметра (дальности или пеленга от наземного средства навигации) или моментом пролета над навигационным средством. 
В случаях, когда момент начала разворота на линию пути приближения не указан, то экипаж руководствуется временем полета по линии пути удаления от траверза контрольной точки ожидания, равным для штилевых условий:
1 мин, если эшелон (высота) ожидания не превышает FL 140 (4250 м);
1,5 мин, если эшелон (высота) ожидания превышает FL 140 (4250 м).
В случае невозможности выполнения требований процедуры ожидания, командир ВС информирует об этом орган ОВД, под управлением, которого находится воздушное судно.
8.9.7.5.	Заход на посадку
Необходимые условия для начала или продолжения захода на посадку по приборам:
(1) Перед заходом на посадку экипаж ВС обязан проверить правильность установки давления на шкалах давлений барометрических высотомеров и сравнить показания всех высотомеров.
(2) Перед заходом на посадку на контролируемом аэродроме командир ВС должен сообщить органу ОВД: 
а)	выбранную систему захода на посадку и получить разрешение на заход на посадку по выбранной системе, если не предполагается наведение ВС по инициативе органа ОВД;
б)	при выполнении захода на посадку с применением визуального маневрирования (circle-to-land) - об установлении визуального контакта с ВПП и (или) ее ориентирами;
в)	при выполнении визуального захода - об установлении визуального контакта с аэродромом.
(3) Запрещается заход на посадку по ППП ниже установленной в документах аэронавигационной информации высоты начала конечного этапа захода на посадку если:
а)	состояние ВПП не соответствует установленным требованиям;
б)	скорость ветра у земли с учетом его направления и порывов, а также значение коэффициента сцепления превышают установленные ограничения.
(4) Если значение сообщённой метеорологической видимости или контрольной RVR ниже эксплуатационного минимума для посадки, заход на посадку по ППП не продолжается ниже установленной в документах аэронавигационной информации высоты начала конечного этапа захода на посадку. 
(5) Если после пролёта этой высоты получено значение метеорологической видимости или RVR ниже эксплуатационного минимума для посадки, заход на посадку может продолжаться до ВПР (DA/H) или МВС (MDА/H). В этом случае, при условии, что до достижения ВПР (DA/H) или МВС (MDА/H), командиром ВС установлен необходимый визуальный контакт с наземными ориентирами (п. 8.9.7.20), КВС имеет право произвести снижение ниже ВПР (DA/H) или МВС (MDА/H) и выполнить посадку. 
Не допускается устанавливать эксплуатационные минимумы аэродрома для посадки при видимости менее 800м, если не предоставляется информация о RVR.
Заход на посадку и посадка по приборам по минимуму ниже САТ I не разрешается, если не предоставляется информация о RVR. При отсутствии информации о RVR заход на посадку и посадка по САТ I выполняется при метеорологической видимости не менее 800 м.
(6) Контрольная RVR определяется по сообщенным значениям RVR в одной или нескольких точках наблюдения за RVR (точка приземления, средняя точка и конец ВПП), используемых в целях определения соблюдения установленных эксплуатационных минимумов. В случае, когда используется информация о RVR в разных точках, контрольная RVR представляет собой RVR в точке приземления, при этом RVR в средней точке и в дальнем конце ВПП должна быть не менее RVR установленного минимума для взлета. 
Примечание. При отсутствии информации о минимумах для взлета с данным посадочным курсом (взлет с данным курсом запрещен) применять значения минимума для взлета, приведенные в таблице «Условные минимумы для взлета» сборников эксплуатационных минимумов аэродромов, рассчитанные согласно Методике определения минимумов Авиакомпании.
Внимание!
«Условные минимумы для взлёта» используются только для определения RVR в средней точке и дальнем конце ВПП при заходе на посадку и не являются основанием для выполнения процедуры взлёта с ВПП, взлёт с которой запрещён.
Запрещается выполнение посадки при наличии информации о сильном дожде и метеорологической видимости менее 600 м без использования бортового радиолокатора и системы заблаговременного предупреждения о сдвиге ветра. 
(7) В любом случае, КВС прекращает заход на посадку на любом аэродроме, если, по его мнению, не обеспечивается безопасность посадки. 
При отсутствии визуального наблюдения пилотом менее одного наземного ориентира в течение времени, достаточного для оценки пилотом местоположения воздушного судна и тенденции его изменения по отношению к заданной траектории полёта, продолжение захода на посадку ниже DA/H или MDA/H является нарушением минимума для посадки.
Указанными ориентирами являются:
а)	при заходе на посадку с применением визуального маневрирования (маневр «circle-to-land») – любые ориентиры, относительно которых представляется возможным определять положение воздушного судна относительно ВПП. Снижение ниже высоты MDA/H, установленной для визуального маневрирования (маневр «circle-to-land»), допускается только при наличии визуального контакта с порогом ВПП или светосигнальными средствами захода на посадку, связанными с ВПП;
б)	при заходе на посадку в условиях не ниже категории I – система огней приближения или её часть, порог ВПП и его маркировка, входные огни ВПП, огни обозначения порога ВПП, система визуальной индикации глиссады, зона приземления, её маркировка, огни зоны приземления, посадочные огни ВПП.
(8) На контролируемом аэродроме разрешение захода на посадку по опубликованной схеме от органа ОВД может быть выдано в любой момент времени, но не позднее выхода ВС на конечный участок захода на посадку (до входа в глиссаду). Заход на посадку по РМС начинается в точке конечного участка захода на посадку, являющейся точкой входа в глиссаду (FAF, FAP).
(9) При полетах на неконтролируемый аэродром или на контролируемый аэродром, на котором временно не производится обслуживание аэродромного (воздушного и (или) наземного) движения, перед заходом на посадку КВС обязан: 
а)	получить информацию о состоянии ВПП и её пригодности для выполнения посадки от органа ОВД аэродрома или выполнить осмотр ВПП с воздуха;
б)	передать сведения о месте и магнитном курсе посадки на частоте связи органа ОВД, в районе ответственности которого он находится. 
После приземления, при наличии связи с органом ОВД, сообщить ему о посадке. 
(10) Командир ВС может выполнять повторный заход на посадку при:
а)	запасе топлива, достаточном для ухода на запасной аэродром с ВПР(DA(H)) или от точки ухода на второй круг после повторного захода (с сохранением Final Reserve Fuel);
б)	фактических условиях захода на посадку и посадки, соответствующих требованиям ФАП.
(11) При отсутствии на аэродроме посадки радиолокационного контроля или невозможности использования угломерно-дальномерной системы, заход на посадку в условиях ППП по установленной схеме, выполняется после выхода на радионавигационное средство, на котором основана схема, на безопасном эшелоне (высоте).
(12) При одновременном визуальном заходе на посадку двух воздушных судов преимущество совершить посадку первым имеет воздушное судно, летящее впереди, слева или ниже.
8.9.7.6.	Указания по выполнению точных и неточных заходов на посадку по приборам
Сокращения
ILS       система посадки по приборам
MLS     микроволновая система посадки (сантиметрового диапазона)
GLS     система посадки с использованием GBAS
GBAS  наземная система функционального дополнения
PAR     посадочный радиолокатор
SBAS   спутниковая система функционального дополнения
APV     схема захода на посадку с вертикальным наведением
VNAV вертикальная навигация
LNAV боковая навигация
LOC   курсовой радиомаяк
DME   дальномерное оборудование
SRA   заход на посадку с помощью обзорного радиолокатора
NPA   неточный заход на посадку
РА      точный заход на посадку
VOR   всенаправленный ОВЧ-радиомаяк
NDB   ненаправленный радиомаяк
RNAV зональная навигация
PBN   навигация, основанная на характеристиках
RNP   требуемые навигационные характеристики
CDFA заход на посадку с непрерывным снижением на конечном участке
FMS    система управления полетом
Заходы на посадку на конечном этапе схемы захода на посадку разделяются на следующие виды:
а)	точный заход на посадку с использованием ILS, MLS, GLS, GBAS (п.8.9.7.8), PAR, SBAS;
б)	заход на посадку с вертикальным наведением (APV) (с использованием баро - VNAV, APV I и APV II);
в)	неточный заход на посадку (с использованием NPA, LOC c DME, VOR, VOR c DME, 2 NDB, NDB, NDB c DME, VDF, RNAV(LNAV), SPA, который в свою очередь подразделяется на «при наличии FAF» и «при отсутствии FAF».
В «Сборниках эксплуатационных минимумов аэродромов для взлета и посадки самолетов» приведены структурные схемы заходов на посадку для различных:
а)	видов захода на посадку;
б)	методов захода на посадку;
в)	схем захода на посадку по ППП;
г)	классификаций минимальных значений эксплуатационных минимумов;
д)	видов конечных этапов захода на посадку;
е)	способов управления траекторией в вертикальной плоскости при заходах NPA;
ж)	заходов на посадку с применением оборудования RNAV.
8.9.7.7.	Классификация заходов на посадку по приборам
Схемы заходов на посадку по приборам классифицируются следующим образом:
а)	NPA- неточные заходы на посадку по приборам для выполнения двухмерных заходов (2D) по типу А;
б)	APV- заходы с вертикальным наведением для выполнения трехмерных заходов (3D) по типу А;
в)	РА - точные заходы на посадку для выполнения трехмерных заходов (3D) по типу А или В.
Классификация заходов на посадку по приборам исходя из расчетных наиболее низких эксплуатационных минимумов, ниже которых заход на посадку продолжается только при необходимом визуальном контакте с ориентирами, следующим образом:
а)	тип A: минимальная относительная высота снижения или минимальная относительная высота принятия решения составляет 75 м (250 фут) или более;
б)	тип B: относительная высота принятия решения составляет менее 75 м (250 фут). Заходы на посадку по приборам типа B подразделяются на следующие категории:
(1) Категория I (САТ I): относительная высота принятия решения не менее 60м (200 фут) и/либо при видимости не менее 800 м, либо при дальности видимости на ВПП не менее 550м (любой точный заход на посадку при DA/H в 60м (200футов) или выше и минимальной видимости RVR в 550м или более определяется как стандартный заход на посадку по САТ I);
(2) Категория II (САТ II): относительная высота принятия решения менее 60м (200фут), но не менее 30м (100 фут) и/или дальность видимости на ВПП не менее 300 м;
(3) Категория IIIA (САТ IIIA): относительная высота принятия решения менее 30м (100фут) или без ограничений по относительной высоте принятия решения и/или дальность видимости на ВПП не менее 175 м;
(4) Категория IIIB (САТ IIIB): относительная высота принятия решения менее 15м (50фут) или без ограничений по относительной высоте принятия решения и/или дальность видимости на ВПП менее 175 м, но не менее 50м;
(5) Категория IIIC (САТ IIIC): без ограничений по относительной высоте принятия решения и дальности видимости на ВПП.
Внимание!
а)	Перед каждым полетом КВС должен проверить состояние ВС по открытым пунктам MEL в листе отложенных неисправностей (HIL).
б)	Если состояние радио или электротехнического оборудования ВПП не отвечает установленным требованиям или отсутствует информация о видимости на ВПП, для принятия решения на производство посадки необходимо руководствоваться таблицами Приложения СП2 РПП части С «Сборник установленных минимумов для захода на посадку и взлета на аэродромах, разрешенных для самолетов категорий В и C». 
Процедуры выполнения точных заходов на посадку описаны в РПП, Часть В типа ВС, глава 2 «Инструкция по взаимодействию и технология работы членов экипажа» (Standard Оperation Procedure).
8.9.7.8.	Точный заход на посадку с использованием Global Navigation Satellite System 
(GNSS) – вариант GBAS (Ground-Based Augmentation System)
(1) Выполнение захода на посадку.
Точный заход на посадку по GBAS выполняется методом, в значительной степени аналогичным точному заходу на посадку по ILS, с использованием бокового наведения на промежуточном участке до входа в глиссаду, после чего для посадки наряду с боковым наведением начинает и продолжает обеспечиваться вертикальное наведение.
(2) Критерии отображения информации при заходе на посадку по GBAS.
GBAS обеспечивает точный заход на посадку, аналогично заходу на посадку по категории I ILS.
Минимальные функциональные возможности отображения аналогичны ILS и предусматривают индикацию отклонения по курсу, индикацию отклонения в вертикальной плоскости, информацию о расстоянии до порога ВПП и флажки сигнализации отказов. При отсутствии на борту навигационного оборудования пилот не обеспечивается информацией о местоположении и навигационной информацией. Предоставляется лишь информация, обеспечивающая наведение по курсу и глиссаде на конечном этапе захода на посадку.
8.9.7.9.	Заход на посадку с вертикальным наведением (APV) 
Заход на посадку по приборам основанный на характеристиках (PBN), предназначенный для выполнения трехмерных (3D) заходов на посадку по приборам типа A.
Заход на посадку с вертикальным наведением с использованием оборудования баро - VNAV
Барометрическая вертикальная навигация (баро-VNAV) представляет собой навигационную систему, которая выдает пилоту вычисленное вертикальное наведение относительно угла траектории в вертикальной плоскости (VРА), номинальное значение которого составляет 3°.
Обеспечиваемое ЭВМ вертикальное наведение основывается на барометрической абсолютной высоте и определяется в виде VPA, начинающегося на относительной высоте опорной точки (RDH).
Схемы APV/баро-VNAV предназначаются для использования ВС, оборудованными FMS или иными системами RNAV, способными вычислять траектории барометрической VNAV и выдавать отклонения от них на индикатор на приборной доске.
Особенности использования баро-VNAV при низких температурах наружного воздуха аэродрома
а)	схемы бapo-VNAV не разрешается использовать в том случае, когда температура на аэродроме ниже опубликованной минимальной температуры для конкретной схемы, если система управления (FMS) не имеет для конечного этапа захода на посадку утвержденной компенсации низких температур. При наличии такой функции минимальную температуру можно не учитывать при условии, что она находится в пределах минимальной сертифицированной температуры для оборудования. Ниже этой температуры и применительно ВС, не имеющим FMS с утвержденной для конечного этапа захода на посадку компенсацией низких температур, может по-прежнему использоваться схема LNAV при условии, что:
б)	для захода на посадку опубликованы обычная неточная схема с применением RNAV и ОСА/Н APV/LNAV;
в)	пилотом применяется соответствующая поправка к высотомеру на низкую температуру ко всем опубликованным минимальным абсолютным/относительным высотам.
Оборудование баро – VNAV
Оборудование баро-VNAV может применяться при выполнении заходов на посадку и посадок двух различных классов:
а)	Заходы на посадку и посадки с вертикальным наведением
При использовании баро-VNAV наведение в боковой плоскости основывается на навигационных спецификациях RNP APCH и RNP AR APCH. Такие схемы публикуются с указанием абсолютной/относительной высоты принятия решения (DA/H). Их не следует путать с классическими схемами неточного захода на посадку (NPA), в которых устанавливается минимальная абсолютная/относительная высота снижения (MDA/H).
Заходы на посадку и посадки с вертикальным наведением обладают значительным преимуществом по сравнению с операциями, используемыми совместно со схемой неточного захода на посадку, т.к. они основываются на специальных критериях построения схем, не требуя перекрестной проверки ограничений в схеме неточного захода на посадку (например, пролет контрольных точек ступенчатого снижения).
Примечание. На основе этих критериев также решаются проблемы:
 потери высоты после начала ухода на второй круг, когда разрешается использовать DA вместо MDA и этим стандартизируется техника пилотирования при выполнении заходов на посадку с вертикальным наведением и точных заходов на посадку;
 пролета препятствий на этапе захода на посадку и посадки, с учетом температурных ограничений до DA, в результате чего обеспечивается более надежная защита от препятствий по сравнению со схемой неточного захода на посадку.
б)	Неточные заходы на посадку и посадки
В этом случае использовать систему баро-VNAV не требуется, но она может применяться в качестве средства вспомогательного содействия методу CDFA (8.9.7.11(а)). Это означает, что консультативное наведение VNAV используется совместно со схемой неточного захода на посадку.
8.9.7.10.	Неточный заход на посадку (NPA) с использованием 2 NDB (ОСП), NDB c DME, VOR, 
                                        LOC, LOC c DME, VOR c DME, SRA, RNAV (LNAV)
(1) Предназначен для выполнения двухмерных (2D) заходов на посадку по приборам типа A
Примечание. При неточном заходе на посадку по системе ОСП, ОПРС (NDB, VOR) при отсутствии дальномерной системы маяки системы ОСП (NDB) располагаются, как правило, в створе ВПП (но не более ± 3º от МПУ залегания ВПП посадки) по курсу захода перед ВПП. Отдельно стоящий маяк ОПРС (VOR) может использоваться при расположении его как в створе ВПП, так и не в створе ВПП (но не более ± 10º от МПУ залегания ВПП посадки). 
Отдельно стоящий маяк (ОПРС, VOR) может быть использован для захода на посадку при его расположении от 10 км. перед ВПП до 1,5 км. за ВПП. 
В каждом из перечисленных вариантов минимум для захода на посадку по приборам устанавливается в зависимости от расположения маяков (ОСП, ОПРС, NDB, VOR) относительно ВПП и рассчитывается в соответствие с Методикой определения эксплуатационных минимумов аэродромов для взлета и посадки ВС, утвержденной для авиакомпании.
(2) Полеты по схемам неточного захода на посадку могут выполняться с использованием метода захода на посадку с непрерывным снижением на конечном участке (CDFA). Операции по методу CDFA с консультативным наведением при VNAV и вычислением параметров бортовым оборудованием считаются трехмерными (3D) заходами на посадку по приборам с использованием высоты DA (H).
Операции по методу CDFA с вычислением параметров на основе неавтоматизированного расчета требуемой вертикальной скорости снижения считаются двухмерными (2D) заходами на посадку по приборам с использованием высоты МDA (H).
(3) Выход на траекторию захода на посадку и её выдерживание в боковой плоскости осуществляется по информации системы посадки, выдаваемой на пилотажные приборы пилотов, или по командам диспетчера зоны посадки при заходе по локатору.
(4) Контроль пути по дальности производится с использованием радионавигационных средств (DME, GPS), счисление пути - по времени пролёта FAF и по информации органа ОВД.

8.9.7.11.	Способы управления траекторией в вертикальной плоскости при неточных заходах:
а)	Заход на посадку с непрерывным снижением на конечном участке при неточном заходе на посадку Continuous Descent final approach (CDFA)
Заход на посадку с непрерывным снижением на конечном участке при неточном заходе на посадку (CDFA) применяется, если имеется опубликованная на схеме точка FAF или имеется возможность определить точку начала снижения на конечном этапе захода на посадку (ТВГ) или расстояния до торца ВПП с помощью FMS, RNAV, DME, SRA.
Данный способ предусматривает непрерывное стабилизированное снижение в полете на конечном этапе захода на посадку, выполняемое с наведением VNAV и вычислением параметров бортовым оборудованием (для 3D заходов) или на основе неавтоматизированного расчета требуемой вертикальной скорости снижения (для 2D заходов), без промежуточных горизонтальных участков. Вертикальная скорость снижения выбирается, корректируется для обеспечения непрерывного снижения до точки, расположенной на высоте примерно 15м (50фут) над посадочным порогом ВПП, или до точки, где для данного типа ВС должен начинаться маневр выравнивания перед посадкой. Снижение рассчитывается и осуществляется таким образом, чтобы обеспечить пролет на минимальной абсолютной высоте или выше её любых контрольных точек ступенчатого снижения.
УХОД НА ВТОРОЙ КРУГ должен начинаться на высоте, предотвращающей снижение ниже МDA (H), но не менее чем за 15м (50Ф) до достижения МDA(H). При раннем уходе на второй круг летный экипаж не должен набирать высоту больше, чем высота контрольной точки конечного этапа захода на посадку до достижения точки MAPt, чтобы не создавать помеха для остальных участников воздушного движения.
Ни в какой момент времени при полете на конечном этапе захода на посадку с использованием метода CDFA ВС не должно выполнять горизонтального полета.
Любые развороты при уходе на второй круг не начинаются до тех пор, пока ВС не достигнет МАРt. Если ВС достигает МАРt раньше, чем DA(H) (МDA(H)) в МАРt должен быть начат уход на второй круг.
                FAF

       	                                                                                                            MAP

                                                  VDA 3,0º


              Высота начала ухода на второй круг             при 2D ЗАХОДАХ
                         
            
 
              MDA (H) или DA (H) (см.п.п. 8.9.7.10.(3))

                                                                                                                         15м (50 фут).
                         Рис.1
б)	Снижение с постоянным углом
Второй способ рассчитан обеспечить постоянный угол снижения от конечной контрольной точки захода на посадку (FAF) или оптимальной точки на схемах без FAF до опорной точки над порогом ВПП, расположенной на высоте 15м (50 фут). При подходе ВС к MDA(H) принимается решение о продолжении снижения с постоянным углом, либо выполнять выравнивание в горизонтальный полет на или выше MDA(H), в зависимости от визуальных условий.
Если визуальные условия являются адекватными, ВС продолжает снижение до ВПП без какого-либо промежуточного выравнивания в полете.
Если визуальные условия являются неадекватными для продолжения снижения, ВС выполняет выравнивание в полете на или выше MDA(H) и продолжает полет по линии пути приближения до тех пор, пока не окажется в визуальных условиях, достаточных для снижения ниже MDA (H) до ВПП, или пока не выполнит уход на второй круг по достижении опубликованной точки ухода на второй круг.
   FAF                                                                VDP	                             MAP

	


                                            VDA 3,00




MDA (H)            
                
                Высота начала выода в ГП


                                                                                                    15м (50 фут)
                                           Рис.2
в)	Ступенчатое снижение
Предусматривает снижение с повышенной вертикальной скоростью до минимальных высот контрольных точек ступенчатого снижения. Как правило, указывается только одна контрольная точка ступенчатого снижения.
В	схеме для VOR/DME может быть установлено несколько контрольных точек DME, каждая с соответствующей минимальной абсолютной высотой пролета.
Пилот начинает снижение после стабилизации на линии пути, выдерживая самолет на опубликованных соотношениях "расстояние по DME/высота" или выше.
Данный способ является приемлемым до тех пор, пока получаемый градиент снижения остается менее 15% и уход на второй круг начинается по достижении МАРt или до МАРt. Этот способ требует уделять особое внимание контролю абсолютной высоты вследствие высоких вертикальных скоростей снижения на участке до достижения MDA(H), а также на последующем участке вследствие повышенного времени полета в зоне препятствий на минимальной абсолютной высоте снижения.

            FAF


высота пролета контрольной точки                                                           MAP

                                                             VDA 3,0º



MDA (H)

                                   высота начала вывода в ГП

	15м (50 фут)
  
Рис.3
В установленной (расчетной) точке входа в глиссаду самолёт переводится на снижение с расчётной вертикальной скоростью VУ.
Eсли к моменту достижения заданной высоты контрольная точка не пройдена, то необходимо установить режим работы двигателей, который был установлен перед началом снижения, и следовать в горизонтальном полёте до пролета контрольной точки.
Вывод ВС в горизонтальный полет начинается за 20 - 30м до достижения высоты пролета контрольной точки.
Если в контрольной точке высота больше расчётной, то производится увеличение VУ снижения на 0,5 ÷1 м/с.
8.9.7.12.	При достижении МDA(H) при неточном заходе на посадку:
если установлен необходимый визуальный контакт с ВПП и положение ВС в пространстве обеспечивает безопасность посадки, снижение для посадки может быть продолжено, при этом не допускается снижения к ВПП с вертикальной скоростью более 5 м/с;
а)	если визуальный контакт с наземными ориентирами не установлен или визуальный контакт недостаточен для принятия решения на посадку, ВС переводится в горизонтальный полет и следует на высоте не ниже МDA(H) в направлении на точку ухода на второй круг(MAPt) до установления необходимого контакта с ВПП;
б)	если до пролета точки VDP необходимый контакт с ВПП не установлен (см. п. 8.9.7.20) или выполнение посадки небезопасно, командир ВС обязан выполнить уход на второй круг с набором высоты в направлении точки MAPt. Боковое маневрирование по схеме ухода разрешается только после пролета MAPt.
На аэродромах, оснащенных системой визуальной индикации глиссады для выдерживания установленной глиссады снижения к ВПП используются показания огней этой системы.
При снижении ПКУ сравнивает установленные (расчетные) и фактические высоты пролета контрольных точек (ОПРС, ДПРМ, БПРМ, маркеров, удалений до ВПП), определяет уклонения ВС от посадочного курса и сообщает о них экипажу.
При пролете контрольных точек схемы захода экипаж должен производить сверку высоты полета по барометрическим высотомерам, с показаниями радиовысотомера (учитывая превышения рельефа местности).
8.9.7.13.	Методика выполнения неточного захода на посадку с использованием VOR/DME
Выполнение захода на посадку по VOR/DME имеет ряд особенностей и отличий, как в порядке использования радионавигационного оборудования, так и построения самой схемы захода.
При выполнении заходов по маякам (VOR/DME) необходимо руководствоваться следующим:
а)	Требования ICAO к размещению маяка VOR и установление линии пути конечного этапа захода на посадку:
	при отсутствии фиксированной точки конечного этапа захода на посадку (ТВГ) VOR должен быть расположен не далее 1,9 км от ближайшей точки ВПП;
	при наличии фиксированной точки конечного этапа захода на посадку (ТВГ) VOR может располагаться не далее 37 км от аэродрома;
	при расположении VOR в стороне от осевой линии ВПП линия пути конечного этапа захода на посадку устанавливается под углом к продолжению осевой линии ВПП, которая может устанавливаться с пересечением и без пересечения осевой линии ВПП перед порогом. Но при этом должны быть соблюдены условия, указанные на рисунках 1 и 2.





 
Рис. 1. Для схем с пересечением осевой линии ВПП
 
Рис. 2. Для схем без пересечения осевой линии ВПП перед порогом.
б)	Особенности захода на посадку при использовании наведения по дуге DME (DME Arc).
Наведение по дуге DME осуществляется на начальном участке захода на посадку. Параметры схем маневрирования указываются на схемах STAR или APPROACH CHART. Маневрирование по дугам DME используется для захода на посадку как по системе VOR/DME, так и по другим системам (ILS и т.д.).
Процедура захода по дуге DME включает в себя следующие этапы:
	подход по заданному значению радиала/пеленга до фиксированного значения дальности по DME;
	разворот на 90 градусов для выхода на дугу заданного радиала;
	следование по дуге до указанного радиала для последующего разворота на посадочную прямую.
Таблица значений кренов / ∆ R в зависимости от W и радиуса дуги DME Arc
                                                                                                                                                 Таблица А8.9-Т5
А RC (NM)	W КМ/Ч (KN)
А RC (КМ)	222(120)	259(140)	296(160)	333(180)	370(200)	407(220)	444(240)
7	13	2/ 4º	2º / 5º	3º / 6º	4º / 8º	5º / 10º	6º / 12º	7º / 14º
10	19	1º / 3º	2º / 3º	2º / 4º	3º / 6º	3º / 7º	4º / 8º	5º / 10º
13	24	1º / 2º	1º / 3º	2º / 4º	2º / 4º	3º / 6º	3º / 7º	4º / 8º
16	30	- / 2º	1º / 2º	1º / 3º	2º / 4ºº	2º / 4º	2º / 5º	3º / 6º
19	35	- / 1º	1º / 2º	1º / 2º	1º / 3º	2º / 4º	2º / 5º	3º 5º
22	41	- / 1º	1º / 2º	1º / 2º	1º / 3º	2º / 3º	2º / 4º	2º / 5º

  
Рис. 3.	При полете по дуге заданного радиуса ВС следует с креном, который вычисляется на НЛ-10м или по табл.№ 1 в зависимости от заданного радиуса и текущего значения путевой скорости.
Наглядным источником информации для выдерживания постоянного радиуса дуги яв- ляются показания ИКУ. При выполнении «правильной» дуги DME Arc в каждый текущий момент времени значение магнитного курса должно быть больше значения радиала на 90 градусов для правой орбиты и на 90 градусов меньше для левой. 
При использовании навигационных приборов с индикацией КУР при полете по дуге с правым креном значение КУР по маяку VOR (NDB) должно быть 90°, при полете с левым креном - 270°.
Совместное использование показаний DME и значений радиалов (КУР) позволяет траекторию ВС при наличии ветра приблизить к дуге постоянного радиуса.
Величину AR (Рис.3) можно взять из Таблицы А8.9-Т5, исходя из конкретных условий полета 
На ВС, оборудованных FMS, обеспечивающими ввод координат пунктов маршрута, и оснащенными приемниками спутниковой навигации, имеются возможности контроля выполнения маневра по дуге DME Arc. Тем не менее основным источником информации являются значения дальности DME и радиала VOR.
На ВС, оборудованных FMS, целесообразно при подготовке к полету разбить дугу DME на несколько промежуточных участков, в зависимости от величины радиуса дуги и величины воздушной скорости, чтобы обеспечить участки полета по 30-40 сек и произвести программирование этих точек. Полет выполнять по запрограммированным участкам.
При использовании приемников спутниковой навигации при подготовке к полету аналогичным образом рассчитываются и вводятся в маршрут полета промежуточные точки дуги DME. При выполнении захода по дуге DME осуществляется последовательный выход во все запрограммированные точки с использованием параметровнаведения пеленг (bеагing) фактический путевой угол (track) с обязательным контролем выполнения маневра по текущим значениям дальности DME, радиала VOR и магнитного курса.
На картах захода на посадку начало разворота для выхода на посадочную прямую, как правило, указывается фиксированным значением радиала, который в отдельных случаях может указываться по категориям ВС.
в)	Общие правила порядка использования бортового навигационного оборудования при заходе на посадку по системам VOR, VOR/DME, NDB.
Настройка аппаратуры должна быть продумана так, чтобы перед четвертым разворотом на первом комплекте оборудования наведения был настроен ILS (VOR, NDB), т.е. то средство, по которому будет осуществляться заход на конечном участке захода. Затем настройка «разворачивается» в обратную сторону по STAR.
Порядок настройки, последовательность переключений навигационных средств и приборов должны быть продуманы на целесообразность, надежность и наглядность отображения траектории ВС на схеме STAR(SID).

\paragraph{} \label{sec:visual}	Выполнение визуального захода на посадку
