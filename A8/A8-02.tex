\section{Предполетная подготовка}

\subsection{Организация предполетной подготовки}


Предполетную подготовку экипажа организует и проводит командир ВС перед каждым полетом с учетом конкретной аэронавигационной обстановки и метеоусловий.

Началом предполетной подготовки является время, установленное технологическим графиком работы экипажа согласно времени отправления рейса (приложение А 8.6), указанного в суточном плане полётов Авиакомпании или время прохождения предполетного медицинского осмотра в аэропорту вылета (что наступит раньше).

Во всех случаях должно быть достаточно времени для качественного выполнения всех процедур, предусмотренных технологическим графиком.

Перед началом полётов, но не ранее чем за 2 часа до вылета, все члены экипажа обязаны пройти предполётный медицинский осмотр у врача (фельдшера). При задержке вылета на 6 часов и более предполётный медосмотр проводится повторно. При нахождении на дежурстве, в резерве экипажи проходят предполётный медицинский осмотр перед заступлением на дежурство, в резерв и перед вылетом, если до вылета прошло 6 часов и более.

При выполнении международных полетов с аэродрома иностранного государства, а также при выполнении авиационных работ и других полетов с аэродромов, где отсутствует медицинский работник, имеющий право проводить медицинский осмотр, а также с посадочных площадок, предполетный медицинский осмотр не проводится, решение о допуске членов экипажа ВС к полетам принимает командир ВС с записью в задании на полёт.

В случае неявки одного из членов экипажа или отстранении его по каким-либо причинам от вылета, КВС, а в случае его отсутствия второй пилот, бортинженер (бортмеханик), обязан сообщить диспетчеру ОКВР о необходимости вызова резервного экипажа не позднее, чем за 50 минут до времени вылета при полетах на внутренних линиях, и за 1 час 20 минут - при полетах на международных линиях.

Перед полетом или серии полетов КВС должен заполнить чек-лист готовности к полету (приложение А 8.4), удостоверяющий тот факт, что он удовлетворен результатами подготовки.
Подписанный чек-лист, экземпляр рабочего плана передаются в представительство Авиакомпании.
При отсутствии в пункте вылета представителя Авиакомпании – представителю обслуживающей компании или аэропорта.

Запрос КВС на запуск двигателя (двигателей) на контролируемом аэродроме или запуск двигателя (двигателей) на неконтролируемом аэродроме свидетельствует о принятии решения на начало полета.


\subsection{Учет ограничений летно-технических характеристик}


\paragraph{} Воздушное судно эксплуатируется в соответствии с положениями сертификата летной годности (удостоверения о годности к полетам), РЛЭ и нормами, применяемыми для установления эксплуатационных ограничений летно-технических характеристик.	

Самолеты с одним двигателем, как правило, эксплуатируются только в таких условиях погоды и освещенности, на таких маршрутах и с таким отклонением от них, которые в случае отказа двигателя позволят безопасно совершить вынужденную посадку. 

Разрешается начинать полет только в том случае, когда информация о летно-технических характеристиках, содержащаяся в РЛЭ, в разделе 3 «Подготовка к полету», указывает на то, что в предстоящем полете могут быть выполнены требования, содержащиеся в пункте \hyperref[sec:8222]{8.2.2.2}

При выполнении указанных требований следует учитывать все факторы, которые влияют на летно-технические характеристики воздушного судна (масса, барометрическая высота, соответствующая превышению аэродрома, температура, уклон ВПП и состояние ВПП, т. е. наличие слякоти, воды и (или) льда). 

\paragraph{} \label{sec:8222}Масса воздушного судна в начале взлета не превышает максимальную взлетную массу, указанную в РЛЭ для барометрической высоты, соответствующей превышению аэродрома, а также для любых других местных атмосферных условий, если они используются в качестве параметров для определения максимальной взлетной массы. 

Расчетная масса воздушного судна к расчетному времени приземления на аэродроме намеченной посадки и на любом запасном аэродроме пункта назначения не превышает максимальную посадочную массу, указанную в РЛЭ для барометрической высоты, соответствующей превышению этих аэродромов, а также для других местных атмосферных условий, если они используются в качестве параметров для определения максимально допустимой массы при посадке. 

Летно-технические характеристики самолета позволяют в случае отказа критического двигателя в любой точке взлета либо прекратить взлет и остановиться в пределах располагаемой дистанции прерванного взлета, либо продолжать взлет и пролететь все препятствия вдоль траектории полета с достаточным запасом. 
 
Летно-технические характеристики самолета позволяют в случае выхода из строя критического двигателя в любой точке на маршруте или запланированных на случай отклонения от него запасных маршрутах продолжать полет до аэродрома, не снижаясь ни в каком месте до высоты ниже минимально разрешенной. 

Летно-технические характеристики самолета позволяют приземлиться на аэродроме намеченной посадки или любом запасном аэродроме после пролета всех препятствий вдоль траектории захода на посадку с минимальным для обеспечения безопасности запасом высоты и с учетом достижения достаточно низкой скорости в пределах располагаемой посадочной дистанции. При этом КВС учитываются предполагаемые различия в технике пилотирования при выполнении захода на посадку и посадки, если это не было учтено при установлении летно-технических характеристик.

РЛЭ эксплуатируемых ВС позволяют определить вышеперечисленные параметры. При этом исходными данными для расчета являются:
\begin{itemize} 
    \item располагаемая длина разбега РДР (Take-off Run Available (TORA));
    \item располагаемая длина прерванного взлета РДПВ (Accelerate Stop Distance Available (ASDA));
    \item располагаемая длина продолженного взлета РДВ (Take-off Distance Available (TODA));
    \item располагаемая посадочная дистанция (Landing Distance Available (LDA));
    \item продольный уклон ВПП;
    \item температура наружного воздуха на аэродроме;
    \item атмосферное давление на аэродроме;
    \item скорость продольной и боковой составляющей ветра;
    \item состояние поверхности ВПП (коэффициент сцепления, вид и толщина слоя осадков);
    \item плоскость ограничения препятствий.
\end{itemize}
При определении располагаемой длины ВПП учитывается возможное ее уменьшение в связи с необходимостью выведения самолета на осевую линию перед взлетом.


\paragraph{} Для определения взлетно-посадочных характеристик с учетом фактических условий, включая состояние ВПП, для ВС иностранного производства, в Авиакомпании применяется программное обеспечение для CL-300. (Инструкции пользователей указанного программного обеспечения публикуются в ИС и в Приложении А 8.12.), для ВС отечественного производства применяются таблицы, графики и номограммы, содержащиеся в РЛЭ для каждого конкретного типа ВС.
При определении градиентов набора высоты должно быть выдержано условие, что минимальный градиент на взлете не менее 3.3\%. Для процедуры ухода на второй круг не менее 2.5\%, если другие значения не оговорены на схемах вылета и захода на посадку. 

\subsection{Действия экипажа в процессе предполетной подготовки}


\textbf{Командир ВС в процессе предполетной подготовки обязан:}


\begin{itemize}
    \item получить задание на полет;
    \item убедиться в соответствии всех записей в задании на полет планируемому полету;
    \item проверить документы членов летного и кабинного экипажа; 
    \item изучить информацию по безопасности полетов, информацию NOTAM, NOTAM Авиакомпании о состоянии аэродромов вылета, назначения, запасных, аэронавигационную обстановку на аэродромах и по трассе; 
    \item получить информацию о технической готовности воздушного судна, наличии и размещении аварийно-спасательного оборудования; 
    \item удостовериться в наличии на борту документов, указанные в подразделе 8.6.4. настоящей главы;
    \item провести анализ записей в бортовом техническом журнале ВС (TLB) с целью определения статуса летной годности ВС, летно-технических характеристик и центровочных данных, возможности выполнения полета при наличии отложенных неисправностей в соответствии с требованиями ПМО/MEL;
    \item \chgdPar{17/07/22}{3}{При любых сомнениях в статусе ВС, подозрениях на техническую неисправность направить соответ-ствующую информацию, при необходимости приложить фото/провести техническую консультацию по телефонам ОКВР через мобильное приложение WhatsApp / Telegram;}
    \item анализ NOTAM аэродромов назначения и запасных, об аэронавигационном обеспечении на аэродромах и по трассе; 
    \item получить информацию о предполагаемой коммерческой загрузке;
    \item изучить метеорологическую обстановку на аэродроме вылета, по маршруту полета, аэродроме назначения и запасных аэродромах;
    \item проверить правильность штурманского расчета и других данных для выполнения полета и уточнить необходимую заправку топливом;
    \item определить конкретные действия экипажа в случае возникновения аварийной обстановки, в том числе при необходимости экстренной посадки после взлета, в зависимости от характера местности, наличия площадок, времени суток и метеоусловий;
    \item принять решение на вылет или отказаться от вылета (перенести вылет) с указанием причин такого решения;
    \item представить в службу движения заполненный план полета (FPL) не позднее, чем за тридцать минут до вылета (при выполнении полетов по МВЛ при наличии брифинга на полет);
    \item провести анализ и передать представителю авиакомпании подписанный рабочий план полета;
    \item проверить наличие необходимого количества топлива, а также центровку и взлетную массу воздушного судна;
    \item готовить ВС и его системы к вылету согласно РЛЭ типа ВС (AFM, FCOM);
    \item провести инструктаж лица, занимающего дополнительное кресло в кабине экипажа, по технике безопасности и использованию аварийно-спасательного оборудования, расположенного около его кресла, ознакомиться с правилами поведения и взаимодействия с экипажем;
    \item принять доклады от каждого члена экипажа о проведенном осмотре и готовности воздушного судна к вылету и выполнить работы, предусмотренные РЛЭ перед вылетом;
    \item проверить срок обновления базы данных при полетах в районы, где необходимо использование  
установленного оборудования; 
\item перед полетом в целях выполнения авиационных работ дополнительно убедиться в том, что на борту установлены приборы и оборудование, необходимые для ожидаемых условий полета. Убедиться в работоспособности указанных приборов и оборудования в соответствии с требованиями РЛЭ.
\end{itemize}


\textbf{Второй пилот в процессе предполетной подготовки обязан:} 
\begin{itemize}
    \item провести анализ NOTAM, информацию о состоянии аэродромов вылета, назначения и запасных, об аэронавигационном обеспечении на аэродромах и по трассе; 
    \item получить информацию о предполагаемой коммерческой загрузке; 
    \item участвовать в изучении метеорологической и аэронавигационной обстановки, выполнить штурманский расчет полета по маршруту, проверить наличие «Инструкции по безопасности» для всех посадочных мест (при полетах на самолете Ан-2); 
    \item лично рассчитать взлетно-посадочные характеристики и предельную коммерческую загрузку, составить центровочный график, когда в аэропорту отсутствует диспетчер по центровке; 
    \item проверить наличие документов аэронавигационной информации на электронных, бумажных носителях и убедиться в их актуальности (при отсутствии штурмана в экипаже);
    \item готовить ВС и его системы к вылету согласно РЛЭ типа ВС; 
    \item проверить размещение груза, багажа и почты по багажным помещениям согласно документам СОПП и докладу бортоператора/бортпроводника, ответственного за загрузку, убедиться в заполнении всех граф сводной сопроводительной ведомости; 
    \item доложить командиру ВС о проведенном осмотре и подготовке ВС и своей готовности к вылету.
\end{itemize}

\textbf{Штурман в процессе предполетной подготовки обязан:}
\begin{itemize}  
    \item	провести анализ NOTAM, информацию о состоянии аэродромов вылета, назначения и запасных, об аэронавигационном обеспечении на аэродромах и по трассе; 
    \item	участвовать в изучении метеорологической информации и представлять командиру ВС свои предложения;
    \item	выполнить навигационный расчет полета или проверить, если необходимо, план, подготовленный АСШР, определить минимальный запас топлива и доложить командиру ВС рассчитанное значение запаса топлива и, если необходимо, рубеж ухода на запасный аэродром;
    \item	составить план полета (FPL) и вручить его командиру ВС. Если полет выполняется по расписанию - проверить RPL на его соответствие плану полета и выписать из него необходимые в полете сведения (скорости, эшелоны, маршрут, время полета и т. п.);
    \item	получить сверенные с контрольными экземплярами документы АНИ;
    \item	готовить оборудование ВС к вылету в соответствии с РЛЭ типа ВС;
    \item	доложить командиру ВС о проведенном осмотре и подготовке ВС и своей готовности к вылету;
    \item	на ВС имеющих оборудование, где применяется электронная навигационная база данных, провести проверку ее на актуальность. 
\end{itemize}

\textbf{Бортмеханик в процессе предполетной подготовки обязан:}
\begin{itemize}
    \item	изучить информацию по безопасности полетов;
    \item	пройти таможенный, паспортный и карантинный контроль, когда это необходимо;
    \item	принять воздушное судно от инженерно-технической службы или от сменяемого экипажа;
    \item	проверить наличие и оформление судовой, технической документации и записей об устранении неисправностей в бортовом журнале;
    \item	выполнить работы, предусмотренные РЛЭ перед вылетом;
    \item	доложить командиру ВС о готовности к полету.
\end{itemize}

\textbf{Бортоператор в процессе предполетной подготовки обязан:}
\begin{itemize}
    \item	изучить информацию по БП;
    \item	уточнить данные о коммерческой загрузке и доложить о них командиру ВС;
    \item	продублировать заявку на получение бортового питания и бытового оборудования (при необходимости), получить медицинскую аптечку;
    \item	провести внешний осмотр ВС, проверить грузовые люки и грузовую кабину на предмет отсутствия повреждений и посторонних предметов, исправность замков и средств крепления груза;
    \item	проверить аварийно-спасательное и кислородное оборудование в грузовой кабине;
    \item	принять бытовое оборудование и бортовое питание согласно предварительной заявке;
    \item	контролировать процесс загрузки и разгрузки для предотвращения повреждения ВС;
    \item	проконтролировать правильность размещения и крепления груза согласно центровочному графику;
    \item	доложить командиру ВС об окончании загрузки, наличии и размещении опасного груза на борту, закрытии грузовых люков и своей готовности вылету.
\end{itemize}

\textbf{Бортпроводник в процессе предполетной подготовки обязан:}
\begin{itemize}
    \item	изучить информацию по безопасности полетов;
    \item	получить документацию, снаряжение и медицинскую аптечку;
    \item	доложить командиру ВС о готовности кабинного экипажа к выполнению рейса (если бригада более одного бортпроводника доклад делает бортпроводник-бригадир); 
    \item	проверить оборудование ВС согласно РЛЭ, проверить наличие «Инструкции по безопасности» для всех посадочных мест;
    \item	выполнить осмотр бортового аварийно-спасательного оборудования внутри ВС;
    \item	осмотреть багажники, проверить узлы и средства крепления груза;
    \item	доложить члену летного экипажа о приемке оборудования ВС; 
    \item	контролировать количество и размещение загрузки на ВС;
    \item	доложить б/проводнику - инструктору о проведенном осмотре и своей готовности к вылету;
    \item	доложить командиру ВС о готовности кабинного экипажа к вылету.
    \item Согласовать с командиром ВС:
    \begin{itemize}  
        \item размещение в салонах ВС больных, инвалидов, несопровождаемых детей, депортированных пассажиров и сотрудников безопасности; 
        \item размещение загрузки в багажных помещениях и размещение опасных грузов (при необходимости).
    \end{itemize}
\end{itemize}

Другие авиационные специалисты, включенные в задание на полет, в период подготовки воздушного судна обязаны выполнить операции, предусмотренные РЛЭ и технологией работы, в части их каса-ющейся.

Полет не начинается до тех пор, пока командир ВС не будет убежден, что:
\begin{itemize}
    \item 	воздушное судно технически готово к полету;
    \item 	бортовое оборудование исправно, за исключением тех неисправностей, которые включены в перечень допустимых отказов (ПДО), перечень минимального оборудования (ПМО, MEL) в соответствии с правилами его применения; 
    \item 	судовая документация, соответствующая целям полета, находится на борту в полном объеме;
    \item 	наземные РТС и оборудование, необходимые для планируемого полета, работоспособны и соответствуют цели полета;
    \item 	условия эксплуатации, указанные в РПП в отношении топлива, масла и кислорода, минимальных высот полета, минимумов аэродромов (вылета, назначения и запасных) могут быть выдержаны при выполнении полета;
    \item 	груз размещен и закреплен в соответствии с требованиями РЛЭ типа ВС;
    \item 	масса ВС не выходит за установленные ограничения;
    \item 	экипаж ознакомлен со срочной информацией по безопасности полетов;
    \item 	члены экипажа способны выполнить данный полет;
    \item 	надлежащим образом оформлено разрешение на вылет.
\end{itemize}

Каждый член экипажа несет персональную ответственность за своевременность, точность, полноту и качество выполнения возложенных на него обязанностей по подготовке к полету.

\subsection{Формы контроля качества и оформления факта и своевременности проведения предполетной подготовки:}
\begin{itemize}
    \item	прохождения медицинского контроля удостоверяется подписью и штампом врача здравпункта в задании на полёт;
    \item	прохождения метеорологической подготовки (если предусмотрено) – подписью и штампом дежурного синоптика в задании на полет или записью в бланке на АМСГ. 
    \item	посадки пассажиров, принятия груза, почты, багажа, бортового питания и т.п. – указанием времени завершения данных операций в бланках документации по этим процедурам.
\end{itemize}

В период предполетной подготовки командир ВС взаимодействует с представителем авиакомпании по вопросам обеспечения рейса и подготовки вылета службами аэропорта. При отсутствии представителя авиакомпании в аэропорту вылета вопросы обеспечения решает командир ВС.

В случае задержек рейса (переносе времени вылета) командир ВС сообщает лично или по установленным каналам связи в АДП аэропорта, информирует диспетчера ОКВР авиакомпании о принятом решении и согласовывает с ними дальнейший порядок выполнения полета.

\subsection{Предполетная подготовка перед полетом по перегонке и перед облетом ВС}


При выполнении предполетной подготовки перед полетом по перегонке ВС без пассажиров для обеспечения необходимого значения центровки необходимо определить вес балластного груза и проконтролировать его правильное размещение. В случае перегона ВС с неисправностями, ухудшающими аэродинамические, летные характеристики ВС (с отказавшем двигателем, выпущенными шасси и т.п.), необходимо определить ограничения по максимальной взлетной массе, скорости на взлете и разобрать технологию взаимодействия при появлении дополнительного (как правило, критического) отказа на важнейших этапах полета.

Перед облетом воздушного судна или его систем необходимо уточнить возможности облета исходя из конкретных и прогнозируемых метеоусловий и установленных ограничений по целям облета, заказать работу (при необходимости) дополнительного наземного оборудования, рассчитать ограничения характеристик ВС при работающих и неработающих системах облета.

\subsection{Предполетная подготовка перед тренировочными или учебными полетами}


Перед тренировочными или учебными полетами необходимо детально изучить метеообстановку в районе выполнения тренировочных полетов с учетом их целей и ограничений по отрабатываемым элементам, согласовать с руководителем полетов план проведения тренировки с обязательным рассмотрением порядка взаимодействия с диспетчерами ОВД при отработке элементов тренировки, ограничениях, отличных от нормальных полетов.

\subsection{Проверяющий в составе экипажа}

При включении проверяющего в состав экипажа проведение предполетной подготовки для командира ВС и членов экипажа не должно затрудняться этим фактором. Задача проверяющего (инструктора) на данном этапе заключается в создании нормальных условий подготовки экипажа. Не допускается проводить проверку знаний у экипажа по вопросам, не относящимся к данному полету или на этапах, отвлекающих или мешающих проведению качественной подготовки, принятию правильного решения.

При включении проверяющего в задание на полет командир ВС является ответственным за качество проведения предполетной подготовки. Проверяющий не должен вмешиваться в процесс анализа и принятия решения на вылет командиром ВС, а при угрозе безопасности полету принимать меры в пределах своей компетенции и полномочий. 

\subsection{Контролирующий в составе экипажа}

При включении контролирующего в задание на полет КВС является ответственным за качество проведения предполетной подготовки. Контролирующий не должен вмешиваться в процесс принятия решения на вылет командиром ВС, а при угрозе безопасности полету принимать меры в пределах своей компетенции и полномочий.

\subsection{Внешний осмотр ВС}

Члены летного экипажа (или назначенный квалифицированный член летного экипажа) обязаны выполнить внешний осмотр ВС перед каждым вылетом (обход экипажем) в соответствии с РЛЭ ВС (AFM, FCOM). Инспекция должна включать следующие позиции, имеющие решающее значение для безопасности полетов:
\begin{itemize}
    \item приемники воздушного давления исправны, свободны от чехлов, заглушек; 
    \item органы управления разблокированы; 
    \item отсутствует наледь, снег или лёд на важнейших поверхностях ВС; 
    \item отсутствуют какие-либо повреждения (структурная целостность) ВС; 
    \item отсутствие подтёков топлива, масла;
    \item состояние шасси.
\end{itemize}

\subsection{Внутренний осмотр ВС}

Летный экипаж (или назначенный квалифицированный член обслуживающего персонала на борту) выполняет осмотр внутреннего аварийного оборудования ВС, включая оборудование, имеющее решающее значение для безопасности полетов: кислородное, медицинское и аварийное оборудование.

Данная проверка проводится при передаче ВС от другого экипажа или от ИАС. Результат проверки фиксируется в бортовом техническом журнале.

Обеспечение выполнения всех необходимых операций и полноты комплекта полетной документации на предполетной подготовке достигается контролем по чек-листу готовности к полету (Приложение А-8.4). 

Члены экипажа ВС при предполетной подготовке несут ответственность за исполнение своих должностных обязанностей и выполнение технологии работы. 

По завершении предполетной подготовки командир ВС, принимает решение на вылет, передаёт информацию о принятом решении в уполномоченное подразделение ОВД.

Запрос КВС разрешения на запуск двигателя (й) на контролируемом аэродроме или запуск двигателя (й) на неконтролируемом аэродроме свидетельствует о принятии решения на начало полета.

\subsection{Взаимодействие с наземными службами}

Экипажи ВС Авиакомпании при проведении предполетной подготовки взаимодействуют с наземными службами аэропорта:
\begin{itemize}
    \item стартовый медицинский пункт; 
    \item уполномоченное подразделение органа ОВД (АДП); 
    \item штурманская служба аэропорта; 
    \item авиационная метеорологическая служба; 
    \item отдел полетно-диспетчерского обеспечения САОП ЛД; 
    \item отдел контроля выполнения рейсов Авиакомпании; 
    \item служба организации перевозок.
\end{itemize}

Взаимодействие с указанными службами в дополнительно оговоренных случаях и аэропортах может осуществляться через уполномоченного представителя Авиакомпании.

При выполнении чартерных рейсов экипаж ВС обеспечивается дополнительной информацией, содержащей указания об особенностях обеспечения и выполнения рейса («брифинг»). Брифинг предоставляется представителем Авиакомпании в пакете документов на предполетной подготовке. При возникновении вопросов по содержанию информации в указанном документе КВС обязан получить разъяснения у специалиста, предоставившего информацию, или по телефону у специалиста отдела контроля выполнения рейсов.

Процедуры взаимодействия экипажа с наземными службами определены в Инструкции по взаимодействию и технологии работы членов экипажа, а также на основании графиков и технологий подготовки ВС в аэропорту обслуживания.

\subsection{Действия экипажа при сбойных ситуациях}

Под сбойными ситуациями подразумеваются случаи посадки на запасном аэродроме, задержка или отмена рейса, замена члена экипажа в день вылета, неисправность воздушного судна, стихийные бедствия и катастрофы.

В этих случаях командир ВС обязан:
\begin{itemize}
    \item как представитель эксплуатанта, установить контакт с обслуживающей фирмой (со службами аэропорта или места оперативного базирования) для своевременного и качественного обеспечения полета;
    \item доложить диспетчеру ОКВР авиакомпании о характере сбойной ситуации;
    \item поддерживать контакт с диспетчером ОКВР, используя все доступные каналы связи, для запроса и получения информации, необходимой для безопасного выполнения полёта, а также при существенных отклонениях от рабочего плана полёта, включая уход на запасной аэродром;
    \item организовать работу и отдых экипажа в соответствие с требованиями руководящих документов и полученными указаниями руководства авиакомпании.
\end{itemize}

\textbf{Действия экипажа в отношении пассажиров при непреднамеренной задержке вылета ВС.}

Предельно возможное время ожидания пассажирами вылета на борту ВС в случае задержки рейса определяет КВС с учетом рабочего времени экипажа ВС.

Во всех случаях, когда вылет ВС предполагается позже времени, указанного в билетах пассажиров, необходимо принести извинения от имени Авиакомпании.

В случае задержки вылета при нахождении пассажиров на борту бортпроводники находятся в салоне, отвечают на вопросы пассажиров, оказывают необходимую помощь.

Во время задержки рейса, когда пассажиры находятся на борту, бортпроводники обязаны:
\begin{itemize}
    \item уточнить у КВС причину и предполагаемое время задержки;
    \item контролировать салон и реакцию пассажиров, при необходимости давая согласованные с КВС сведения о задержке вылета;
    \item давать дополнительные разъяснения пассажирам;
    \item при желании пассажиров отказаться от перевозки и покинуть ВС в индивидуальном порядке информировать пассажиров о возможных затруднениях и/или невозможности осуществить в последующем процедуру снятия пассажиров и их багажа с задержанного рейса по причине технологических особенностей обслуживания рейсов в аэропорту;
    \item держать связь с летным экипажем для оперативного получения информации о продолжении полета и своевременного информирования пассажиров;
    \item проводить индивидуальное обслуживание, своевременно реагировать на вызовы из пассажирского салона;
    \item уделять особое внимание пассажирам с детьми, пассажиров с ограниченными физическими возможностями и пожилым людям (индивидуально предложить воду, плед, указать место расположения туалетных комнат);
    \item по согласованию с КВС в случае задержки вылета более двух часов обслужить пассажиров прохладительными напитками;
    \item по согласованию с КВС в случае задержки вылета более четырех часов (с учетом сроков годности бортового питания) обслужить пассажиров выданным на рейс рационом питания на земле, в случае отсутствия рациона питания на рейсе произвести заказ питания для обслуживания пассажиров;
    \item по согласованию с КВС дать в обслуживающую компанию информацию о необходимости дозагрузки/ замены бортового питания и напитков, предупредить представителя Авиакомпании (при наличии).
\end{itemize}

Во время задержки рейса, когда пассажиры находятся на борту, летный экипаж обязан:
\begin{itemize}
    \item информировать пассажиров о причине задержки рейса с указанием предполагаемого времени вылета, а также вносить корректировки при получении новой информации;
    \item информировать бортпроводников обо всех изменениях предполагаемого времени вылета.
\end{itemize}

В случае задержки высадки пассажиров по прибытию напитки и питание могут быть предоставлены из объема, имеющегося на борту на обратный рейс (с учетом сроков годности бортового питания), при этом время предоставления питания должно соответствовать установленным требованиям в полете (при продолжительности полета свыше трех часов и далее каждые четыре часа - в дневное время и каждые шесть часов - в ночное время).

При принятии решения обслужить пассажиров напитками и питанием необходимо учитывать возможность загрузки бортпитания в данном аэропорту (по информации, полученной от КВС) и оперативно дать заявку на загрузку бортпитания, предупредить представителя Авиакомпании (при наличии).


\subsection{Предстартовая подготовка}
\begin{enumerate}
    \item Бортпроводник/бортоператор:
    \begin{enumerate}
        \item После окончания посадки пассажиров, загрузки грузов:
        \begin{enumerate}
            \item докладывает командиру ВС:
            \begin{itemize}
                \item количество пассажиров на борту и их размещение в салоне ВС, количество мест груза;
                \item коммерческую загрузку и ее размещение в багажных / грузовых отсеках;
                \item наличие служебной корреспонденции, опасных грузов, оружия и/или боеприпасов;
                \item наличие депортированных, больных пассажиров, инвалидов и сопровождающих их лиц;
                \item наличие и количество находящихся на борту вооружённых лиц, имеющих разрешение на перевозку оружия, их расположение в салоне самолёта;
                \item наличие несопровождаемых детей;
            \end{itemize}
            \item согласовывает:
            \begin{itemize}
                \item время и очередность приема пищи членами летного экипажа;
                \item порядок входа в кабину летного экипажа;
                \item условные сигналы о начале взлета, наборе критической высоты полета;
                \item момент доведения информации для пассажиров командиром ВС.
            \end{itemize}    
        \end{enumerate}
        Старший бортпроводник должен получить от командира ВС разрешение на закрытие дверей, если это предусмотрено технологией работы конкретного типа ВС.

        \item После закрытия дверей, до запуска двигателей, в соответствии с технологией работы бортпроводник (и) (а где его нет в составе экипажа – член экипажа по указанию КВС) информирует (ют) пассажиров об установленных авиакомпанией правилах, демонстрирует пассажирам правила применения аварийно-спасательных средств.
        \item До взлета старший бортпроводник (бортпроводник) (а где его нет в составе экипажа – член экипажа по указанию КВС) обязан убедиться в готовности пассажирского салона к взлету. 
        Готовность пассажирского салона к взлету означает, что:
        \begin{itemize}
            \item все пассажиры находятся на своих местах, ремни безопасности пристегнуты (дети на руках);
            \item буфетно-кухонное оборудование закреплено;
            \item в салонах все предметы закреплены;
            \item пути эвакуации свободны;
            \item шторки иллюминаторов открыты;
            \item освещение салонов притушено.
        \end{itemize}
        \item Перед взлетом, на установленном рубеже, старший бортпроводник (бортпроводник) /бортоператор должен:
        \begin{itemize}
            \item доложить командиру ВС о готовности пассажирской кабины к взлету;
            \item получить от командира ВС условный сигнал кабинному экипажу о необходимости занять свои места и пристегнуться привязными ремнями.
        \end{itemize}
    \end{enumerate}
    \item Летный экипаж. Предполетный инструктаж.
    \begin{enumerate}
        \item Каждый член летного экипажа на ВС перед вылетом выполняет предписанные ему процедуры согласно РЛЭ ВС (AFM, FCOM), стандартным эксплуатационным процедурам (SOP), технологиям работы.
        \item После доклада бортпроводника/ бортоператора и членов летного экипажа о готовности к полету командир ВС проводит предполетный инструктаж летного экипажа, в процессе которого сообщает:
        \begin{itemize}
            \item дату выполнения полета, назначенный номер рейса, маршрут полета;
            \item запасные аэродромы для взлета, по маршруту и для аэродрома назначения;
            \item эшелон (высоту) полета;
            \item запас топлива;
            \item погоду на аэродроме вылета, аэродроме назначения и запасных аэродромах; 
            \item ВПП для взлета, процедуры взлета;
            \item особенности руления; 
            \item имеющиеся предупреждения, ограничения в NOTAM;
            \item особенности полета при наличии отложенных неисправностей по ПМО/MEL/CDL;
            \item порядок пилотирования;
            \item порядок ведения радиосвязи;
            \item особенности на взлете (в том числе при низкой видимости, наличие или прогнозирование сдвига ветра, с кратковременной остановкой, с уменьшением шума и т.п.);
            \item взлетные данные (механизация крыла, характерные скорости на взлете); 
            \item порядок выхода (процедура – SID), установка режима работы двигателей, управление механизацией, высота перехода, рельеф местности;
            \item рельеф местности, безопасные высоты в районе аэродрома и по маршруту выхода из района аэродрома;
            \item запасные площадки на случай вынужденной посадки, особенности маневрирования при этом;
            \item при наличии в составе экипажа специалиста на дополнительном откидном сидении (обзёрвер, стажер, проверяющий, контролирующий) дополнительно проводится его инструктаж по технике безопасности и использованию аварийно-спасательного оборудования, расположенного около его кресла. Ознакомить его с правилами поведения и взаимодействия с членами экипажа.
            Объем информации определяется РЛЭ ВС (AFM, FCOM), технологией работы членов экипажа типа ВС(SOP).
        \end{itemize}
        \item Предстартовая подготовка заканчивается контролем готовности экипажа к запуску двигателей по карте контрольных проверок.
    \end{enumerate}
\end{enumerate}


\subsection{Предполетная подготовка экипажа перед взлетом в сторону больших водных поверхностей}

\begin{enumerate}
    \item В процессе подготовки к вылету ВС и экипажа, дополнительно ко всем необходимым операциям, требуемым технологией работы экипажа, проводятся:
    \begin{enumerate}
        \item В комнате предполетной подготовки. Командир ВС, второй пилот, штурман – изучают береговую черту в зоне взлета, ветровой режим и состояние водной поверхности на случай приводнения. Особое внимание уделяется подбору возможных мест вынужденной посадки (характер береговой черты, наличие островов с удобными берегами и т.д.).
        
        По прибытию командира ВС на самолет бортмеханик и бортпроводник (бортоператор) в своих докладах о готовности к полету сообщают о наличии и размещении аварийно-спасательных средств на ВС.
        \item На воздушном судне. Обеспечение и размещение индивидуальных и групповых спасательных средств возлагается на ИАС, обеспечивающую подготовку к вылету. Бортмеханик и бортпроводник (бортоператор) экипажа проверяют наличие и размещение индивидуальных и групповых спасательных средств, запаса продуктов питания и питьевой воды, аварийных радиостанций и сигнальных средств на борту ВС, а в случае необходимости требуют от ИАС устранения недостатков по подготовке ВС для полета над водной поверхностью.
        \item Перед посадкой пассажиров. Командир ВС путем устного опроса членов летного экипажа и бортпроводника проверяет их знания по действиям по аварийному расписанию на случай вынужденной посадки на воду и информирует об особенностях предстоящего взлета, обращая внимание всех, что аварийная ситуация на взлете скоротечна и требует, в случае ее возникновения, четких и своевременных действий от каждого члена экипажа. Также доводятся вероятные места приводнения, расстояние от аэродрома до береговой черты, возможное состояние водной поверхности и порядок эвакуации пассажиров и использования аварийных средств, исходя из конкретных условий данного полета.
        \item Предстартовая подготовка
        
        Бортпроводники: после закрытия дверей, до запуска двигателей в соответствие с технологией работы, демонстрирует пассажирам правила применения спасательных жилетов и других аварийно-спасательных средств.

        Летный экипаж: 
        КВС в процессе доведения предполетной информации дополнительно сообщает экипажу:
        \begin{itemize}
            \item возможные места аварийного приводнения;
            \item расстояние от ВПП (VOR/DME) до береговой черты;
            \item направление и скорость ветра для примерного определения курса приводнения.
        \end{itemize}

        Предстартовая подготовка заканчивается докладом каждого члена экипажа готовностью к полету.

        \textbf{Примечание.} Общее количество спасательных средств должно обеспечивать всех лиц, находящихся на ВС. Надувные плоты должны находиться на борту ВС, если это требуется по условиям выполнения полета. Ответственность за сохранность аварийно-спасательных жилетов возлагается на бортпроводника №1. 
    \end{enumerate}
    \item Дополнительное аварийно-спасательное оборудование при полетах над водной                                         поверхностью
    \begin{enumerate}
        \item \label{sec:oneeng} На ВС  с максимальной взлетной массой более 5700 кг или с количеством посадочных мест более 9 и всех воздушных судах при осуществлении коммерческой воздушной перевозки пассажиров, выполняющих полеты по маршруту над водным пространством на расстоянии от берега, превышающем предельную дальность полета в режиме планирования, а также при взлете и посадке на аэродроме, где траектория взлета и захода на посадку проходит над водным пространством таким образом, что в случае какого - либо происшествия имеется вероятность вынужденной посадки на воду – размещается, как минимум, по одному спасательному жилету на каждого находящегося на борту человека, расположенных на самолетах таким образом, чтобы человек, для которого он предназначен, мог легко достать его со своего кресла или спального места;
        
        Каждый спасательный жилет оснащается средствами электрического освещения для облегчения обнаружения людей.
        \item Дополнительно к требованиям п. \hyperref[sec:oneeng]{(a)} при эксплуатации самолетов с двумя двигателями при отказе критического двигателя или самолетов с тремя и более двигателями при отказе любых двух двигателей, в любой точке на маршруте или запланированных запасных маршрутах, летно-технические характеристики которых обеспечивают продолжение полета до аэродрома назначения или запасного не снижаясь ни в каком месте до высоты ниже минимально разрешенной и выполнение посадки после пролета всех препятствий вдоль траектории захода на посадку с минимальным запасом высоты, при полете над водным пространством на удалении от береговой черты на расстоянии более чем 120 минут полета в режиме крейсерской скорости или 740 км (400 м. миль), (что наступит раньше) на борту ВС должны находиться спасательные плоты для размещения всех находящихся на борту лиц и пиротехнические устройства для подачи сигналов бедствия ВС.
        \item Для всех других ВС, летно-технические характеристики которых не соответствуют требованиям п.\hyperref[sec:oneeng]{(a)}, в дополнение к оборудованию, предусмотренному в п.п. \hyperref[sec:oneeng]{(a)}, на борту ВС должны находиться спасательные плоты для размещения всех находящихся на борту лиц и пиротехнические устройства для подачи сигналов бедствия воздушного судна, выполняющего полеты над водным пространством на удалении от береговой черты на расстоянии более, чем 30 минут полета или 190 км (100м. миль), что наступит раньше.
        \item Плоты располагаются таким образом, чтобы упростить их использование в аварийной обстановке, и оснащаются таким спасательным оборудованием, включая средства жизнеобеспечения людей, которое отвечает условиям выполняемого полета
    \end{enumerate}
\end{enumerate}
 
