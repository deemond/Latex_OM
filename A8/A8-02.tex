\section{Предполетная подготовка}

\subsection{Организация предполетной подготовки}


Предполетную подготовку экипажа организует и проводит командир ВС перед каждым полетом с учетом конкретной аэронавигационной обстановки и метеоусловий.

Началом предполетной подготовки является время, установленное технологическим графиком работы экипажа согласно времени отправления рейса (приложение А 8.6), указанного в суточном плане полётов Авиакомпании или время прохождения предполетного медицинского осмотра в аэропорту вылета (что наступит раньше).

Во всех случаях должно быть достаточно времени для качественного выполнения всех процедур, предусмотренных технологическим графиком.

Перед началом полётов, но не ранее чем за 2 часа до вылета, все члены экипажа обязаны пройти предполётный медицинский осмотр у врача (фельдшера). При задержке вылета на 6 часов и более предполётный медосмотр проводится повторно. При нахождении на дежурстве, в резерве экипажи проходят предполётный медицинский осмотр перед заступлением на дежурство, в резерв и перед вылетом, если до вылета прошло 6 часов и более.

При выполнении международных полетов с аэродрома иностранного государства, а также при выполнении авиационных работ и других полетов с аэродромов, где отсутствует медицинский работник, имеющий право проводить медицинский осмотр, а также с посадочных площадок, предполетный медицинский осмотр не проводится, решение о допуске членов экипажа ВС к полетам принимает командир ВС с записью в задании на полёт.

В случае неявки одного из членов экипажа или отстранении его по каким-либо причинам от вылета, КВС, а в случае его отсутствия второй пилот, бортинженер (бортмеханик), обязан сообщить диспетчеру ОКВР о необходимости вызова резервного экипажа не позднее, чем за 50 минут до времени вылета при полетах на внутренних линиях, и за 1 час 20 минут - при полетах на международных линиях.

Перед полетом или серии полетов КВС должен заполнить чек-лист готовности к полету (приложение А 8.4), удостоверяющий тот факт, что он удовлетворен результатами подготовки.
Подписанный чек-лист, экземпляр рабочего плана передаются в представительство Авиакомпании.
При отсутствии в пункте вылета представителя Авиакомпании – представителю обслуживающей компании или аэропорта.

Запрос КВС на запуск двигателя (двигателей) на контролируемом аэродроме или запуск двигателя (двигателей) на неконтролируемом аэродроме свидетельствует о принятии решения на начало полета.


\subsection{Учет ограничений летно-технических характеристик}


\paragraph{} Воздушное судно эксплуатируется в соответствии с положениями сертификата летной годности (удостоверения о годности к полетам), РЛЭ и нормами, применяемыми для установления эксплуатационных ограничений летно-технических характеристик.	

Самолеты с одним двигателем, как правило, эксплуатируются только в таких условиях погоды и освещенности, на таких маршрутах и с таким отклонением от них, которые в случае отказа двигателя позволят безопасно совершить вынужденную посадку. 

Разрешается начинать полет только в том случае, когда информация о летно-технических характеристиках, содержащаяся в РЛЭ, в разделе 3 «Подготовка к полету», указывает на то, что в предстоящем полете могут быть выполнены требования, содержащиеся в пункте \hyperref[sec:8222]{\ref* {sec:8222}} \footnote{Тут мы ссылаемся на пункт в текущей главе/файле}

При выполнении указанных требований следует учитывать все факторы, которые влияют на летно-технические характеристики воздушного судна (масса, барометрическая высота, соответствующая превышению аэродрома, температура, уклон ВПП и состояние ВПП, т. е. наличие слякоти, воды и (или) льда). 

\paragraph{} \label{sec:8222}Масса воздушного судна в начале взлета не превышает максимальную взлетную массу, указанную в РЛЭ для барометрической высоты, соответствующей превышению аэродрома, а также для любых других местных атмосферных условий, если они используются в качестве параметров для определения максимальной взлетной массы. 

Расчетная масса воздушного судна к расчетному времени приземления на аэродроме намеченной посадки и на любом запасном аэродроме пункта назначения не превышает максимальную посадочную массу, указанную в РЛЭ для барометрической высоты, соответствующей превышению этих аэродромов, а также для других местных атмосферных условий, если они используются в качестве параметров для определения максимально допустимой массы при посадке. 

Летно-технические характеристики самолета позволяют в случае отказа критического двигателя в любой точке взлета либо прекратить взлет и остановиться в пределах располагаемой дистанции прерванного взлета, либо продолжать взлет и пролететь все препятствия вдоль траектории полета с достаточным запасом. 
 
Летно-технические характеристики самолета позволяют в случае выхода из строя критического двигателя в любой точке на маршруте или запланированных на случай отклонения от него запасных маршрутах продолжать полет до аэродрома, не снижаясь ни в каком месте до высоты ниже минимально разрешенной. 

Летно-технические характеристики самолета позволяют приземлиться на аэродроме намеченной посадки или любом запасном аэродроме после пролета всех препятствий вдоль траектории захода на посадку с минимальным для обеспечения безопасности запасом высоты и с учетом достижения достаточно низкой скорости в пределах располагаемой посадочной дистанции. При этом КВС учитываются предполагаемые различия в технике пилотирования при выполнении захода на посадку и посадки, если это не было учтено при установлении летно-технических характеристик.

РЛЭ эксплуатируемых ВС позволяют определить вышеперечисленные параметры. При этом исходными данными для расчета являются:
\begin{itemize} 
    \item располагаемая длина разбега РДР (Take-off Run Available (TORA));
    \item располагаемая длина прерванного взлета РДПВ (Accelerate Stop Distance Available (ASDA));
    \item располагаемая длина продолженного взлета РДВ (Take-off Distance Available (TODA));
    \item располагаемая посадочная дистанция (Landing Distance Available (LDA));
    \item продольный уклон ВПП;
    \item температура наружного воздуха на аэродроме;
    \item атмосферное давление на аэродроме;
    \item скорость продольной и боковой составляющей ветра;
    \item состояние поверхности ВПП (коэффициент сцепления, вид и толщина слоя осадков);
    \item плоскость ограничения препятствий.
\end{itemize}
При определении располагаемой длины ВПП учитывается возможное ее уменьшение в связи с необходимостью выведения самолета на осевую линию перед взлетом.


\paragraph{} Для определения взлетно-посадочных характеристик с учетом фактических условий, включая состояние ВПП, для ВС иностранного производства, в Авиакомпании применяется программное обеспечение для CL-300. (Инструкции пользователей указанного программного обеспечения публикуются в ИС и в Приложении А 8.12.), для ВС отечественного производства применяются таблицы, графики и номограммы, содержащиеся в РЛЭ для каждого конкретного типа ВС.
При определении градиентов набора высоты должно быть выдержано условие, что минимальный градиент на взлете не менее 3.3\%. Для процедуры ухода на второй круг не менее 2.5\%, если другие значения не оговорены на схемах вылета и захода на посадку. 

\textcolor[rgb]{1,0,0}{В этом параграфе демонстрируется ссылка на другую главу/другой файл документа:}
\hyperref[ssec:pre-check]{смотрим в раздел \ref*{ssec:pre-check}}

При определении градиентов набора высоты должно быть выдержано условие, что минимальный градиент на взлете не менее 3.3\%. Для процедуры ухода на второй круг не менее 2.5\%, если другие значения не оговорены на схемах вылета и захода на посадку. 

\subsection{Действия экипажа в процессе предполетной подготовки}


\textbf{Командир ВС в процессе предполетной подготовки обязан:}


\begin{itemize}
    \item получить задание на полет;
    \item убедиться в соответствии всех записей в задании на полет планируемому полету;
    \item проверить документы членов летного и кабинного экипажа; 
    \item изучить информацию по безопасности полетов, информацию NOTAM, NOTAM Авиакомпании о состоянии аэродромов вылета, назначения, запасных, аэронавигационную обстановку на аэродромах и по трассе; 
    \item получить информацию о технической готовности воздушного судна, наличии и размещении аварийно-спасательного оборудования; 
    \item удостовериться в наличии на борту документов, указанные в подразделе 8.6.4. настоящей главы;
    \item провести анализ записей в бортовом техническом журнале ВС (TLB) с целью определения статуса летной годности ВС, летно-технических характеристик и центровочных данных, возможности выполнения полета при наличии отложенных неисправностей в соответствии с требованиями ПМО/MEL;
    \item При любых сомнениях в статусе ВС, подозрениях на техническую неисправность направить соответ-ствующую информацию, при необходимости приложить фото/провести техническую консультацию по телефонам ОКВР через мобильное приложение WhatsApp / Telegram;
    \item анализ NOTAM аэродромов назначения и запасных, об аэронавигационном обеспечении на аэродромах и по трассе; 
    \item получить информацию о предполагаемой коммерческой загрузке;
    \item изучить метеорологическую обстановку на аэродроме вылета, по маршруту полета, аэродроме назначения и запасных аэродромах;
    \item проверить правильность штурманского расчета и других данных для выполнения полета и уточнить необходимую заправку топливом;
    \item определить конкретные действия экипажа в случае возникновения аварийной обстановки, в том числе при необходимости экстренной посадки после взлета, в зависимости от характера местности, наличия площадок, времени суток и метеоусловий;
    \item принять решение на вылет или отказаться от вылета (перенести вылет) с указанием причин такого решения;
    \item представить в службу движения заполненный план полета (FPL) не позднее, чем за тридцать минут до вылета (при выполнении полетов по МВЛ при наличии брифинга на полет);
    \item провести анализ и передать представителю авиакомпании подписанный рабочий план полета;
    \item проверить наличие необходимого количества топлива, а также центровку и взлетную массу воздушного судна;
    \item готовить ВС и его системы к вылету согласно РЛЭ типа ВС (AFM, FCOM);
    \item провести инструктаж лица, занимающего дополнительное кресло в кабине экипажа, по технике безопасности и использованию аварийно-спасательного оборудования, расположенного около его кресла, ознакомиться с правилами поведения и взаимодействия с экипажем;
    \item принять доклады от каждого члена экипажа о проведенном осмотре и готовности воздушного судна к вылету и выполнить работы, предусмотренные РЛЭ перед вылетом;
    \item проверить срок обновления базы данных при полетах в районы, где необходимо использование  
установленного оборудования; 
\item перед полетом в целях выполнения авиационных работ дополнительно убедиться в том, что на борту установлены приборы и оборудование, необходимые для ожидаемых условий полета. Убедиться в работоспособности указанных приборов и оборудования в соответствии с требованиями РЛЭ.
\end{itemize}


\textbf{Второй пилот в процессе предполетной подготовки обязан:} 
\begin{itemize}
    \item провести анализ NOTAM, информацию о состоянии аэродромов вылета, назначения и запасных, об аэронавигационном обеспечении на аэродромах и по трассе; 
    \item получить информацию о предполагаемой коммерческой загрузке; 
    \item участвовать в изучении метеорологической и аэронавигационной обстановки, выполнить штурманский расчет полета по маршруту, проверить наличие «Инструкции по безопасности» для всех посадочных мест (при полетах на самолете Ан-2); 
    \item лично рассчитать взлетно-посадочные характеристики и предельную коммерческую загрузку, составить центровочный график, когда в аэропорту отсутствует диспетчер по центровке; 
    \item проверить наличие документов аэронавигационной информации на электронных, бумажных носителях и убедиться в их актуальности (при отсутствии штурмана в экипаже);
    \item готовить ВС и его системы к вылету согласно РЛЭ типа ВС; 
    \item проверить размещение груза, багажа и почты по багажным помещениям согласно документам СОПП и докладу бортоператора/бортпроводника, ответственного за загрузку, убедиться в заполнении всех граф сводной сопроводительной ведомости; 
    \item доложить командиру ВС о проведенном осмотре и подготовке ВС и своей готовности к вылету.
\end{itemize}

\dots

\dots

Каждый член экипажа несет персональную ответственность за своевременность, точность, полноту и качество выполнения возложенных на него обязанностей по подготовке к полету.

\subsection{Формы контроля качества и оформления факта и своевременности проведения предполетной подготовки:}
\begin{itemize}
    \item	прохождения медицинского контроля удостоверяется подписью и штампом врача здравпункта в задании на полёт;
    \item	прохождения метеорологической подготовки (если предусмотрено) – подписью и штампом дежурного синоптика в задании на полет или записью в бланке на АМСГ. 
    \item	посадки пассажиров, принятия груза, почты, багажа, бортового питания и т.п. – указанием времени завершения данных операций в бланках документации по этим процедурам.
\end{itemize}

В период предполетной подготовки командир ВС взаимодействует с представителем авиакомпании по вопросам обеспечения рейса и подготовки вылета службами аэропорта. При отсутствии представителя авиакомпании в аэропорту вылета вопросы обеспечения решает командир ВС.

В случае задержек рейса (переносе времени вылета) командир ВС сообщает лично или по установленным каналам связи в АДП аэропорта, информирует диспетчера ОКВР авиакомпании о принятом решении и согласовывает с ними дальнейший порядок выполнения полета.

\subsection{Предполетная подготовка перед полетом по перегонке и перед облетом ВС}


При выполнении предполетной подготовки перед полетом по перегонке ВС без пассажиров для обеспечения необходимого значения центровки необходимо определить вес балластного груза и проконтролировать его правильное размещение. В случае перегона ВС с неисправностями, ухудшающими аэродинамические, летные характеристики ВС (с отказавшем двигателем, выпущенными шасси и т.п.), необходимо определить ограничения по максимальной взлетной массе, скорости на взлете и разобрать технологию взаимодействия при появлении дополнительного (как правило, критического) отказа на важнейших этапах полета.

Перед облетом воздушного судна или его систем необходимо уточнить возможности облета исходя из конкретных и прогнозируемых метеоусловий и установленных ограничений по целям облета, заказать работу (при необходимости) дополнительного наземного оборудования, рассчитать ограничения характеристик ВС при работающих и неработающих системах облета.
