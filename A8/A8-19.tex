\section{Правила и процедуры контроля и управление расходом топлива во время полета}

Общее количество топлива на предстоящий полёт должно быть не менее рассчитанного в соответствии с требованиями, указанными в \hyperref[par:849]{пункте \ref*{par:849}}.

Контроль количества и температуры топлива в полете выполняется экипажем, как правило, над поворотными пунктами маршрута или их траверзе, но не реже одного раза в 30 минут. Для контроля количества топлива используется рабочий план полета, в котором ведутся все необходимые записи.

\subsection{Процедура контроля и управления расходом топлива в полете}

Процедура включает:

\begin{itemize}
    \item контроль количества израсходованного топлива по показаниям расходомеров;
    \item контроль фактического (текущего) остатка топлива по показаниям топливомеров и оценка его соответствия расчетному количеству топлива, указанному в рабочем плане полета;
    \item контроль соответствия израсходованного топлива по расходомерам и фактического остатка по топливомерам;
    \item определение расчетного остатка топлива в аэропорту назначения и принятие необходимых мер в случаях, когда расчетный остаток топлива в аэропорту назначения меньше, чем планировалось перед вылетом.
\end{itemize}

Если фактический расход топлива в полёте значительно превышает расчетный – экипаж определяет причину и принимает меры по уменьшению расхода топлива:

\begin{itemize}
    \item выполняет процедуры по определению возможной утечки топлива в соответствии с РЛЭ (QRH);
    \item меняет эшелон полёта на оптимальный по ветровому режиму или по температурным условиям (ISA);
    \item запрашивает разрешение у органа ОВД на спрямление маршрута полёта.
\end{itemize}

При подтверждении признаков значительной утечки топлива экипаж должен принять решение о посадке на ближайшем запасном аэродроме по маршруту.

Если расчетный остаток топлива на аэродроме назначения меньше, чем планировалось перед вылетом, экипаж:

\begin{itemize}
    \item принимает меры по уменьшению расхода топлива (как указано выше); или
    \item выбирает маршрутный запасной аэродром для промежуточной посадки; или
    \item рассчитывает рубеж ухода на запасной аэродром; или
    \item продолжает полет по плану, повысив контроль над расходом и остатком топлива и за погодой на аэродроме назначения.
\end{itemize}

Действия экипажа при дефиците топлива в маршрутном полёте.

При попадании ВС в нерасчетные условия полета (нерасчетный эшелон, нештатная конфигурация ВС, усиление встречного ветра, обледенение и т.п.), при обнаружении течи топлива, при невозможности использования части имеющегося топлива из-за отказов элементов топливной системы требуется уход на запасной аэродром, а расчетный остаток топлива на аэродроме назначения не обеспечивает уход на запасной с ВПР, КВС должен принять решение:
\begin{itemize}
    \item произвести техническую посадку на ближайшем пригодном аэродроме для дозаправки, как правило, на маршрутном запасном аэродроме; или
    \item следовать до рубежа ухода на запасной аэродром для аэродрома назначения и до пересечения рубежа ухода доложить органу ОВД о принятом решении–следовать на аэродром назначения или на запасной.
\end{itemize}

\subsection{Сообщение экипажа о минимальном остатке топлива - MINIMUM FUEL}

\paragraph{} Командир ВС постоянно следит за тем, чтобы запас топлива на борту был не меньше запаса топлива, который требуется для продолжения полета до аэродрома назначения или запасного при сохранении после посадки количества топлива для полета в течение 30 мин со скоростью полета в зоне ожидания на высоте 450 м (1500 фут) над превышением аэродрома в условиях стандартной атмосферы (Final Reserve Fuel).

 Командир ВС запрашивает у службы ОВД информацию о задержке на прилет, когда непредвиденные обстоятельства могут привести к посадке на планируемом аэродроме с меньшим запасом топлива, чем сумма Final Reserve Fuel и топлива, требующегося для выполнения полета до запасного аэродрома.

 \paragraph{} Действия экипажа при задержке на прилет:

\begin{itemize}
    \item MINIMUM FUEL REQUIRED FOR MISSED APPROACH (MFRMA) - (в CFP обозначается как MIN FUEL RQ.MIS APP) - расчётное количество топлива, указанное в СFP, которое обеспечивает безопасный уход на запланированный запасной аэродром пункта назначения и является суммой ALTERNATE FUEL и FINAL RESERVE. 
    \item В случае, когда по расчету экипажа количество топлива к моменту посадки на аэродроме назначения равно MFRMA и любые непредвиденные обстоятельства (полет в зоне ожидания, временный режим и т.д.) могут привести к посадке на аэродроме назначения с меньшим запасом топлива, чем MFRMA, экипажу необходимо запросить у органа ОВД информацию о вероятности и длительности возможной задержки на прилет.
    \item На основании оценки эксплуатационных условий (статус ВС, состояние ВПП, регламент работы, временные ограничения и т.д.), метеорологической и воздушной обстановки в районе аэродрома назначения и запасного аэродрома, КВС имеет право выбрать аэродром, на который он будет следовать при достижении количества топлива, равного MFRMA (это может быть аэродром назначения, или запасной аэродром по плану, или любой другой запасной аэродром, выбранный в полёте).
    \item В любом случае к моменту достижения количества топлива на борту, равному MFRMA, КВС должен принять решение о следовании на конкретный выбранный им аэродром, т.е. или на аэродром назначения или на запасной аэродром.
    \item Если принято решение следовать на аэродром назначения, то метеоусловия на нем должны быть не ниже эксплуатационного минимума для посадки. В этом случае топливо, предназначенное для перелета на запасной аэродром (ALTERNATE FUEL), может быть израсходовано на предполагаемое ожидание.
\end{itemize}

\paragraph{} \label{par:1934}КВС передает сообщением \textbf{«PAN–PAN–PAN-MINIMUM FUEL»} службе ОВД об остатке минимального запаса топлива, когда он должен выполнить посадку на конкретном аэродроме и рассчитывает, что любое последующее изменение выданного разрешения для полета на этот аэродром может привести к посадке с меньшим запасом топлива, чем Final Reserve Fuel.

Сообщение MINIMUM FUEL информирует службу ОВД о том, что все запланированные варианты использования аэродромов (назначения или запасных) сводятся к возможности посадки только на одном конкретном аэродроме и любое изменение полученного разрешения для полета на этот аэродром может привести к выполнению посадки с запасом топлива меньше, чем Final Reserve.

Примечание. Это не означает аварийную ситуацию, а лишь указывает на возможность возникновения аварийной обстановки, если произойдет какая - либо непредвиденная задержка.

\paragraph{} Сообщение экипажа об аварийном остатке топлива - EМЕGENCY FUEL

КВС объявляет об аварийной ситуации, связанной с запасом топлива на борту сообщением \textbf{«MAYDAY MAYDAY MAYDAY FUEL»}, когда расчет предполагаемого запаса топлива на борту показывает, что после посадки на ближайшем аэродроме, на котором можно совершить безопасную посадку, запас топлива окажется ниже запланированного уровня Final Reserve Fuel.

Значение Final Reserve Fuel для фактических условий конкретного рейса отображается в СFP, рассчитывается для посадочной массы на аэродроме запасном или аэродроме назначения при выполнении полета без запасного аэродрома.

\subsection{Учёт низкой температуры топлива (Температуры начала кристаллизации Fuel Freeze Temperature - FFT)}

\paragraph{} Учёт низкой температуры топлива В полете температура топлива постепенно снижается до значения ТАТ. В некоторых географических районах температура топлива может снизиться до значений температуры начала кристаллизации (FFT). Это характерно для полетов в зимнее время года над Сибирью, Арктической частью России, Канады, Гренландии.

Температура начала кристаллизации топлива не связана с замерзанием воды в топливе и образованием ледяных кристаллов, а связана с образованием парафинов твердой фракции в виде осадков, входящих в состав топлива.

\paragraph{} Учёт низкой температуры топлива Температура начала кристаллизации топлива не является критической с точки зрения потери тяги из-за нарушения потока топлива через насосы к двигателям. Но дальнейшее снижение температуры топлива ведёт к переходу топлива в полутвердое состояние и нарушению потока топлива через насосы. Это происходит при температурах примерно на 6ºС ниже FFT.
Темп снижения температуры топлива около 3°С/час, но может достигать и 12°С/час 


\begin{table}[H]
    
    \begin{center}
        \small  
        \caption{Температура начала кристаллизации (FFT) различных сортов топлива}     
        \begin{tabular}{|p{0.12\textwidth}p{0.08\textwidth}|p{0.12\textwidth}p{0.08\textwidth}|p{0.12\textwidth}p{0.08\textwidth}|}
            \hline
            JET A	&- 40ºС		&JP5 (воен)	&- 46ºС		&TS-1/RT	&- 50ºС\\
            JET A1	&- 47ºС		&JET B	    &- 50ºС		&TH	        &- 53ºС\\
            \hline\hline            
        \end{tabular}        
    \end{center}
\end{table}


\paragraph{} Заправка различными сортами топлива приводит к тому, что в баках ВС образуется смесь с неизвестной FFT. При подготовке к полету температурой начала кристаллизации топлива на борту необходимо считать наивысшую из температур начала кристаллизации сорта топлива, заправленного в 3 последовательных заправках.

Например, в случае 3 последовательных заправок топливом JET A, JET A1, ТС-1 температура начала кристаллизации будет - 40ºС, что соответствует JET A. Чтобы считать FFT = - 50ºС, необходимо произвести 3 последовательных заправки топливом ТС-1. В соответствии с установленными ограничениями, на всех ВС Авиакомпании, не допускается в полете понижение температуры топлива до величины выше на 3ºС температуры начала кристаллизации топлива.

\paragraph{} На этапе расчета СFP и предполетной подготовки рекомендуется:

\begin{itemize}
    \item учитывать географию выполнения рейсов, для анализа возможной кристаллизации топлива в полете в зимнее время. При полетах в ВП РФ и СНГ рассматривать марку топлива ТС-1. Учитывать ограничения по FFT. При планировании полета значение ТАТ должно быть на 4º выше FFT (1 градус на точность прогноза);
    \item определить прогнозируемое значение ТАТ по рабочему плану полёта для тех поворотных пунктов, над которыми прогнозируется Тнв (SAT/OAT) ниже -64ºС. При возможном снижении температуры топлива ниже указанных ограничений, необходимо подготовить новый рабочий план полёта (СFP), предусматривающий иной маршрут полета, иной профиль полета по высотам, возможное увеличения числа М;
    \item определить возможность выполнения рейса по требуемому значению заправляемого топлива и величине загрузки.
\end{itemize}

\paragraph{} Действия экипажа в полете:

\begin{itemize}
    \item учитывать тенденцию к снижению температуры топлива в баках до значения TАТ;
    \item для исключения снижения температуры топлива ниже допустимой выполнять следующие процедуры:    
    \begin{itemize}
        \item увеличить (по возможности) число М полета. Увеличение числа М на 0.01 приводит к повышению ТАТ на 0.5–0.7º. Изменение температуры топлива при изменении ТАТ проявляется в течение периода от 15 до 60 минут;
        \item изменение маршрута полета в сторону более теплых воздушных масс;
        \item снижение до более теплых воздушных масс. Обычно отклонение от оптимального эшелона составляет 3000 -5000 футов. При экстремально низких температурах, возможно, потребуется снижение до 25000 футов.
    \end{itemize}
\end{itemize}
