%\ProvidesPackage{UTairJSC}

\documentclass[a4paper,10pt,twoside]{article}
\usepackage{cmap}
\usepackage[utf8]{inputenc} % Кодировка utf8
\usepackage{amsmath,amssymb}
\usepackage{indentfirst}
\usepackage[english,russian]{babel}
\usepackage[T2A]{fontenc}

\usepackage[inner=2.3cm, outer=1.25cm,top=18mm,bottom=18mm]{geometry}%top=headheight+margin4mm+8
\usepackage{graphicx}%Вставка картинок правильная
\usepackage{float}%"Плавающие" картинки
\usepackage{color}
\usepackage{lastpage}
\usepackage{multirow}
\usepackage[colorlinks=true, linkcolor=blue]{hyperref} %==подключение и определение гиперссылок в доках
\usepackage{setspace}

%===параметры абзаца =================================
\setlength{\parindent}{1cm} %размер отступа первой строки абзаца
\setlength{\parskip}{6pt} %=отступы после абзацев


%=оформление списков:==============
%\usepackage{enumerate}
\usepackage{enumitem}
\setlist{noitemsep}
\setlist[itemize]{leftmargin=10mm}
\setlist[enumerate]{leftmargin=10mm}
%\setlist{leftmargin=10mm}
\setlist{labelindent=40mm} % < Usually a good idea
%\setlist[itemize,1]{label=$\triangleleft$}
%\setlist[enumerate,1]{label=\arabic*., ref=\arabic*}
%==конец оформления списков=========

%===========Колонтитулы====================
\usepackage{fancybox,fancyhdr}
\renewcommand{\headrulewidth}{1pt}
\renewcommand{\footrulewidth}{1pt}
%==Команды для вставки в колонтитулы переменных данных
\newcommand{\UTdocHead}[1]{\fancyhead[L]{\includegraphics[width=0.13\textwidth]{../utlogo.png}}
    \fancyhead[C]{}
    \fancyhead[R]{\textcolor{blue}{\small{ РПП АО ЮТэйр\\#1}}}}%==Название документа

\newcommand{\UTdocPageFooter}[2]{\fancyfoot[LO,RE]{Дата редакции:#2}
    \fancyfoot[C]{}
    \fancyfoot[LE,RO]{#1.\arabic{page} из \pageref{LastPage}~}}%==страницы
%=============конец колонтитулов===============


%== нумерует новый параграф, увеличивает счетик на единицу. 
%== Subsect должен обнуляться принудительно обнулятьсяпосле объявления раздела
\newcommand{\newPar}{\paragraph{\arabic{section}.\arabic{subsect} } \stepcounter{subsect}}


%== для внесения изменений в текст. Оформляется синим, сбоку вертикальная линия
%== вносятся данные в нижний колонтитул и он определяется только для страницы с изм параграфом
\newcommand{\chgdPar}[3]{\marginpar{\rule[-8ex]{1pt}{7ex}}%
     \textcolor[rgb]{0,0,1}{#3}
     \fancypagestyle{fplain}{
     \fancyfoot[LO,RE]{Дата редакции:#1}
     \fancyfoot[C]{Изменение №:#2}}
     \thispagestyle{fplain} }

     \setcounter{secnumdepth}{6}%глубина нумерования

\begin{appendix}

    \UTdocHead{Приложение А8.1 Методы определения минимальных абсолютных высот полета}
    \UTdocPageFooter{8.1}{01.01.2022}
    %\setcounter{page}{1}
    \pagestyle{fancy}
    \tableofcontents
   % \newcounter {subsect}   
    \setlength{\headheight}{22pt}.
    \addtolength{\topmargin}{-2pt}


\section{Методы определения минимальных абсолютных высот полета}    
\subsection{Методика расчета минимальных высот полета}

\setcounter{subsect}{0}

 В целях предотвращения столкновений ВС с наземными препятствиями авиакомпания устанавливает правила расчета и использования минимальных безопасных высот полета и эшелонов полета.

 Минимальные высоты полета контролируются по барометрическим высотомерам (основным и контрольным), установленным по давлению QNH, QFE или QNE в соответствии с методикой, приведенной в данном разделе. 

 Рассчитанные значения минимальных высот полета могут корректироваться в случаях, когда изменились данные АНИ, на которых основаны их расчеты, либо при обнаружении любых неточностей в расчетах.

 \chgdPar{17/07/22}{3}{Экипаж в процессе предварительной или предполетной подготовки должен самостоятельно определить значения минимальных безопасных высот\footnote[1]{ООООООчень высоко} в соответствии с "Методикой определения минимальных высот полета".} 



 Независимо от имеющихся в распоряжении экипажа сведений о заранее рассчитанных минимальных высотах полета, на борту ВС при выполнении полета должны быть такие документы АНИ, которые позволят экипажу самостоятеkьно определить минимальную высоту полета при любом возможном отклонении от запланированного маршрута полета.
 
    \subsection{Минимальные высоты при полетах по РФ}
    \setcounter{subsect}{0}
    
     За исключением случаев, в которых это необходимо при осуществлении взлета, посадки или указанных в пункте 1.2, запрещено выполнять полет воздушного судна: 
    а)  над территориями населенных пунктов и над местами скопления людей при проведении массовых мероприятий - ниже высоты, допускающей в случае отказа двигателя аварийную посадку без создания чрезмерной опасности для людей и имущества на земле, и ниже высоты 300 м над самым высоким препятствием в пределах горизонтального радиуса в 500 м вокруг данного воздушного судна; 
    б)  в местах, не указанных в подпункте «а», на высоте менее 150 м. 
    
     Полеты с отклонением от требований подпункта «b», пункта 1.1 разрешены в случаях, когда это не создает опасности для людей или имущества на земле при выполнении авиационных работ или летном обучении под наблюдением пилота-инструктора. 
    
     В зависимости от правил полетов, рельефа местности, скорости полета устанавливается минимальный запас высоты полета (МЗВ) воздушного судна над наивысшими препятствиями в соответствии с таблицей:
    
    \begin{table}[H]
        \begin{center}
        \begin{tabular}{|p{0.32\textwidth}|p{0.17\textwidth}|p{0.10\textwidth}|p{0.17\textwidth}|p{0.10\textwidth}|}
        \hline
            & \multicolumn{2}{c|}{ППП}& \multicolumn{2}{c|}{ПВП} \\
            \cline{2-5}
            %\raisebox{1.5ex}[0cm][0cm]{Безопасные высоты (эшелоны) полета}
            \textbf{Безопасные высоты (эшелоны) полета}&\textbf{Учет препятствий}&\textbf{МЗВ, м}&\textbf{Учет препятствий}&\textbf{МЗВ, м}\\
            \hline
            Безопасная высота круга (Нб кр)&5 км от оси маршрута полетов по кругу&100&5 км от оси маршрута полетов по кругу&200\\
            \hline
            Безопасная высота полетов ниже нижнего (безопасного) эшелона (Нб ниж (без) эш. над:&\multirow[c]{5}{=}{днем-в пределах ширины воздушной трассы, маршрута; ночью -  $\pm$8 км от оси воздушной трассы, маршрута.}&&\multirow[c]{5}{=}{$\pm$25 км от оси воздушной трассы, маршрута}&\\
            над равнинной и холмистой местностью на скоростях: Vист300км/ч&&100&&200\\
            горной местностью, горы до 2000м&&300&&300\\
            горы выше 2000м&&600&&600\\[20pt]
            &&&&\\
            \hline
            Нижний безопасный эшелон полета (Н ниж.без. эш)&в пределах ширины воздушной трассы или маршрута полета.&600&$\pm25$ км от оси воздушной трассы, маршрута&600\\
            \hline
            Безопасная высота полетов ниже нижнего (безопасного) эшелона при выполнении полетов с целью оказания срочной медицинской помощи, поисково-спасательных работ и тренировочных полетов:&&&&\\
            над равнинной и холмистой местностью на скоростях V 300 км/ч&&&&\\
            - днем&$\pm5$ км от оси маршрута&не ниже
            50&&\\
            - ночью&$\pm8$ км от оси маршрута &не ниже
            250&&\\
            \hline
        \end{tabular}
        \end{center}
    \end{table}
    
    Перед каждым полетом по ППП в пределах СНГ экипаж обязан определить по сборникам АНИ и знать:
    Нб кр – высоту полета по кругу полетов над аэродромом;
    Нперех – высоту перехода на аэродроме вылета;
    Нб района аэр – безопасную высоту полета района аэродрома в радиусе 50 км от  КТА (район аэроузла).
    
    
     В связи с тем, что измерение высот полета производится барометрическими высотомерами, которые при изменении температуры относительно МСА имеют методическую погрешность и при температурах наружного воздуха менее стандартной + 15ºС завышают значение измеренной высоты, необходимо при расчете различного рода безопасных высот, минимально допустимых высот полета на схемах захода на посадку, учитывать соответствующую температурную поправку. 
    
    Данные высоты рассчитываются по минимальным температурам на аэродроме, отмеченной за период многолетних наблюдений и не требуют корректировки по фактическим условиям полета.
    
     Расчет относительной минимальной безопасной высоты круга полетов над аэродромом ($H _{\textcyrillic{МБВ к }  QFE}$) осуществляется по формуле:
    
    $$
    H _{\textcyrillic{МБВкQFE}}=\Delta H_{\textcyrillic{преп}} +\textcyrillic{МЗВ}+\Delta H_t
    $$
    
    где:
    \begin{itemize}
        \item$\Delta H_{\textcyrillic{преп}}$- относительная высота наивысшего препятствия от низшего порога взлетно-посадочной полосы в полосе шириной 10 км (по 5 км в обе стороны от оси маршрута полета по кругу), округляемая до 30 футов (10 м) в сторону увеличения;
        \item МЗВ - минимальный запас высоты над наивысшим препятствием в зоне учета препятствий: при полете по правилам визуальных полетов - 330 футов (100 м); при полете по правилам полетов по приборам - 660 футов (200 м);
        \item$\Delta H_t$- температурная поправка высотомера, определяемая по формуле:
    \end{itemize}
    $$
    \Delta H_t = H\times\left(\frac{15-t_0}{273+t_0-0{,}5\times L_0\times  (H+H_\textcyrillic{аэр})}\right)
    $$
    
    где:
    \begin{itemize}
        \item $t_0 = t_\textcyrillic{аэр} + L_0\times Н_\textcyrillic{аэр}$ - температура на аэродроме, приведенная к среднему уровню моря;
        \item $t_\textcyrillic{аэр}$ - минимальная по многолетним наблюдениям температура воздуха у земли на аэродроме за период не менее 5 лет. Значение t аэр указывается на картах захода на посадку;
        \item $L_0$ - температурный градиент 0,0065°С/м;
        \item $H_\textcyrillic{аэр}$ - абсолютная высота низшего порога взлетно-посадочной полосы.
    \end{itemize}
    Полученное значение относительной минимальной безопасной высоты круга полетов округляется в большую сторону с кратностью 100 футов (50 м) и публикуется на карте захода на посадку.
    
    
     
    Расчет абсолютной минимальной безопасной высоты круга полетов на аэродромах гражданской авиации, используемых для первоначального обучения пилотов гражданских воздушных судов ($H_{\textcyrillic{МБВ к }QNH}$), осуществляется по формуле:
    $$
    H _{\textcyrillic{МБВ к }  QNH}=\Delta H_{\textcyrillic{преп}} +\textcyrillic{МЗВ}+\Delta H_t
    $$
    где:
    \begin{itemize}
        \item$\Delta H_{\textcyrillic{преп}}$- абсолютная высота наивысшего препятствия в зоне учета препятствий в полосе шириной 10 км (по 5 км в обе стороны от оси маршрута полета по кругу), округляемая до 30 футов (10 м) в сторону увеличения;
        \item МЗВ - минимальный запас высоты над наивысшим препятствием в зоне учета препятствий:
        при полете по правилам визуальных полетов - 330 футов (100 м);
        при полете по правилам полетов по приборам - 660 футов (200 м);
        \item$\Delta H_t$- температурная поправка высотомера, приведенной в пункте 1 настоящего приложения. Значение t аэр указывается на картах захода на посадку.
    \end{itemize}
    
    Полученное значение абсолютной минимальной безопасной высоты круга полетов округляется в большую сторону с кратностью 100 футов (50 м) и публикуется на карте захода на посадку.  
    
    
     Расчет минимальной относительной безопасной высоты полета в районе аэродрома (аэроузла) \\ ($H _{\textcyrillic{МБВpa}QFE}$) осуществляется по формуле:
    $$
    H _{\textcyrillic{МБВpa}QFE}=\Delta H_{\textcyrillic{преп}} +\textcyrillic{МЗВ}+\Delta H_t
    $$
    где:
    \begin{itemize}
    \item$\Delta H_{\textcyrillic{преп}}$ относительная высота наивысшего препятствия от низшего порога взлетно-посадочной полосы в районе аэродрома в радиусе не более 46 км от контрольной точки аэродрома (КТА) с учетом буферной зоны шириной 9 км, устанавливаемой вокруг любого заданного сектора. Если высота наивысшего препятствия относительно низшего порога взлетно-посадочной полосы в буферной зоне превышает препятствия в основной зоне, то оно используется для расчета;
    \item МЗВ - минимальный запас высоты над наивысшим препятствием в районе аэродрома в радиусе не более 46 км от контрольной точки аэродрома (КТА) с учетом буферной зоны:
    \begin{itemize}
        \item в горной местности (местности с абсолютным превышением над средним уровнем моря 1000 м и более, а также с пересеченным рельефом и относительными превышениями 500 м и более в радиусе 25 км) составляет 2000 футов (600 м);
        \item в равнинной местности (местности с относительными превышениями рельефа менее 200 м в радиусе 25 км) и холмистой местности (местности с пересеченным рельефом и относительными превышениями от 200 м до 500 м в радиусе 25 км) составляет 1000 футов (300 м);
    \end{itemize}
    
    \item$\Delta H_t$- температурная поправка высотомера, определяемая по формуле, приведенной в пункте 1 настоящего приложения. Значение t аэр указывается на схемах стандартного маршрута вылета по приборам, схемах стандартного маршрута прибытия по приборам и на карте захода на посадку.
    \end{itemize}
    
    В зависимости от расположения препятствий минимальная относительная безопасная высота полета определяется по секторам.
    При разнице между относительными высотами менее 330 футов (100 м) может устанавливаться минимальная относительная высота, применимая ко всем секторам.
    
    Полученное значение минимальной относительной безопасной высоты полета в районе аэродрома округляется в большую сторону с кратностью 100 футов (50 м) и публикуется на схемах стандартного маршрута вылета по приборам, схемах стандартного маршрута прибытия по приборам и на карте захода на посадку.
    
    Минимальная относительная безопасная высота полета в районе аэроузла устанавливается по наибольшему значению минимальной относительной безопасной высоты полета в районах аэродромов, входящих в аэроузел.
    
    
    
     Расчет минимальной абсолютной безопасной высоты полета в районе аэродрома (районе аэроузла) ($H _{\textcyrillic{МБВpa}QNH}$) осуществляется по формуле:
    
    $$
    H _{\textcyrillic{МБВpa}QNH}=H_{\textcyrillic{преп}} +\textcyrillic{МЗВ}+\Delta H_t
    $$
    
    где:
    
    \begin{itemize}
        \item $H_{\textcyrillic{преп}}$ - абсолютная высота наивысшего препятствия в районе аэродрома в радиусе не более 46 км от контрольной точки аэродрома (КТА) с учетом буферной зоны шириной 9 км. Если высота наивысшего препятствия в буферной зоне превышает высоту препятствия в основной зоне, то оно используется для расчета;
        \item МЗВ - минимальный запас высоты над наивысшим препятствием в районе аэродрома в радиусе не более 46 км от контрольной точки аэродрома (КТА) с учетом буферной зоны:
        \begin{itemize}
            \item в горной местности составляет 2000 футов (600 м);
            \item в равнинной и холмистой местности составляет 1000 футов (300 м);
    \end{itemize}
        \item $\Delta H_t$- температурная поправка высотомера, определяемая по формуле, приведенной в пункте 1 настоящего приложения, для которой
        $\Delta H_{\textcyrillic{преп}}$ - относительная высота наивысшего препятствия от низшего порога взлетно-посадочной полосы в радиусе не более 46 км от контрольной точки аэродрома (КТА) с учетом буферной зоны шириной 9 км. Значение $t_\textcyrillic{аэр}$ публикуется на схемах стандартного маршрута вылета по приборам, схемах стандартного маршрута прибытия по приборам и на карте захода на посадку.
    \end{itemize}
    
    В зависимости от расположения препятствий минимальная абсолютная безопасная высота полета определяется по секторам.
    
    При разнице между относительными высотами менее 330 футов (100 м) может устанавливаться минимальная относительная высота, применимая ко всем секторам.
    
    Полученное значение минимальной абсолютной безопасной высоты полета в районе аэродрома округляется в большую сторону с кратностью 100 футов (50 м) и публикуется на схемах стандартного маршрута вылета по приборам, схемах стандартного маршрута прибытия по приборам и на карте захода на посадку.
    
    Минимальная абсолютная безопасная высота полета в районе аэроузла устанавливается по наибольшему значению минимальной абсолютной безопасной высоты полета в районах аэродромов, входящих в аэроузел.
    
    
     Определение высоты перехода.
    
    Относительная высота перехода ($H_{\textcyrillic{перехQFE}}$) устанавливается не ниже наивысшей минимальной относительной безопасной высоты полета в районе аэродрома (районе аэроузла), определяемой в соответствии с пунктом 3 настоящего приложения.
    
    Абсолютная высота перехода ($H_{\textcyrillic{перехQNH}}$) устанавливается не ниже наивысшей минимальной абсолютной безопасной высоты полета в районе аэродрома (районе аэроузла), определяемой в соответствии с пунктом 4 настоящего приложения.
    
     Расчет абсолютной безопасной высоты полета ниже нижнего (безопасного) эшелона при полете по правилам полетов по приборам при установке на высотомере давления QNH района ($H_{\textcyrillic{БНQNH}}$) осуществляется по формуле:
    $$
    H_{\textcyrillic{БНQNH}}=\left( H_{\textcyrillic{преп}}+ \textcyrillic{МЗВ}\right) \times\left( \frac{285}{273+t_{\textcyrillic{з}}}\right) 
    $$
    где:
    \begin{itemize}
        \item $H_{\textcyrillic{преп}}$ - абсолютная высота наивысшего препятствия на участке маршрута в пределах ширины не менее 16 км (по 8 км в обе стороны от оси маршрута);
        \item МЗВ - минимальный запас высоты над наивысшим препятствием:
        в горной местности составляет 2000 футов (600 м);
        в равнинной и холмистой местностях составляет 1000 футов (300 м);
        \item $t_{\textcyrillic{з}}$- наименьшая температура воздуха у земли по маршруту полета (местной воздушной линии) в районе наивысшего препятствия.
    \end{itemize}
    Абсолютная безопасная высота полета ниже нижнего (безопасного) эшелона при установке на высотомере давления QNH района может быть рассчитана с применением навигационной линейки.
    
    
     Расчет нижнего (безопасного) эшелона полета ($H_{\textcyrillic{НЭQNE}}$) осуществляется по формуле:
    $$
        H_{\textcyrillic{НЭQNE}}\ge \left( H_{\textcyrillic{преп}}+ \textcyrillic{МЗВ}+\Delta H_{\textcyrillic{бар}}\right) \times\left( \frac{285}{273+t_{\textcyrillic{з}}}\right) 
    $$
    
    где:
    \begin{itemize}
        \item $H_{\textcyrillic{преп}}$ - абсолютная высота наивысшего препятствия в пределах:
        маршрута обслуживания воздушного движения (полета) при полете по правилам визуальных полетов;
        не менее 16 км (по 8 км в обе стороны от оси маршрута обслуживания воздушного движения) при полете по правилам полетов по приборам;
        \item  МЗВ - минимальный запас высоты над наивысшим препятствием 2000 футов (600 м);
        \item $\Delta H_{\textcyrillic{бар}} = (QNE - QNH_{\textcyrillic{района}})\times \Delta h$ 
        где:
        \begin{itemize}
            \item $QNH_{\textcyrillic{района}}$ - минимальное давление, приведенное к уровню моря по стандартной атмосфере по району полета или по маршруту обслуживания воздушного движения;
            \item $\Delta h$ - барометрическая ступень. При установке на шкале высотомера: давления 1013,2 гПа  $\Delta h$= 8.3 м/гПа, 760 мм ртутного столба $\Delta h$ = 11 м/мм ртутного столба;
        \end{itemize}
        \item $t_{\textcyrillic{з}}$ - наименьшая температура воздуха у земли по маршруту обслуживания воздушного движения (полета) в районе наивысшего препятствия.
    \end{itemize}
    Полученное значение увеличивается до ближайшего эшелона. 
    
    
    Расчет высоты эшелона перехода района аэродрома в радиусе не более 46 км от контрольной точки аэродрома (КТА) осуществляется:
    
    %\begin{enumerate}[label=\protect\fcolorbox{blue}{yellow}{\protect\ding{\value*}}]
    \begin{enumerate}[label=\alph*), ref=\alph*]
        \item по давлению QFE ($H_{\textcyrillic{ЭперехQFE}}$):
        $$
        H_{\textcyrillic{ЭперехQFE}}\ge H_{\textcyrillic{перехQFE}}+\textcyrillic{ЗПС}+H_{\textcyrillic{аэр}}
        $$
        где:
        \begin{itemize}
            \item $H_{\textcyrillic{аэр}}$ - значение относительной высоты перехода в районе аэродрома в соответствии с пунктом 5 настоящего приложения;
            \item ЗПС - установленное значение переходного слоя 1000 футов (300 м);
        \end{itemize}
        \item по давлению QNH аэродрома ($H_{\textcyrillic{ЭперехQNH}}$):
        $$
        H_{\textcyrillic{ЭперехQNH}}\ge H_{\textcyrillic{перехQNH}}+\textcyrillic{ЗПС}
        $$
        где:
        \begin{itemize}
            \item $H_{\textcyrillic{ЭперехQNH}}$ - значение относительной высоты перехода в районе аэродрома в соответствии с пунктом 5 настоящего приложения;
            \item ЗПС - установленное значение переходного слоя 1000 футов (300 м).
        \end{itemize}
    \end{enumerate}
    
    Расчет применяется при условии, что атмосферное давление аэродрома, приведенное к уровню моря, равняется давлению QNE.
    
    При значении давления QNH аэродрома (давления QFE) меньше давления QNE на величину не более 36 гПа/27 мм ртутного столба в качестве нижнего (безопасного) эшелона устанавливается следующий верхний эшелон, а более 36 гПа/27 мм ртутного столба - очередной верхний эшелон.
    
    Нижний (безопасный) эшелон (эшелон перехода) района аэроузла устанавливается не ниже наибольшего значения нижнего (безопасного) эшелона (эшелона перехода) районов аэродромов, входящих в аэроузел.
    
    
     Расчет абсолютной высоты перехода района Единой системы (установленной части района Единой системы) ($H_{\textcyrillic{перехЕСОрВДQNH}}$) осуществляется по формуле:
    $$
    H_{\textcyrillic{перехЕСОрВДQNH}}=\left(H_{\textcyrillic{преп}}+\textcyrillic{МЗВ}\right)\times \left(\frac{285}{273+t_{\textcyrillic{з}}}\right)
    $$
    где:
    \begin{itemize}
        \item $H_{\textcyrillic{преп}}$ - абсолютная высота наивысшего препятствия в пределах района Единой системы (установленной части района Единой системы);
        \item МЗВ - минимальный запас высоты над наивысшим препятствием в пределах района Единой системы (установленной части района Единой системы) 2000 футов (600 м);
        \item $t_{\textcyrillic{з}}$ - минимальная температура воздуха у земли в районе наивысшего препятствия в пределах района Единой системы (установленной части района Единой системы).
    \end{itemize}
    
    Абсолютная высота перехода района Единой системы (установленной части района Единой системы) с учетом температурной поправки высотомера может быть определена с применением навигационной линейки.
    
    
     Расчет высоты эшелона перехода в районе Единой системы ($H_{\textcyrillic{перехЕСОрВД}}$) осуществляется по формуле:
    $$
        H_{\textcyrillic{перехЕСОрВД}}=H_{\textcyrillic{перехЕСОрВДQNH}}+1000
    $$
    где:
    \begin{itemize}
        \item $H_{\textcyrillic{перехЕСОрВД}}$- значение абсолютной высоты перехода в пределах района Единой системы (установленной части района Единой системы), определяемой в соответствии с пунктом 9 настоящего приложения;
        \item 1000 футов (300 м) - значение установленной величины переходного слоя.
    \end{itemize}
    
    
    Расчет для условия, что атмосферное давление в районе Единой системы (установленной части района Единой системы), приведенное к уровню моря по стандартной атмосфере, соответствует давлению QNE.
    
    При значении давления в районе Единой системы (установленной части района Единой системы), приведенного к уровню моря по стандартной атмосфере, меньше давления QNE на величину более 13 гПа/10 мм ртутного столба, но не более 36 гПа/27 мм ртутного столба, в качестве нижнего (безопасного) эшелона устанавливается следующий верхний эшелон, а более 36 гПа/27 мм ртутного столба - очередной верхний эшелон.
    
    
     Расчет минимальной абсолютной высоты полета в зоне, образованной линиями параллелей и меридианов картографической сетки ($H_{\textcyrillic{Змин}}$), осуществляется по формуле:
    $$
    H_{\textcyrillic{Змин}}=H_{\textcyrillic{рел}}+\textcyrillic{МЗВ}
    $$
    где:
    \begin{itemize}
        \item $H_{\textcyrillic{рел}}$ - абсолютная высота наивысшего препятствия в пределах зоны, образованной линиями параллелей и меридианов картографической сетки. До северной широты 70° шаг сетки 1° по широте и долготе, свыше 75° - 5° по долготе и 1° по широте, свыше широты 85° не применяется;
        \item МЗВ - установленное значение запаса высоты над препятствием при полетах по правилам полетов по приборам вне маршрутов обслуживания воздушного движения в пределах зоны, образованной линиями параллелей и меридианов картографической сетки, в горной местности - 2000 футов (600 м), в равнинной и холмистой местностях - 1000 футов (300 м).
        
    \end{itemize}
    

    

\subsection{Минимальные высоты при полетах по МВЛ}
\setcounter{subsect}{0}

 Минимальные абсолютные высоты полета устанавливаются каждым государством над его территорией в зависимости от правил полетов и характера местности с учетом искусственных препятствий.


 В соответствии с правилами ICAO по выдерживанию высот, фирма «Jeppesen» наносит на карты высоты, приведенные в таблице А-2. 

Высоты, указываемые на картах фирмы «Jeppesen».

\begin{table}[H]
    \centering
    \scriptsize
    \begin{tabular}{|p{0.09\textwidth}|p{0.19\textwidth}|*{4}{p{0.10\textwidth}|}p{0.13\textwidth}|}
    \hline
    \textbf{Название}&\textbf{Полное наименование}&\textbf{Обозн. на картах}&\textbf{МЗВ}&\textbf{Полоса учета преп.}&\textbf{Уровень отсчета}&\textbf{Примечания}\\
        \hline
        MORA&Minimum off Route Altitude – мин. абсолютная безопасная высота полета вне маршрута&4500a&\multirow{6}{=}{300м (1000ft)в равнинной и холмистой местности (5000ft и менее).        
        
        600м (2000ft) в горной  местности (более 5000ft)  МЗВ}&$\pm10$nm&\multirow{3}{=}{QNH}&\\
        \cline{1-3} \cline{5-5} \cline{7-7}
        Grid MORA&Сеточная MORA&$2_{7}$(в сотнях футов 
        2700 ft)&&Площадь&&Если указан $\pm$, то MORA определяется приблизительно, но безопасность обеспечивается.\\
        \cline{1-3} \cline{5-5} \cline{7-7}
        MEA&\multirow{2}{=}{Minimum Enroute IFR Altitude – мин. безопасная высота полета по маршруту по ППП.}&7600&&$\pm8$км&&\multirow{2}{=}{Обеспечивает уверенный прием сигнала от РТС по всему участку полета}\\[50pt]
        &&FL 80&&&QNE&\\
        \cline{1-3} \cline{5-7}
        MOCA&Minimum Obstruction Clearance Altitude - мин. абсолютная безопасная высота пролета препятствий в полосе ширины маршрута.&5000T&&$\pm8$км&\multirow{5}{=}{QNH}&В США обеспечивает уровень приема сигнала на удалении 22nm от VOR\\
        \cline{1-3} \cline{5-5} \cline{7-7}
        MHA&Minimum Holding Altitude - мин. абсолютная безопасная высота   в зоне ожидания.&MHA 3000&&В пределах границ зоны ожидания&&\\
        \cline{1-5} \cline{7-7}
        MSA&Minimum Sector Altitude - мин. абсолютная безопасная высота в секторе.  &2000 MSA&300м 1000ft&Радиус 25nm&&\\
        \cline{1-5} \cline{7-7}
        MRA&Minimum Reception Altitude - мин. высота уверенного приема сигнала.&MRA 7000&---&---&&Указывается у пункта приема сигнала.\\
        \cline{1-5} \cline{7-7}
         MCA&Minimum Crossing Altitude - мин. абсолютная высота полета, на которой ВС должно пересекать контрольные точки при следовании в сторону большей MEA.&MCA 5000&---&---&&\\
         \hline
    \end{tabular}
\end{table}

\begin{table}[H]
    \centering
    \scriptsize
    \begin{tabular}{|p{0.09\textwidth}|p{0.19\textwidth}|*{4}{p{0.10\textwidth}|}p{0.13\textwidth}|}
    \hline
    \textbf{Название}&\textbf{Полное наименование}&\textbf{Обозн. на картах}&\textbf{МЗВ}&\textbf{Полоса учета преп.}&\textbf{Уровень отсчета}&\textbf{Примечания}\\
        \hline
    MAA&Maximum Authorized Altitude – опубликованные максимально разрешенные высоты&MAA FL290  MAA 25000&---&---&QNE&Если не опубли-кована MAA, можно использовать верхний эшелон верхнего (нижн.) воздушного пространства.\\
    \hline    
    \end{tabular}
\end{table}

\textbf{Примечание:}
\textit{В соответствии с правилами ICAO горная местность – местность, где наивысшее препятствие 1524м (5000 ft) от среднего уровня моря или выше. }


    
\subsection{Ошибки барометрических высотомеров}
\setcounter{subsect}{0}

 Для расчета дополнительных запасов по высоте необходимо учитывать, что:
\begin{enumerate}[label=\alph*), ref=\alph*]
    \item в случае резкого падения барометрического давления, ограниченного по времени (15 минут) и в пространстве, вызванного аномальными явлениями погоды (ураган, шторм, торнадо и др.) барометрические высотомеры могут индицировать высоту, превышающую истинное ее значение на 30-60 м (100-200 ft);
    \item при полете над горной местностью на высотах, близких к минимальным безопасным в условиях сильного ветра, барометрические высотомеры могут индицировать высоты, превышающие истинные значения высот
\end{enumerate}
   

    
\subsection{Практическое использование минимальных абсолютных высот полета}
\setcounter{subsect}{0}

 Перед каждым полетом по ППП экипаж ВС:
\begin{enumerate}[label=\alph*), ref=\alph*]
    \item по сборникам аэронавигационной информации в процессе предполетной подготовки определяет высоту полета по схеме захода на посадку, минимальную безопасную высоту в районе аэродрома (МБВ), безопасную высоту полета в районе подхода, безопасную высоту круга;
    \item рассчитывает высоту нижнего безопасного эшелона;
    \item определяет MEA, MOCA, MORA, MVA.
\end{enumerate}


Перед каждым полетом по ПВП экипаж ВС рассчитывает:
\begin{enumerate}[label=\alph*), ref=\alph*]
    \item безопасную высоту в районе аэродрома при полете ниже нижнего эшелона;
    \item минимальную безопасную высоту полета по маршруту (району авиационных работ) ниже нижнего эшелона;
    \item высоту нижнего безопасного эшелона.    
\end{enumerate}


Полеты ниже высот MEA, MOCA, MORA, MVA производятся только в нижнем воздушном пространстве ниже нижнего эшелона по правилам визуальных полетов, но не ниже безопасной высоты, установленной для такого вида полетов.


Запрещается выполнять полеты ниже минимальных безопасных высот, за исключением этапов захода на посадку (на конечном участке захода посадку до посадки), ухода на второй круг (от точки начала ухода до завершения маневра ухода) и при выполнении схемы выхода (от взлета до точки, где ВС достигло значения минимальной безопасной высоты полета), при выполнении аварийной вынужденной посадки вне аэродрома.

Экипаж обязан владеть информацией о значениях минимальных высот полета на всех его участках полета.


 Снижение воздушного судна ниже минимальной безопасной высоты при заходе на посадку по ППП допускается только при:
\begin{enumerate}[label=\alph*), ref=\alph*]
    \item  устойчивой работе бортового навигационного оборудования;
    \item  устойчивой работе наземного навигационного оборудования;
    \item  знании экипажем и диспетчером местоположения ВС.       
\end{enumerate}


Снижение ниже ВПР (DA(H)), минимальной абсолютной/относительной высоты снижения (МВС(MDA(H)) в любом случае не производится до тех пор, пока: 
\begin{enumerate}[label=\alph*), ref=\alph*]
    \item  не будет установлен и поддерживаться надежный визуальный контакт с ориентирами;
    \item  запас высоты над препятствиями, положение ВС в пространстве, направление его движения не будет обеспечивать безопасного выполнения посадки.       
\end{enumerate}
	
\textbf{За соблюдение минимальных безопасных высот полета ответственность несет командир воздушного судна.}




\end{appendix}
